%% The following is a directive for TeXShop to indicate the main file
%%!TEX root = diss.tex

\chapter{Glossary}

%%\makeglossaries


% use \acrodef to define an acronym, but no listing
%\acrodef{UI}{user interface}
%\acrodef{UBC}{University of British Columbia}

% The acronym environment will typeset only those acronyms that were
% *actually used* in the course of the document
%\begin{acronym}[ANOVA]
%\acro{ANOVA}[ANOVA]{Analysis of Variance, statistical 
%techniques to identify sources of variability between groups}

% \acro{ $\overline{w^{,} \theta^{,}_{s}}$ }{Surface heat flux in $\frac{Watts}{meters^{2}}$}

% \acro{$\gamma$}{Initial and upper potential temperature lapse rate in Kelvin}

\begin{acronym}
\acro{CBL}[CBL]{Convective Boundary Layer}
\end{acronym}

\begin{acronym}
\acro{DNS}[DNS]{Direct Numerical Simulation}
\end{acronym}

\begin{acronym}
\acro{EL}[EL]{Entrainment Layer}
\end{acronym}

\begin{acronym}
\acro{EZ}[EZ]{Entrainment Zone}
\end{acronym}


\begin{acronym}
\acro{FA}[FA]{Free Atmosphere}
\end{acronym}

\begin{acronym}
\acro{FFT}[FFT]{Fast Fourrier Transform}
\end{acronym}

\begin{acronym}
\acro{GCM}[GCM]{General Circulation Model}
\end{acronym}

\begin{acronym}
\acro{ML}[ML]{Mixed Layer}
\end{acronym}

\begin{acronym}
\acro{LES}[LES]{Large Eddy Simulation}
\end{acronym}

\begin{acronym}
\acro{Ri}[Ri]{Richardson Number\acroextra{, the bulk Richardson Number is $\frac{gh}{\overline{\theta}_{ML}} \frac{\Delta \theta}{w^{*2}}$, 
$\Delta \theta = \overline{\theta}(h_{1})-\overline{\theta}(h_{0})$}}
\end{acronym}

\begin{acronym}
\acro{TKE}[TKE]{Turbulence Kinetic Energy}
\end{acronym}

%\acro{DNS}{DNS}{Direct Numerical Simulation}
%\acro{GCM}{GCM}{Global Circulation Model}
%\acro{TKE}{TKE}{Turbulence Kinetic Energy}

%\begin{acronym}
%=======
%\acro{LES}{LES}{Large Eddy Simulation}
%\acro{DNS}{DNS}{Direct Numerical Simulation}
%\acro{GCM}{GCM}{Global Circulation Model}
%\acro{TKE}{TKE}{Turbulence Kinetic Energy}
%\acro{Ri}[Ri]{Richardson Number 
%\acroextra{, the bulk Richardson Number is 
%$\frac{gh}{\overline{\theta}_{ML}} \frac{\Delta \theta}{w^{*2}}$, 
%$\Delta \theta = \overline{\theta}(h_{1})-\overline{\theta}(h_{0})$ 
%}}
%\end{acronym}

%\acro{FFT}[FFT]{Fast Fourrier Transform}

%\end{acronym}
\endinput


%\acro{FoGS}[FoGS]{The Faculty of Graduate Studies}
%\acro{PDF}{Portable Document Format}
%\acro{RCS}[RCS]{Revision control system\acroextra{, a software
%    tool for tracking changes to a set of files}}
%\acro{TLX}[TLX]{Task Load Index\acroextra{, an instrument for gauging
%  the subjective mental workload experienced by a human in performing
%  a task}}
%\acro{UML}{Unified Modelling Language\acroextra{, a visual language
%    for modelling the structure of software artefacts}}
%\acro{URL}{Unique Resource Locator\acroextra{, used to describe a
%    means for obtaining some resource on the world wide web}}
%\acro{W3C}[W3C]{\acroextra{the }World Wide Web Consortium\acroextra{,
%    the standards body for web technologies}}
%\acro{XML}{Extensible Markup Language}


% You can also use \newacro{}{} to only define acronyms
% but without explictly creating a glossary
% 
% \newacro{ANOVA}[ANOVA]{Analysis of Variance\acroextra{, a set of
%   statistical techniques to identify sources of variability between groups.}}
% \newacro{API}[API]{application programming interface}
% \newacro{GOMS}[GOMS]{Goals, Operators, Methods, and Selection\acroextra{,
%   a framework for usability analysis.}}
% \newacro{TLX}[TLX]{Task Load Index\acroextra{, an instrument for gauging
%   the subjective mental workload experienced by a human in performing
%   a task.}}
% \newacro{UI}[UI]{user interface}
% \newacro{UML}[UML]{Unified Modelling Language}
% \newacro{W3C}[W3C]{World Wide Web Consortium}
% \newacro{XML}[XML]{Extensible Markup Language}
