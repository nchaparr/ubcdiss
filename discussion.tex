%% The following is a directive for TeXShop to indicate the main file
%%!TEX root = diss.tex

\chapter{discussion}
\label{ch:results}
\setlength{\parindent}{0cm}

\section{Description of Runs}
\FloatBarrier

The domain for each individual case is small relative to that used by \citeauthor{SullMoengStev} in \cite{SullMoengStev}, \citeauthor{FedConzMir04} in \cite{FedConzMir04} and \citeauthor{BrooksFowler2} in \cite{BrooksFowler2} i.e. $5Km \times 5 Km$ in the horizontal.  \citeauthor{SullMoengStev} (\cite{SullMoengStev}) did a higher resolution run on a $3 Km \times 3 Km$ horizontal domain and noticed a lower convective boundary layer height ($h$) but similar slope in $h$ with respect to scaled time when compared with the analogous run on a larger domain with lower resolution.  They speculated the smaller domain enforced a smaller limit on plume size, thus influencing $h$. But according to \citeauthor{SullPat} (\cite{SullPat}) grid size also impacts $h$. \\

\citeauthor{SullMoengStev}'s (\cite{SullMoengStev}) grid spacing for most of their runs was $\Delta x, y= 33.3$, $\Delta z=10$ except for the run on the smaller domain which had $\Delta x, y = 15$, $\Delta z=6.67$.  The highest resolution \citeauthor{FedConzMir04} used in \cite{FedConzMir04} was $\Delta x, y = 50$ and $\Delta z = 20$.  \citeauthor{BrooksFowler2} in \cite{BrooksFowler2} used $\Delta x, y = 50$ and $\Delta z = 12$ except in resolution test runs where they used $\Delta x, y = 25$ and $\Delta z = 7.27$.  So the vertical resolution around the entrainment region in this study ($\Delta z= 5m$) is higher that the other LES studies. Both \citeauthor{SullMoengStev} (\cite{SullMoengStev}) and \citeauthor{BrooksFowler2} (\cite{BrooksFowler2}) use varying grids in the vertical, such that the region around the entrainment layer (\acs{EL}) is of higher resolution than elsewhere. We do the same in this study and noticed slight kinks in some of the profiles where the transition to and form higher resolution occurs. We will perform one run on a uniform vertical grid at $\Delta z=5m$ to verify that this does not effect the results.\\          

\citeauthor{SullMoengStev}'s (\cite{SullMoengStev}) initialized with a layer of constant potential temperature topped by an inversion topped by a constant lapse rate ($\gamma \approx 2.5 K/Km$). They applied constant average surface heat fluxes ($\overline{w^{'}\theta^{'}}_{s}$) ranging from about $20 - 450 \ Watts/m^{2}$. \citeauthor{BrooksFowler2} (\cite{BrooksFowler2}) followed suit, in that their range of Richardson numbers (\acs{Ri}) resulted from variation of initial inversion ($\Delta \theta$) strength and average surface heat flux ($\overline{w^{'}\theta^{'}}_{s}$).  \citeauthor{FedConzMir04} in \cite{FedConzMir04} start with a finite layer of constant average potential temperature ($\overline{\theta}$) above which there was a constant lapse rate which they varied from  $1 - 10 \ K/Km$. In this study we begin with a constant $\overline{w^{'}\theta^{'}}_{s}$ acting against uniform potential temperature lapse rate.  \citeauthor{SchmidtSchu} point out in \cite{SchmidtSchu} that as a convectively mixed layer (\acs{ML}) grows against a stable lapse rate ($\gamma$) overshoot of the plumes to buoyancy levels above their own, and subsequent entrainment causes a sharp temperature gradient. (see Table \ref{fig:tableofruns}) 

%\clearpage

\section{Relevant Definitions}

\FloatBarrier

See Table \ref{table:reldefs}.\\

\citeauthor{SullMoengStev} (\cite{SullMoengStev}) compared four methods of determining \acs{CBL} height, two of which they based on the vertical average heat flux ($\overline{w^{'}\theta^{'}}$) profile. For both, the time-series were a lotless smooth than that for $z_{f}$ determined in this study.  Their gradient and contour methods produced smoother time-series plots.  The former, they determined from the horizontal average of the local heights of maximum vertical potential temperature gradient.  Description of the contour method will be omitted since it is not directly useful. Their gradient based height is consistently higher than the heat flux based definitions i.e. the flux based definition overall is about 0.9 times the gradient definition. This is in line with the findings of this study. They did not focus on \acs{EL} depth. For their Richardson number (\acs{Ri}) they calculated $\Delta \theta = \overline{\theta}(z_{f1})-\overline{\theta}(z_{f})$.  This value is likely to be smaller than, and not necessarily proportionally to the $\Delta \theta$ used in this study. \\

\citeauthor{FedConzMir04} in \cite{FedConzMir04} determined \acs{CBL} height and \acs{EL} depth from the horizontal and time ($100 \times 2s$) averaged vertical $\overline{w^{'}\theta^{'}}$ profiles.  They used two difference buoyancy ($\frac{g\overline{\theta}}{\overline{\theta_{ML}}}$) jumps: $\Delta b = \overline{\theta}_{0}z_{f} - \overline{\theta}z_{f0}$ for comparison with the zero order model and $\delta \theta = \overline{\theta}z_{f1} - \overline{\theta}z_{f0}$ for comparison with the first order model and analysis of the \acs{EL}.\\  

%They observed a decrease in \acs{EL} depth with increasing \acs{Ri} with possibly a $-1$ or $-\frac{1}{2}$ power law.  In our study all three flux based heights seem to remain a constant fraction of $h$, so at this point our results diverge from theirs.  Their grid spacing could be compared to two of the lower resolutions runs from \citeauthor{SullPat}'s (\cite{SullPat}) resolution study, whereas ours compares with the lowest resolution run of those with vertical flux profiles that converge.  Assuming higher resolution produces more physical results, their flux profiles may be unrealistically wide.  According to \citeauthor{SullPat} (\cite{SullPat}) \acs{CBL} growth rate may also be impacted by such differences in resolution.  Furthermore to compare their plots of scaled entrainment rate vs \acs{Ri}, we would need to reproduce their \acs{Ri} s in our framework.


\citeauthor{BrooksFowler2} (\cite{BrooksFowler2}) used tracer concentration profiles and compare a number of different corresponding \acs{CBL} height definitions.  Although their height and temperature jump used to calculate the Richardson number (\acs{Ri}) are quite different, their scaling relations based on the fluxed based definitions can be compared to those in this study. For example the corresponding scaled entrainment rate vs \acs{Ri} plot has a lot of scatter.\\

The definitions that perform best in relation to \acs{Ri} for \citeauthor{BrooksFowler2} (\cite{BrooksFowler2}) are those based on the means of locally determined heights.  That based on the domain averaged tracer profile, ie the point of maximum vertical gradient, is directly comparable to our $h$. Although, this last definition does not produce a plot as correlated as ours.\\  

Their scaled statistical \acs{EL} definitions based on the local vertical gradient and the local wavelet covariance decrease with increasing \acs{Ri} similarly to ours, but their flux based definition ($2\times(z_{f1}-z_{f})$) show slight and opposite trends when averaged differently.  The latter is in line with what we found.\\


The height definitions in this study are all based on the average vertical potential temperature gradient ($\overline{\theta}$).  It seems to be assumed that the region, where the average potential temperature increases significantly from its mixed layer (\acs{ML}) value through the maximum to that of the free atmosphere, corresponds to the \acs{EL} as enclosed by the zero levels in the average potential temperature flux profiles (\citeauthor{Deardorff79} \cite{Deardorff79}, \citeauthor{FedConzMir04} 
\cite{FedConzMir04}, \citeauthor{GarciaMellado} \cite{GarciaMellado}).  But the average potential temperature profile is not used to quantitatively define the \acs{EL}.\\

\citeauthor{BrooksFowler2} (\cite{BrooksFowler2}) discuss the draw-backs of defining the \acs{EL} based on the 
gradient of an average tracer profile.  Specifically the inconsistency in the size of the gradient
relative to a maximum, at the average \acs{EL} limits as defined based on the local limits. They
found the relative size had significant scatter and varied according to \acs{Ri}. Their maximum and the manner in which they determine is not reproducible in our framework but their conclusion could serve as a caution.\\  

Since in the \acs{ML} on average there is a gradual increase through zero in average potential temperature above the surface layer, rather than a region where the gradient is zero.  So a threshold value must be chosen to identify the lower limit of the \acs{EL}.  This threshold should be less than the upper lapse rate ($\gamma$), positive and consistent for all runs.  It was chosen by looking at the gradient profile and selecting a point which looked reasonable. The principal result was plotted at three different thresholds based on the unscaled gradient ($\frac{\partial \overline{\theta}}{\partial z}$) profiles.\\

The upper \acs{EL} limit is defined as the point at which the average vertical potential temperature gradient resumes $\gamma$.  These two limits then represent: the point above the surface layer at which the air on average begins to be less turbulently mixed, and the lowest point at which the air is unaffected as yet by the convected turbulence.  Our principal length scale $h$ is the point at which the gradient is maximum i.e. the point at which on average the air differs greatest from that directly above it. Our $\Delta \theta$ is the difference in average potential temperature ($\overline{\theta}$) over the \acs{EL}.  We compare $h$ with the fluxed based definitions.
  
%\clearpage

\section{Verifying the Model Output}
\label{sec:CheckingtheModel}
\subsection{Time till well-mixed}%Spin Up Time according to the Convective Time Scale $\tau$}
\FloatBarrier

To establish statistically steady turbulent flow \citeauthor{SullMoengStev} in \cite{SullMoengStev}
ran from the same random initial conditions on their coarse grid for more than ten eddy turnover times.
Then they switched on the nested high resolution grid and continued for another 4 Odie turnovers.  
\citeauthor{BrooksFowler2} (\cite{BrooksFowler2}) waited 2 simulated hours before they
judged the turbulence to be fully developed.  To initialize turbulence they added a small random perturbation
to the temperature field.\\ 

\citeauthor{FedConzMir04} (\cite{FedConzMir04}) focus on the attainment of a quasi-steady
state regime within which their zero order entrainment equation holds.  Their derivation also hinges
upon parametrizations for turbulent kinetic energy ($e$) and dissipation ($\epsilon$):

\begin{equation}
e=w^{*2}\Psi_{e}\left( \frac{z}{z_{i}} \right) \ \epsilon=\frac{w^{*3}}{z_{i}}\Psi_{\epsilon}\left( \frac{z}{z_{i}}\right)
\end{equation}

Where the two functions of dimensionless height integrate over the \acs{CBL} to constants, for example

\begin{equation}
\int^{z_{i}}_{0}\frac{e}{w^{*2}}dz = C_{e}
\end{equation}

In the referenced regime, \acs{CBL} growth is much slower than the convective velocity scale ($w^{*}$),
there is a constant entrainment ratio $-\frac{\overline{w^{'}\theta^{'}}_{min}}{\overline{w^{'}\theta^{'}}_{s}}$
and change in the total $e$ and it's escape from the boundary layer through waves are negligible relative
to the buoyant production and dissipation rate.  The resulting entrainment equation predicts a $\frac{1}{2}$
power law relationship between the normalized height, $z_{i}B_{s}^{-\frac{1}{2}} N^{\frac{3}{2}}$ and time
$tN$.  Since variation in $\overline{\theta}$ results in
less than 3 percent variation in $N$, when the surface heat flux $B_{s}$ and $\gamma$ are constant
this roughly translates to a $\frac{1}{2}$ power law relationship between $h$ and time.  In our study we find this to be the case (see Figure \ref{fig:hvstime1}).\\
  
We also observe self similarity of the scaled flux profiles, and so a constant entrainment ratio (see Figure \ref{fig:scaledfluxprofs15010}). By 2 hours of simulated time, at least 10 eddy turnover times have elapsed and by 3 hours the \acs{EL} is fully within the region of high vertical resolution.  Worth noting is the collapse in scaled time curves from 7 to 3 according to upper lapse rate ($\gamma$) (see Figure \ref{fig:ScaledTimevsTime}).

%\clearpage
\subsection{FFT Energy Spectra}
\FloatBarrier

Based on the scalar \acs{FFT} energy plots taken at the top of the \acs{ML} there is a cascade from the larger
to the smaller scales following the $-\frac{5}{3}$ power law (see Figure \ref{fig:2fftw602point5}). The \acs{CBL} is fully turbulent at this point
but further into the entrainment layer (\acs{EL}) there are large areas of little or no vertical velocity
interspersed with isolated impinging plumes.  So the dominant structures are smaller
and there is a steeper decay to the lower scales.  In this the \acs{FFT} plots and the contour plots in Figures \ref{fig:conts1} and \ref{fig:conts} compliment
each-other.  Furthermore there seems to be adequate scale separation between the dominant turbulent structures and the grid size, as well as isotropic turbulence.

%\clearpage

\subsection{Ensemble and horizontally averaged vertical Potential Temperature $\overline{\theta}$ 
and Heat Flux profiles $\overline{w^{'}\theta^{'}}$}
%Average Potential Temperature, Heat Flux and Kinetic Energy}
\FloatBarrier

\citeauthor{SchmidtSchu} point out in \cite{SchmidtSchu} that as a convectively mixed layer (\acs{ML})
grows against a stable lapse rate ($\gamma$) overshoot of the plumes to levels above their buoyancy causes
a sharp temperature gradient.  The sharpest vertical gradient in the area averaged potential temperature ($\overline{\theta}$)
profile corresponds to the vertical level at which the average potential temperature (Figure \ref{fig:pottempprofs2hrs}) differs greatest from that one level above.  Once a plume has overshot, envelopment or pinching off (\citeauthor{SullMoengStev} 
\cite{SullMoengStev}) of warm air from above causes a more gradual increase in temperature.  
Where this occurs is regarded here as the entrainment layer \acs{EL}.  In the averaged potential temperature 
profile it is represented by an increase in the vertical gradient. On the horizontal plane it would be composed 
of areas of \acs{ML} air interspersed with pockets of warmer air from above.  The ratio of \acs{ML} to stable air 
increases with proximity to the \acs{ML}.  This progression is seen in the average profile as a decrease in the
vertical gradient to close to zero (Figure \ref{fig:hdefs}).  Our average potential temperature profiles in Figure \ref{fig:pottempprofs2hrs} show a well mixed \acs{ML} overshooting
and growing against $\gamma$.  \acs{CBL} growth increases with $\overline{w^{'}\theta^{'}_{s}}$ and is inhibited 
by $\gamma$.  The \acs{ML} warming rate is strongly influenced by $\overline{w^{'}\theta^{'}_{s}}$ and $\gamma$.\\

The vertical $\overline{w^{'}\theta^{'}}$ profiles in Figure \ref{fig:fluxprofs2hrs} assume the expected shape becoming negative in the \acs{EL}
where the upward moving thermals are relatively cooler than the horizontal average and there is also downward
moving warmer air that has been pinched off or folded in.  Like \citeauthor{SullMoengStev} in \cite{SullMoengStev}
and \citeauthor{FedConzMir04} in \cite{FedConzMir04} we notice the entrainment ratio is less than .2 ($\approx .1$) 
for all runs but seems to increase with increased $\gamma$ inline with \citeauthor{Sorbjan}'s assertion in \cite{Sorbjan} 
that moments of $\theta^{'}$ depend on $\gamma$. Otherwise, there seems to be self similarity in time and across runs
when scaled by $\overline{w^{'}\theta^{'}_{s}}$ and plotted against scaled height.  So the scaled depth of the 
region of negative  $\overline{w^{'}\theta^{'}}$ seems more or less constant whereas \citeauthor{FedConzMir04} in 
\cite{FedConzMir04} seem to show a decrease from about .6 to about .2 with increasing \acs{Ri} and \citeauthor{BrooksFowler2}
with their slightly different definition in \cite{BrooksFowler2} seem to observe slight and contrasting trends with
respect to \acs{Ri} depending on whether the output is time averaged or not.



%\clearpage

\subsection{Visualization of Structures within the Entrainment Layer}
\FloatBarrier

\citeauthor{SullMoengStev} in \cite{SullMoengStev} show both horizontal and vertical cross sections
of their domain within the \acs{EL} around the inversion ($h$).  Horizontal cross sections of vertical
velocity and temperature perturbations clearly show coherent structures with both relatively
warm and cool air, associated with up-and-downward velocity.  Vertical cross sections show
impinging plumes and pockets of trapped warmer air.  The weak inversion case seems to show
convective overturning with apparent folding of warm stable air.  The strong inversion
case shows less deformation of the inversion interface and the entrainment event 
shown in the vertical cross section seems to occur via a narrow downward wisp associated
with an impinging plume.  In both cases, the downward motion of air from above is closely associated
with upward moving impinging plumes.\\

In our contours of $w^{'}$ and $\theta^{'}$ we see the almost spoke like pattern characteristic of the
mixed layer (\citeauthor{SchmidtSchu} \cite{SchmidtSchu}) at the lower limit of the \acs{EL}
and then distinct plumes become clearer at the inversion and above, where there are coherent
areas of warmer and cooler air associated up and downward vertical velocity perturbations (Figures \ref{fig:conts} and \ref{fig:conts1}).  
This progression is similar to that seen in \cite{GarciaMellado} by \citeauthor{GarciaMellado}.  
We do see bigger clearer regions of upward moving air in the weak stability case as compared to the
the strong stability case. There are pockets of warmer air close to and around the impinging cooler plumes, in line with the concept of wisps being pinched off, or enfolded.

%\clearpage


\section{Local Mixed Layer Heights ($h_{0}^{l}$)}
\label{sec:locmlh}     
\FloatBarrier

%The gradient method of determining \acs{CBL} height proved to be flawed due to the high gradients
%well into the stable layer which exceeded that of the transition
%from \acs{ML} to free atmosphere at points outside an active impinging plume.  

\citeauthor{SullMoengStev} \cite{SullMoengStev} used a centred differencing gradient method for determining local \acs{CBL} height  and observed the distributions
of $z_{i}^{'} = z_{i}-<z_{i}>$.  They observed positive skew in their weak stability cases which
they speculated was due to a small number of high reaching plumes.  We initially tried a similar
method and noticed positive skew, which we found corresponded to local points where the upper variability
exceeded the gradient between the \acs{ML} and the upper atmosphere. So for our purposes the gradient method was rendered unusable\\

The point of maximum vertical gradient in  a tracer profile should correspond to that in a potential 
temperature profile but the profiles can be quite different.  For example a Lidar back-scatter profile
which corresponds directly to tracer concentration profile, has a high value in the \acs{ML} and a low
value in the upper atmosphere, similar to step function.  Usually the variability within these regions 
is a lotsmaller than that over the transition region between the two.  So the transition region can be 
identified using a wavelet of dilation corresponding the the depth of the transition zone.  This is 
clearly shown by \citeauthor{Brooks} in \cite{Brooks} who uses such a wavelet to identify the local \acs{EL}
and then one with narrower dilation to identify the \acs{EL} limits.  The gradient method can also be applied
to a Lidar profile but again this can be noisy. %[references p247 \cite{BrooksFowler2}].  
\citeauthor{SteynBaldHoff}
in \cite{SteynBaldHoff} overcame this by fitting smooth idealized curve to the profile.\\

In line with this last method, we fit a three lines to the local profile representing the \acs{ML}, 
\acs{EL} and upper layer of constant $\gamma$ based on the multilinear regression method outlined by \citeauthor{Vieth} in 
\cite{Vieth}.  This works well with our very simple set up, IE, each local profile consists
of a distinct \acs{ML} and upper region of constant $\gamma$. Locally there is not always
a clear \acs{EL}.  At points where there is neither a sharp gradient nor a clear \acs{EL} 
and some variation in the slope within the \acs{ML}, a test was needed on the slope of 
the second line to see if it was significantly less $\gamma$.  If so, it was considered
to be part of the \acs{ML}.\\

\citeauthor{BrooksFowler2}'s three statistically based entrainment zone limits in 
\cite{BrooksFowler2} showed decreasing trend with increase in \acs{Ri}. Their resulting
scaled \acs{EL} is a lotnarrower than that based on our $\frac{\partial \overline{\theta}}{\partial z}$ profile i.e. .05 - 1.5, and even seems narrower than
what would be the 5th and 95th percentile of our local \acs{ML} heights (see Figure \ref{fig:localhpdf}).  Their lowest
inversion strength seems to be 1 degree over 100 meters (IE .01 per meter) which is
the same as our maximum stability, except of course ours is constant, and their highest is
10 times that. But their lapse rate above is a lot lower ($3k/Km$).  So, this difference cannot simply be explained in terms of inversion strength.\\  %These authors dismiss using the averaged tracer profile based on how the gradients, at the limits based on the local height percentiles on the average profile, relate to the, further scaled by the peak gradient determined from the local peaks in gradient, decrease with decreasing \acs{Ri}. This roughly corresponds to the conceptual difference between the heights of the lowest plumes, and the region where air is mostly (for example on average 95 percent) \acs{ML} air.  It serves as a cautionary note for our study.\\

We see that the local profiles are very different to the average profile and that local
profiles differ from each other (Figures \ref{fig:rssfitshigh} and \ref{fig:rssfitslow}).  The \acs{EL} is an inherently average phenomena i.e.
the range in space or, the range in time, over which the plume heights vary.  So
it is possible to see a local \acs{EL}.  For example in Figure \ref{fig:rssfitslow} (a)
we see a region above the \acs{ML} which is clearly not part of the stable
air above.  Here, we can speculate that a plume previously had reached that point
and some entrainment of warmer air from above had occurred.\\
 
Overall \citeauthor{SullMoengStev} 
\cite{SullMoengStev} show decreased variation in the local heights, with increased \acs{Ri} as we do.  Based on the histograms of our local \acs{ML} heights in Figure \ref{fig:localhhist} we see the range or spread
increases with increased $\overline{w^{'}\theta^{'}}_{s}$ and decreases with increased
$\gamma$.  When scaled by $h$ in Figure \ref{fig:localhpdf} the spread seems only influenced by $\gamma$.  So once
again there is a cancellation of the effects of $\overline{w^{'}\theta^{'}}_{s}$ once
$h$ is introduced.

%\clearpage

\section{Flux Quadrants}
\label{sec:fluxquadrants}     
\FloatBarrier

The shape of the average potential temperature profile evolves according to the temperature flux
profile. In particular warming in the entrainment layer (\acs{EL}), and upper mixed layer (\acs{ML})
is related to the flux of warmer air up or down to that region.  Lower in the \acs{ML} warming is 
from the thermals or plumes originating at the surface.  These plumes become cooler than the horizontal
average in the \acs{EL} where upper stability above the inversion interface causes them to turn downward.
Here there are accompanying downward moving pockets of warm air associated with the upward moving
plumes.  All of this was seen in the visual aids presented by \citeauthor{SullMoengStev} in \cite{SullMoengStev}.\\

In \cite{MahrtPaum} \citeauthor{MahrtPaum} examined the joint distributions of $w^{'}$ and $\theta^{'}$ 
from measurements taken of mixed layers developed in the flow of cold air masses over a warm current.
Their two dimensional representations clearly show the four quadrants: upward warm, upward cool, downward cool
and downward warm.\\

\citeauthor{Sorbjan} in \cite{Sorbjan} asserted and demonstrated that the moments involving $\theta^{'}$
particularly in the upper \acs{ML} and \acs{EL} are strongly influenced by the upper lapse rate $\gamma$.
Whereas moments of $w^{'}$ were less so.  These effects were seen when the corresponding vertical profiles
were scaled by the convective scales ($\theta^{*}$ and $w^{*}$).\\

Bearing the above three studies in mind we separate the $w^{'}\theta^{'}$ into the four quadrants and plot
the average vertical scaled profiles as well as the 2d histograms at $h$ and the \acs{EL} limits.  We
can confirm that the upper extrema of the four individual quadrants exceed that of the average
and are higher i.e. close to $h$ (Figure \ref{fig:fluxqadprofs}).  Higher stability results in a more pronounced peak particularly in the
upward cool quadrant profile which corresponds to increased damping and a sharper decrease in velocity.
Since warming in this region is associated with downward movement of air from above, the downward warm
quadrant is important.\\

The 2d histograms at each level show increased spread of both $\theta^{'}$ and $w^{'}$ with increased 
$\overline{w^{'}\theta^{'}}_{s}$ (Figures \ref{fig:fluxquadsh}, \ref{fig:fluxquadsh0}, \ref{fig:fluxquadsh1}).  There is damping of $w^{'}$ with increased $\gamma$.  To isolate the 
effects of increased $\gamma$ we should scale by the convective scales ($\theta^{*}$ and $w^{*}$) .

%\clearpage

\section{$h$ and  $\Delta h$ based on Average Profiles}
\label{sec:hdeltahavprofs}
\subsection{Reminder of Relevant Definitions}
\FloatBarrier

Our heights are defined based on the average vertical temperature gradient the principle length scale being $h$ the vertical location of the maximum.  Flux based heights are scaled by $h$ to enable comparison with the frameworks of other studies.  

%\clearpage
\subsection{$\frac{w_{e}}{w^{*}}$ vs $Ri^{-1}$}
\FloatBarrier

In Figure \ref{fig:hvstimeloglog} $h$ shows a $\frac{1}{2}$ power law relationship to time indicating we are in the regime outlined by \citeauthor{FedConzMir04} in \cite{FedConzMir04}.  Self similarity of the scaled heat flux profiles vs scaled height in Figure \ref{fig:scaledfluxprofs15010} indicate a more or less constant entrainment ratio, but also a more or less constant scaled entrainment depth with respect to time. Our Richardson numbers (\acs{Ri}s) increase with respect to time and again grouping according to $\gamma$ is evident (Figure \ref{fig:invristime}).\\


\citeauthor{KatoPhil} successfully related
the scaled entrainment rate of penetrative shear driven turbulence
in their water-tank experiment in \cite{KatoPhil}
to a  dimensionless group formed from the three main characteristics
of the flow : the buoyancy jump across the interface, the turbulent velocity of the \acs{ML} and the depth of the \acs{ML}. IE

\begin{equation}
\frac{u_{e}}{u^{*}} \propto \frac{\rho_{0} u^{*2}}{g \delta \rho D} 
\end{equation}

\citeauthor{DearWill80} related their scaled entrainment of penetrative convection to this dimensionless group, substituting the shear driven velocity scale for the convective one, thus forming the now commonly used Richardson number (\acs{Ri})for the \acs{CBL}.  Their heights were determined from the vertical heat flux profiles.  The heat flux profiles in turn were derived from two successive potential temperature profiles.  The resulting relationship between scaled entrainment rate and \acs{Ri} appears to potentially exhibit both $-1$ and $\frac{-3}{2}$ power laws.\\

\citeauthor{SullMoengStev}'s data in \cite{SullMoengStev} showed some scatter and they speculated that a power law other than $-1$ may have described the relationship at \acs{Ri}s smaller than 14.  They compare the data to this fit:

\begin{equation}
\frac{w_{e}}{w^{*}}=0.2 Ri^{-1}
\end{equation}

\citeauthor{Turner86} in \cite{Turner86} attribute the $-\frac{3}{2}$ power law to mixing that depends on the recoil of impinging eddies.  Whereas \citeauthor{FedConzMir04} in \cite{FedConzMir04} derive it from a best fit approximation of the \acs{Ri} calculated using the buoyancy jump across the \acs{EL} to scaled time (after $tN>100$) and applying the zero order model relationship.\\

\citeauthor{BrooksFowler2}'s plot in \cite{BrooksFowler2} has relatively little scatter and exhibits a linear relationship ($-1$ power law) whereas \citeauthor{GarciaMellado}'s data in \cite{GarciaMellado} seems asymptotic to a linear relationship.\\

Our data based on the temperature jump across the entire \acs{EL} shows a seemingly linear relationship (Figure \ref{fig:scaledweinvri}). 

%\clearpage

\subsection{$\frac{\Delta h}{h}$ vs $Ri^{-1}$}
\label{subsec:deltahri}
\FloatBarrier

The \acs{EL} tops as defined by the point at which the temperature gradient resumes $\gamma$ seem to be scaled well by $h$ (Figure \ref{fig:scaledELlims}).  This seems in contrast to the assertion of \citeauthor{GarciaMellado} in \cite{GarciaMellado} about the upper \acs{EL} i.e. that length and buoyancy in this region are not scaled by the the \acs{CBL} convective scales. The \acs{EL} top as defined where the point at which the buoyancy flux decreases to close to zero, when scaled by $h$ is comparable, but has greater scatter (Figure \ref{fig:scaledELlims1}). But in both cases, the top limit is about 1.15 $\times \ h$, and there is a barely perceptible, possible negative trend.\\  

The scaled lower \acs{EL} limits based on the increase in potential temperature gradient from zero, show a clearer increase but don't show the same kind of collapse across runs as the upper limit does (Figure \ref{fig:scaledELlims}).  The scaled lower limit based on the flux profiles however, do collapse well (Figure \ref{fig:scaledELlims1}).  So we could say with some confidence that $\frac{h-z_{f0}}{h} \approx .2$ and this is comparable to \citeauthor{GarciaMellado}'s lower \acs{EL} sublayer.\\
  
So the scaled \acs{EL} as defined by the vertical gradient in the potential temperature profile certainly decreases with respect to time.  The scaled \acs{EL} based on the flux profiles shows slight or no change with respect to time.  This is in line to the findings of \citeauthor{BrooksFowler2} in \cite{BrooksFowler2} even though their definition is slightly different IE $2 \times (z_{f}-z_{f0})$.  But it is in stark contrast to what \citeauthor{FedConzMir04} show in \cite{FedConzMir04} i.e. $\frac{z_{f1}-z_{f0}}{z_{f}}$ decreasing from about .6 to about 0.1.  This could in part be explained by the difference in vertical resolution since according to \citeauthor{SullPat} in \cite{SullPat} the shape of average heat flux profile in the \acs{EL} is sensitive to grid size.\\

\citeauthor{Sorbjan} in \cite{Sorbjan} and \cite{Sorbjan1} demonstrates how the surface and lower \acs{ML} portions of the temperature gradient profile is scaled well by the convective scales but $\gamma$ becomes more important in the \acs{EL}.  From our potential temperature profiles in Figure \ref{fig:pottempprofs2hrs} we see that both $\gamma$ and $\overline{w^{'}\theta^{'}}_{s}$ influence the warming of the \acs{ML}. So this should reflect in particular in the downward flux of warm air from the inversion IE at $h$. That is, increasing $\gamma$ seems to result in an increased slightly positive gradient in the upper \acs{ML} and this should relate to an increase in the downward flux warm air above it, for example at $h$.\\

So, first we define the \acs{EL} lower limit as the point at which the vertical gradient exceeds a positive threshold that's less than $\gamma$ and the same for all runs, at all times. We try three different values and note that there is a seeming decrease in the scaled magnitude with respect to \acs{Ri}, bearing in mind the definition of the \acs{EL} is included in the calculation of $\Delta \theta$ for $Ri$. Grouping according to $\gamma$ is evident.\\

Scaling the vertical potential temperature gradient profiles by $\gamma$ results in collapse to more or less one curve.  The gradient profiles seem to show an increase in the peak gradient as the \acs{EL} seems to narrow. This trend is apparent with respect to time and across runs.  This portion of the profile has been scaled effectively by \citeauthor{Sorbjan1} in \cite{Sorbjan1} using $\frac{\Delta \theta}{\Delta h}$ and \citeauthor{GarciaMellado} using their buoyancy scale $b \approx N^{2} \delta + [\overline{b_{0}}(h) - \overline{b}(h)]$ where $\delta \propto \frac{w^{*}}{N}$ is their length-scale for the upper \acs{EL} sublayer.  Related to $\frac{\Delta \theta}{\Delta h}$ is the entrainment layer stratification parameter $G = \gamma \frac{\Delta h}{\Delta \theta}$ which \citeauthor{FedConzMir04} found to be constant throughout the quasi-steady state regime IE, $\Delta \theta \propto \Delta h$.  This seems to contradict the apparent increase in maximum gradient with decrease in \acs{EL} depth.


\endinput

Any text after an \endinput is ignored.
You could put scraps here or things in progress.
