%% The following is a directive for TeXShop to indicate the main file
%%!TEX root = diss.tex

\chapter{discussion}
\label{ch:results}
\setlength{\parindent}{0cm}

\section{Comparison of general Set-up}
\FloatBarrier

\citeauthor{SullPat} (\citeyear{SullPat}) found that the shapes of the vertical average potential temperature ($\overline{\theta}$) and average vertical heat flux ($\overline{w^{'}}\theta^{'}$) profiles, as well as the measured \acs{CBL} height vary depending on grid size.  The resolution at which convergence begins is listed in Table \ref{table:gridcomp}.  Untill this point the $\overline{\theta}$ and $\overline{w^{'}}\theta^{'}$ profiles are such that the \acs{EL} is a larger portion of the \acs{CBL} and measured \acs{CBL} height is higher overall.  Overall they concluded that vertical resolution was more critical.  This compliments the conclusion \citeauthor{BrooksFowler2} (\citeyear{BrooksFowler2}) when discussing their resolution test.  That is, to capture the steep vertical gradients in the \acs{EL} requires high resolution.\\

\begin{table}[htbp]
    \begin{center}
%\centerline{
    \begin{tabular}{ p{5cm} p{4cm}}
    %\hline
Publication & \acs{EL} $\Delta x$, $\Delta y$, $\Delta z$ (m)\\ \hline
      \citeauthor{SullMoengStev} (\citeyear{SullMoengStev}) & 33, 33, 10  \\ \hline 
      {FedConzMir04} (\citeyear{FedConzMir04}) & 100, 100, 20  \\ [.3cm] %\hline
      \citeauthor{BrooksFowler2} (\citeyear{BrooksFowler2}) & 50, 50, 12\\ \hline
    \citeauthor{SullPat} (\citeyear{SullPat}) &  20, 20, 8  \\ \hline
    This study & 25, 25, 5\\ \hline 
      
    \end{tabular}
%}
\caption[]{Grid spacing around the \acs{EL} used in comparable \acs{LES} studies. Those used for resolution tests are not listed here.  For \citeauthor{SullPat}'s \citeyear{SullPat} resolution study I list the grid sizes at which profiles within the \acs{EL} and \acs{CBL} height evolution began to converge.}
\label{table:gridcomp}   
\end{center}    
\end{table}

The \acs{FFT} energy spectra of horizontal slices at the top of the \acs{ML} show a substantial resolved intertial subrange giving confidence in the choice of horizontal grid size used here. In the \acs{EL} where turbulence is intermittent, the dominant energy containing structures are smaller, and decay to the grid-size is steeper.\\

As \citeauthor{Turner86} discusses in his \citeyear{Turner86} review of turbulent entrainment, the important of smaller scale processes, ie at the molecular level are relatively unimportant.  Large scale engulfment and trapping between thermals dominates. Yet, the steep vertical, and surely horizontal, gradients within the \acs{EL} remain motivational for \acs{DNS} studies such as \citeauthor{GarciaMellado} (\citeyear{GarciaMellado}).  Among these author's conclusions was that the production and destruction rates of \acs{TKE}, as well as the entrainment ratio used to calculate the entrainment rate, were effectively independent of molecular scale processes.\\  

The horizontal domain in this study is smaller than those used in the other studies listed in Table \ref{table:gridcomp}. \citeauthor{SullMoengStev} (\citeyear{SullMoengStev}) carried out one run on a smaller domain with higher resolution, noticed it resulted in lower \acs{CBL} height and concluded this was due to restricted plume size. However, given the results of \citeauthor{SullPat} (\citeyear{SullPat}) it could have been an effect of the grid.  Visualizations of horizontal and vertical slices clearly showed multiple thermals with diameter increasing with increased \acs{CBL} height, but remaining less than or on the order of 100 meters.\\   

A comparison of how the \acs{Ri}s were calculated will be left for a later section, but the range was dependent on variation in $\gamma$ and less so on $\overline{w^{'}\theta^{'}}$.  \citeauthor{BrooksFowler2} (\citeyear{BrooksFowler2}) and \citeauthor{SullMoengStev} (\citeyear{SullMoengStev}) imposed a $\theta$ jump of varying strength topped by a constant $\gamma$.  Whereas \citeauthor{FedConzMir04} (\citeyear{FedConzMir04}) initialized with a layer of uniform $\theta$ topped by a constant $\gamma$ which was different for each run.  Thesis authors did not vary average surface heat flux ($(\overline{w^{'}\theta^{'}})_{s}$).  They also used a timescale based on $\gamma$ rather than the convective timescale $\tau$.  The results of this study support this, in that the effects of varying $(\overline{w^{'}\theta^{'}})_{s}$ seem to be offset by $h$ and $\tau = \frac{h}{w^{*}}$ depends solely on $\gamma$.\\    

%\clearpage

\section{Local \acs{ML} heights}

\citeauthor{SullMoengStev} (\citeyear{SullMoengStev}) determined local \acs{CBL} height by locating the point of maximum gradient.  Analysis of the resulting distributions showed dependence of standard deviation and skewness on Richardson number.  The normalized standard deviation decreased with increased \acs{Ri} whereas skewness was alomost bimodal; being negative at high \acs{Ri} and positive and low \acs{Ri}.  Iniatially in this study, I applied a similar method and found distributions with lower \acs{Ri} to have positive skew.  Upon exhaustive inspection of local vertical $\theta$  profiles, it became evident that at certain horizontal points high gradients well into the free atmposphere exceded those closer to the location of the \acs{CBL} height reasonably identified by eye.\\

Locating the \acs{ML} height using the multi-linear regression method employed proved more reliable, based on inspection of hundreds of local vertical $\theta$  profiles.  For a large proportion of these profiles it was impossible even by eye to locate a reliable \acs{CBL} height based on a maximum in the vertical gradient. The distributions were seen to broaden with increased $ (\overline{w^{'}\theta^{'})_{s} $ and narrow with increased $ \gamma $.  When normalized by the height of the maximum average vertical potential gradient ($h$) what apparantly remains is the effect of $\gamma$ on the lower limit or lowest percentile.  The result is an overall narrowing of the scaled distributions with $\gamma$.\\  

Potential temperature and vertical velocity fluctuations ($\theta^{'}$ and $w^{'}$) at several vertical levels around the \acs{EL} were plotted as 2 dimensional histograms.  At $z_{f}$ and $h$, ie within the \acs{EL} the quadrants of largest magnitude were upward and downward moving relatively cool, thermal,  air ($w^{'-}\theta^{'+}$ and $w^{'+}\theta^{'+}$). The $w^{'-}\theta^{'-}$, $w^{'+}\theta^{'-}$ and $w^{'-}\theta^{'+}$ quadrants do approximately cancel.  The convective velocity scales ($\theta^{*}$ and $w^{*}$) were appplied to isolate the effects of $\gamma$, although it is accounted for indirectly via $h$ (see Section \ref{subsec:scales}). As shown in \citeauthor{Sorbjan}'s (\citeyear{Sorbjan}) $\theta^{'}$ is influenced  by $\gamma$.  For example at $h$ there is an apparant increase in the spread, as well as a shift thowards the positive.  So positive fluctuations due representing air from the \acs{FA} are more positive and negative fluctuations representing thermals are less negative.  The former can easily be explained in terms of an increased lapse rate above ($\gamma$).\\

The downward moving warm quadrant ($w^{'-}\theta^{'+}$) at $h$ represents warmer free atmosphere air that is being entrained.  So it's magnitude at a certain point in time is a measure of heating at a successive time in the region below.  In Figure \ref{fig:downwarm} the magnitude increases with respect to time is grouped according to $(\overline{w^{'}\theta^{'}})_{s}$.  Indeed it is an increasing proportion of  $(\overline{w^{'}\theta^{'}})_{s}$ and Figures \ref{fig:downwarm_wvel} and \ref{fig:downwarm_theta} show that its increase is primarily due to the increased positive temperature variance ($\theta^{'+}$).  While the velocity of downward warm quadrant $w^{'-}$ quickly approaches a constant proportion of $w^{*}$, the magnitude of temperature fluctuation approaches a constant proportion of $\gamma \Delta h$ rather than the convective temperature scale $\theta^{*}$.  So the positive horizontal variance in temperature is related to difference in temperature over $\Delta h$ of the inital lapse rate $\gamma$.  The relationship between the horizontal and vertical variance in temperature is clearly shown in the plots of each in \citeauthor{Sorbjan}'s (\citeyear{Sorbjan}) and \citeauthor{SullMoengStev}'s (\citeyear{SullMoengStev}) and \citeauthor{GarciaMellado}'s (\citeyear{GarciaMellado}). Their peaks within the acs{EL} seem to coincide.  In the mixed layer vigorous horizontal and vertical motion renders both close to zero.\\             

\subsection{Relationship of Entrainment Layer Depth to Richardson Number}

None of the comparable \acs{LES} studies define the \acs{EL} based ont the vertical $\frac{\partial \overline{\theta}}{\partial z}$ profile.  Yet, here, when the profile is scaled by $\gamma$ the resulting scaled \acs{EL} depth as defined in Figure \ref{fig:hdefs1} shows a dependence on Richardson number (\acs{Ri}) 

\begin{equation}
\frac{\Delta h}{h} \propto Ri ^{b} \tag{\ref{eq:dhvsri}}
\end{equation}


which appears to have an exponent of $-\frac{1}{2}$, as predicted and seen by \citeauthor{Boers89} (\citeyear{Boers89}), at lower \acs{Ri} (higher \acs{Ri}$^{-1}$) possibly increasing to $-1$ at higher \acs{Ri}.  It is possible that there is a change in entrainment mechanism. \citeauthor{SullMoengStev} (\citeyear{SullMoengStev}) observed enfolding and engulfment at lower \acs{Ri}.  Whereas at higher \acs{Ri} when motion is more restricted, entraiment seemed to occur via trapping of thinner whisps at the edge of an upward moving thermal.  \citeauthor{Turner86} (\citeyear{Turner86}) also distiguishes between entrainment by convective overturning and recoil.\\    


Although their heights are defined differently,  \citeauthor{FedConzMir04}'s (\citeyear{FedConzMir04}) see an exponent of of $-\frac{1}{2}$.  They say the decrease in the overall depth is due to a slight decrease in the scaled top limit over time.  However based on their plot, it seems to go from about .5 to .2           


\begin{table}[htbp]
\begin{center}
%\centerline{
\begin{tabular}{ p{5cm} p{4cm}}
    %\hline
Publication & \acs{EL} & \acs{CBL} height & $\Delta \theta$\\ \hline
\citeauthor{FedConzMir04} (\citeyear{FedConzMir04}) & $z_{f1} - z_{f0}$ & $z_{f}$ &  $\overline{\theta}(z_{f1})-\overline{\theta}(z_{f0})$\\ [.3cm] %\hline
\citeauthor{BrooksFowler2} (\citeyear{BrooksFowler2}) & $2 \times (z_{f} - z_{f0})$ & $z_{f}$ & average of local values\\ \hline
\end{tabular}
 
\FloatBarrier


\endinput

Any text after an \endinput is ignored.
You could put scraps here or things in progress.
