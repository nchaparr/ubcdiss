%% The following is a directive for TeXShop to indicate the main file
%%!TEX root = diss.tex

\chapter{Conclusion and Future Work}
\label{ch:conc}
\setlength{\parindent}{0cm}

\section{\acs{CBL} Height and \acs{EZ} Depth can be defined based on the average Potential Temperature Profile}

The $\overline{\theta}$ profile characterizes the dry, idealized \acs{CBL} and links bulk models to soundings via an \acs{LES}.  Both the \acs{EZ} depth and \acs{CBL} height based on the average $\frac{\frac{\partial \overline{\theta}}{\partial z}}{\gamma}$ profile show dependence on $\acs{Ri}$ as seen in other studies and justified theoretically.  So this is a valid way of defining the \acs{CBL} and its \acs{EZ}.  A change in entrainment mechanism or regime with increased $\acs{Ri}$ has been, observed in measurement as well as \acs{LES} based studies, and justified theoretically. I suggest the change in the exponents of Equations \ref{eq:dhvsri} and \ref{eq:ervsri}, seen here, represent this.\\

\section{Upper Lapse-rate strongly influences dry, idealized \acs{CBL} Entrainment}

The magnitude and variance, of local height, increase with increasing $\overline{w^{'}\theta^{'}}_{s}$ and decrease with increasing $\gamma$.  The same can be said for the vertical velocity fluctuations ($w^{'}$) in the \acs{EZ}.  However, increased $\gamma$ results in an increase in the positive potential temperature fluctuations ($\theta^{'+}$) at $h$. The magnitude of ($\theta^{'+}$) at points where $w^{'}$ is negative represents downward moving entrained air and depend on $\gamma$.  Below $h$, in the lower \acs{EZ}, the average vertical potential temperature gradient ($\frac{\partial \overline{\theta}}{\partial z}$) also depends $\gamma$. So, the growth of the idealized dry \acs{CBL} is driven by $\overline{w^{'}\theta^{'}}_{s}$ and suppressed by stability ($\gamma$). But \acs{CBL} warming is due, in part, to the entrainment of air from aloft the potential temperature of which in turn depends on $\gamma$.\\

Throughout this entire study, threads the influence of this paramater.  Distributions of scaled local \acs{ML} heights approach apparent similarity, when $\gamma$ is constant but $\overline{w^{'}\theta^{'}}_{s}$ is varied.  Curves representing Equation \ref{eq:dhvsri} group according to $\gamma$ when based on the $\frac{\partial \overline{\theta}}{\partial z}$ profile, but become similar once based on $\frac{\frac{\partial \overline{\theta}}{\partial z}}{\gamma}$.  The convective time scale $\tau = \frac{w^{*}}{h}$ and $\acs{Ri}$ group according $\gamma$ lending support to \citeauthor{FedConzMir04} (\citeyear{FedConzMir04})'s use of a the Brunt-Vaisala time scale.  It seems that once the effect of the surface heat flux ($\overline{w^{'}\theta^{'}}_{s}$) is accounted for through $h$, $\gamma$ emerges as the dominant parameter in dry, idealized \acs{CBL} entrainment.\\ 

\FloatBarrier


\endinput

Any text after an \endinput is ignored.
You could put scraps here or things in progress.
