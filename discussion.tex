%% The following is a directive for TeXShop to indicate the main file
%%!TEX root = diss.tex

\chapter{Results in Context}
\label{ch:results}
\setlength{\parindent}{0cm}

\section{Comparison of general Set-up}
\FloatBarrier

\citeauthor{SullPat} (\citeyear{SullPat}) found that the shapes of the vertical average potential temperature ($\overline{\theta}$) and average vertical heat flux ($\overline{w^{'}}\theta^{'}$) profiles, as well as the measured \acs{CBL} height vary depending on grid size.  The resolution at which convergence begins is listed in Table \ref{table:gridcomp}.  Untill this point the $\overline{\theta}$ and $\overline{w^{'}}\theta^{'}$ profiles are such that the \acs{EL} is a larger portion of the \acs{CBL} and measured \acs{CBL} height is higher.  Overall they concluded that vertical resolution was more critical.  This compliments the conclusion \citeauthor{BrooksFowler2} (\citeyear{BrooksFowler2}) when discussing their resolution test.  That is, to capture the steep vertical gradients in the \acs{EL} requires high resolution. \\

As \citeauthor{Turner86} discusses in his \citeyear{Turner86} review of turbulent entrainment, the important of smaller scale processes, ie at the molecular level are relatively unimportant.  Large scale engulfment and trapping between thermals dominates.  If $\overline{\theta^{'2}}$ is calculated basedon differences from horizontally averaged $\theta$, ie if the ergodic assumption is drawn upon, then it can be a measure of horizontal variance. Although \citeauthor{SullPat} (\citeyear{SullPat}) found the vertical distance over which $\overline{\theta^{'2}}$ varies more or less converged at the resolution shown in Table \ref{table:gridcomp} the maximum continued to increaase up to their finest grid spacing (5, 5, 2).  This question of whether the mxing within the \acs{EZ} is adequately resolved serves as motivation for \acs{DNS}  studies such as that of \citeauthor{GarciaMellado} (\citeyear{GarciaMellado}). These authors found the entrainment ratio $\frac{\overline{w^{'}\theta^{'}}_{z_{f}}}{\overline{w^{'}\theta^{'}}_{s}}$ to be about 0.1 which is lower than for example what {FedConzMir04} (\citeyear{FedConzMir04}) observed, but close to what was seen in this study. Although they did not directly measure \acs{EZ} depth, based on their heat flux profiles the region of negative flux seems to be about .3 of the height of minimum flux, again this is comparable to what I saw.  Among these author's other conclusions was that the production and destruction rates of \acs{TKE}, as well as the entrainment ratio used to calculate the entrainment rate, were effectively independent of molecular scale processes.\\  
  

\begin{table}[htbp]
\caption[]{Grid spacing around the \acs{EL} used in comparable \acs{LES} studies. Those used for resolution tests are not listed here.  For \citeauthor{SullPat}'s \citeyear{SullPat} resolution study I list the grid sizes at which profiles within the \acs{EL} and \acs{CBL} height evolution began to converge.}

    \begin{center}
%\centerline{
    \begin{tabular}{ p{5cm} p{3cm} p{3cm}}
    %\hline
Publication & \acs{EZ} $\Delta x$, $\Delta y$, $\Delta z$ (m) & Horizonal Domain \\ \hline
      \citeauthor{SullMoengStev} (\citeyear{SullMoengStev}) & 33, 33, 10 & 5 x 5 km \\ %\hline 
      \citeauthor{FedConzMir04} (\citeyear{FedConzMir04}) & 100, 100, 20 & 5 x 5 km \\ [.3cm] %\hline
      \citeauthor{BrooksFowler2} (\citeyear{BrooksFowler2}) & 50, 50, 12 & 5 x 5 km %\hline
    \citeauthor{SullPat} (\citeyear{SullPat}) &  20, 20, 8 & 5 x 5 km\\ %\hline
    This study & 25, 25, 5 &  3.4 x 4.8 km\\ \hline 
      
    \end{tabular}
%}
\label{table:gridcomp}   
\end{center}    
\end{table}

The \acs{FFT} energy spectra of horizontal slices at the top of the \acs{ML} show a substantial resolved intertial subrange giving confidence in the choice of horizontal grid size used here. In the \acs{EL} where turbulence is intermittent, the dominant energy containing structures are smaller, and decay to the grid-size is steeper. This confirms the assertion of \citeauthor{GarciaMellado} (\citeyear{GarciaMellado}) that the \acs{EZ} is separated into two sublayers in terms of turbulence scales.\\


The horizontal domain in this study is smaller than those used in the other studies listed in Table \ref{table:gridcomp}. \citeauthor{SullMoengStev} (\citeyear{SullMoengStev}) carried out one run on a smaller domain with higher resolution, noticed it resulted in lower \acs{CBL} height and concluded this was due to restricted plume size. However, given the results of \citeauthor{SullPat} (\citeyear{SullPat}) it could have been an effect of the grid.  Visualizations of horizontal and vertical slices clearly showed multiple thermals with diameter increasing with increased \acs{CBL} height, but remaining less than or on the order of 100 meters.\\   

Brooks and Fowler tried basing heights on average profiles but got lots of scatter and concluded that it was best to use statistics of locally determined heights rather than the mean profile.  Sullivan and moeng's heights based on average profiles produced very jagged oscillating versus time plots. Whereas the heights based on average profiles here, using an ensemble of cases, varied smoothly in time.  This could be attributed to a smoother average based on more horizontal points (10*128*192 vs 100*100 vs 151*151).\\

A comparison of how the \acs{Ri}s were calculated will be left for a later section, but the range was dependent on variation in $\gamma$ and less so on $\overline{w^{'}\theta^{'}}$.  \citeauthor{BrooksFowler2} (\citeyear{BrooksFowler2}) and \citeauthor{SullMoengStev} (\citeyear{SullMoengStev}) imposed a $\theta$ jump of varying strength topped by a constant $\gamma$.  Whereas \citeauthor{FedConzMir04} (\citeyear{FedConzMir04}) initialized with a layer of uniform $\theta$ topped by a constant $\gamma$ which was different for each run.  Thesis authors did not vary average surface heat flux ($(\overline{w^{'}\theta^{'}})_{s}$).  They also used a timescale based on $\gamma$ rather than the convective timescale $\tau$.  The results of this study support this, in that the effects of varying $(\overline{w^{'}\theta^{'}})_{s}$ seem to be offset by $h$ and $\tau = \frac{h}{w^{*}}$ depends solely on $\gamma$.\\    

\begin{table}[htbp]
\caption[]{Grid spacing around the \acs{EL} used in comparable \acs{LES} studies. Those used for resolution tests are not listed here.  For \citeauthor{SullPat}'s \citeyear{SullPat} resolution study I list the grid sizes at which profiles within the \acs{EL} and \acs{CBL} height evolution began to converge.}

    \begin{center}
%\centerline{
    \begin{tabular}{ p{5cm} p{2cm} p{2cm} p{2cm} p{2cm}}
    %\hline
Publication & $\overline{w^{'}\theta^{'}}_{s}$ W/m2& $\gamma$ K/km & Jump & $\acs{Ri}$ range \\ \hline
      \citeauthor{SullMoengStev} (\citeyear{SullMoengStev}) & 20 - 450& 3  &.436 - 5.17 &
%\\ \hline 
      \citeauthor{FedConzMir04} (\citeyear{FedConzMir04}) & 300 & 1 - 10 & na & 10 - 40\\ [.3cm] %\hline
      \citeauthor{BrooksFowler2} (\citeyear{BrooksFowler2}) &  10 -100 &  3& 1 - 10 &%\hline
      This study & 60 - 150 & 2.5 - 10& na & 10 - 30\\ \hline 
      
    \end{tabular}
%}
\label{table:initconditcomp}   
\end{center}    
\end{table}



%\clearpage

\section{Local \acs{ML} heights}

\citeauthor{SullMoengStev} (\citeyear{SullMoengStev}) determined local \acs{CBL} height by locating the point of maximum gradient.  Analysis of the resulting distributions showed dependence of standard deviation and skewness on Richardson number.  The normalized standard deviation decreased with increased \acs{Ri} whereas skewness was alomost bimodal; being negative at high \acs{Ri} and positive and low \acs{Ri}.  Iniatially in this study, I applied a similar method and found distributions with lower \acs{Ri} to have positive skew.  Upon exhaustive inspection of local vertical $\theta$  profiles, it became evident that at certain horizontal points high gradients well into the free atmposphere exceded those closer to the location of the \acs{CBL} height reasonably identified by eye.\\

Locating the \acs{ML} height using the multi-linear regression method employed proved more reliable, based on inspection of hundreds of local vertical $\theta$  profiles.  For a number of these profiles, ie those at points outside an actively impinging thermal, it was impossible even by eye to locate a reliable \acs{CBL} height based on a maximum in the vertical gradient but there as a discernible \acs{EZ}.  Conversely at points within an active thermal the \acs{ML} tops were characterized by a sharp vertical $\theta$ gradient and the absence of an \acs{EZ} similar to the zero-order model, since here entrainment has yet to begin.  All inspected local profiles showed a clear \acs{ML} and $h^{l}_{0}$. The distributions were seen to broaden with increased $(\overline{w^{'}\theta^{'}})_{s} $ and narrow with increased $ \gamma $.  When normalized by the height of the maximum average vertical potential gradient ($h$) what apparantly remains is the effect of $\gamma$ on the lower limit or lowest percentile. That is, the lowest \acs{ML} heights become lower with at lower $\gamma$ resulting in an apparant increased negative skew.  So, there is an overall narrowing of the scaled distributions with increased $\gamma$. This supports decreaased o f scaled \acs{EZ} depth with increased $\acs{Ri}$ of Section \ref{subsec:ellimscaledprof} since the $\acs{Ri}$s tend to group according to $\gamma$. The approximate range of the scaled \acs{EZ} basd on the $h^{l}_{0}$ distributions is about 0.2 - 0.4 of $h$ whereas that calculatted by  \citeauthor{BrooksFowler2} (\citeyear{BrooksFowler2}} based on distributions of local maximums in gradient was smaller (.05 - .2).  This could be partially explained by the local maximum gradient of the tracer profile being within the \acs{EZ} of points for example outside an actively impinging plume and so higher than  the\acs{ML} top defined here. \\  

Potential temperature and vertical velocity fluctuations ($\theta^{'}$ and $w^{'}$) at several vertical levels around the \acs{EL} were plotted as 2 dimensional histograms.  Increased $\overline{w^{'}\theta^{'}}_{s}$ causes an increase in positive temperature fluctuations and verical velocity as thermals become more vigorous, causing a higher $h$ so and a deeper \acs{EZ} over which relatively warmer air is pulled down.  The convective velocity scales ($\theta^{*}$ and $w^{*}$) were appplied to isolate the effects of $\gamma$, although it is accounted for indirectly via $h$ (see Section \ref{subsec:scales}). As shown in \citeauthor{Sorbjan}'s (\citeyear{Sorbjan}) $\theta^{'}$ is influenced  by $\gamma$.  For example at $h$ there is an apparant increase in the spread, as well as a shift thowards the positive.  So downward moving positive potential temperature fluctuations representing air from the \acs{FA} are more positive and negative fluctuations representing thermals are less negative, when scaled.  The former can easily be attributed to relatively warmer air over a shorter vertical distance (\acs{EZ}) being brought down but an explaination for the latter remains elusive.  I conclude that with incresed $\gamma$ there is a positive deviation from the convective temperature scale.  Also, there is an apparant damping of the scaled velocities associated with positive temperature fluctuations.  So the $w^{*}$ scales these less effectively with increased $\gamma$.  \\


At $z_{f}$ and $h$, ie within the \acs{EL} the quadrants of largest magnitude were upward and downward moving relatively cool, thermal,  air ($w^{'-}\theta^{'+}$ and $w^{'+}\theta^{'+}$). The $w^{'-}\theta^{'-}$, $w^{'+}\theta^{'-}$ and $w^{'-}\theta^{'+}$ quadrants do approximately cancel as \citeauthor{SullMoengStev} (\citeyear{SullMoengStev}) concluded. The downward moving warm quadrant ($w^{'-}\theta^{'+}$) at $h$ represents warmer free atmosphere air that is being entrained.  So it's magnitude at a certain point in time is an indication of the increased at a successive time in the region below.  In Figure \ref{fig:downwarm} the magnitude increases with respect to time is grouped according to $(\overline{w^{'}\theta^{'}})_{s}$.  Indeed it is an increasing proportion of  $(\overline{w^{'}\theta^{'}})_{s}$ and Figures \ref{fig:downwarm_wvel} and \ref{fig:downwarm_theta} show that its increase is primarily due to the increased positive temperature variance ($\theta^{'+}$).  While the velocity of downward warm quadrant $w^{'-}$ quickly approaches a constant proportion of $w^{*}$, the magnitude of temperature fluctuation approaches a constant proportion of $\gamma \Delta h$ or $\gamma (h_{1}-h)$ rather than the convective temperature scale $\theta^{*}$.  This is similar to the bouyancy scale for the upper \acs{EZ} used by \citeauthor{GarciaMellado}'s (\citeyear{GarciaMellado}). So the positive variance in temperature is related to difference in temperature over $\Delta h$ (or $h_{1} - h$) of the inital lapse rate $\gamma$.  A broad explaination is that air from $h_{1}$ (ie at potential temperature $\overline{\theta_{0}}(h_{1})$) is brought down by the motions described in \ref{subsec:cblgrowth} to $h$ at which point much of the air is still at $\overline{\theta_{0}}(h)$.  \citeauthor{GarciaMellado}'s (\citeyear{GarciaMellado} describe this region as the upper sub layer within the \acs{EZ} where the buoyancy and turbulence are strongly influenced by $\gamma$. The relationship between the variance and vertical gradient in temperature is clearly shown in the plots of each in \citeauthor{Sorbjan}'s (\citeyear{Sorbjan}) and \citeauthor{SullMoengStev}'s (\citeyear{SullMoengStev}) and \citeauthor{GarciaMellado}'s (\citeyear{GarciaMellado}). Their peaks within the acs{EL} seem to coincide.  In the mixed layer vigorous horizontal and vertical motion renders both close to zero.  The increased in average potential temperature in the lower \acs{EZ} (and upper \acs{ML}) is in part due to the entrainment of warm air from aloft which can be represented by the magnitude of ($\theta^{'+}_{h}$) coming down from above and this depends on $\gamma$. Another perspective is noticing the increased magnitude of the minimum $\overline{w^{'}\theta^{'}}$, seen here and in both \citeauthor{Sorbjan}'s (\citeyear{Sorbjan}) and {FedConzMir04} (\citeyear{FedConzMir04}), in the \acs{EZ} which then leads to increased $-\frac{\partial \overline{w^{'}\theta^{'}}}{\partial z}$ in the lower \acs{EZ}, since the profiles are otherwise similar, and so increased warming per Equation \ref{eq:warming}. \\             

\subsection{Relationship of Entrainment Layer Depth to Richardson Number}

The \acs{EZ} is inhomogenous and on average is a region of transition as clearly represented by the $\overline{\theta}$ profile
and where thermals overturn or recoil initiating entrainment as represented by the heat flux profile.  The $\overline{\theta}$ profile
partially characterises the thermodynamic state of the \acs{CBL} as well defining its three layer structure.  It is directly comparable to both bulk models and local $\theta$ profiles which in turn are comparable to a sounding unlike a heat flux profile which is an inherently average quantity.  Neither of the two comparable \acs{LES} (see Table \ref{table:el}) studies define the \acs{EL} based on the vertical $\frac{\partial \overline{\theta}}{\partial z}$ profile.  So, for to enable direct comparison heights were based on the heat flux profile.  Based in this framework \citeauthor{FedConzMir04}'s (\citeyear{FedConzMir04}) show decreasing scaled \acs{EZ} with increaseing $\acs{Ri }$ (decreasing $\acs{Ri}^{-1}$) and an exponent of of $-\frac{1}{2}$.  They say the decrease in the overall depth is due to a slight decrease in the scaled top limit over time.  However based on their plot, it seems to go from about .5 to .2. \citeauthor{BrooksFowler2} (\citeyear{BrooksFowler2}) found no clear \acs{Ri} dependence of the scaled \acs{EL} depth based on the heat flux profle, but their definition was based on the lower part, i.e, the portion that according to \citeauthor{FedConzMir04}'s (\citeyear{FedConzMir04}) does not vary in time.  In this study when based on the heat flux profiles as \citeauthor{FedConzMir04}'s (\citeyear{FedConzMir04}) did, there is no clear dependence on \acs{Ri}. This is supported by the similarity in time and accross runs of the vertical turbulent heat flux profiles when scaled by $(\overline{w^{'}\theta^{'}})_{s}$. The most obvious possible cause for this disagreement with the findings of citeauthor{FedConzMir04}'s (\citeyear{FedConzMir04}) is the difference in grid size shown in Table \ref{table:gridcomp}.  Inspection of their heat flux profiles confirms a relatively deeper region of negatflux as compared with those seen here (~.4 vs ~.25). Their surface heat flux was twice the highest used here, but their range of $\acs{Ri}$ is comparable to that of this study, which serves as confirmation that $\gamma$ is the more influential parameter.\\              

\begin{table}[htbp]
\label{table:el}
\begin{center}
%\centerline{
\begin{tabular}{ p{5cm} p{4cm} p{2cm} p{2cm}}
    %\hline
Publication & \acs{EL} & \acs{CBL} height & $\Delta \theta$\\ \hline
\citeauthor{FedConzMir04} (\citeyear{FedConzMir04}) & $z_{f1} - z_{f0}$ & $z_{f}$ &  $\overline{\theta}(z_{f1})-\overline{\theta}(z_{f0})$\\ [.3cm] %\hline
\citeauthor{BrooksFowler2} (\citeyear{BrooksFowler2}) & $2 \times (z_{f} - z_{f0})$ & $z_{f}$ & average of local values\\ \hline
\end{tabular}
\end{center}    
\end{table}


Here, when heights are defined on the vertical potential temperature profile cruves representing Equation \ref{eq:dhvsri} 

\begin{equation}
\frac{\Delta h}{h} \propto Ri ^{b} \tag{\ref{eq:dhvsri}}
\end{equation}

are grouped according to $\gamma$ and show an (or slope) exponent which increases in magnitude with increasing $\gamma$.  Overall there is a clear narrowing of the scaled \acs{EZ} depth with increased \acs{Ri} (decreased $\acs{Ri}^{-1}$).   When the $\frac{\partial \overline{\theta}}{\partial z}$ profile is scaled by $\gamma$ the relative magnitudes of the gradient in the upper \acs{EZ} are reversed.  This reflects as a higher \acs{EZ} bottom limit ($h_{0}$) and so a narrower \acs{EZ} with increased $\gamma$.  The resulting scaled \acs{EL} depth as defined in Figure \ref{fig:hdefs1} shows a dependence on Richardson number (\acs{Ri}) which appears to have an exponent of $-\frac{1}{2}$, as predicted and seen by \citeauthor{Boers89} (\citeyear{Boers89}), at lower \acs{Ri} (higher \acs{Ri}$^{-1}$) possibly increasing to $-1$ at higher \acs{Ri}.  Although based on different height definitions \citeauthor{FedConzMir04} (\citeyear{FedConzMir04}) concluded an $a = -\frac{1}{2}$ and \citeauthor{BrooksFowler2}'s (\citeyear{BrooksFowler2}) plots show curves with an apparant exponent less in magnitude than $-1$. The curves seem to separate out for each run.  In this study, before scaling the $\frac{\partial \overline{\theta}}{\partial z}$ profile curves separate out, but in the reverse order.  I see runs with higher stability exhibiting larger \acs{EL} depths.  \citeauthor{BrooksFowler2} (\citeyear{BrooksFowler2})'s runs with initially lower \acs{Ri} (higher \acs{Ri}$^{-1}$) have larger \acs{EL} depths than runs with initially higher \acs{Ri}s, even where \acs{Ri} values overlap.\\     

Neither study addresses a change in exponent with increased $\acs{Ri}$.  It is possible that there is a change in entrainment mechanism. \citeauthor{SullMoengStev} (\citeyear{SullMoengStev}) observed enfolding and engulfment at lower \acs{Ri}.  Whereas at higher \acs{Ri} when motion is more restricted, entraiment seemed to occur via trapping of thinner whisps at the edge of an upward moving thermal.  \citeauthor{Turner86} (\citeyear{Turner86}) also distiguishes between entrainment by convective overturning and recoil. \citeauthor{GarciaMellado}'s (\citeyear{GarciaMellado}) refer to a change in entrainment rate due to the effects of increased stability on the upper \acs{EZ} sublayer.  In this study, although the narrowing of the \acs{EZ} as defined here depends predominantly on the magnitude of the potential temperature gradient in the lower \acs{EZ} and upper \acs{ML} the scaled magnitude upper limit based on the $\frac{\partial \overline{\theta}}{\partial z}$ does appear to decrease slightly in time.\\

\section{Relationship of Entrainment Rate to Richardson Number (Q3)}

The magnitudes of \acs{Ri} numbers determined in this and the comparable studies are primarily influenced by the magnitude of the $\theta$ jump. So with different definitions you can have the same conditions represented by a different \acs{Ri} value. Here, I define it in two ways as both \citeauthor{FedConzMir04} (\citeyear{FedConzMir04}) and \citeauthor{GarciaMellado} (\citeyear{GarciaMellado}) did.  I do this based on the flux profile for the purpose of direct comparison with these studies and to observe how the change in definition effects Equation \ref{eq:ervsri}.\\

\begin{equation}
\frac{w_{e}}{w^{*}} \propto Ri^{a} \tag{\ref{eq:ervsri}}
\end{equation}

As in these studies, the larger jump ie that taken accross the \acs{EZ} ($\Delta \theta$) yields a larger $a$ as \citeauthor{FedConzMir04} (\citeyear{FedConzMir04}) conclude.  Although \citeauthor{GarciaMellado} (\citeyear{GarciaMellado}) interpret their curves as both asymptoting to a straight line ($a=-1$) as the upper \acs{EZ} sublayer narrows. Based on their plots alone, ie in the absence of their justification based on the derivation of the entrainment relation, I would not conclude this asymtotic behavior.  Rather for $Delta \theta$ I see a curve with increasing exponent exceeding magnitude $-1$ at higher $\acs{Ri}$.  For $\delta \theta$ I see a curve with exponent less in magnitude than $-1$.\\

That there is an analogous distinction between plots of Equation \ref{eq:ervsri} using $\Delta \theta$ vs $\delta \theta$ when all heights are defined on the $\frac{\frac{\partial \overline{\theta}}{\partial z}}{\gamma}$ profile lends some credence to this framework.  Scatter is least when the $\theta$ jump is defined accross the \acs{EL}.  In this plot $-\frac{3}{2}$ fits at higher \acs{Ri} (lower \acs{Ri}$^{-1}$) and $-1$ seems to fit at lower \acs{Ri}.  Combined with the apparant change in $b$ for Equation \ref{eq:dhvsri} this could be interpreted as an indication of a change in entrainment regime at increased \acs{Ri}.\\ 

\section{Conclusion}

The magnitude and variance of local height increase with increasing $\overline{w^{'}\theta^{'}}_{s}$ and decrease with increasing $\gamma$.  Similarly for the vertical velocity fluctuations in the \acs{EZ}.  However increased $\gamma$ results in an increase in the positive temperature fluctuations at $h$. The magnitude of these positive temperature perturbations at points where vertical velocity is negative represents downward moving entrained air.  Below $h$ in the lower \acs{EZ} the temperature gradient increases with increasing $\gamma$. So, the growth of the idealized dry \acs{CBL} is driven by $\overline{w^{'}\theta^{'}}_{s}$ and suppressed by $\gamma$. But warming is due to $\overline{w^{'}\theta^{'}}_{s}$ and the entrainment of air from aloft the temperature of which in turn depends on $\gamma$.\\

Both the \acs{EZ} depth and \acs{CBL} height based on the average $\frac{\frac{\partial \overline{\theta}}{\partial z}}{\gamma}$ profile show dependence on \acs{Ri} as seen in other studies and justified theoretically.  This profile serves to characterize the \acs{CBL} and links bulk models to soundings via an \acs{LES}.  So it is a valid way of defining the \acs{CBL} and its \acs{EZ}. Plots of Equations exhibit changes in exponent.  A change in entrainment mechanism or regime with increased $\acs{Ri}$ has been observed in measurement as well as \acs{LES} and justified theoretically. I suggest the change in exponents represent this.\\

Throughout this study threads the influence of $\gamma$.  The convecive time scale $\tau = \frac{w^{*}}{h}$ and $\acs{Ri}$ group according to this parameter.  This justifies  the use of the brunt vaisalla time scale as well as the constant heat flux with varying lapse rate of 
\citeauthor{FedConzMir04} (\citeyear{FedConzMir04}).  I conclude once the effects of $\overline{w^{'}\theta^{'}}_{s}$ are accounted for through $h$, $\gamma$ remains the dominant parameter in idealized \acs{CBL} entrainment.\\ 

\FloatBarrier


\endinput

Any text after an \endinput is ignored.
You could put scraps here or things in progress.
