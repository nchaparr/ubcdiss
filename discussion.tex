%% The following is a directive for TeXShop to indicate the main file
%%!TEX root = diss.tex

\chapter{Results in Context}
\label{ch:results}
\setlength{\parindent}{0cm}

Much work has been done to develop our understanding of \acs{CBL} entrainment, so this chapter will focus on how my results fit into the discussion established in the literature.  I hone in on five closely related publications for comarison.  \citeauthor{SullMoengStev}'s \citeyear{SullMoengStev} \acs{LES} study was seminal in shedding light on \acs{CBL} entrainment zone structure.  Whereas \citeauthor{BrooksFowler2}'s (\citeyear{BrooksFowler2}) work contains the most recent \acs{LES} results on the topic framed within an up to date review of \acs{CBL} height and \acs{EZ} definitions.  \citeauthor{FedConzMir04} (\citeyear{FedConzMir04}) bridges \acs{LES} and bulk models, while the closely related \acs{DNS} study of \citeauthor{GarciaMellado} (\citeyear{GarciaMellado}) introduces the two-layer \acs{EZ} concept and answers questions regarding the scale resolution required to realistically capture \acs{CBL} growth and \acs{EZ} structure.  \citeauthor{SullPat} (\citeyear{SullPat}) addressed this last point using an \acs{LES}, and was pivotal in guiding the choice of grid-size in the study described in this thesis.  Finally \citeauthor{Sorbjan1} (\citeyear{Sorbjan1}) focused on the effects of upper lapse rate on the turbulence in the upper \acs{CBL} and provided ideas upon which I based Section \ref{sec:q1}.\\

Section \ref{sec:gensetup} draws upon the results of \citeauthor{SullPat} (\citeyear{SullPat}) to address the need for high resolution in the entrainment zone (\acs{EZ}) such that the steep gradients are sufficiently represented. I present and compare those of my results that are pertinent and refer to how \citeauthor{GarciaMellado} (\citeyear{GarciaMellado}) speaks to this point.  Finally, I touch upon how my domain size and initial conditions compare with those of the similar \acs{LES} studies the bearing in mind possible implications of the similarities and differences.\\

In Section \ref{sec:entzonestruc} I describe problems encountered when using the gradient method, as well as discuss the results obtained using my chosen method of determining local \acs{ML} height.  All of this is set in context with the results of \citeauthor{SullMoengStev}'s \citeyear{SullMoengStev} and \citeauthor{BrooksFowler2}'s (\citeyear{BrooksFowler2}).  The influence of $\gamma$ on the turbulent fluctuations of vertical velocity and potential temperature is discussed and compared with the results of \citeauthor{Sorbjan1} (\citeyear{Sorbjan1}) before addressing the dependence of the downward moving positive potential temperature fluctuations at $h$ on this parameter.  An explaination of the potential temperature fluctuation scale $\delta h \gamma$ follows.\\ 

A primary goal of this thesis was to test the average $\theta$ profile as a basis for defining the \acs{EZ} boundaries.  Before comparing the results using heights thus defined in Section \ref{sec:ezbound}, I base all heights on the vertical heat flux ($\overline{w^{'}\theta^{'}}$) profile to enable direct comparison with the results of \citeauthor{BrooksFowler2} (\citeyear{BrooksFowler2}) and \citeauthor{FedConzMir04} (\citeyear{FedConzMir04}).  I discuss similarities, differences and possible reasons for the latter. I then compare results based on the potential temperature profile focusing on the exponent $b$ in Equation \ref{eq:dhvsri} and how it varies depending on $\acs{Ri}$.\\


Section \ref{sec:erparam} contains an analogous comparison to that described above.  Heights are defined, first based on the vertical heat flux profile for direct comparison with the results of \citeauthor{FedConzMir04} (\citeyear{FedConzMir04}) and \citeauthor{GarciaMellado} (\citeyear{GarciaMellado}) and then based on the average $\theta$ profile.  Each of these two comparisons is further broken into two, in order to address the effect of, definining the $\theta$ jump accross the \acs{EZ} as in Figure \ref{fig:1storder}, vs at $h$ as in Figure \ref{fig:0order}.  In all, there are four plots of Equation \ref{eq:ervsri} for the purpose of observing how the exponent $a$ varies depending on $\theta$ jump definition, as well as with $\acs{Ri}$.  This variation is discussed in the context of the results of the other comparable studies.\\          
%need chapter map here

\section{Comparison of general Set-up}
\label{sec:gensetup}
\FloatBarrier

\subsection{Significance of Grid-size}

\citeauthor{SullPat} (\citeyear{SullPat}) found that the shapes of the average potential temperature ($\overline{\theta}$) and heat flux ($\overline{w^{'}\theta^{'}}$) profiles, as well as the measured \acs{CBL} height vary depending on grid size.  The resolution at which convergence begins is listed in Table \ref{table:gridcomp}.  At lower resolution the $\overline{\theta}$ and $\overline{w^{'}\theta^{'}}$ profiles are such that the entrainment zone (\acs{EZ}) is a larger portion of the \acs{CBL} and measured \acs{CBL} height is higher.  Overall they concluded that vertical resolution was critical.  This compliments the conclusion \citeauthor{BrooksFowler2} (\citeyear{BrooksFowler2}) reached when discussing their resolution test.  That is, to capture the steep vertical gradients in the \acs{EZ} requires high vertical resolution. \\

\begin{table}[htbp]
\caption[]{Grid spacing around the \acs{EL} used in comparable \acs{LES} studies. Those used for resolution tests are not listed here.  For \citeauthor{SullPat}'s \citeyear{SullPat} resolution study I list the grid sizes at which profiles within the \acs{EL} and \acs{CBL} height evolution began to converge.}

    \begin{center}
%\centerline{
    \begin{tabular}{ p{5cm} p{3cm} p{3cm}}
    %\hline
Publication & $\Delta x$, $\Delta y$, $\Delta z$ & Horizontal \\ \hline
Publication & in the \acs{EZ} (m)& Domain (km$^{2}$) \\ \hline
      \citeauthor{SullMoengStev} (\citeyear{SullMoengStev}) & 33, 33, 10 & 5 x 5 \\ %\hline 
      \citeauthor{FedConzMir04} (\citeyear{FedConzMir04}) & 100, 100, 20 & 5 x 5 \\ [.3cm] %\hline
      \citeauthor{BrooksFowler2} (\citeyear{BrooksFowler2}) & 50, 50, 12 & 5 x 5 \\%\hline
      \citeauthor{SullPat} (\citeyear{SullPat}) &  20, 20, 8 & 5 x 5\\ %\hline
      This study & 25, 25, 5 &  3.4 x 4.8\\ \hline       
    \end{tabular}
%}
\label{table:gridcomp}   
\end{center}    
\end{table}


As \citeauthor{Turner86} discusses in his \citeyear{Turner86} review of turbulent entrainment, smaller scale processes such as those at the molecular level are relatively unimportant.  Large scale engulfment and trapping between thermals dominates.  If the ergodic assumption holds and potential temperature variance ($\overline{\theta^{'2}}$) is calculated based on the difference at a point from the horizontal average, it is a measure of horizontal variance at a point in time.  \citeauthor{SullPat} (\citeyear{SullPat}) found that the vertical distance over which $\overline{\theta^{'2}}$ varied significantly, more or less converged at the resolution shown in Table \ref{table:gridcomp}.  But the maximum $\overline{\theta^{'2}}$ continued to increase up to their finest grid spacing ($\Delta x=5$, $Delta y = 5$, $Delta z = 2$).\\

The question as to whether mixing and gradients within the \acs{EZ} are adequately resolved motivates \acs{DNS}  studies such as that of \citeauthor{GarciaMellado} (\citeyear{GarciaMellado}). These authors found the entrainment ratio ($\frac{\overline{w^{'}\theta^{'}}_{z_{f}}}{\overline{w^{'}\theta^{'}}_{s}}$) to be about 0.1 which is lower than observed by \citeauthor{FedConzMir04} (\citeyear{FedConzMir04}), but close to what was seen here in Figures \ref{fig:fluxprofs2hrs} and \ref{fig:tempgradfluxprofs1005}.  Based on their $\overline{w^{'}\theta^{'}}$ profiles the depth of the region of negative flux is comparable to what is shown in Figure \ref{fig:scaledELlims}.  Furthermore, these authors concluded that the production and destruction rates of turbulence kinetic energy (\acs{TKE}), as well as the entrainment ratio used to calculate the entrainment rate, were effectively independent of molecular scale processes.\\  
  
The \acs{FFT} energy spectra of the turbulent velocities at the top of the \acs{ML} show a substantial resolved inertial subrange giving confidence in the choice of horizontal grid size used. In the \acs{EZ} where turbulence is intermittent, the dominant energy containing structures are smaller, and decay to the smallest resolved turbulent structures is steeper. This confirms the assertion of \citeauthor{GarciaMellado} (\citeyear{GarciaMellado}) that the \acs{EZ} is separated into two sub-layers in terms of turbulence scales.\\

\subsection{Horizontal Domain}

The horizontal domain in this study is relatively small (see Table \ref{table:gridcomp}). However, visualizations of horizontal and vertical slices clearly showed multiple resolved thermals.  Their diameters increased with \acs{CBL} height, but remained less than or on the order of 100 meters.  \citeauthor{SullMoengStev} (\citeyear{SullMoengStev}) carried out one run on a smaller domain with higher resolution, noticed it resulted in lower \acs{CBL} height and concluded this was due to restricted horizontal thermal size. However, given the results of \citeauthor{SullPat} (\citeyear{SullPat}) it could have been an effect of grid-size.\\   

When defining heights based on average profiles \citeauthor{SullMoengStev} (\citeyear{SullMoengStev}) produced jagged, oscillating time-series and \citeauthor{BrooksFowler2} (\citeyear{BrooksFowler2}) encountered significant scatter in plots of Equation \ref{eq:ervsri}.  But the heights based on average profiles here, using an ensemble of cases, varied smoothly in time.  This could be attributed to a smoother profile based on a greater number horizontal points (10*128*192).\\

\subsection{Initial Conditions}

The principle parameter describing the balance of forces in dry, idealized \acs{CBL} entrainment is the Richardson number ($\acs{Ri}$) and its magnitude depends on the way in which the $\theta$ jump is defined.  Varying the $\theta$ jump definition causes identical conditions to be described by different $\acs{Ri}$ values.  The $\acs{Ri}$ range in this study was dependent on variation in $\gamma$ (see Figure \ref{fig:invristime}).  \citeauthor{BrooksFowler2} (\citeyear{BrooksFowler2}) and \citeauthor{SullMoengStev} (\citeyear{SullMoengStev}) imposed a $\theta$ jump of varying strength topped by a constant $\gamma$.  Whereas \citeauthor{FedConzMir04} (\citeyear{FedConzMir04}) initialized with a layer of uniform $\theta$.  They varied $\gamma$ and kept $\overline{w^{'}\theta^{'}}_{s}$ constant for each run.  Their initial conditions, definitions of the $\theta$ jump and $\acs{Ri}$ range are directly comparable to those of this study, whereas those of \citeauthor{BrooksFowler2} (\citeyear{BrooksFowler2}) and \citeauthor{SullMoengStev} (\citeyear{SullMoengStev}) are quite different.\\    

\begin{table}[htbp]
\caption[]{Initial conditions used in comparable \acs{LES} studies.}

    \begin{center}
%\centerline{
    \begin{tabular}{ p{4cm} p{1.4cm} p{1.4cm} p{1.7cm} p{1.8cm}}
    %\hline
Publication & $\overline{w^{'}\theta^{'}}_{s}$& $\gamma$& Initial $\theta$ & $\acs{Ri}$ \\ 
& $Wm^{-2}$ & $Kkm^{-1}$ & Jump K & range \\ \hline
      \citeauthor{SullMoengStev} (\citeyear{SullMoengStev}) & 20 - 450& 3  &.436 - 5.17 & 1 - 100\\ %\hline 
      \citeauthor{FedConzMir04} (\citeyear{FedConzMir04}) & 300 & 1 - 10 & NA & 10 - 40\\ %[.3cm] %\hline
      \citeauthor{BrooksFowler2} (\citeyear{BrooksFowler2}) &  10 -100 &  3& 1 - 10 &10 - 100 \\ %\hline
      This study & 60 - 150 & 2.5 - 10& NA & 10 - 30\\ \hline 
      
    \end{tabular}
%}
\label{table:initconditcomp}   
\end{center}    
\end{table}

%\clearpage

\section{Entrainment Zone Structure}
\label{sec:entzonestruc}
%\subsection{The Gradient Method is problematic}
\subsection{Local \acs{ML} Heights}

\citeauthor{SullMoengStev} (\citeyear{SullMoengStev}) determined local \acs{CBL} height by locating the point of maximum $\theta$ gradient.  Analysis of the resulting distributions showed dependence of standard deviation and skewness on Richardson number ($\acs{Ri}$).  The normalized standard deviation decreased with increased \acs{Ri} whereas skewness was almost bimodal; being negative at high $\acs{Ri}$ and positive and low \acs{Ri}.  Initially in this study, I applied a similar method and found local \acs{CBL} height distributions with lower $\acs{Ri}$ to have positive skew.  Upon exhaustive inspection of local vertical $\theta$  profiles such as those in Figure \ref{fig:rssfitslow}, it became evident that at certain horizontal points high gradients well into the free atmosphere exceeded those closer to the location of the \acs{CBL} height reasonably identified by eye.\\

\subsection{Local \acs{ML} Height Distributions}

Locating the local \acs{ML} height ($h^{l}_{0}$) using the multi-linear regression method described in Chapter 2 proved more reliable.  The resulting distributions, normalized by \acs{CBL} height ($h$) in Figure \ref{fig:localhpdf},  showed a decrease in the lowest $\frac{h^{l}_{0}}{h}$ resulting in an apparent increased negative skew with decreasing stability (decreasing $\acs{Ri}$). This, combined with a widening of the distribution agrees, with the findings of \citeauthor{SullMoengStev} (\citeyear{SullMoengStev}) and supports the results based on the average profiles in Section \ref{sec:deltahri}.  The approximate scaled \acs{EZ} based on the $\frac{h^{l}_{0}}{h}$ distributions is about 0.2 - 0.4 whereas that based on distributions of local maximum tracer gradients by  \citeauthor{BrooksFowler2} (\citeyear{BrooksFowler2}) was smaller (.05 - .2).  However, the local maximum gradient of the tracer profile would likely be within the \acs{EZ} at points outside an actively impinging plume and so higher than $h^{l}_{0}$ defined here. \\  

%Potential temperature and vertical velocity fluctuations ($\theta^{'}$ and $w^{'}$) at several vertical levels around the \acs{EL} were plotted as 2 dimensional histograms.  Increased $\overline{w^{'}\theta^{'}}_{s}$ causes an increase in positive temperature fluctuations and verical velocity as thermals become more vigorous, causing a higher $h$ so and a deeper \acs{EZ} over which relatively warmer air is pulled down.  The convective velocity scales ($\theta^{*}$ and $w^{*}$) were appplied to isolate the effects of $\gamma$, although it is accounted for indirectly via $h$ (see Section \ref{subsec:scales}). As shown in \citeauthor{Sorbjan}'s (\citeyear{Sorbjan}) $\theta^{'}$ is influenced  by $\gamma$.  For example at $h$ there is an apparant increase in the spread, as well as a shift thowards the positive.  So downward moving positive potential temperature fluctuations representing air from the \acs{FA} are more positive and negative fluctuations representing thermals are less negative, when scaled.  The former can easily be attributed to relatively warmer air over a shorter vertical distance (\acs{EZ}) being brought down but an explaination for the latter remains elusive.  I conclude that with incresed $\gamma$ there is a positive deviation from the convective temperature scale.  Also, there is an apparant damping of the scaled velocities associated with positive temperature fluctuations.  So the $w^{*}$ scales these less effectively with increased $\gamma$.  \\

\subsection{Local vertical Velocity and Potential Temperature Fluctuations}

As expected, with increased $\overline{w^{'}\theta^{'}}_{s}$ the variance and magnitude of the vertical velocity fluctuations within and at the limits of the \acs{EZ} increase.  Greater turbulent velocity causes a higher \acs{CBL} and a deeper \acs{EZ} over which: relatively warmer air from higher up is brought down; and relatively cooler air from below is brought up.  So the magnitude of the potential temperature fluctuations ($\theta^{'}$) and the width of their distribution increases. All of this agrees with the findings of \citeauthor{Sorbjan} (\citeyear{Sorbjan}), but the portion of the scaled $w^{'}$ ($\frac{w^{'}}{w^{*}}$) distribution where scaled $\theta^{'}$ ($\frac{\theta^{'}}{\theta^{*}}$) is positive, in Figure \ref{fig:scaled_fluxquadsh1}, appears to narrow as $\gamma$ increases. This seems to contradict his assertion that $w^{'}$ is independent of this parameter while the effectiveness of $w^{*}$ as a scale for $w^{'-}$ where $\theta^{'}>0$ in Figure \ref{fig:downwarm} supports it.\\

\subsection{Downward moving warm Air at $h$}

 Although the motion of the thermals dominates within the \acs{EZ}, the $\overline{w^{'-}\theta^{'-}}$, $\overline{w^{'+}\theta^{'-}}$ and $\overline{w^{'-}\theta^{'+}}$ quadrants do approximately cancel leaving $\overline{w^{'-}\theta^{'+}}$ as the net dynamic, as \citeauthor{SullMoengStev} (\citeyear{SullMoengStev}) concluded. The downward moving warm quadrant at $h$ ($\overline{w^{'-}\theta^{'+}}_{h}$) represents warmer free atmosphere (\acs{FA}) air that is being entrained.  So its magnitude, at a certain point in time, is an indication of how much the region below will be warmed due to entrainment at a successive time.  The increase of $\overline{w^{'-}\theta^{'+}}_{h}$ in time is primarily due to the increased average positive potential temperature fluctuation at $h$ ($\overline{\theta^{'+}}_{h}$) which is effectively scaled by the temperature scale $(h_{1}-h)\gamma = \delta h \gamma$ (see the Figures of Section \ref{subsec:downwarm}).  A similar scale was introduced by \citeauthor{GarciaMellado} (\citeyear{GarciaMellado}) to further their line of reasoning that the buoyancy in the upper \acs{EZ} is determined by $\gamma$. Figure \ref{fig:deltagamma} illustrates a broad qualitative explanation.  At $h$ much of the air is at the background (or initial) potential temperature $\overline{\theta}_{0}(h)$. 

\begin{figure}[htbp]
    \centering
    %plot_height.py[master 1573b9d] h vs time plot
    \includegraphics[scale=0.4]{/newtera/tera/phil/nchaparr/tera2_cp/nchaparr/ubcdiss/pngs/deltagamma}
    \caption[Illustration of \acs{EZ} Potential Temperature Scale based on $\gamma$]{Illustration of the potential temperature scale $(h_{1}-h)\gamma = \delta h \gamma$: The curves represent a vertical cross-section of thermal tops.  Between them is stable air at the initial lapse rate $\gamma$. $h_{1}$ and $h$ correspond to the highest and average thermal height respectively and $h_{0}$ is the top of the well mixed region (\acs{ML}).  The horizontally uniform, initial potential temperature is $\theta_{0} = \overline{\theta}_{0}$. A thermal will initiate the downward movement of air from $h_{1}$ to $h$, and the difference between its potential temperature and that of the background stable air at $h$ is $(h_{1}-h)\gamma = \delta h \gamma$.}
    \label{fig:deltagamma}   % label should change
\end{figure}
%diagram with plumes and as and $\gamma$

Some air at potential temperature $\theta = \overline{\theta}_{0}(h_{1})$ is brought down from the upper \acs{EZ} limit ($h_{1}$) resulting in positive potential temperature fluctuations ($\theta^{'+}$) at $h$.\\

\citeauthor{GarciaMellado} (\citeyear{GarciaMellado}) suggest that the buoyancy in the lower portion of the \acs{EZ}, i.e. from a point just below $h$ down, is more strongly influenced by the vigorous turbulence of the \acs{ML} than by $\gamma$.  So mixing reduces the difference between, the potential temperature at the top of the \acs{ML}, and that at or just below $h$.  However the observation in Section \ref{subsec:ellimscaledprof}, that the magnitude of the average vertical potential temperature gradient ($\frac{\partial \overline{\theta}}{\partial z}$) in the upper \acs{ML} increases with increasing $\gamma$, indicates that the influence of this parameter extends further down.  On a related note, the magnitude of the minimum heat flux ($\overline{w^{'}\theta^{'}}_{z_{f}}$) is seen to increase with increased $\gamma$, here and in both \citeauthor{Sorbjan} (\citeyear{Sorbjan}) and \citeauthor{FedConzMir04} (\citeyear{FedConzMir04}).  It is reasonable to suggest this leads to an increased negative vertical heat flux gradient ($-\frac{\partial \overline{w^{'}\theta^{'}}}{\partial z}$) in the lower \acs{EZ} and so increased warming per Equation \ref{eq:warming1}.

\begin{equation}
\frac{\partial \overline{\theta}}{\partial t} = -\frac{\partial}{\partial z}\overline{w^{'}\theta^{'}} \tag{\ref{eq:warming1}}
\end{equation}

\section{Entrainment Zone Boundaries}
\label{sec:ezbound}
The \acs{EZ} is inhomogeneous, but on average is a region of transition as clearly represented by the $\overline{\theta}$ profile.  It's where relatively cooler thermals overturn or recoil initiating entrainment as represented by the vertical heat flux ($\overline{w^{'}\theta^{'}}$) profile.  The $\overline{\theta}$ profile partially characterizes the thermodynamic state of the \acs{CBL} as well defining its three layer structure.  It is directly comparable to both bulk models and local $\theta$ profiles which in turn are comparable to a sounding, unlike a $\overline{w^{'}\theta^{'}}$ profile which is an inherently average quantity.\\

\subsection{Direct Comparison based on the vertical Heat Flux Profile}

Neither of the two comparable \acs{LES} studies in Table 3.3 define the \acs{EL} based on the $\frac{\partial \overline{\theta}}{\partial z}$ profile.  So to enable direct comparison, heights were based on the heat flux ($\overline{w^{'}\theta^{'}}$) profile as in Figure \ref{fig:hdefs1}.  In this framework \citeauthor{FedConzMir04} (\citeyear{FedConzMir04}) show decreasing scaled \acs{EZ} with increasing $\acs{Ri}$ and conclude an exponent of of $b = -\frac{1}{2}$.  They attribute the decrease in the overall scaled depth to a slight decrease in the scaled upper boundary over time.  However based on their plot in Figure \ref{fig:FedEZRi} the decrease seems more than slight, varying from about 0.5 to 0.2.\\

\begin{figure}[htbp]
    \centering
    %plot_height.py[master 1573b9d] h vs time plot
    \includegraphics[scale=1]{/newtera/tera/phil/nchaparr/tera2_cp/nchaparr/ubcdiss/pngs/FedEZRi}
    \caption[Plot of the relationship between scaled \acs{EZ} depth and Richardson number from \citeauthor{FedConzMir04}'s (\citeyear{FedConzMir04})]{Figure 9 from \citeauthor{FedConzMir04} (\citeyear{FedConzMir04}) representing Equation \ref{eq:dhvsri} using three different Richardson numbers, in log-log coordinates.  Heights are based on the $\overline{w^{'}\theta^{'}}$ profile as in Figure \ref{fig:hdefs1} and their $z_{i}$ is my $z_{f}$. $\frac{\delta z_{i}}{z_{i}}$. $Ri_{\Delta b}$ (circles) and $Ri_{\delta b}$ (crosses) correspond directly to those determined here using $\delta \theta$ and $\Delta \theta$.  Note that their $\Delta$ refers to the smaller jump measured at $z_{f}$, whereas I use it for the larger.  $Ri_{N}$ (triangles) is the Richardson number defined in Equation \ref{eq:gradri}, with $w^{*}$ and $z_{f}$ as the velocity and length scale.}
    \label{fig:FedEZRi}   % label should change
\end{figure}


 \citeauthor{BrooksFowler2} (\citeyear{BrooksFowler2}) found no clear $\acs{Ri}$ dependence of the scaled \acs{EZ} depth defined based on the $\overline{w^{'}\theta^{'}}$ profile.  But their definition hinged solely upon the lower part ($z_{f1} - z_{f}$) which according to \citeauthor{FedConzMir04} (\citeyear{FedConzMir04}) does not vary in time.  Figure \ref{fig:deltahinvri_scaled} of this thesis shows that when I defined the \acs{EZ} based on the $\overline{w^{'}\theta^{'}}$ profile as \citeauthor{FedConzMir04} (\citeyear{FedConzMir04}) did, the scaled \acs{EZ} depth had no clear dependence on $\acs{Ri}$. This is supported by the similarity in time and across runs of the vertical turbulent heat flux profiles when scaled by $(\overline{w^{'}\theta^{'}})_{s}$ in Figures \ref{fig:scaledfluxprofs15010} and \ref{fig:fluxprofs2hrs}.\\

The most obvious possible cause for disagreement with the results of \citeauthor{FedConzMir04} (\citeyear{FedConzMir04}) is the difference in grid size shown in Table \ref{table:gridcomp}.  Inspection of their $\overline{w^{'}\theta^{'}}$ profiles confirms a relatively deeper scaled region of negative flux as compared with those seen here (~.4 vs ~.25). Their surface heat flux $\overline{w^{'}\theta^{'}}_{s}$ was twice the highest used here, but their range of $\acs{Ri}$ is comparable to that of this study.  The latter point although not directly relevant here, serves as confirmation that $\gamma$ is the more influential parameter.\\              

\begin{table}[htbp]
\label{table:elandri}
\caption[\acs{EZ} Definitions used in comparable Studies]{\acs{EZ} Definitions used in comparable Studies}

\begin{center}
%\centerline{
\begin{tabular}{ p{4cm} p{2cm} p{1.5cm} p{3cm}}
    %\hline

Publication & \acs{EZ} Depth & \acs{CBL} height & $\theta$ Jump\\ \hline
\citeauthor{FedConzMir04} (\citeyear{FedConzMir04}) & $z_{f1} - z_{f0}$ & $z_{f}$ &  $\overline{\theta}(z_{f1})-\overline{\theta}(z_{f0})$\\ [.3cm] %\hline
\citeauthor{BrooksFowler2} (\citeyear{BrooksFowler2}) & $2 \times (z_{f} - z_{f0})$ & $z_{f}$ & average of local values\\ \hline

\end{tabular}
\end{center}    
\end{table}

\subsection{General Comparison using the Potential Temperature Profile}

Here, when heights are defined based on the scaled vertical potential temperature gradient profile $\frac{\frac{\partial \overline{\theta}}{\partial z}}{\gamma}$ the curve representing Equation \ref{eq:dhvsri} 

\begin{equation}
\frac{\Delta h}{h} \propto Ri^{b} \tag{\ref{eq:dhvsri}}
\end{equation}

shows an exponent $b$ which increases in magnitude, from about $-\frac{1}{2}$ as predicted and seen by \citeauthor{Boers89} (\citeyear{Boers89}), to about $-1$ as justified in \citeauthor{StullNelEl} (\citeyear{StullNelEl}),  with increasing $\acs{Ri}$ (decreasing $\acs{Ri}^{-1}$).  Overall there is a clear narrowing of the scaled \acs{EZ} depth with increasing $\acs{Ri}$ (decreasing $\acs{Ri}^{-1}$) as supported by the local height distributions in Section \ref{subsec:locmlh}.  Although based on different height definitions, \citeauthor{FedConzMir04} (\citeyear{FedConzMir04}) concluded an exponent $b = -\frac{1}{2}$ and \citeauthor{BrooksFowler2}'s (\citeyear{BrooksFowler2}) plots show curves with an apparent exponent less in magnitude than $-1$, in Figure \ref{fig:BandFEZ}. \\

\begin{figure}[htbp]
    \centering
    %plot_height.py[master 1573b9d] h vs time plot
    \includegraphics[scale=1.3]{/newtera/tera/phil/nchaparr/tera2_cp/nchaparr/ubcdiss/pngs/BandFEZ}
    \caption[Relationship of Scaled \acs{EZ} depth to Richardson number from \citeauthor{BrooksFowler2}'s (\citeyear{BrooksFowler2})]{Panel (a) from Figure 5 in \citeauthor{BrooksFowler2} (\citeyear{BrooksFowler2}) representing Equation \ref{eq:dhvsri}:The normalized \acs{EZ} depth is determined in three ways (i) the upper and lower percentiles from the distribution of local \acs{CBL} height (maximum tracer gradient), normalized by the average of the local heights (pale grey) (ii) the average of local scaled \acs{EZ} depths based on wavelet covariance (dark grey) and (iii) the average of the locally determined \acs{EZ} depths scaled by the average of the locally determined heights (black), based on wavelet covariance.  Their $\theta$ jump is an average of the potential temperature differences across the local \acs{EZ} depths.}
    \label{fig:BandFEZ}   % label should change
\end{figure}

The curves representing each run in Figure \ref{fig:BandFEZ} fan out.  In Figure \ref{fig:scaledeltahinvri} of this thesis, before scaling, the $\frac{\partial \overline{\theta}}{\partial z}$ profile curves separate out, but in the reverse order.  \acs{CBL}s under higher stability, and so higher $\acs{Ri}$, have larger scaled \acs{EZ} depths.  Whereas \citeauthor{BrooksFowler2}'s (\citeyear{BrooksFowler2}) runs with initially lower $\acs{Ri}$ have larger scaled \acs{EZ} depths than those with higher, even where $\acs{Ri}$ values overlap. Nonetheless, that there appears a family of separate but similar curves rather than a single curve hints at an underlying scaling parameter.\\     

Neither study referenced in Table 3.3 addresses the change in exponent with increased $\acs{Ri}$ that I observe in Figure \ref{fig:loglogdeltahinvri}.  It is reasonable to suggest that this represents a change in entrainment mechanism. \citeauthor{SullMoengStev} (\citeyear{SullMoengStev}) observed enfolding and engulfment at lower $\acs{Ri}$.  Whereas at higher $\acs{Ri}$ when motion is more restricted, entrainment seemed to occur via trapping of thinner wisps at the edge of an upward moving thermal.  \citeauthor{Turner86} (\citeyear{Turner86}) also distinguishes between entrainment by convective overturning and recoil. \citeauthor{GarciaMellado} (\citeyear{GarciaMellado}) refer to a change in entrainment rate due to the effects of increased stability on the upper \acs{EZ} sub-layer.  In this study, the narrowing of the \acs{EZ} depends predominantly on the magnitude of the average vertical potential temperature gradient $\frac{\partial \overline{\theta}}{\partial z}$ in the lower \acs{EZ} and upper \acs{ML}.  However, the scaled magnitude of upper limit in Figure \ref{fig:scaledELlims} (a) does appear to decrease slightly in time.  This could correspond to the slowly decreasing upper sub layer of the \acs{EZ} mentioned in both \citeauthor{GarciaMellado} (\citeyear{GarciaMellado}) and \citeauthor{FedConzMir04} (\citeyear{FedConzMir04}).\\

\section{Entrainment Rate Parameterization}
\label{sec:erparam}
$\acs{Ri}$ magnitude determined in this and the comparable studies is primarily influenced by the magnitude of the $\theta$ jump.  Here, I define it in two ways as \citeauthor{FedConzMir04} (\citeyear{FedConzMir04}) did.  I do this based on the $\overline{w^{'}\theta^{'}}$ profile, as in Figure \ref{fig:hdefs1} and Table \ref{table:reldefs} for the purpose of direct comparison and to observe how the change in definition effects Equation \ref{eq:ervsri}.\\

\begin{equation}
\frac{w_{e}}{w^{*}} \propto Ri^{a} \tag{\ref{eq:ervsri}}
\end{equation}

\subsection{Direct Comparison based on the vertical Heat Flux Profile}
The larger jump, i.e. that taken across the \acs{EZ} ($\Delta \theta$) as in Figure \ref{fig:1storder}, yields a larger value of $a$ as \citeauthor{FedConzMir04} (\citeyear{FedConzMir04}) conclude.  \citeauthor{GarciaMellado} (\citeyear{GarciaMellado}) interpret both curves as asymptotic to straight lines ($a=-1$) as the upper \acs{EZ} sub-layer narrows. Based on their plots in Figure \ref{fig:GarcMelERRi}, in the absence of their justification based on the derivation of the entrainment relation, for $\Delta \theta$ I see a curve (grey and blue) with increasing exponent exceeding magnitude $-1$ at higher $\acs{Ri}$.  For their version of $\delta \theta$ I see a curve (grey and red) with exponent less in magnitude than $-1$.\\

\begin{figure}[htbp]
    \centering
    %plot_height.py[master 1573b9d] h vs time plot
    \includegraphics[scale=1]{/newtera/tera/phil/nchaparr/tera2_cp/nchaparr/ubcdiss/pngs/GaMeERRi}
    \caption[Plots of scaled entrainment rate vs Richardson number from \citeauthor{GarciaMellado} (\citeyear{GarciaMellado})]{Figure 11 from \citeauthor{GarciaMellado} (\citeyear{GarciaMellado}) and representing equation \ref{eq:ervsri} based on the two $\theta$ jumps.  The grey and blue curve is based on $\Delta \theta$ and the (grey and) red curve is based on $\overline{\theta}(h) - \overline{\theta}_{0}(h)$ which is slightly different to the $\delta \theta$ defined here and in \citeauthor{FedConzMir04} (\citeyear{FedConzMir04}). The dashed and continuous black lines represent the straight lines to which the curves asymptote according to their analysis. Their heights are comparable to those based on the heat flux ($\overline{w^{'}\theta^{'}}$) profile in Figure \ref{fig:hdefs1}.}
    \label{fig:GarcMelERRi}   % label should change
\end{figure}

\subsection{Extending Comparison to the average Potential Temperature Profile}

There is an analogous distinction between curves representing Equation \ref{eq:ervsri} using $\Delta \theta$ and those using $\delta \theta$, when all heights are based on the $\frac{\frac{\partial \overline{\theta}}{\partial z}}{\gamma}$ profile.  Scatter is least when the $\theta$ jump is defined across the \acs{EZ}.  In Figure \ref{fig:weinvri} $a=-\frac{3}{2}$ fits at higher $\acs{Ri}$ (lower $\acs{Ri}^{-1}$) and $a=-1$ seems to fit at lower $\acs{Ri}$.  Combined with the apparent change in $b$ for Equation \ref{eq:dhvsri} I interpret this as an indication of a change in entrainment regime at increasing $\acs{Ri}$.\\ 

\FloatBarrier


\endinput

Any text after an \endinput is ignored.
You could put scraps here or things in progress.
