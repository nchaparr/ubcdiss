%% The following is a directive for TeXShop to indicate the main file
%%!TEX root = diss.tex

\chapter{Results in Context}
\label{ch:results}
\setlength{\parindent}{0cm}

\section{Comparison of general Set-up}
\FloatBarrier

\citeauthor{SullPat} (\citeyear{SullPat}) found that the shapes of the average potential temperature ($\overline{\theta}$) and average vertical heat flux ($\overline{w^{'}}\theta^{'}$) profiles, as well as the measured \acs{CBL} height vary depending on grid size.  The resolution at which convergence begins is listed in Table \ref{table:gridcomp}.  At lower resolution the $\overline{\theta}$ and $\overline{w^{'}\theta^{'}}$ profiles are such that the \acs{EL} is a larger portion of the \acs{CBL} and measured \acs{CBL} height is higher.  Overall they concluded that vertical resolution was more critical.  This compliments the conclusion \citeauthor{BrooksFowler2} (\citeyear{BrooksFowler2}) reached when discussing their resolution test.  That is, to capture the steep vertical gradients in the \acs{EL} requires high vertical resolution. \\

\begin{table}[htbp]
\caption[]{Grid spacing around the \acs{EL} used in comparable \acs{LES} studies. Those used for resolution tests are not listed here.  For \citeauthor{SullPat}'s \citeyear{SullPat} resolution study I list the grid sizes at which profiles within the \acs{EL} and \acs{CBL} height evolution began to converge.}

    \begin{center}
%\centerline{
    \begin{tabular}{ p{5cm} p{3cm} p{3cm}}
    %\hline
Publication & $\Delta x$, $\Delta y$, $\Delta z$ & Horizonal \\ \hline
Publication & in the \acs{EZ} (m)& Domain (km$^{2}$) \\ \hline
      \citeauthor{SullMoengStev} (\citeyear{SullMoengStev}) & 33, 33, 10 & 5 x 5 \\ %\hline 
      \citeauthor{FedConzMir04} (\citeyear{FedConzMir04}) & 100, 100, 20 & 5 x 5 \\ [.3cm] %\hline
      \citeauthor{BrooksFowler2} (\citeyear{BrooksFowler2}) & 50, 50, 12 & 5 x 5 \\%\hline
      \citeauthor{SullPat} (\citeyear{SullPat}) &  20, 20, 8 & 5 x 5\\ %\hline
      This study & 25, 25, 5 &  3.4 x 4.8\\ \hline       
    \end{tabular}
%}
\label{table:gridcomp}   
\end{center}    
\end{table}


As \citeauthor{Turner86} discusses in his \citeyear{Turner86} review of turbulent entrainment, smaller scale processes, such as those at the molecular level are relatively unimportant.  Large scale engulfment and trapping between thermals dominates.  If $\overline{\theta^{'2}}$ is calculated based on differences from horizontally averaged $\theta$, applying the ergodic assumption, then it is a measure of horizontal variance at a point in time. Although \citeauthor{SullPat} (\citeyear{SullPat}) found that the vertical distance over which $\overline{\theta^{'2}}$ varied significantly more or less converged at the resolution shown in Table \ref{table:gridcomp} the maximum continued to increaase up to their finest grid spacing (5, 5, 2).\\

The question as to whether mixing and gradients within the \acs{EZ} are adequately resolved serves as motivation for \acs{DNS}  studies such as that of \citeauthor{GarciaMellado} (\citeyear{GarciaMellado}). These authors found the entrainment ratio $\frac{\overline{w^{'}\theta^{'}}_{z_{f}}}{\overline{w^{'}\theta^{'}}_{s}}$ to be about 0.1 which is lower than for example what \citeauthor{FedConzMir04} (\citeyear{FedConzMir04}) observed, but close to what was seen in Figures \ref{fig:fluxprofs2hrs} and \ref{fig:tempgradfluxprofs1005}.  Based on their $\overline{w^{'}\theta^{'}}$ profiles the depth of the region of negative flux is comparable to what's shown in Figure \ref{fig:scaledELlims}.  Furthermore, these author's concluded that the production and destruction rates of \acs{TKE}, as well as the entrainment ratio used to calculate the entrainment rate, were effectively independent of molecular scale processes.\\  
  
The \acs{FFT} energy spectra of the turbulent velocities at the top of the \acs{ML} show a substantial resolved intertial subrange giving confidence in the choice of horizontal grid size used. In the \acs{EZ} where turbulence is intermittent, the dominant energy containing structures are smaller, and decay to the smallest resolved turbulent structures is steeper. This confirms the assertion of \citeauthor{GarciaMellado} (\citeyear{GarciaMellado}) that the \acs{EZ} is separated into two sublayers in terms of turbulence scales.\\


The horizontal domain in this study is relatively small (see Table \ref{table:gridcomp}). However, visualizations of horizontal and vertical slices clearly showed multiple resolved thermals with diameter increasing with increased \acs{CBL} height, but remaining less than or on the order of 100 meters.  \citeauthor{SullMoengStev} (\citeyear{SullMoengStev}) carried out one run on a smaller domain with higher resolution, noticed it resulted in lower \acs{CBL} height and concluded this was due to restricted plume size. However, given the results of \citeauthor{SullPat} (\citeyear{SullPat}) it could have been an effect of the grid size.\\   

\citeauthor{BrooksFowler2} (\citeyear{BrooksFowler2}) encountered significant scatter when basing heights of average profiles.  \citeauthor{SullMoengStev} (\citeyear{SullMoengStev})'s heights based on average profiles produced very jagged oscillating timeseries. But the heights based on average profiles here, using an ensemble of cases, varied smoothly in time.  This could be attributed to a smoother profile based on a greater number horizontal points (10*128*192).\\

The principle parameter describing the balance of forces in idealized \acs{CBL} entrainment is the Richardson number $\acs{Ri}$ and its magnitude depends on the way in which the $\theta$ jump is defined.  Varying this can cause identical conditions can be described by quite different $acs{Ri}$ s.  The $\acs{Ri}$ range in this study was dependent on variation in $\gamma$ and less so on 
$\overline{w^{'}\theta^{'}}_{s}$.  \citeauthor{BrooksFowler2} (\citeyear{BrooksFowler2}) and \citeauthor{SullMoengStev} (\citeyear{SullMoengStev}) imposed a $\theta$ jump of varying strength topped by a constant $\gamma$.  Whereas \citeauthor{FedConzMir04} (\citeyear{FedConzMir04}) initialized with a constant heat flux, with a layer of uniform $\theta$, topped by a constant $\gamma$.  They varied $\gamma$ and kept $\overline{w^{'}\theta^{'}}_{s}$ constant for each run.  Their range of $\gamma$, definitions of the $\theta$ jump and $\acs{Ri}$ range are directly comparable to those of this study, whereas those of \citeauthor{BrooksFowler2} (\citeyear{BrooksFowler2}) and \citeauthor{SullMoengStev} (\citeyear{SullMoengStev}) are quite different.  The $\theta$ jumps defined in these sudies were smaller.\\    

\begin{table}[htbp]
\caption[]{Initial conditions used in comparable \acs{LES} studies.}

    \begin{center}
%\centerline{
    \begin{tabular}{ p{4cm} p{1.4cm} p{1.4cm} p{1.7cm} p{1.8cm}}
    %\hline
Publication & $\overline{w^{'}\theta^{'}}_{s}$& $\gamma$& Initial $\theta$ & $\acs{Ri}$ \\ 
& W/m$^{2}$ & K/km & Jump K & range \\ \hline
      \citeauthor{SullMoengStev} (\citeyear{SullMoengStev}) & 20 - 450& 3  &.436 - 5.17 & 1 - 100\\ %\hline 
      \citeauthor{FedConzMir04} (\citeyear{FedConzMir04}) & 300 & 1 - 10 & na & 10 - 40\\ %[.3cm] %\hline
      \citeauthor{BrooksFowler2} (\citeyear{BrooksFowler2}) &  10 -100 &  3& 1 - 10 &10 - 100 \\ %\hline
      This study & 60 - 150 & 2.5 - 10& na & 10 - 30\\ \hline 
      
    \end{tabular}
%}
\label{table:initconditcomp}   
\end{center}    
\end{table}



%\clearpage

\section{Local \acs{ML} Heights}

\citeauthor{SullMoengStev} (\citeyear{SullMoengStev}) determined local \acs{CBL} height by locating the point of maximum gradient.  Analysis of the resulting distributions showed dependence of standard deviation and skewness on Richardson number.  The normalized standard deviation decreased with increased \acs{Ri} whereas skewness was alomost bimodal; being negative at high \acs{Ri} and positive and low \acs{Ri}.  Iniatially in this study, I applied a similar method and found distributions with lower \acs{Ri} to have positive skew.  Upon exhaustive inspection of local vertical $\theta$  profiles, it became evident that at certain horizontal points high gradients well into the free atmposphere exceded those closer to the location of the \acs{CBL} height reasonably identified by eye.\\

Locating the \acs{ML} height using the multi-linear regression method employed proved more reliable.  The resulting distributions, normalized by $h$ showed a decrease in the lowest $\frac{h^{l}_{0}}{h}$ resulting in an apparant increased negative skew with decreasing stability (decreasing $\acs{Ri}$). This combined with an increase in spread agrees with the findings \citeauthor{SullMoengStev} (\citeyear{SullMoengStev}) and supports the results of \ref{sec:deltahri}.  The approximate scaled \acs{EZ} based on the $\frac{h^{l}_{0}}{h}$ distributions is about 0.2 - 0.4 whereas that based on distributions of local maximum tracer gradients by  \citeauthor{BrooksFowler2} (\citeyear{BrooksFowler2}) was smaller (.05 - .2).  But the local maximum gradient of the tracer profile would likely be within the \acs{EZ} at points outside an actively impinging plume and so higher than the \acs{ML} top ($h^{l}_{0}$) defined here. \\  

%Potential temperature and vertical velocity fluctuations ($\theta^{'}$ and $w^{'}$) at several vertical levels around the \acs{EL} were plotted as 2 dimensional histograms.  Increased $\overline{w^{'}\theta^{'}}_{s}$ causes an increase in positive temperature fluctuations and verical velocity as thermals become more vigorous, causing a higher $h$ so and a deeper \acs{EZ} over which relatively warmer air is pulled down.  The convective velocity scales ($\theta^{*}$ and $w^{*}$) were appplied to isolate the effects of $\gamma$, although it is accounted for indirectly via $h$ (see Section \ref{subsec:scales}). As shown in \citeauthor{Sorbjan}'s (\citeyear{Sorbjan}) $\theta^{'}$ is influenced  by $\gamma$.  For example at $h$ there is an apparant increase in the spread, as well as a shift thowards the positive.  So downward moving positive potential temperature fluctuations representing air from the \acs{FA} are more positive and negative fluctuations representing thermals are less negative, when scaled.  The former can easily be attributed to relatively warmer air over a shorter vertical distance (\acs{EZ}) being brought down but an explaination for the latter remains elusive.  I conclude that with incresed $\gamma$ there is a positive deviation from the convective temperature scale.  Also, there is an apparant damping of the scaled velocities associated with positive temperature fluctuations.  So the $w^{*}$ scales these less effectively with increased $\gamma$.  \\

As expected, with increased $\overline{w^{'}\theta^{'}}_{s}$ the variance and magnitude of the vertical velocity fluctuations within and at the limits of the \acs{EZ} increase.  Greater turbulent velocity causes results in a higher \acs{CBL} and a thicker \acs{EZ} over which warmer air from higher up is brought down and relatively cooler air from below is brought up.  So the magnitude and spread of $\theta^{'}$ increases. All of this agrees with the findings of \citeauthor{Sorbjan} (\citeyear{Sorbjan}), but the portion of the $\frac{w^{'}}{w^{*}}$ distribution where $\theta^{'}$ is positive appears to narrow as $\gamma$ increases. This contradicts his assertion that velocities are uninfluenced by this parameter while the effectiveness of $w^{*}$ as a scale for $w^{'-}$ where $\theta^{'}>0$ supports it.\\

 Although the motion of the thermals dominates the dynamics witin the \acs{EZ}, the $w^{'-}\theta^{'-}$, $w^{'+}\theta^{'-}$ and $w^{'-}\theta^{'+}$ quadrants do approximately cancel as \citeauthor{SullMoengStev} (\citeyear{SullMoengStev}) concluded. The downward moving warm quadrant ($w^{'-}\theta^{'+}$) for example, at $h$, represents warmer free atmosphere air that is being entrained.  So it's magnitude at a certain point in time is an indication of how much the region below will be warmed due to entrainment at a successive time.  The increase of $\overline{w^{'-}\theta^{'+}}_{h}$ in time is primarily due to $\theta^{'+}$ which is effectively scaled by the temperature scale $(h_{1}-h)\gamma$.  A very similar scale was introduced by \citeauthor{GarciaMellado} (\cite{GarciaMellado}) to further their line of reasoning that the buoyancy in the upper \acs{EZ} is determined by $\gamma$. Figure \ref{fig:deltagamma} illustrates a broad qualitative explaination for its effectiveness.  At $h$ much of the air is at the background (or initial) potential temperature $\overline{\theta}_{0}(h)$, 

\begin{figure}[htbp]
    \centering
    %plot_height.py[master 1573b9d] h vs time plot
    \includegraphics[scale=0.5]{/newtera/tera/phil/nchaparr/tera2_cp/nchaparr/ubcdiss/pngs/deltagamma}
    \caption[Illustration of \acs{EZ} Potential Temperature Scale based on $\gamma$]{Illustration of the potential temperature scale $(h_{1}-h)\gamma = \delta \gamma$: The curves represent vertical a cross-section of thermal tops.  Between them is stable air at the initial lapse rate $\gamma$. $h_{1}$, $h$ correspond to the highest and average thermal height respectively and $h_{0}$ is the top of the well mixed region (\acs{ML}).  The initial temperature is $\theta_{0} = \overline{\theta}_{0}$. A thermal will initiate the downward movement of air from $h_{1}$ to $h$, and the difference between its potential temperature and that of the background stable air at $h$ is $(h_{1}-h)\gamma = \delta \gamma$.}
    \label{fig:deltagamma}   % label should change
\end{figure}
%diagram with plumes and hs and $\gamma$

but some air of $\theta = \overline{\theta}_{0}(h_{1})$ is brought down from $h_{1}$ resulting in positive temperature fluctuations at $h$ ($\theta^{'+}$).\\

\citeauthor{GarciaMellado} (\cite{GarciaMellado}) suggest that the buoyancy in the lower portion of the \acs{EZ} ie from a point just below $h$ down is more influenced by the vigourous turbulence of the \acs{ML} so mixing reduces the difference between the temperature at the top of the \acs{ML} and that at or just below $h$.  However, the observation in Section \ref{subsec:ellimscaledprof} that the magnitude of the vertical potential temperature gradient in the upper \acs{ML} increases with increasing $\gamma$ indicates that the influence of this parameter extends further.  Related, is the increased magnitude of the minimum $\overline{w^{'}\theta^{'}}$ with $\gamma$, seen here and in both \citeauthor{Sorbjan}'s (\citeyear{Sorbjan}) and {FedConzMir04} (\citeyear{FedConzMir04}), which leads to an increased $-\frac{\partial \overline{w^{'}\theta^{'}}}{\partial z}$ in the lower \acs{EZ} and so increased warming per Equation \ref{eq:warming}.

\begin{equation}
\frac{\partial \overline{\theta}}{\partial t} = -\frac{\partial}{\partial z}\overline{w^{'}\theta^{'}} \tag{\ref{eq:warming1}}
\end{equation}

\subsection{Relationship of Entrainment Layer Depth to Richardson Number}

The \acs{EZ} is inhomogenous and on average is a region of transition as clearly represented by the $\overline{\theta}$ profile.  It's where relatively cooler thermals overturn or recoil initiating entrainment as represented by the heat flux profile.  The $\overline{\theta}$ profile partially characterizes the thermodynamic state of the \acs{CBL} as well defining its three layer structure.  It is directly comparable to both bulk models and local $\theta$ profiles which in turn are comparable to a sounding, unlike a heat flux profile which is an inherently average quantity.\\

Neither of the two comparable \acs{LES} (see Table \ref{table:el}) studies define the \acs{EL} based on the vertical $\frac{\partial \overline{\theta}}{\partial z}$ profile.  So, to enable direct comparison heights were based on the heat flux profile.  In this framework \citeauthor{FedConzMir04}'s (\citeyear{FedConzMir04}) show decreasing scaled \acs{EZ} with increaseing $\acs{Ri }$ (decreasing $\acs{Ri}^{-1}$) and conclude an exponent of of $-\frac{1}{2}$.  They attribute the decrease in the overall depth to a slight decrease in the scaled top limit over time.  However based on their plot, it seems to go from about 0.5 to 0.2.\\

\begin{figure}[htbp]
    \centering
    %plot_height.py[master 1573b9d] h vs time plot
    \includegraphics[scale=1]{/newtera/tera/phil/nchaparr/tera2_cp/nchaparr/ubcdiss/pngs/FedEZRi}
    \caption[Plot of the relationship between scaled \acs{EZ} depth and Richardson number from \citeauthor{FedConzMir04}'s (\citeyear{FedConzMir04})]{Figure 9 from \citeauthor{FedConzMir04}'s (\citeyear{FedConzMir04}) representing Equation \ref{eq:dhvsri} using three different Richardson numbers, in log-log coordinates.  $Ri_{\Delta b}$ (circles) and $Ri_{\delta b}$ (crosses) correspond directly to those determined here using $\delta \theta$ and $\Delta \theta$ based on the heat flux profiles.  Note their $\Delta$ refers to the smaller $\theta$ jump, ie that at $z_{f}$, whereas I use it for the larger.  Then $Ri_{N}$ (triangles) is the Richardson number defined in Equation \ref{eq:gradri}, with $w^{*}$ and $z_{f}$ as the velocity and length scale.}
    \label{fig:FedEZRi}   % label should change
\end{figure}


 \citeauthor{BrooksFowler2} (\citeyear{BrooksFowler2}) found no clear \acs{Ri} dependence of the scaled \acs{EL} depth based on the heat flux profle, but their definition was based on the lower part ($z_{f1} - z_{f}$).  According to \citeauthor{FedConzMir04} (\citeyear{FedConzMir04}) this lower sublayer does not vary in time.  Figure \ref{fig:deltahinvri_scaled} shows that when based on the heat flux profiles as \citeauthor{FedConzMir04}'s (\citeyear{FedConzMir04}) did, there is no clear dependence on \acs{Ri}. This is supported by the similarity in time and accross runs of the vertical turbulent heat flux profiles when scaled by $(\overline{w^{'}\theta^{'}})_{s}$.\\

The most obvious possible cause for this disagreement is the difference in grid size shown in Table \ref{table:gridcomp}.  Inspection of their heat flux profiles confirms a relatively deeper region of negative flux as compared with those seen here (~.4 vs ~.25). Their surface heat flux was twice the highest used here, but their range of $\acs{Ri}$ is comparable to that of this study.  This latter point although not directly relevent here, serves as confirmation that $\gamma$ is the more influential parameter.\\              

\begin{table}[htbp]
\label{table:el}
\begin{center}
%\centerline{
\begin{tabular}{ p{4cm} p{2cm} p{1.5cm} p{3cm}}
    %\hline
Publication & \acs{EZ} Depth & \acs{CBL} height & $\theta$ Jump\\ \hline
\citeauthor{FedConzMir04} (\citeyear{FedConzMir04}) & $z_{f1} - z_{f0}$ & $z_{f}$ &  $\overline{\theta}(z_{f1})-\overline{\theta}(z_{f0})$\\ [.3cm] %\hline
\citeauthor{BrooksFowler2} (\citeyear{BrooksFowler2}) & $2 \times (z_{f} - z_{f0})$ & $z_{f}$ & average of local values\\ \hline
\label{table:el}
\end{tabular}
\end{center}    
\end{table}


Here, when heights are defined on the scaled vertical potential temperature gradient profile $\frac{\frac{\partial \overline{\theta}}{\partial z}}{\gamma}$ the curve representing Equation \ref{eq:dhvsri} 

\begin{equation}
\frac{\Delta h}{h} \propto Ri^{b} \tag{\ref{eq:dhvsri}}
\end{equation}

and show an (or slope) exponent $b$ which increases in magnitude from about $-\frac{1}{2}$, as predicted and seen by \citeauthor{Boers89} (\citeyear{Boers89}), to about $-1$ as justified in \citeauthor{StullNelEl} (\citeyear{StullNelEl}),  with increasing $\acs{Ri}$ (decreasing $\acs{Ri}^{-1}$).  Overall there is a clear narrowing of the scaled \acs{EZ} depth with increased \acs{Ri} (decreased $\acs{Ri}^{-1}$) as indicated by the local height distributions in Section \ref{subsec:locmlh}.  Although based on different height definitions \citeauthor{FedConzMir04} (\citeyear{FedConzMir04}) concluded an exponent $b = -\frac{1}{2}$ and \citeauthor{BrooksFowler2}'s (\citeyear{BrooksFowler2}) plots show curves with an apparant exponent less in magnitude than $-1$. 

\begin{figure}[htbp]
    \centering
    %plot_height.py[master 1573b9d] h vs time plot
    \includegraphics[scale=1]{/newtera/tera/phil/nchaparr/tera2_cp/nchaparr/ubcdiss/pngs/BandFEZ}
    \caption[Relationship of Scaled \acs{EZ} depth to Richardson number from \citeauthor{BrooksFowler2}'s (\citeyear{BrooksFowler2})]{This is panel (a) from Figure 5 in \citeauthor{BrooksFowler2}'s (\citeyear{BrooksFowler2}) and represents relationship \ref{eq:dhvsri}.  Heights are based on the distributions of local maximum tracer gradients, ie upper and lower percentiles.  Their $\theta$ jump is an average of the potential temperature differences accross the regions of significant wavelet covariance in the local tracer profiles.}
    \label{fig:BandFEZ}   % label should change
\end{figure}


The curves in Figure \ref{fig:BandFEZ} seem to separate out for each run.  In this study, before scaling the $\frac{\partial \overline{\theta}}{\partial z}$ profile curves in Figure \ref{fig:scaledeltahinvri} separate out, but in the reverse order.  Runs under higher stability have larger scaled \acs{EZ} dephts.  \citeauthor{BrooksFowler2} (\citeyear{BrooksFowler2})'s runs with initially lower \acs{Ri} (higher \acs{Ri}$^{-1}$) have larger scaled \acs{EZ} depths than runs with initially higher \acs{Ri}s, even where \acs{Ri} values overlap.\\     

Neither study addresses a change in exponent with increased $\acs{Ri}$.  It is possible that there is a change in entrainment mechanism. \citeauthor{SullMoengStev} (\citeyear{SullMoengStev}) observed enfolding and engulfment at lower \acs{Ri}.  Whereas at higher \acs{Ri} when motion is more restricted, entraiment seemed to occur via trapping of thinner whisps at the edge of an upward moving thermal.  \citeauthor{Turner86} (\citeyear{Turner86}) also distiguishes between entrainment by convective overturning and recoil. \citeauthor{GarciaMellado}'s (\citeyear{GarciaMellado}) refer to a change in entrainment rate due to the effects of increased stability on the upper \acs{EZ} sublayer.  In this study, although the narrowing of the \acs{EZ} as defined here depends predominantly on the magnitude of the potential temperature gradient in the lower \acs{EZ} and upper \acs{ML} the scaled magnitude of upper limit based on the $\frac{\partial \overline{\theta}}{\partial z}$ does appear to decrease slightly in time.  So this could correspond to the mention of a slowly decreasing upper sub layer in the \acs{EZ} in both \citeauthor{GarciaMellado} (\citeyear{GarciaMellado}) and \citeauthor{FedConzMir04} (\citeyear{FedConzMir04}).\\

\section{Relationship of Entrainment Rate to Richardson Number (Q3)}

The magnitudes of \acs{Ri} numbers determined in this and the comparable studies are primarily influenced by the magnitude of the $\theta$ jump. So with different definitions you can have the same conditions represented by a different \acs{Ri} value. Here, I define it in two ways as both \citeauthor{FedConzMir04} (\citeyear{FedConzMir04}) and \citeauthor{GarciaMellado} (\citeyear{GarciaMellado}) did.  I do this based on the flux profile for the purpose of direct comparison with these studies and to observe how the change in definition effects Equation \ref{eq:ervsri}.\\

\begin{equation}
\frac{w_{e}}{w^{*}} \propto Ri^{a} \tag{\ref{eq:ervsri}}
\end{equation}

As in these studies, the larger jump ie that taken accross the \acs{EZ} ($\Delta \theta$) yields a larger $a$ as \citeauthor{FedConzMir04} (\citeyear{FedConzMir04}) conclude.  Although \citeauthor{GarciaMellado} (\citeyear{GarciaMellado}) interpret their curves as both asymptoting to a straight line ($a=-1$) as the upper \acs{EZ} sublayer narrows. Based on their plots alone, ie in the absence of their justification based on the derivation of the entrainment relation, I would not conclude this asymtotic behavior.  Rather for $Delta \theta$ I see a curve with increasing exponent exceeding magnitude $-1$ at higher $\acs{Ri}$.  For $\delta \theta$ I see a curve with exponent less in magnitude than $-1$.\\

\begin{figure}[htbp]
    \centering
    %plot_height.py[master 1573b9d] h vs time plot
    \includegraphics[scale=1]{/newtera/tera/phil/nchaparr/tera2_cp/nchaparr/ubcdiss/pngs/GaMeERRi}
    \caption[Plots of scaled entrainment rate vs Richardson number from \citeauthor{GarciaMellado} (\citeyear{GarciaMellado})]{This is Figure 11 from \citeauthor{GarciaMellado} (\citeyear{GarciaMellado}) and represents equation \ref{eq:ervsri} based on the two $\theta$ jumps.  The grey and blue curve is based on $\Delta \theta$ and the red curve is based on something similar to $\delta \theta$, ie $\overline{\theta}(h) - \overline{\theta}_{0}(h)$. The dashed and continuous black lines represent the straight lines two which the curves assymptote according to their analysis. Their defined heights are comparable to those based on the heat flux profile in Figure \ref{fig:hdefs1}.}
    \label{fig:GarcMelERRi}   % label should change
\end{figure}


That there is an analogous distinction between plots of Equation \ref{eq:ervsri} using $\Delta \theta$ vs $\delta \theta$ when all heights are defined on the $\frac{\frac{\partial \overline{\theta}}{\partial z}}{\gamma}$ profile lends some credence to this framework.  Scatter is least when the $\theta$ jump is defined accross the \acs{EL}.  In this plot $-\frac{3}{2}$ fits at higher \acs{Ri} (lower \acs{Ri}$^{-1}$) and $-1$ seems to fit at lower \acs{Ri}.  Combined with the apparant change in $b$ for Equation \ref{eq:dhvsri} this could be interpreted as an indication of a change in entrainment regime at increased \acs{Ri}.\\ 

\section{Conclusion}

The magnitude and variance of local height increase with increasing $\overline{w^{'}\theta^{'}}_{s}$ and decrease with increasing $\gamma$.  Similarly for the vertical velocity fluctuations in the \acs{EZ}.  However increased $\gamma$ results in an increase in the positive temperature fluctuations at $h$. The magnitude of these positive temperature perturbations at points where vertical velocity is negative represents downward moving entrained air.  Below $h$ in the lower \acs{EZ} the temperature gradient increases with increasing $\gamma$. So, the growth of the idealized dry \acs{CBL} is driven by $\overline{w^{'}\theta^{'}}_{s}$ and suppressed by $\gamma$. But warming is due to $\overline{w^{'}\theta^{'}}_{s}$ and the entrainment of air from aloft the temperature of which in turn depends on $\gamma$.\\

Both the \acs{EZ} depth and \acs{CBL} height based on the average $\frac{\frac{\partial \overline{\theta}}{\partial z}}{\gamma}$ profile show dependence on \acs{Ri} as seen in other studies and justified theoretically.  This profile serves to characterize the \acs{CBL} and links bulk models to soundings via an \acs{LES}.  So it is a valid way of defining the \acs{CBL} and its \acs{EZ}. Plots of Equations exhibit changes in exponent.  A change in entrainment mechanism or regime with increased $\acs{Ri}$ has been observed in measurement as well as \acs{LES} and justified theoretically. I suggest the change in exponents represent this.\\

Throughout this study threads the influence of $\gamma$.  The convecive time scale $\tau = \frac{w^{*}}{h}$ and $\acs{Ri}$ group according to this parameter.  This justifies  the use of the brunt vaisalla time scale as well as the constant heat flux with varying lapse rate of 
\citeauthor{FedConzMir04} (\citeyear{FedConzMir04}).  I conclude once the effects of $\overline{w^{'}\theta^{'}}_{s}$ are accounted for through $h$, $\gamma$ remains the dominant parameter in idealized \acs{CBL} entrainment.\\ 

\FloatBarrier


\endinput

Any text after an \endinput is ignored.
You could put scraps here or things in progress.
