%% The following is a directive for TeXShop to indicate the main file
%%!TEX root = diss.tex

\chapter{Conclusion}
\label{ch:conc}
\setlength{\parindent}{0cm}

\section{Major Findings}

The dry idealized convective atmospheric boundary layer (\acs{CBL}) was modeled using large eddy simulation (\acs{LES}).  Although this has been done before and a broad understanding of the dynamics and scaling behaviour has been established, discussion of details continues.  This study was intended to contribute to this discussion and shed light on some of these details.  It was guided by the questions outlined in Section \ref{sec:resquest} and answered in Chapter \ref{ch:results} and concludes with the following four points:    


\subsection{The gradient Method for determining local Heights based on the $\theta$ Profile is problematic.}

Local $\theta$ profiles vary depending on location.  The top of an active thermal impinging on the free atmosphere (\acs{FA}) as in Figure \ref{fig:rssfitshigh} is characterized by a steep gradient comparable to the zero-order model representation in Figure \ref{fig:0order}.  At other locations, for example where a thermal has overturned or recoiled and some entrainment has been initiated as in Figure \ref{fig:rssfitslow}, there is a region over which the $\theta$ profile transitions to the upper lapse rate ($\gamma$). That is, there is a local entrainment zone (\acs{EZ}).  At such locations, there are gradients well into the \acs{FA} that exceed any within the \acs{EZ}, as well as an absence of a well-defined local \acs{CBL} height.  This presents both a practical and conceptual challenge to the gradient method, while determination of the \acs{ML} using piecewise linear regression is more reliable. 

\subsection{\acs{CBL} Height and \acs{EZ} Boundaries can be defined based on the average Potential Temperature Profile}

The $\overline{\theta}$ profile characterizes the dry, idealized \acs{CBL} and links bulk models to soundings via an \acs{LES}.  Both the \acs{EZ} depth and \acs{CBL} height based on the average $\frac{\frac{\partial \overline{\theta}}{\partial z}}{\gamma}$ profile showed dependence on $\acs{Ri}$ (Sections \ref{sec:deltahri} and \ref{sec:weri}) as seen in other studies and justified theoretically.  So this is a valid way of defining the \acs{CBL} and its \acs{EZ}.  

\subsection{Upper Lapse-Rate is a critical Parameter in idealized \acs{CBL} Entrainment}

The magnitude and variance of local \acs{ML} height, increase with increasing $\overline{w^{'}\theta^{'}}_{s}$, and decrease with increasing $\gamma$.  The same can be said for the vertical velocity fluctuations ($w^{'}$) in the \acs{EZ}.  However, increased $\gamma$ results in an increase in the positive potential temperature fluctuations ($\theta^{'+}$) at $h$. The magnitude of ($\theta^{'+}$) at points where $w^{'}$ is negative represents downward moving entrained air and depends on $\gamma$ (Section \ref{sec:q1}).  Below $h$, in the lower \acs{EZ}, the average vertical potential temperature gradient ($\frac{\partial \overline{\theta}}{\partial z}$) also depends on $\gamma$ (Section \ref{subsec:ellimscaledprof}). So, the growth of the idealized dry \acs{CBL} is driven by $\overline{w^{'}\theta^{'}}_{s}$ and suppressed by stability ($\gamma$). But \acs{CBL} warming is due, in part, to the entrainment of air from aloft the potential temperature of which in turn depends on $\gamma$.\\

The influence of $\gamma$ threads throughout this study.  Distributions of scaled local \acs{ML} heights approach similarity, when $\gamma$ is constant but $\overline{w^{'}\theta^{'}}_{s}$ is varied (Figure \ref{fig:localh}).  Curves representing Equation \ref{eq:dhvsri} group according to $\gamma$ when based on the $\frac{\partial \overline{\theta}}{\partial z}$ profile, but collapse once based on $\frac{\frac{\partial \overline{\theta}}{\partial z}}{\gamma}$ (Section \ref{subsec:ellimscaledprof}).  The convective time scale $\tau = \frac{w^{*}}{h}$ and $\acs{Ri}$ group according to $\gamma$ (Figure \ref{fig:ScaledTimevsTime}) lending support to \citeauthor{FedConzMir04}'s (\citeyear{FedConzMir04}) use of the Brunt-Vaisala time scale.  It seems that once the effect of the surface heat flux ($\overline{w^{'}\theta^{'}}_{s}$) is accounted for through $h$, $\gamma$ emerges as the dominant parameter in dry, idealized \acs{CBL} entrainment.\\ 

\subsection{There are two \acs{CBL} Entrainment Regimes}

\citeauthor{Turner86} (\citeyear{Turner86}) outlined and theoretically justified two distinct convective boundary layer entrainment regimes wherein the scaled entrainment rates have different $\acs{Ri}$ dependence. The \acs{LES} flow visualizations of \citeauthor{SullMoengStev} (\citeyear{SullMoengStev}) showed large scale engulfment at lower $\acs{Ri}$.  At higher $\acs{Ri}$, trapping of smaller volumes of stable air between and at the edges of impinging thermals appeared to be the dominant mechanism. The \acs{CBL} entrainment zone measurements analyzed in \citeauthor{Traum11} (\citeyear{Traum11}) further support the concept of varying entrainment mechanism depending on the strength of the upper lapse rate $\gamma$.  Finally, both \citeauthor{FedConzMir04} (\citeyear{FedConzMir04})  and \citeauthor{GarciaMellado} (\citeyear{GarciaMellado}) discuss the varying dependence of the scaled entrainment rate on $\acs{Ri}$ as the effects of upper stability become more important.  On these grounds I attribute the change in exponent in the plots of equations \ref{eq:dhvsri} and \ref{eq:ervsri} in Figures \ref{fig:loglogdeltahinvri} and \ref{fig:weinvri} to a change in entrainment regime as $\acs{Ri}$ increases.   


\section{Future Work}

Some ideas as to how the work in this thesis could be completed or extended are as follows:

\subsection{Expand local \acs{CBL} height determination Method}

To further the tri-linear regression method described in Section \ref{subsec:trilin}, the \acs{EZ} could be approximated by a suitable polynomial, fit using an appropriate regression method.  It then could be possible to determine a local \acs{CBL} height at the point of maximum gradient on this fitted curve.\\

\subsection{Examine Resolution Effects}

Runs could be carried out at lower resolution, to examine the effects on the curves in Figure \ref{fig:deltahinvri_scaled} and eliminate or confirm this as a cause for disagreement with the results of \citeauthor{FedConzMir04} (\citeyear{FedConzMir04}) discussed in Section \ref{subsec:comphfl}.  The height and $\theta$ jump definitions of \citeauthor{GarciaMellado} (\citeyear{GarciaMellado}) could be matched to facilitate direct comparison with their results and speak more to the need for increased resolution.

\subsection{Further explore Entrainment Regimes}

As discussed already in this Section, there is sufficient cause to assume there is a change in entrainment mechanism as $\acs{Ri}$ increases.  Animated visualizations of two-dimensional horizontal slices showed thermals regularly impinging on the \acs{FA} with associated, intermittent periods of vigorous activity.  A possible way to observe these mixing events and how they change with respect to $\acs{Ri}$ and time, is to measure turbulent velocity and vorticity at local points, or sub-domain regions.  The connection between increased horizontal and downward motions in the \acs{EZ}, and \acs{CBL} growth can, easily be tested by concurrently measuring the local height. Furthermore, turbulent activity measured at different levels within the \acs{EZ} could shed further light on the turbulence characteristics of \citeauthor{GarciaMellado}'s (\citeyear{GarciaMellado}) suggested two layer structure.\\     

\subsection{Apply a Mass Flux Scheme}

The robustness of equations \ref{eq:dhvsri} and \ref{eq:ervsri} could be tested by first establishing criteria for identifying \acs{CBL} air and then calculating the entrainment rate based on the increase in its volume using the method described in \citeauthor{DawAus} (\citeyear{DawAus}).  \acs{CBL} air could be identified using a passive tracer, and or potential temperature.  The \acs{EZ} could, for example, be defined based on statistics of local \acs{CBL} heights.   

\FloatBarrier


\endinput

Any text after an \endinput is ignored.
You could put scraps here or things in progress.
