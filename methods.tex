%% The following is a directive for TeXShop to indicate the main file
%%!TEX root = diss.tex

\chapter{Tools}
\label{ch:tools}
\setlength{\parindent}{0cm}

\section{Large Eddy Simulation}
\label{sec:LargeEddieSimulation}

System for Atmospheric Modelling (SAM) is a Large Eddy Simulation with cloud resolving capability (\citeauthor{KhairRand} \citeyear{KhairRand}). The dynamical framework uses the anelastic equations of motion, which in tensor notation are:

\begin{equation}
\frac{\partial u_{i}}{\partial t} = -\frac{1}{\overline{\rho}}\frac{\partial}{\partial x_{i}}(\overline{\rho}u_{i}u_{j} + \tau_{ij}) - \frac{\partial}{\partial x_{i}}\frac{p^{'}}{\overline{\rho}} + \delta_{i3}B + \epsilon_{ij3}f(u_{j} - U_{gj}) + \left( \frac{\partial u_{i}}{\partial t} \right)_{l.s.}
\end{equation}

and

\begin{equation}
\frac{\partial}{\partial x_{i}}\overline{\rho}u_{i}=0
\end{equation}


The over-bar denotes the horizontal average and prime denotes fluctuations from the average. B is buoyancy $=-g\frac{\rho^{'}}{\rho}$,  $U_{g}$ is the prescribed geostrophic wind and $f$ is the Coriolis parameter.  $\tau_{ij}$ is the sub-grid scale stress tensor and the subscript $l.s.$ denotes the prescribed large scale tendency.\\      

The prognostic thermodynamical variable is the liquid water/ice moist static energy ($h_{L}$). 

\begin{equation}
\frac{\partial h_{L}}{\partial t} = -\frac{1}{\overline{\rho}}\frac{\partial}{\partial x_{i}}(\overline{\rho} u_{i}h_{L} + F_{h_{L}i}) - \left( \frac{\partial h_{L}}{\partial t} \right)_{rad} + \left( \frac{\partial h_{L}}{\partial t} \right)_{mic}
\end{equation}

where

\begin{equation}
h_{L} = c_{p}T + gz
\end{equation}

So temperature and potential temperature are diagnozed based on this variable, at each time-step.  $F_{h_{L}i}$ is the sub-grid scale flux.  A simple first-order Smagorinski closure scheme is used to parameterize the sub-grid stresses and scalar fluxes. The eddy diffusivity coefficient is based on the grid scale.\\

The model equations are represented discretely on a fully staggered Arakawa C-type grid which is uniform in the horizontal and stretched in the vertical. Integration is performed using a third-order Adams-Bashforth scheme with variable time step.  Momentum is advected in flux form with second order differencing and conservation of kinetic energy. 
Prognosed scalars are advected using a three dimensional positive definite, monotonic scheme.  Lateral boundaries are periodic.  The top is bounded by a rigid lid, and
Newtonian damping is applied in the top third of the domain to reduce the effects of gravity waves.  Surface fluxes are computed using Monin-Obvukhov similarity.\\

\section{Handling of Output}

The model was run on parallel computers in a Linux environment using Message Passing Interface (MPI). 3d variable fields were output every 10 - 15 minutes in binary form and converted to Network Common Data Form (NetCDF).  The use of Python was enabled using the netcdf4 interface.  Plotting was done using matplotlib. All analysis, except for the tri-linear regression method for determining local \acs{ML} height, were performed using NumPy and SciPy.

\subsection{Tri-Linear Fit for Determining local \acs{ML} height $h_{0}$}

The following is a modified version of the piecewise linear regression method used in \citeauthor{Vieth} (\citeyear{Vieth}) and was implemented using Cython.  Potential temperature is assumed to be linear function of height 

\begin{equation}
\theta = bz + a 
\end{equation}.

Each local $\theta$ profile was assumed to have three linear portions, with slopes ($b_{1}$, $b_{2}$, $b_{3}$) and intercepts ($a_{1}$, $a_{2}$, $a_{3}$) as follows:

\begin{equation}
b_{1} = \frac{\sum^{j}_{0}z(i) \theta (i) - \frac{1}{j}\sum^{j}_{0}z(i)\sum^{j}_{0}\theta}{\sum^{j}_{0}z(i)^{2} - \frac{1}{j}(\sum^{j}_{0}z(i))^{2}}
\end{equation}
\begin{equation}
a_{1} = \frac{\sum^{j}_{0}z(i)\theta(i)}{\sum^{j}_{0}z(i)} - b_{1}\frac{\sum^{j}_{0}z(i)^{2}}{\sum^{j}_{0}z(i)}
\end{equation}

\begin{equation}
b_{2} = \frac{\sum^{k}_{j}z(i) \theta(i) - (k-j) a_{1}+b_{1}z(j)}{\sum^{k}_{j}z(i) - (k-j)z(j)}
\end{equation}
\begin{equation}
a_{2} = \frac{\sum^{k}_{j}z(i)\theta(i)}{\sum^{k}_{j}z(i)} - b_{2}\frac{\sum^{k}_{j}z(i)^{2}}{\sum^{k}_{j}z(i)}
\end{equation}

\begin{equation}
b_{3} = \frac{\sum^{n}_{k}z(i) \theta(i) - (k-j) a_{1}+b_{1}z(j)}{\sum^{k}_{j}z(i) - (k-j)z(j)}
\end{equation}
\begin{equation}
a_{3} = \frac{\sum^{n}_{k}z(i)\theta(i)}{\sum^{n}_{k}z(i)} - b_{3}\frac{\sum^{n}_{k}z(i)^{2}}{\sum^{n}_{j}z(i)}
\end{equation}

where $z(i)$ and $\theta(i)$ are a local height and potential temperature value at a particular height index $i$.  $j$ is the height index of the \acs{ML} top, $h_{0}$. $k$ is the height index for the top of the \acs{EZ}, $h_{0}$. $n$ is the total number of height levels.  The best fit is that with the smallest residual sum of squares   

\begin{equation}
RSS(j,k) = \sum^{j}_{0}(\theta(i) - (a_{1} + b_{1}z(i)))^{2} + \sum^{k}_{j}(\theta(i) - (a_{2} + b_{2}z(i)))^{2} + \sum^{n}_{k}(\theta(i) - (a_{3} + b_{3}z(i)))^{2}
\end{equation}.
 

\endinput

Any text after an \endinput is ignored.
You could put scraps here or things in progress.
