%% The following is a directive for TeXShop to indicate the main file
%%!TEX root = diss.tex

\chapter{Tools}
\label{ch:methods}

\section{Large Eddy Simulation}
\label{sec:LargeEddieSimulation}
\subsection{Description of The Model}
System for Atmospheric Modelling (SAM) is a Larg Eddy Simulation with cloud resolving capability.

The dynamical framework of the model uses the anelastic equations of motion.

\begin{equation}
\frac{\partial u_{i}}{\partial t} = -\frac{1}{\overline{\rho}}\frac{\partial}{\partial x_{i}}(\overline{\rho}u_{i}u_{j} + \tau_{ij}) - \frac{\partial}{\partial x_{i}}\frac{p^{'}}{\overline{\rho}} + \delta_{i3}B + \epsilon_{ij3}f(u_{j} - U_{gj}) + \left( \frac{\partial u_{i}}{\partial t} \right)_{l.s.}
\end{equation}
\begin{equation}
\frac{\partial}{\partial x_{i}}\overline{\rho}u_{i}=0
\end{equation}

The prognostic thermodynamical variables are the liquid water/ice moist static energy and water.  Our cases are all dry so the
those for water are not used and the following pde reduces to the dry version.

\begin{equation}
\frac{\partial h_{L}}{\partial t} = -\frac{1}{\overline{\rho}}\frac{\partial}{\partial x_{i}}(\overline{\rho} u_{i}h_{L} + F_{hLi}) - \frac{1}{\overline{\rho}}\frac{\partial}{\partial z}(L_{c}P_{r} + L_{s}P_{s} + L_{s}P_{s}) + \left( \frac{\partial h_{L}}{\partial t} \right)_{rad} + \left( \frac{\partial h_{L}}{\partial t} \right)_{mic}
\end{equation}

where

\begin{equation}
h_{L} = c_{p}T + gz - L_{c}(q_{c} + q_{r}) - L_{s}(q_{i} + q_{s} + q_{g})
\end{equation}

A simple first-order Smagorinski closure scheme is used to parametrize the subgrid stresses and scalar fluxes. The eddy diffusivity coefficient is calculated using the
grid scale and only in the vertical if the grid is highly anisotropic.  This is to prevent artificially excessive mixing in the vertical.\\

(Tempeture then diagnosed?)

The model equations are represented discretely on a full staggered Arakawa C-type grid which is uniform in the horizontal and stretched in the vertical. Integration is performed
using a a third -order Adams-Bashforth scheme with variable time step.  Momentum is advected in flux form with second oreder differencing and conservation of kinetic energy. 
Prognosed scalars are advected using a three dimensional positive definite, monotonic scheme.  Lateral boundaries are periodic.  The top is bounded by a rigid lid, and
Newtonian damping is applied in the top third of the domain to reduce the effects of gravity waves.  Surface fluxes are computed using Monin-Obvukhov similarity.

Radiation is not used in this study.\\


\subsection{Ensemble Cases Initial Conditions}

10 cases were run simultaneously, with intial potential temperature profiles differing by a small perturbation based in
a uniform distribtion, and centered about zero.  All runs were initialized with a surface temerature of 300K, a constant
lapse rate ($\gamma$) and a constant average surface heat flux, and zero wind.\\

\begin{table}[!ht]
    \begin{center}
    \begin{tabular}{ | l | l | l | l |}
    \hline
     Domain & Grid Size & Time Step & Run Time  \\ \hline
     128 x 194 x 312 & 25m x 25m x 5 - 100m  & 1s & 8 hrs \\ \hline
     %\end   
\end{tabular}
\caption{}
\label{fig:}   
\end{center}    
\end{table}

The model was run on parallel computers using the Message Passing Interface
(MPI) and grid numbers were chosen based on available processors and computation
 time. 

\subsection{OutPut: NetCDF}

(3d and horizontally averaged Output was dumped every 10 - 15 minutes in binary form.  Output was convertied
to Network Common Data Form (NetCDF).  To analyze and plot the output in Python the NetCDF4 package was used)

\section{Data Analysis}
\label{sec:DataAnalysis}

\subsection{Tri-Linear Fit for Determining $h_{0}$}
\begin{equation}
\theta = bz + a 
\end{equation}
\begin{equation}
b_{1} = \frac{\sum^{j}_{0}z(i) \theta - \frac{1}{j}\sum^{j}_{0}z(i)\sum^{j}_{0}\theta}{\sum^{j}_{0}z(i)^{2} - \frac{1}{j}(\sum^{j}_{0}z(i))^{2}}
\end{equation}
\begin{equation}
a_{1} = \frac{\sum^{j}_{0}z(i)\theta(i)}{\sum^{j}_{0}z(i)} - b_{1}\frac{\sum^{j}_{0}z(i)^{2}}{\sum^{j}_{0}z(i)}
\end{equation}

\begin{equation}
b_{2} = \frac{\sum^{k}_{j}z(i) \theta - (k-j) a_{1}+b_{1}z(j)}{\sum^{k}_{j}z(i) - (k-j)z(j)}
\end{equation}
\begin{equation}
a_{2} = \frac{\sum^{k}_{j}z(i)\theta(i)}{\sum^{k}_{j}z(i)} - b_{2}\frac{\sum^{k}_{j}z(i)^{2}}{\sum^{k}_{j}z(i)}
\end{equation}

\begin{equation}
b_{3} = \frac{\sum^{n}_{k}z(i) \theta - (k-j) a_{1}+b_{1}z(j)}{\sum^{k}_{j}z(i) - (k-j)z(j)}
\end{equation}
\begin{equation}
a_{3} = \frac{\sum^{n}_{k}z(i)\theta(i)}{\sum^{n}_{k}z(i)} - b_{3}\frac{\sum^{n}_{k}z(i)^{2}}{\sum^{n}_{j}z(i)}
\end{equation}

\begin{equation}
RSS(j,k) = \sum^{j}_{0}(\theta(i) - (a_{1} + b_{1}z(i)))^{2} + \sum^{k}_{j}(\theta(i) - (a_{2} + b_{2}z(i)))^{2} + \sum^{n}_{k}(\theta(i) - (a_{3} + b_{3}z(i)))^{2}
\end{equation}

The top of the mixed layer (\acs{ML}) is then z(j) such that RSS(j, k) is minimum.  Unless there is more variation in slope within the \acs{ML} than
in the region between the \acs{ML} and the point where $\gamma$ is resumed, ie no identifiable local \acs{EL}, in which case if the slope of the second
line is small relative to $\gamma$ z(k) is taken to be $h^{l}_{0}$.\\ 

The above equations were written in cython.\\

\subsection{Plots}

The 3 dimensional fields were pulled from the NetCDF files using the 

\endinput

Any text after an \endinput is ignored.
You could put scraps here or things in progress.
