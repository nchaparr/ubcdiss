%% The following is a directive for TeXShop to indicate the main file 
%%!TEX root = diss.tex

\chapter{Research Answers}
\label{ch:results}
\setlength{\parindent}{0cm}

Section \ref{sec:q1} will focus on answering \textbf{Q1 (Entrainment Zone Structure):} \\ 

\textbf{How do the distributions of local \acs{CBL} height, and the joint distributions of $w^{'}$ and $\theta^{'}$ within the \acs{EZ}, vary with $(\overline{w^{'}\theta^{'}})_{s}$ and $\gamma$?}\\

The distributions of local \acs{ML} heights at each horizontal point, in each ensemble member, will be plotted as histograms to visualize the effects of $(\overline{w^{'}\theta^{'}})_{s}$ and $\gamma$.  For the same reason the joint distributions of local potential temperature and velocity fluctuations ($\theta^{'}$ and $w^{'}$) at $h$ will be plotted.  Focus will then be narrowed to the average downward moving warm quadrant at $h$ ($\overline{w^{'-}\theta^{'+}}_{h}$, $\overline{w^{'-}}_{h}$  where $ \theta^{'} >0$ and $\overline{\theta^{'+}}_{h} $ where $ w^{'} < 0 $) to examine the direct influence of $\gamma$ on entrainment.\\       

To answer \textbf{Q2 (Entrainment Zone Boundaries):}\\ 

\textbf{Can the \acs{EZ} boundaries be defined based on the $\overline{\theta}$ profile and what is the relationship} 

\begin{equation}
\frac{\Delta h}{h} \propto Ri^{b}\tag{\ref{eq:dhvsri}}
\end{equation}

\textbf{of the resulting scaled depth ($\frac{\Delta h}{h}$) to \acs{Ri}?}\\

in Section \ref{sec:deltahri}, Equation \ref{eq:dhvsri} will be plotted using height definitions based on the $\frac{\partial \overline{\theta}}{\partial z}$ profile as in Figure \ref{fig:hdefs} and Table \ref{table:reldefs}.  Since the choice of a threshold to determine the lower \acs{EZ} boundary is somewhat arbitrary, plots will be reproduced using two additional values. For comparison with the results of \citeauthor{FedConzMir04} (\citeyear{FedConzMir04}) and \citeauthor{BrooksFowler2} (\citeyear{BrooksFowler2}), Equation \ref{eq:dhvsri} will be plotted using heights based on the average heat flux ($\overline{w^{'}\theta^{'}}$) profile.\\   

In Section \ref{sec:weri} the temperature jump will be defined in four ways to answer 

\textbf{Q3 (Entrainment Rate Parameterization):}\\
 
\textbf{How does defining the $\theta$ jump based on the vertical $\overline{\theta}$ profile across the \acs{EZ} as in Figure \ref{fig:1storder} vs at the inversion ($h$) as in Figure \ref{fig:0order}, affect the entrainment relation}
 \begin{equation}
\frac{w_{e}}{w^{*}} \propto Ri^{a} \tag{\ref{eq:ervsri}}
\end{equation}
\textbf{and in particular $a$?}\\

This analysis will involve observing how \acs{CBL} height evolves in time and culminate in four plots representing Equation \ref{eq:ervsri} in log-log coordinates such that the most suitable values of the exponent $a$ can be identified. 
\\ 

\clearpage

\section{Entrainment Zone Structure}
\label{sec:q1}

\subsection{Local Mixed Layer Heights ($h_{0}^{l}$)}
\label{subsec:locmlh}     
\FloatBarrier

In Figures \ref{fig:rssfitshigh} and \ref{fig:rssfitslow} the local vertical $\theta$ profiles,  each at a single horizontal point in an individual case, exhibit a distinct \acs{ML}.  Above, there are sharp changes in the profile well into the free 
atmosphere, due possibly to waves.  These render the gradient method for determining a local \acs{CBL} height, $h^{l}$, 
unreliable.  Instead a linear regression method is used, whereby three lines representing: the
 \acs{ML}, the \acs{EZ} and the upper lapse rate ($\gamma$), are fit to the profile according 
to the minimum residual sum of squares.  Determining local \acs{ML} height ($h_{0}^{l}$) in this way was 
more straight forward than the local height of maximum potential temperature gradient 
($h^{l}$) for the reasons stated above.\\  


%For a number of these profiles, i.e. those at points outside an actively impinging thermal as in Figure \ref{fig:rssfitslow}, it was impossible even by eye to locate a reliable \acs{CBL} height based on a maximum in the vertical gradient.  But there was a discernible \acs{EZ}.  Conversely at points within an active thermal as in Figure \ref{fig:rssfitshigh} the \acs{ML} tops were characterized by a sharp vertical $\theta$ gradient and the absence of an \acs{EZ} similar to the zero-order model.  Here, entrainment has yet to begin.  All inspected local profiles showed a clear \acs{ML} and $h^{l}_{0}$. The distributions were seen to broaden with increased $(\overline{w^{'}\theta^{'}})_{s} $ and narrow with increased $ \gamma $.  When normalized by the height of the maximum average vertical potential gradient ($h$) what apparantly remains is the effect of $\gamma$ on the lower boundary or lowest percentile. That is, the lowest \acs{ML} heights become lower under weaker stability ($\gamma$) resulting in an apparant increased negative skew. 


Figure \ref{fig:rssfitshigh} shows two local $\theta$ profiles where $h_{0}^{l}$ is relatively high.  
A sharp interface is evident indicating that this is within an active thermal impinging on the stable layer.
In Figure \ref{fig:rssfitslow}, where $h_{0}^{l}$ is relatively low, a less defined interface indicates 
a point now outside a rising thermal.  Under weaker stability ($\gamma$), as in Figure \ref{fig:rssfitslow} (a), 
these inactive locations show a larger vertical region that could be called a local \acs{EZ}.  
In two-dimensional horizontal plots, not shown here, regions of high 
$h_{0}^{l}$ corresponded to regions of upward moving relatively cool air at $h$.\\

The distribution of $h_{0}^{l}$ represents the range over which \acs{CBL} height varies in space, so as 
discussed in Section \ref{subsec:cblel}, relates to the depth of the entrainment zone (\acs{EZ}).
Figure \ref{fig:localh} (a), (b) and (c) illustrate that the distribution widens with increasing $(\overline{w^{'}\theta^{'}})_{s}$ and narrows with increasing $\gamma$.  When scaled by $h$, the local \acs{ML} height distribution 
narrows with increased $\gamma$, in Figure \ref{fig:localh} (d), (e) and (f).  The upper boundary seems to be constant at about 1.1($\times h$) , whereas the lower boundary clearly increases.  When $h_{0}^{l}$s are lower and their distribution is narrower, the scaled versions have relatively larger spacing between bins and so higher numbers in each bin. In Figure \ref{fig:localh} (d), at higher $(w^{'}\theta^{'})_{s}$ there are fewer $\frac{h_{0}^{l}}{h}$ values with higher probabilities, but the width of the distributions is more or less constant regardless of $(w^{'}\theta^{'})_{s}$.\\

\begin{figure}[htbp]
%Pcolor_Peaks.py [master 61491be] rss_fit plots
\begin{minipage}[b]{0.5\linewidth}
        %
        \subfloat[]{\label{main:a}
                \includegraphics[scale=.335]{/newtera/tera/phil/nchaparr/python/Plotting/Dec252013/pngs/rss_fit_high}}\\
        \end{minipage}             
\quad
\begin{minipage}[b]{0.5\linewidth}
        \subfloat[]{\label{main:b}          
          
                \includegraphics[scale=.335]{/newtera/tera/phil/nchaparr/python/Plotting/Mar52014/pngs/rss_fit_high}}\\
       
       \end{minipage}
\caption[High local \acs{ML}]{Local vertical $\theta$ profiles with 3-line fit for the 60/2.5 (a) and 150/10 (b) runs at 
points where $h^{l}_{0}$ is high.}
        
        \label{fig:rssfitshigh}
\end{figure}

\begin{figure}[htbp]
        
%Pcolor_Peaks.py [master 61491be] rss_fit plots
\begin{minipage}[b]{0.5\linewidth}
        %
        \subfloat[]{\label{main:a}

                \includegraphics[scale=.335]{/newtera/tera/phil/nchaparr/python/Plotting/Dec252013/pngs/rss_fit_low}}\\

        \end{minipage}             
\quad
\begin{minipage}[b]{0.5\linewidth}
        \subfloat[]{\label{main:b}          
          
                \includegraphics[scale=.335]{/newtera/tera/phil/nchaparr/python/Plotting/Mar52014/pngs/rss_fit_low}}\\
       
       \end{minipage}
\caption[Low local \acs{ML}]{Local vertical $\theta$ profiles with 3-line fit for the 60/2.5 (a) and 150/10 (b) runs at 
points where $h^{l}_{0}$ is low. The red line represents the \acs{ML}, the blue represents the \acs{EZ} and the green represents the \acs{FA}.}
        \label{fig:rssfitslow}
\end{figure}

\afterpage{%
\thispagestyle{empty}
\begin{figure}[htbp]

\begin{minipage}[b]{0.5\linewidth}
        \subfloat[]{\label{main:a}
                \includegraphics[scale=.32]{/newtera/tera/phil/nchaparr/python/Plotting/Dec252013/pngs/ML_Height_hist_10}}\\
         \subfloat[]{\label{main:e}
                \includegraphics[scale=.32]{/newtera/tera/phil/nchaparr/python/Plotting/Dec252013/pngs/ML_Height_hist_5}}\\
         \subfloat[]{\label{main:e}
                \includegraphics[scale=.32]{/newtera/tera/phil/nchaparr/python/Plotting/Dec252013/pngs/ML_Height_hist_2point5}}\\
\end{minipage}
\quad
\begin{minipage}[b]{0.5\linewidth}
        \subfloat[]{\label{main:b}   
          \includegraphics[scale=.32]{/newtera/tera/phil/nchaparr/python/Plotting/Dec252013/pngs/Scaled_ML_Height_hist_10}}\\
         \subfloat[]{\label{main:e}
                \includegraphics[scale=.32]{/newtera/tera/phil/nchaparr/python/Plotting/Dec252013/pngs/Scaled_ML_Height_hist_5}}\\     
          \subfloat[]{\label{main:e}
                \includegraphics[scale=.32]{/newtera/tera/phil/nchaparr/python/Plotting/Dec252013/pngs/Scaled_ML_Height_hist_2point5}}\\    
\end{minipage}
\caption[Local \acs{ML} Height Distributions]{Histograms of local \acs{ML} heights ($h^{l}_{0}$) are shown in (a), (b) and (c).  Probability distributions of scaled local \acs{ML} height ($\frac{h^{l}_{0}}{h}$) are shown in (d), (e) and (f). Both sets of plots are taken at 5 hours and darker shading represents higher $\overline{w^{'}\theta^{'}}_{s}$. Stability decreases from top to bottom i.e (a) and (d) represents runs with the highest stability ($\gamma=10Kkm^{-1}$).}
    
\label{fig:localh}
\end{figure}
\clearpage
}

%\begin{figure}[htbp]
 %   \centering
    %plot_height.py [master 199de9a7cf]  
  %  \includegraphics[scale=.5]{/newtera/tera/phil/nchaparr/python/Plotting/Dec252013/pngs/varvsinvri}
  %  \caption{Variance vs \acs{Ri}$^{-1}$ at 5 hours}
   % \label{fig:varsvsinvri}   % label should change
%\end{figure}

\clearpage

\subsection{Local turbulent Velocity and Potential Temperature Fluctuations}
\label{subsec:fluxquadrants}     
\FloatBarrier
%\subsection{2-Dimensional Histograms of $\theta^{'}$ and $w^{'}$}

The two-dimensional histograms of $\theta^{'}$ and $w^{'}$, at each horizontal point in each ensemble case, for all runs at $h$ are plotted in Figure \ref{fig:fluxquadsh} to visualize how the distributions are influenced by changes in $(\overline{w^{'} \theta^{'}})_{s}$ and $\gamma$.  In order to isolate the effects of $\gamma$,  $w^{'}$ and $\theta^{'}$ are scaled by the convective velocity and temperature scales ($w^{*}$ and $\theta^{*}$) respectively and plotted in Figure \ref{fig:scaled_fluxquadhs}.\\

Distributions of both $w^{'}$ and $\theta^{'}$ widen with increasing $(\overline{w^{'}\theta^{'}})_{s}$.  Whereas that of $\theta^{'}$ increases only slightly with increasing stability ($\gamma$) in Figure \ref{fig:fluxquadsh}.  As expected, $\gamma$ inhibits both upward and downward $w^{'}$. The scaled version in Figure \ref{fig:scaled_fluxquadhs} shows damping of $w^{'}$ where potential temperature fluctuations are positive.  This can be seen as the horizontal tick marks bounding the $\frac{w^{'}}{w^{*}}$ distribution become less obscured as $\gamma$ increases.  Concurrently, the coolest negative $\frac{\theta^{'}}{\theta^{*}}$ become less cool, and the warmest become warmer.  So the $\frac{\theta^{'}}{\theta^{*}}$ distribution shifts positively with increasing $\gamma$.\\ 

Although the quadrant of overall largest magnitude is that of upward moving cool air ($w^{'+}\theta^{'-}$), in the \acs{EZ} the net heat flux is downward moving warm ($w^{'-}\theta^{'+}$) air as the other three quadrants approximately cancel.  This is in line with the findings of \citeauthor{SullMoengStev} (\citeyear{SullMoengStev}). \\


\begin{figure}[htbp]
\centering
 \includegraphics[scale=.8]{/newtera/tera/phil/nchaparr/python/Plotting/Dec252013/pngs/fluxquadhist1}                 
\caption[Two-dimensional Distributions of $w^{'}$ and $\theta^{'}$ for all Runs]{Two-dimensional histograms of $w^{'}$ and $\theta^{'}$ at $h$ for $w^{'}\theta^{'} = 150 \ - \ 60\ (Wm^{-2})$ (top - bottom) and $\gamma = 10 \ - \  2.5 (Kkm^{-1})$ (left - right) at five hours}
\label{fig:fluxquadsh}
\end{figure}

\begin{figure}[htbp]
\centering
 \includegraphics[scale=.8]{/newtera/tera/phil/nchaparr/python/Plotting/Dec252013/pngs/scaled_fluxquadhist1}                 
\caption[Two-dimensional Distributions of $\frac{w^{'}}{w^{*}}$ and $\frac{\theta^{'}}{\theta^{*}}$ for all Runs]{Two-dimensional distributions of $\frac{w^{'}}{w^{*}}$ and $\frac{\theta^{'}}{\theta^{*}}$ at $h$ for $(\overline{w^{'}\theta^{'}})_{s} = 150 \ - \ 60 (Wm^{-2})$ (top - bottom) and $\gamma = 10 \ - \  2.5(Kkm^{-1})$ (left - right) at 5 hours. Tick-marks are thickened to show the narrowing of the $\frac{w^{'}}{w^{*}}$ distribution where $\frac{\theta^{'}}{\theta^{*}}$ is positive, as well as the positive shift in $\frac{\theta^{'}}{\theta^{*}}$, as $\gamma$ increases.}
\label{fig:scaled_fluxquadhs}
\end{figure}


\clearpage

\subsection{Downward-moving warm Air at $h$}
\label{subsec:downwarm}

The average downward moving warm quadrant ($\overline{w^{'-}\theta^{'+}}$) at $h$ represents the pockets of trapped or engulfed warm air that become mixed into the growing \acs{CBL}.  So its magnitude can be taken as a measure of entrainment.  Figure \ref{fig:downwarm} shows that this increases, in magnitude, in time, as well as with increasing $(\overline{w^{'}\theta^{'}})_{s}$.  Grouping according to $(\overline{w^{'}\theta^{'}})_{s}$ is evident and there is further collapse when this is applied as scale in Figure \ref{fig:downwarm} (b).  Further partitioning $(\overline{w^{'-}\theta^{'+}})_{h}$ into its velocity and temperature components reveals additional complexity.\\

\begin{figure}[htbp]
\begin{minipage}[b]{0.5\linewidth}
        %plot_height [master c7af4de] scaleddeltahinvri
        \subfloat[]{\label{main:a}
                \includegraphics[scale=.34]{/newtera/tera/phil/nchaparr/python/Plotting/Dec252013/pngs/downwarm.pdf}}\\
        \end{minipage}             
\quad
\begin{minipage}[b]{0.5\linewidth}
        \subfloat[]{\label{main:d}          
          %plot_height [master b9c30ad] scaleddeltahinvri1
                \includegraphics[scale=.34]{/newtera/tera/phil/nchaparr/python/Plotting/Dec252013/pngs/scaled_downwarm.pdf}}\\
     
       \end{minipage}
        \caption[Downward-moving warm Air at $h$]{Plots of (a) the average downward moving warm air at $h$ $(\overline{w^{\prime-}\theta^{\prime+}})_{h}$ and (b) $(\overline{w^{\prime-}\theta^{\prime+}})_{h}$ scaled by the average vertical turbulent heat flux $ ( \overline{ w^{'} \theta^{'} } )_{s} $ vs time}
        \label{fig:downwarm}
\end{figure}

Figure \ref{fig:downwarm_wvel} shows that the velocity component $\overline{w^{'-}}_{h}$  where $ \overline{\theta^{'}}_{h}>0$, is effectively scaled by $w^{*}$.  The curves representing $\overline{\theta^{'+}}_{h}$ where $\overline{w^{'}}_{h}>0$ vs time do collapse when scaled by $\theta^{*}$ in Figure \ref{fig:downwarm_theta}.  However, Figure \ref{fig:downwarm_theta1} shows this component is scaled more effectively by the potential temperature scale introduced in Section \ref{subsubsec:tempscales}, $\delta h \gamma$, thus indicating that the effects of $\gamma$ on the positive potential temperature fluctuations at $h$ are more important than $(\overline{w^{'}\theta^{'}})_{s}$.\\ 
\\   

\begin{figure}[htbp]
\begin{minipage}[b]{0.5\linewidth}
        %plot_height [master c7af4de] scaleddeltahinvri
        \subfloat[]{\label{main:a}
                \includegraphics[scale=.34]{/newtera/tera/phil/nchaparr/python/Plotting/Dec252013/pngs/downwarm_wvel.pdf}}\\
        \end{minipage}             
\quad
\begin{minipage}[b]{0.5\linewidth}
        \subfloat[]{\label{main:d}          
          %plot_height [master b9c30ad] scaleddeltahinvri1
                \includegraphics[scale=.34]{/newtera/tera/phil/nchaparr/python/Plotting/Dec252013/pngs/scaled_downwarm_wvel.pdf}}\\       
       \end{minipage}
        \caption[Downward turbulent velocity Fluctuations at $h$]{(a) Average negative vertical turbulent velocity fluctuation at $h$ ($\overline{w^{\prime-}}_{h}$) at points where $\theta^{\prime}>0$ and (b) $\overline{w^{\prime-}}_{h}$ where $\theta^{\prime}>0$ scaled by $w^{*}$.}
        \label{fig:downwarm_wvel}
\end{figure}

\begin{figure}[htbp]
\begin{minipage}[b]{0.5\linewidth}
        %plot_height [master c7af4de] scaleddeltahinvri
        \subfloat[]{\label{main:a}
                \includegraphics[scale=.34]{/newtera/tera/phil/nchaparr/python/Plotting/Dec252013/pngs/downwarm_theta.pdf}}\\
        \end{minipage}             
\quad
\begin{minipage}[b]{0.5\linewidth}
        \subfloat[]{\label{main:d}          
          %plot_height [master b9c30ad] scaleddeltahinvri1
                \includegraphics[scale=.34]{/newtera/tera/phil/nchaparr/python/Plotting/Dec252013/pngs/scaled_downwarm_theta.pdf}}\\      
       \end{minipage}
        \caption[Positive Potential Temperature Fluctuation at $h$ (i)]{(a) Average positive potential temperature fluctuation at $h$ ($\overline{\theta^{\prime+}}_{h}$) at points where $w^{\prime}<0$ and (b) $\overline{\theta^{\prime+}}_{h}$ where $w^{\prime}<0$ scaled by $\theta^{*}$.}
        \label{fig:downwarm_theta}
\end{figure}


\begin{figure}[htbp]
\begin{minipage}[b]{0.5\linewidth}
        %plot_height [master c7af4de] scaleddeltahinvri
        \subfloat[]{\label{main:a}
                \includegraphics[scale=.34]{/newtera/tera/phil/nchaparr/python/Plotting/Dec252013/pngs/downwarm_theta.pdf}}\\
        \end{minipage}             
\quad
\begin{minipage}[b]{0.5\linewidth}
        \subfloat[]{\label{main:d}          
          %plot_height [master b9c30ad] scaleddeltahinvri1
                \includegraphics[scale=.34]{/newtera/tera/phil/nchaparr/python/Plotting/Dec252013/pngs/scaled_downwarm_theta2.pdf}}\\      
       \end{minipage}
        \caption[Positive Potential Temperature Fluctuation at $h$ (ii)]{(a) $\overline{\theta^{\prime+}}_{h}$ at points where $w^{\prime}<0$ and (b) $\overline{\theta^{\prime+}}_{h}$ where $w^{\prime}<0$ scaled by $\delta h \gamma$.}
        \label{fig:downwarm_theta1}
\end{figure}
\clearpage
\subsection{Answer to Q1}

Using a multi-linear regression method, the local \acs{ML} heights ($h^{l}_{0}$) 
were determined.  Although at each horizontal point an \acs{ML} of almost uniform $\theta$ based on the local profiles is evident, the region directly above it differs depending on location as well as from the average profile. 
Since there is no reliable, local definition of $h$, 
I take the distributions of local \acs{ML} height ($h^{l}_{0}$) to be a measure of \acs{CBL} height variance in space and so the \acs{EZ}.
 These distributions approached similarity when scaled by $h$, showing an increase in the lower 
boundary (or percentile) with increased $\gamma$.  \bf{I interpret this result as an indication that increased $\gamma$ results in a narrower scaled \acs{EZ} depth}.\\

\bf{Two dimensional distributions of the local turbulent fluctuations, $w^{'}$ and $\theta^{'}$ at $h$, show some variation with $\gamma$ when scaled by the convective scales $w^{*}$ and $\theta^{*}$.  The distribution of $\frac{w^{'}}{w^{*}}$ where $\frac{\theta^{'}}{\theta^{*}}$ is positive narrows, while $\frac{\theta^{'}}{\theta^{*}}$ shifts positively.}\\

Plots of the average downward moving warm quadrant at $h$ ( $(\overline{w^{'-}\theta^{'+}})_{h}$ ) show dependence on $(\overline{w^{'}\theta^{'}})_{s}$. Breaking $(\overline{w^{'-}\theta^{'+}})_{h}$ into its two components reveals dependence on both $(\overline{w^{'}\theta^{'}})_{s}$ and $\gamma$. The average downward moving velocity at $h$ ( $(\overline{w^{'-}})_{h}$ ), at points where there is a positive potential temperature fluctuation ($\theta^{'+}$), show clear dependence on $w^{*}$.  Whereas the average positive potential temperature fluctuation $\overline{\theta^{'+}}_{h}$ where $w^{'}$ is negative seems to approach a constant value of $\delta h \gamma$. So as one would expect, the potential temperature of the entrained warm air depends on $\gamma$.

\clearpage

%\section{$h$ and  $\Delta h$ based on Average Profiles}
%\label{sec:hdeltahavprofs}

%\FloatBarrier

\section{Entrainment Zone Boundaries}
\label{sec:deltahri}
\FloatBarrier
\subsection{\acs{EZ} Boundaries based on the average vertical Potential Temperature Gradient Profiles}
In Figure \ref{fig:scaledEZlims} (a) the scaled upper \acs{EZ} boundaries collapse to an initial value of approximately 1.15, decreasing to about 1.10.  The scaled lower \acs{EZ} boundaries appear grouped according to $\gamma$ and increase with respect to time.  So overall the scaled \acs{EZ} depth ($\frac{\Delta h}{h} = \frac{h_{1}}{h}$ - $\frac{h_{0}}{h}$) narrows with time.\\

\begin{figure}[htbp]
    \centering
    %plot_height.py [master fd5c6b1] delta h vs time1  
    \includegraphics[scale=.5]{/newtera/tera/phil/nchaparr/python/Plotting/Dec252013/pngs/scaleddeltahstime}
    \caption[Plot of scaled \acs{EZ} upper ($\frac{h_{1}}{h}$) and lower ($\frac{h_{0}}{h}$) Boundaries]{Plot of scaled \acs{EZ} upper ($\frac{h_{1}}{h}$) and lower ($\frac{h_{0}}{h}$) boundaries based on the average vertical potential temperature gradient profile.}
    \label{fig:scaledEZlims}   % label should change
\end{figure}

The lower entrainment zone boundary $h_{0}$, as illustrated in Figure \ref{fig:thresh} is the point at which the vertical 
$\frac{\partial \overline{\theta}}{\partial z}$ profile exceeds a threshold ($.0002K km^{-1}$), chosen such that
it is positive and at least an order of magnitude smaller than $\gamma$.   
As suggested by the results of Section \ref{subsec:locmlh}, the scaled \acs{EZ} depth decreases with increasing Richardson number ($\acs{Ri} = \frac{\frac{g}{\overline{\theta}_{ML}}\Delta \theta h}{w^{*2}}$ as in Table \ref{tab:reldefs}).  However, grouping of the curves representing the relationship of scaled \acs{EZ} depth to Richardson number
\begin{equation}
\frac{\Delta h}{h} \propto Ri ^{b} \tag{\ref{eq:dhvsri}}
\end{equation}
according to $\gamma$ is evident in Figure \ref{fig:scaledeltahinvri}.\\


\begin{figure}[htbp]
    \centering
    %plot_height.py [master fd5c6b1] delta h vs time1  
    \includegraphics[scale=.5]{/newtera/tera/phil/nchaparr/python/Plotting/Dec252013/pngs/theta_grad_profs}
    \caption[$\frac{\partial \overline{\theta}}{\partial z}$ profiles with threshold at $.0002Kkm^{-1}$]{$\frac{\partial \overline{\theta}}{\partial z}$ profiles with threshold at $.0002Kkm^{-1}$.  Black lines represent the threshold at $\frac{\partial \overline{\theta}}{\partial z} = 0.0002$ for the lower \acs{EZ} boundary, as well as the three lapse rates ($0.0025, \ 0.005$ and $.01 \ Km^{-1}$ ).}
    \label{fig:thresh}   % label should change
\end{figure}

\begin{figure}[htbp]
\centering
%\begin{minipage}[b]{0.5\linewidth}
        %plot_height [master c7af4de] scaleddeltahinvri
 %       \subfloat[]{\label{main:a}
 \includegraphics[scale=.5]{/newtera/tera/phil/nchaparr/python/Plotting/Dec252013/pngs/scaleddeltahinvri}
  %      \end{minipage}             
%\quad
%\begin{minipage}[b]{0.5\linewidth}
 %       \subfloat[]{\label{main:d}          
          %plot_height [master b9c30ad] scaleddeltahinvri1
  %              \includegraphics[scale=.36]{/newtera/tera/phil/nchaparr/python/Plotting/Dec252013/pngs/scaleddeltahinvri1}}\\
       
   %    \end{minipage}
        \caption{Scaled EZ depth ($\frac{h_{1}-h_{0}}{h}$) vs inverse Richardson number ($\acs{Ri}_{Delta}^{-1}$) with threshold at $.0002Kkm^{-1}$}
         \label{fig:scaledeltahinvri}
\end{figure}

\clearpage
\subsubsection{Threshold Test for lower \acs{EZ} Boundary, $h_{0}$}
To explore how varying the threshold value effects Equation \ref{eq:dhvsri}, plots analogous to Figure \ref{fig:scaledeltahinvri} were produced at two 
additional thresholds.  In Figure \ref{fig:scaledeltahinvri1}, a higher value ($.0004Kkm^{-1}$) results in a higher $h_{0}$   
and so a narrower \acs{EZ}.
In Figure \ref{fig:scaledeltahinvri2}, a lower threshold value ($.0001Kkm^{-1}$) results in a lower $h_{0}$.  Both of these threshold values result in grouping according to $\gamma$.\\

\begin{figure}[htbp]
    \centering
    %plot_height.py [master fd5c6b1] delta h vs time1  
    \includegraphics[scale=.5]{/newtera/tera/phil/nchaparr/python/Plotting/Dec252013/pngs/scaleddeltahinvri_5}
    \caption{Scaled EZ depth vs inverse Richardson Number with threshold at $.0004Kkm^{-1}$}
    \label{fig:scaledeltahinvri1}   % label should change
\end{figure}

\begin{figure}[htbp]
    \centering
    %plot_height.py [master fd5c6b1] delta h vs time1  
    \includegraphics[scale=.5]{/newtera/tera/phil/nchaparr/python/Plotting/Dec252013/pngs/scaleddeltahinvri_6}
    \caption{Scaled EZ depth vs inverse Richardson number (\acs{Ri}$^{-1}$) with threshold at $.0001Kkm^{-1}$}
    \label{fig:scaledeltahinvri2}   % label should change
\end{figure}

\clearpage
\subsection{\acs{EZ} Boundaries based on scaled vertical Potential Temperature Gradient Profiles}
\label{subsec:ellimscaledprof}

The curves representing Equation \ref{eq:dhvsri} 

\begin{equation}
\frac{\Delta h}{h} \propto Ri^{b} \tag{\ref{eq:dhvsri}}
\end{equation}

collapse when the heights are defined based on the scaled vertical potential temperature gradient 
($\frac{\frac{\partial \overline{\theta}}{\partial z}}{\gamma}$) profile in Figure \ref{fig:deltahinvri_scaled}.  Here $Ri = \acs{Ri}_{\delta}=\frac{\frac{g}{\overline{\theta}_{ML}} \Delta \theta h}{w^{*2}}$ and $\Delta \theta = \overline{\theta}(h_{1}) - \overline{\theta}(h_{0})$.  This stems
from a switch in the relative magnitudes of the vertical potential temperature gradient in the upper \acs{ML} which can be seen when Figure \ref{fig:thresh3} is compared to Figure \ref{fig:thresh}. So from here on all heights will be defined based on the scaled average profiles.   
\\

\begin{figure}[htbp]
    \centering
    %plot_height.py [master fd5c6b1] delta h vs time1  
    \includegraphics[scale=.5]{/newtera/tera/phil/nchaparr/python/Plotting/Dec252013/pngs/scaled_theta_grad_profs}
    \caption{Scaled $\frac{\partial \overline{\theta}}{\partial z}$ profiles with threshold at .03}
    \label{fig:thresh3}   % label should change
\end{figure}

Figure \ref{fig:loglogdeltahinvri} supports an exponent $b = -\frac{1}{2}$ at lower values of \acs{Ri}, increasing to $b = -1$ at higher \acs{Ri}.    

\clearpage

\begin{figure}[t]
    \centering
    %plot_height.py [master fd5c6b1] delta h vs time1  
    \includegraphics[scale=.44]{/newtera/tera/phil/nchaparr/python/Plotting/Dec252013/pngs/scaleddeltahinvri_4}
    \caption[scaled \acs{EZ} depth vs \acs{Ri}$^{-1}$]{Plot of scaled \acs{EZ} depth vs \acs{Ri}$^{-1}$. \acs{EZ} boundaries and so $\Delta \theta = \overline{\theta}(h_{1}) - \overline{\theta}(h_{0})$ are based on the $\frac{\frac{\partial \overline{\theta}}{\partial z}}{\gamma}$ profile.}
    \label{fig:deltahinvri_scaled}   % label should change
\end{figure}                      
\vspace{-50mm}
\begin{figure}[b]
\centering
\includegraphics[scale=.44]{/newtera/tera/phil/nchaparr/python/Plotting/Dec252013/pngs/loglog_scaleddeltahinvri_4}\\
\caption[Log-log plot of scaled \acs{EZ} depth vs \acs{Ri}$^{-1}$]{Scaled \acs{EZ} depth vs \acs{Ri}$^{-1}$ based on the $\frac{\frac{\partial \overline{\theta}}{\partial z}}{\gamma}$ profile in log-log coordinates to identify likely values of the exponent $b$}
\label{fig:loglogdeltahinvri}
\end{figure}

\clearpage

\subsection{\acs{EZ} Boundaries based on scaled Heat Flux Profiles}
\label{susec:fluxbound}
When based on the vertical heat flux profile, the scaled \acs{EZ} depth ($\frac{z_{f0}-z_{f1}}{z_{f}}$) remains more less constant with respect to time in Figure \ref{fig:scaledEZlims1}. Figure \ref{fig:deltahinvri_scaled} shows little or no \acs{Ri} dependence. 

\begin{figure}[htbp]
    \centering
\includegraphics[scale=.5]{/newtera/tera/phil/nchaparr/python/Plotting/Dec252013/pngs/scaled_h_f0_time}              
\caption[Scaled \acs{EZ} Boundaries based on the vertical heat flux profile]{Plot of scaled upper ($\frac{z_{f1}}{z_{f}}$) and lower ($\frac{z_{f0}}{z_{f}}$) \acs{EZ} boundaries based on vertical heat flux profile}
    \label{fig:scaledEZlims1}   % label should change
\end{figure}

\begin{figure}[htbp]
    \centering
    %plot_height.py [master fd5c6b1] delta h vs time1  
    \includegraphics[scale=.5]{/newtera/tera/phil/nchaparr/python/Plotting/Dec252013/pngs/scaleddeltahinvri_f}
    \caption[scaled \acs{EZ} depth vs \acs{Ri}$^{-1}$ based on the vertical heat flux profile]{Plots of scaled \acs{EZ} depth vs $\acs{Ri}_{\Delta}^{-1}$. \acs{EZ} boundaries and so $\Delta \theta$ are based on the $\frac{\overline{w^{'}\theta^{'}}}{(\overline{w^{'}\theta^{'}})_{s}}$ profile.}
    \label{fig:deltahinvri_scaled}   % label should change
\end{figure}

\clearpage

\subsection{Answer to Q2}

Initially, I define \acs{CBL} height and \acs{EZ} boundaries based on the $\frac{\partial \overline{\theta}}{\partial z}$ profile.  As \citeauthor{BrooksFowler2} (\citeyear{BrooksFowler2}) point out, when using an average vertical tracer profile there is no universal criterion for a significant gradient.  So a threshold value for the lower \acs{EZ} boundary ($h_{0}$) was chosen such that it was positive, small (i.e. an order of magnitude less than $\gamma$) and the same for all runs.  For the sake of rigor, plots of the relationship

\begin{equation}
\frac{\Delta h}{h} \propto Ri ^{b} \tag{\ref{eq:dhvsri}}
\end{equation}

were produced based on two additional threshold values yielding analogous results.  In all three cases curves representing Equation \ref{eq:dhvsri} grouped according to $\gamma$\\

The importance of $\gamma$ is revealed again as the curves representing equation \ref{eq:dhvsri} become similar when heights are based on the scaled $\frac{\partial \overline{\theta}}{\partial z}$ profile, $\frac{\frac{\partial \overline{\theta}}{\partial z}}{\gamma}$. Further inspection shows that this change primarily occurs at the lower \acs{EZ} boundary ($h_{0}$) when $\frac{\partial \overline{\theta}}{\partial z}$ is measured as proportion of $\gamma$. The influence of $\gamma$ on $\frac{\partial \overline{\theta}}{\partial z}$ at $h_{0}$ ties in with the influence of $\gamma$ on downward moving $\theta^{'+}$ at $h$ shown in Section \ref{subsec:downwarm}.  This prompts the use of the scaled profiles for the heights ($h_{0}$, $h$, $h_{1}$ and $z_{f0}$, $z_{f}$, $z_{f1}$) in the subsequent section.\\

\bf{These results support a varying exponent $b$ in Equation \ref{eq:dhvsri} which is lower in magnitude ($-\frac{1}{2}$) at lower \acs{Ri} and approaches $-1$ at higher \acs{Ri}.  This is in line with theory and the results of comparable studies so the \acs{EZ} boundary definitions based on the $\frac{\frac{\partial \overline{\theta}}{\partial z}}{\gamma}$ profile are valid.}  For comparison with results from other studies these heights are also based on the vertical $\overline{w^{'}\theta^{'}}$ profiles as shown in Figure \ref{fig:hdefs}. I find no clear dependence of the scaled \acs{EZ} depth on \acs{Ri} within this framework. \\

\clearpage

\section{Entrainment Rate Parameterization}
\label{sec:weri}
\FloatBarrier


\subsection{Reminder of Definitions}

A key finding of Section \ref{subsec:ellimscaledprof} was that curves representing Equation \ref{eq:dhvsri} group according
to $\gamma$ when heights are based on the unscaled $\frac{\partial \overline{\theta}}{\partial z}$ profile and then become similar
when heights are based on $\frac{\frac{\partial \overline{\theta}}{\partial z}}{\gamma}$.  So from here on all heights will be as in
Figure \ref{fig:hdefs1} and the corresponding Richardson numbers (\acs{Ri}) will be as in Table \ref{tab:reldefs}.\\ 

\begin{figure}[htbp]
    \centering
    %plot_height.py[master 1573b9d] h vs time plot
    \includegraphics[scale=.5]{/newtera/tera/phil/nchaparr/python/Plotting/Dec252013/pngs/height_defs_1.pdf}
    \caption[Height definitions]{Height definitions based on the scaled average vertical profiles. $\theta_{0}$ is the initial potential temperature.}
    \label{fig:hdefs1}   % label should change
\end{figure}

\subsection{\acs{CBL} Growth}

Convective boundary layer height ($h$) in Figure \ref{fig:hvstime} increases, rapidly at first, with a steadily decreasing rate and relates to the square-root of time in Figure \ref{fig:loghvstime}.  \citeauthor{FedConzMir04} (\citeyear{FedConzMir04})
focus on the attainment of a quasi-steady state regime in which their zero-order model applies.  Within this regime scaled \acs{CBL} height , $hB_{s}^{-\frac{1}{2}}N^{\frac{3}{2}}$ where $B_{s}$ is the surface buoyancy flux, relates to the square-root of their scaled time, $tN$. Over the time of the runs $B_{s}$ is constant and $N$ varies much more slowly than h.  So based on Figure \ref{fig:loghvstime} I conclude that over the period during which I obtain output, all runs are in this quasi-steady state. The height of minimum vertical heat flux $z_{f}$ is a constant proportion of $h$ in Figure \ref{fig:zfvstime} indicating that this point advances more slowly than $h$.\\
  
\begin{figure}[htbp]
    \centering
    %plot_height.py[master 1573b9d] h vs time plot
    \includegraphics[scale=.5]{/newtera/tera/phil/nchaparr/python/Plotting/Dec252013/pngs/hvstime}
    \caption{$h$ vs time for all runs}
    \label{fig:hvstime}   % label should change
\end{figure}

\begin{figure}[htbp]
    \centering
    %plot_height.py[master 1573b9d] h vs time plot
    \includegraphics[scale=.5]{/newtera/tera/phil/nchaparr/python/Plotting/Dec252013/pngs/hstimelog}
    \caption{$h$ vs time for all runs on log-log coordinates}
    \label{fig:loghvstime}   % label should change
\end{figure}

%need flux height scaled by h plot

\begin{figure}[htbp]
    \centering
    %plot_height.py[master 1573b9d] h vs time plot
    \includegraphics[scale=.5]{/newtera/tera/phil/nchaparr/python/Plotting/Dec252013/pngs/scaled_h_f_time}
    \caption{$\frac{z_{f}}{h}$ vs Time}
    \label{fig:zfvstime}   % label should change
\end{figure}

\clearpage

\subsection{Heights based on the scaled vertical average Potential Temperature Profile}
\label{subsec:thetari}

The inverse Richardson numbers (\acs{Ri}$_{\Delta}^{-1}$ and \acs{Ri}$_{\delta}^{-1}$) in Figure \ref{fig:invristime} decrease in time and group according to $\gamma$. There is an overall difference in magnitude since $\Delta \theta > \delta \theta$.\\  

\begin{figure}[htbp]

\begin{minipage}[b]{0.5\linewidth}
         
        \subfloat[]{\label{main:a}
                \includegraphics[scale=.34]{/newtera/tera/phil/nchaparr/python/Plotting/Dec252013/pngs/invristime_Delta}}\\
        \end{minipage}             
\quad
\begin{minipage}[b]{0.5\linewidth}
        \subfloat[]{\label{main:d}
          %plot_height [master fbd2dfd] invristime1, see branches named accordingly
                \includegraphics[scale=.34]{/newtera/tera/phil/nchaparr/python/Plotting/Dec252013/pngs/invristime_delta}}\\
       
       \end{minipage}
        \caption[Richardson numbers based on $\frac{\frac{\partial \overline{\theta}}{\partial z}}{\gamma}$]{Inverse Richardson number vs time based on the $\frac{\frac{\partial \overline{\theta}}{\partial z}}{\gamma}$
profile using $\Delta \theta$ across the \acs{EZ} in (a) and $\delta \theta$ at $h$ in (b).  See Table \ref{tab:reldefs}.}
        \label{fig:invristime}
\end{figure}

The entrainment rate ($w_{e}= \frac{dh}{dt}$) is determined from the slope of a second order polynomial fit to $h(time)$ in Figure \ref{fig:hvstime}.  When $w_{e}$ is scaled by $w^{*}$ the resulting relationship to \acs{Ri}$_{\Delta}$  

\begin{equation}
\frac{w_{e}}{w^{*}} \propto Ri_{\Delta}^{a},
\end{equation}

plotted in log-log coordinates in Figure \ref{fig:weinvri} (a), has an exponent $a = -1$ at lower \acs{Ri}$_{\Delta}$ and $a = -\frac{3}{2}$ at higher \acs{Ri}$_{\Delta}$.\\

In Figure \ref{fig:weinvri} (b) the relationship

\begin{equation}
\frac{w_{e}}{w^{*}} \propto Ri_{\delta}^{a}
\end{equation}

possibly approaches a value of $a = -1$ at higher \acs{Ri}$_{\delta}$ but a value of lower magnitude would fit better overall. \\    

\begin{figure}[htbp]
\begin{minipage}[b]{0.5\linewidth}
        %plot_height [master fbd2dfd] invristime1
        \subfloat[]{\label{main:a}
                \includegraphics[scale=.34]{/newtera/tera/phil/nchaparr/python/Plotting/Dec252013/pngs/scaledweinvri_Delta}}\\
        \end{minipage}             
\quad
\begin{minipage}[b]{0.5\linewidth}
        \subfloat[]{\label{main:d}          
          %plot_height [master 01c3721] scaledweinvri1
                \includegraphics[scale=.34]{/newtera/tera/phil/nchaparr/python/Plotting/Dec252013/pngs/scaledweinvri_delta}}\\       
       \end{minipage}
        \caption[Scaled entrainment rate vs inverse Richardson number]{Scaled entrainment rate vs inverse Richardson number (\acs{Ri}$^{-1}$), in log-log coordinates, where \acs{Ri} is based on the $\frac{\frac{\partial \overline{\theta}}{\partial z}}{\gamma}$ profile using $\Delta \theta$ across the \acs{EZ} in (a) and $\delta \theta$ at $h$ in (b). See Figure \ref{fig:hdefs1}.}
        \label{fig:weinvri}
\end{figure}

\clearpage

\subsection{Heights based on the scaled vertical average Heat Flux Profile}

Richardson numbers with $\Delta \theta$ and $\delta \theta$ based on the $\overline{w^{'}\theta^{'}}$ profile are comparable with those in Section \ref{subsec:thetari} although \acs{Ri}$_{\Delta}$ shows considerable scatter in Figure \ref{fig:invristime_f} (a).

\begin{figure}[htbp]
\begin{minipage}[b]{0.5\linewidth}
         
        \subfloat[]{\label{main:a}
                \includegraphics[scale=.34]{/newtera/tera/phil/nchaparr/python/Plotting/Dec252013/pngs/invristime_Delta_f}}\\
        \end{minipage}             
\quad
\begin{minipage}[b]{0.5\linewidth}
        \subfloat[]{\label{main:d}
          %plot_height [master fbd2dfd] invristime1, see branches named accordingly
                \includegraphics[scale=.34]{/newtera/tera/phil/nchaparr/python/Plotting/Dec252013/pngs/invristime_delta_f}}\\
       
       \end{minipage}
        \caption[Richardson numbers based on $\frac{\overline{w^{'}\theta^{'}}}{\overline{w^{'}\theta^{'}}_{s}}$]{Inverse Richardson number vs time based on the $\frac{\overline{w^{'}\theta^{'}}}{\overline{w^{'}\theta^{'}}_{s}}$
profile using $\Delta \theta$ across the \acs{EZ} in (a) and $\delta \theta$ at $z_{f}$ in (b).  See Figure \ref{fig:hdefs} and Table \ref{tab:reldefs}.}
        \label{fig:invristime_f}
\end{figure}

In Figure \ref{fig:weinvri_f} the axes are in log-log coordinates and all heights are based on the scaled $\overline{w^{'}\theta^{'}}$ profile. The relationship of scaled entrainment rate to \acs{Ri}$_{\Delta}$ in (a) shows scatter and either value of $a$ or a value in between could fit.  Whereas the exponent in the relationship to \acs{Ri}$_{\delta}$ in (b) seems to change throughout the run(s) and a value less (in magnitude) than $-1$ might fit better. \\    

\begin{figure}[htbp]
\begin{minipage}[b]{0.5\linewidth}
        %plot_height [master fbd2dfd] invristime1
        \subfloat[]{\label{main:a}
                \includegraphics[scale=.335]{/newtera/tera/phil/nchaparr/python/Plotting/Dec252013/pngs/scaledweinvri_Delta_f}}\\
        \end{minipage}             
\quad
\begin{minipage}[b]{0.5\linewidth}
        \subfloat[]{\label{main:d}          
          %plot_height [master 01c3721] scaledweinvri1
                \includegraphics[scale=.335]{/newtera/tera/phil/nchaparr/python/Plotting/Dec252013/pngs/scaledweinvri_delta_f}}\\
       
       \end{minipage}
        \caption[Scaled entrainment rate vs inverse Richardson number (ii)]{Scaled entrainment rate vs inverse Richardson number (\acs{Ri}$^{-1}$), in log-log coordinates, where \acs{Ri} is based on the $\frac{\overline{w^{'}\theta^{'}}}{(\overline{w^{'}\theta^{'}})_{s}}$
profile using $\Delta h$ across the \acs{EZ} in (a) and $\delta \theta$ at $z_{f}$ in (b).  See Figure \ref{fig:hdefs} and Table \ref{tab:reldefs}.}
        \label{fig:weinvri_f}
\end{figure}

\subsection{Answer to Q3}
In conclusion the relationship of scaled entrainment rate to \acs{Ri}$_{\Delta}$ based on the $\frac{\frac{\partial \overline{\theta}}{\partial z}}{\gamma}$ profile shows the least scatter over time and between runs in Figure \ref{fig:weinvri}.  Here the exponent seems to start at a value close to $-1$ increasing in magnitude, with higher \acs{Ri}, to close to $-\frac{3}{2}$.  This apparent change with increased \acs{Ri} mirrors that seen with Equation \ref{eq:dhvsri} in Figure \ref{fig:loglogdeltahinvri}.  It's possible that it represents a change in entrainment mechanism as discussed in Section \ref{subsec:scales}.  \bf{Overall the definition of the temperature jump certainly has an effect, $\Delta \theta$ yielding a value of $a$ higher in magnitude than $\delta \theta$.}

\endinput

Any text after an \endinput is ignored.
You could put scraps here or things in progress.
