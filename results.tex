%% The following is a directive for TeXShop to indicate the main file
%%!TEX root = diss.tex

\chapter{Results}
\label{ch:results}

\section{Runs}

%description of runs ie 10 member ensembles each had delta x, delta y=25 and a region of delta z=5m enclosing the 
%Entrainment zone, z=25 below  and streched to 100 above. 

\label{sec:Runs}
\begin{table}
    \begin{tabular}{ | l | l | l | l |}
    \hline
    $\overline{w^{'}\theta^{'}_{s}}$ / $\gamma$ & 10 (K/Km) & 5 (K/Km) & 2.5 (K/Km) \\ \hline
     150 (W/m2)& \hspace{5mm} \ding{51} &\hspace{5mm} \ding{51}\footnotemark &  \\ \hline
     100 (W/m2)& \hspace{5mm} \ding{51} & \hspace{5mm} \ding{51} & \\ \hline
     60 (W/m2) & \hspace{5mm} \ding{51} & \hspace{5mm} \ding{51} & \hspace{5mm} \ding{51}\\ \hline
    \end{tabular}\\
    \footnotetext{Incomplete run: EL exceded high resolution vertical grid after 7 hours}
\end{table}


\section{Checking the Model based on Ensemble Averaged, Horizontally Averaged Profiles}
\label{sec:CheckingtheModel}
\subsection{Spin Up Time, Convective Time Scale $\tau$}

Time must be allowed for spin-up and turbulent mixing.  \citeauthor{SullMoengStev} in \cite{SullMoengStev}
recommended 10 eddie turnover times, and \citeauthor{BrooksFowler2} in \cite{BrooksFowler2} chose a 
simulated time of 2 hours.  For all of the runs, 10 eddie turnover times were completed by 2 hours (Figure \ref{fig:ScaledTimevsTime}).\\


\begin{figure}[!ht]
    \centering
    % For the sake of this example, we'll just use text
    \includegraphics[scale=.5]{/tera/phil/nchaparr/python/Plotting/Dec252013/pngs/scaledtimevstime}
    \caption{Plots of Scaled Time vs Time for all runs}
    \label{fig:ScaledTimevsTime}   % label should change
\end{figure}

A Measureable well mixed layer and entrainment layer develops after 2 hours (Figure \ref{fig:tempgradfluxprofs15010}).\\

\begin{figure}[!ht]
    \centering
    % For the sake of this example, we'll just use text
    \includegraphics[scale=.5]{/tera/phil/nchaparr/python/Plotting/Mar52014/pngs/theta_flux_profs}
    \caption{Potential Temperature, it's vertical gradient and Flux Profiles for the 150/10 Run}
    \label{fig:tempgradfluxprofs15010}   % label should change
\end{figure}

Fluxes and root mean squared vertical velocity (Figure \ref{fig:rmswvelprofs15010}) are scaled well by the surface flux ($\overline{w^{,}\theta^{,}}_{s}$) and convective velocity scale ($w^{*}$) value after 2 hours (Figure \ref{fig:scaledtempgradfluxprofs15010}).\\


\begin{figure}[!ht]
    \centering
    % For the sake of this example, we'll just use text
    \includegraphics[scale=.5]{/tera/phil/nchaparr/python/Plotting/Mar52014/pngs/scaled_theta_flux_profs}
    \caption{Potential Temperature, it's scaled vertical gradient and scaled Flux Profiles for the 150/10 Run}
    \label{fig:scaledtempgradfluxprofs15010}   % label should change
\end{figure}

\begin{figure}[!ht]
    \centering
    % For the sake of this example, we'll just use text
    \includegraphics[scale=.5]{/tera/phil/nchaparr/python/Plotting/Mar52014/pngs/rmswvels}
    \caption{Root Mean Vertical Velocity Squared Profiles for 150/10 Run}
    \label{fig:rmswvelprofs15010}   % label should change
\end{figure}

\subsection{Temperature, Heat Flux and Kinetic Energy}

The horizontally averaged, ensemble averaged $\theta$ profiles exhibit a mixed layer of height dependent on
$\overline{w^{,}\theta^{,}}_{s}$ and $\gamma$, topped by a region where the vertical gradient exceeds zero,
passing through a maximum (h) before resuming $\gamma$ (Figure \ref{fig:pottempprofs2hrs}).\\

\begin{figure}[!ht]
    \centering
    % For the sake of this example, we'll just use text
    \includegraphics[scale=.5]{/tera/phil/nchaparr/python/Plotting/Dec252013/pngs/theta_profs2hrs}
    \caption{Potential Tempertature Profiles at 2 hours}
    \label{fig:pottempprofs2hrs}   % label should change
\end{figure}


The horizonally averaged, ensemble averaged $\overline{w^{,}\theta^{,}}$ profiles decrease from the surface value, passing through zero to a minumum and increase to zero.(Figure \ref{fig:fluxprofs2hrs})\\

\begin{figure}[!ht]
    \centering
    % For the sake of this example, we'll just use text
    \includegraphics[scale=.5]{/tera/phil/nchaparr/python/Plotting/Dec252013/pngs/flux_profs2hrs}
    \caption{$\overline{w^{,}\theta^{,}}_{s}$ Profiles at 2 hours}
    \label{fig:fluxprofs2hrs}   % label should change
\end{figure}

Each of the $\overline{\theta}$ and $\overline{w^{,}\theta^{,}}$ profiles has a region that can be defined as the Entrainment Zones.  The point of minimum $\overline{w^{,}\theta^{,}}$
is lower than h.  \citeauthor{SullMoengStev} in \cite{SullMoengStev} noted that the minima of the individual flux quadrant profiles are closer or even coincide with h.\\

Root mean velocity squared profiles show a dominance of vertical velocity in the mixed layer, with a peak in horizontal velocity within the entrainment layer where the vertical is inhibited (Figure \ref{fig:rmsvel150102hrs}). \\

Two dimensional fft spectra of horizontal slices of the horizontal (\ref{fig:2dfftu602point5}) and vertical (\ref{fig:2dfftw602point5}) velocity perturbations taken at three
different levels (below, within and above the Entrainment Layer) show peaks in energy at the larger scales.  Within the Mixed
and Entrainment Layers the energy cascades to the smaller scales roughly according to a $\frac{-8}{3}$ slope.  Whereas
above (or at the top of) the Entrainment Layer, where stability supresses turbulence, this slope is steeper.  (TODO: why non zero less than k=1, see scaling factor in script.  must justify slope.)  

\begin{figure}[!ht]
    \centering
    % For the sake of this example, we'll just use text
    \includegraphics[scale=.5]{/tera/phil/nchaparr/python/Plotting/Mar52014/pngs/rmsvel2}
    \caption{$\frac{\sqrt[]{u^{,2}}}{w^{*}}$ Profiles at 2 hours for the 150/10 Run}
    \label{fig:rmsvel150102hrs}   % label should change
\end{figure}

\begin{figure}[!ht]
    \centering
    % For the sake of this example, we'll just use text
    \includegraphics[scale=.5]{/tera/phil/nchaparr/python/Plotting/Dec252013/pngs/2dffts_w}
    \caption{Vertical Velocity Averaged 2D FFT Energy Densities vs Wavenumber ($k = \sqrt{k_{x}^{2}+k_{y}^{2}}$) for 60 / 2.5 Run}
    \label{fig:2dfftw602point5}   % label should change
\end{figure}

\subsection{Visualization of Structures Within the Entrainment Layer}

\begin{figure}
        \centering
        \begin{subfigure}[b]{0.5\textwidth}
                \includegraphics[width=\textwidth]{/tera/phil/nchaparr/python/Plotting/Mar52014/pngs/theta_cont0}
                \caption{}
                \label{fig:}
        \end{subfigure}%
        ~ %add desired spacing between images, e. g. ~, \quad, \qquad etc.
          %(or a blank line to force the subfigure onto a new line)
        \begin{subfigure}[b]{0.5\textwidth}
                \includegraphics[width=\textwidth]{/tera/phil/nchaparr/python/Plotting/Mar52014/pngs/wvel_cont0}
                \caption{}
                \label{fig:tiger}
        \end{subfigure}
        \caption{}\label{fig:}
\end{figure}


\section{h and  $\Delta h$ based on Average Profiles}
\label{sec:hdeltahavprofs}     



%\subsection{$\theta$ and Flux}

%Profiles of horizonatally averaged ensemble averaged $\theta$.\\

%We expect to see formation of a measureable mixed layer with uniform temperature topped by an entrainment region.\\

%Get $\theta_{t+1} -  \theta_{t}$ profiles to predict flux profiles: $\frac{d\theta}{dz} = \overline{w^{'}\theta^{'}}$ so points at where sucessive $\theta$ profiles intersect should more or less correspond to points of zero flux crossing.  Points at which $\theta$s decrease most from one time to the next, should correspond to negative peak in fluxes.\\

%Compare Flux profiles to predicted flux profiles to $\theta$ profiles.\\

%Some type of quadrant analysis : plots of horizontally averaged ensemble averaged upwarm, downwarm, upcold, downcold alongside average flux.\\

%\subsection{2d FFTs}

%\subsection{Root Mean Squared Velocity Profiles}

%\section{h, and EL Limits based on Average Profiles}

%\subsection{Definitions}

%\subsection{Plots}

%\section{Scaling Relationships of $W_{e}$ and $\Delta h$}
%\label{sec:Scaling Relationships of $W_{e}$ and $\Delta h$}

%\section{Local Mixed Layer Height Distriubutions}

%\section{Flux Quadrant Analysis}

\endinput

Any text after an \endinput is ignored.
You could put scraps here or things in progress.
