%% The following is a directive for TeXShop to indicate the main file
%%!TEX root = diss.tex

\chapter{Results}
\label{ch:results}
\setlength{\parindent}{0cm}

\section{Runs}

%description of runs ie 10 member ensembles each had delta x, delta y=25 and a region of delta z=5m enclosing the 
%Entrainment zone, z=25 below  and streched to 100 above. 

\label{sec:Runs}

\begin{table}[!ht]
    \begin{center}
    \begin{tabular}{ | l | l | l | l |}
    \hline
    $\overline{w^{'}\theta^{'}_{s}}$ / $\gamma$ & 10 (K/Km) & 5 (K/Km) & 2.5 (K/Km) \\ \hline
     150 (W/m2)& \hspace{5mm} \ding{51} &\hspace{5mm} \ding{51}\footnotemark &  \\ \hline
     100 (W/m2)& \hspace{5mm} \ding{51} & \hspace{5mm} \ding{51} & \\ \hline
     60 (W/m2) & \hspace{5mm} \ding{51} & \hspace{5mm} \ding{51} & \hspace{5mm} \ding{51}\\ \hline
        
\end{tabular}\\
\end{center}    
\end{table}
\footnotetext{Incomplete run: EL exceded high resolution vertical grid after 7 hours}

\section{Checking the Model based on Ensemble Averaged, Horizontally Averaged Profiles}
\label{sec:CheckingtheModel}
\subsection{Spin Up Time, Convective Time Scale $\tau$}
\FloatBarrier
Time must be allowed for spin-up and turbulent mixing.  \citeauthor{SullMoengStev} in \cite{SullMoengStev}
recommended 10 eddie turnover times, and \citeauthor{BrooksFowler2} in \cite{BrooksFowler2} chose a 
simulated time of 2 hours.  For all of the runs, 10 eddie turnover times were completed by 2 hours (Figure \ref{fig:ScaledTimevsTime}).\\


\begin{figure}[!h]
    \centering
    % plot_height.py [master 03d2835] Round 1 of Plots in Results 
    \includegraphics[scale=.5]{/tera/phil/nchaparr/python/Plotting/Dec252013/pngs/scaledtimevstime}
    \caption{Plots of Scaled Time vs Time for all runs}
    \label{fig:ScaledTimevsTime}   
\end{figure}

A Measureable well mixed layer and entrainment layer develops after 2 hours (Figure \ref{fig:tempgradfluxprofs15010}).\\

\begin{figure}[htbp]
    \centering
    % plot_dthetaflux.py [master 03d2835] Round 1 of Plots in Results
    \includegraphics[scale=.5]{/tera/phil/nchaparr/python/Plotting/Mar52014/pngs/theta_flux_profs}
    \caption{Potential Temperature, it's vertical gradient and Flux Profiles for the 150/10 Run}
    \label{fig:tempgradfluxprofs15010}   % label should change
\end{figure}

Fluxes and root mean squared vertical velocity (Figure \ref{fig:rmswvelprofs15010}) are scaled well by the surface flux ($\overline{w^{,}\theta^{,}}_{s}$) and convective velocity scale ($w^{*}$) value after 2 hours (Figure \ref{fig:scaledtempgradfluxprofs15010}).\\


\begin{figure}[htbp]
    \centering
    % plot_dthetaflux.py [master 9883fda] Round 2 of Plots in Results
    \includegraphics[scale=.5]{/tera/phil/nchaparr/python/Plotting/Mar52014/pngs/scaled_theta_flux_profs}
    \caption{Potential Temperature, it's scaled vertical gradient and scaled Flux Profiles for the 150/10 Run}
    \label{fig:scaledtempgradfluxprofs15010}   % label should change
\end{figure}

\begin{figure}[htbp]
    \centering
    %w_analysis.py [master 03d2835] Round 1 of Plots in Results
    \includegraphics[scale=.5]{/tera/phil/nchaparr/python/Plotting/Mar52014/pngs/rmswvels}
    \caption{Root Mean Vertical Velocity Squared Profiles for 150/10 Run}
    \label{fig:rmswvelprofs15010}   % label should change
\end{figure}

\subsection{Temperature, Heat Flux and Kinetic Energy}
\FloatBarrier
The horizontally averaged, ensemble averaged $\theta$ profiles exhibit a mixed layer of height dependent on
$\overline{w^{,}\theta^{,}}_{s}$ and $\gamma$, topped by a region where the vertical gradient exceeds zero,
passing through a maximum (h) before resuming $\gamma$ (Figure \ref{fig:pottempprofs2hrs}).\\

\begin{figure}[htbp]
    \centering
    % plot_theta_profs.py [master 9883fda] Round 2 of Plots in Results
    \includegraphics[scale=.5]{/tera/phil/nchaparr/python/Plotting/Dec252013/pngs/theta_profs2hrs}
    \caption{Potential Tempertature Profiles at 2 hours}
    \label{fig:pottempprofs2hrs}   % label should change
\end{figure}


The horizonally averaged, ensemble averaged $\overline{w^{,}\theta^{,}}$ profiles decrease from the surface value, passing through zero to a minumum and increase to zero.(Figure \ref{fig:fluxprofs2hrs})\\

\begin{figure}[htbp]
    \centering
    %plot_theta_profs.py [master 03d2835] Round 1 of Plots in Results
    \includegraphics[scale=.5]{/tera/phil/nchaparr/python/Plotting/Dec252013/pngs/flux_profs2hrs}
    \caption{$\overline{w^{,}\theta^{,}}_{s}$ Profiles at 2 hours}
    \label{fig:fluxprofs2hrs}   % label should change
\end{figure}

Each of the $\overline{\theta}$ and $\overline{w^{,}\theta^{,}}$ profiles has a region that can be defined as the Entrainment Zones.  The point of minimum $\overline{w^{,}\theta^{,}}$
is lower than h.  \citeauthor{SullMoengStev} in \cite{SullMoengStev} noted that the minima of the individual flux quadrant profiles are closer or even coincide with h.\\

Root mean velocity squared profiles show a dominance of vertical velocity in the mixed layer, with a peak in horizontal velocity within the entrainment layer where the vertical is inhibited (Figure \ref{fig:rmsvel150102hrs}). \\

Two dimensional fft power spectra of horizontal slices of vertical (\ref{fig:scalardfftw602point5}) velocity perturbations taken at three
different levels ($h_{0}$, $h$ and $h_{1}$) are collapsed to one dimension by integrating around a semi-circle (positive wave-numbers).
Isotropy in all radial directions is assumed and $k = \sqrt{k_{x}^{2} + k_{y}^{2}}$.  The resulting scalar density spectra show peaks in 
energy at the larger scales, cascading to the lower scales roughly according to a $\frac{-5}{3}$ slope lower in the Entrainment Layer.  At
the top of the entrainment layer where turbulence is supressed by stability, the slope is steeper.


\begin{figure}[htbp]
    \centering
    %plot_vars.py [master 03d2835] Round 1 of Plots in Results
    \includegraphics[scale=.5]{/tera/phil/nchaparr/python/Plotting/Mar52014/pngs/rmsvel2}
    \caption{$\frac{\sqrt[]{u^{,2}}}{w^{*}}$ Profiles at 2 hours for the 150/10 Run}
    \label{fig:rmsvel150102hrs}   % label should change
\end{figure}

\begin{figure}[htbp]
    \centering
    %fft_nchap.py [master 03d2835] Round 1 of Plots in Results
    \includegraphics[scale=.5]{/tera/phil/nchaparr/python/Plotting/Dec252013/pngs/2dfftpow}
    \caption{2D FFT Energy Densities of $w^{,}$ at $h_{0}$ vs Wavenumber for 60 / 2.5 Run}
    \label{fig:scalardfftw602point5}   % label should change
\end{figure}

\begin{figure}[htbp]
    \centering
    % fft_chap.py [master 03d2835] Round 1 of Plots in Results
    \includegraphics[scale=.5]{/tera/phil/nchaparr/python/Plotting/Dec252013/pngs/scalarfftpow}
    \caption{Scalar FFT  Energy Densities vs Wavenumber ($k = \sqrt{k_{x}^{2}+k_{y}^{2}}$) for 60 / 2.5 Run}
    \label{fig:2fftw602point5}   % label should change
\end{figure}

\subsection{Visualization of Structures Within the Entrainment Layer}

At the bottom of the Entrainment Layer ($h_{0}$) in the 150/10 run (Figure \ref{fig:conts15010run} (a) and (d)) coherent
areas of positive and negative temperature perturbations correspond roughly to areas of upward and downward 
moving air.  The individual plumes of cool (relative to the surrounding) air are more evident at the inversion($h$)
((b) and (e)) and their locations correspond to areas of upward (relative) motion.  Most of the upward moving
 cool areas are adjacent to (in some cases encircled by) smaller areas of downward moving warm air. At $h_{1}$ ((c) and (f)) peaks of cool
air are associated with up and downwelling.\\  

In the 60/2.5 run (Figure \ref{fig:conts15010run}) a similar progression is evident but the impinging (cool upward moving)
plumes are more defined.\\   

\begin{figure}[htbp]

\begin{minipage}[b]{0.5\linewidth}
  
        %Flux_Quads.py [master 03d2835] Round 1 of Plots in Results
        \subfloat[]{\label{main:a}
                \includegraphics[scale=.36]{/tera/phil/nchaparr/python/Plotting/Mar52014/pngs/theta_cont0}}\\
        \subfloat[]{\label{main:b}      
                \includegraphics[scale=.36]{/tera/phil/nchaparr/python/Plotting/Mar52014/pngs/theta_cont1}}\\ 
        \subfloat[]{\label{main:c}      
                \includegraphics[scale=.36]{/tera/phil/nchaparr/python/Plotting/Mar52014/pngs/theta_cont2}} 
 \end{minipage}             
\quad
\begin{minipage}[b]{0.5\linewidth}
        %Flux_Quads.py [master 9883fda] Round 2 of Plots in Results
        \subfloat[]{\label{main:d}
                \includegraphics[scale=.36]{/tera/phil/nchaparr/python/Plotting/Mar52014/pngs/wvel_cont0}}\\
       
       \subfloat[]{\label{main:e}
                \includegraphics[scale=.36]{/tera/phil/nchaparr/python/Plotting/Mar52014/pngs/wvel_cont1}}\\
        
       \subfloat[]{\label{main:f}
                \includegraphics[scale=.36]{/tera/phil/nchaparr/python/Plotting/Mar52014/pngs/wvel_cont2}}                 
\end{minipage}
        \caption{$\theta^{'}$ (left) and $w^{'}$ (right) at 2 hours at $h_{0}$ (a,d), $h$ (c,e) and $h_{1}$ (d,f)}
        \label{fig:conts15010run}
\end{figure}

\begin{figure}[htbp]

\begin{minipage}[b]{0.5\linewidth}
  
        
        \subfloat[]{\label{main:a}
                \includegraphics[scale=.36]{/tera/phil/nchaparr/python/Plotting/Dec252013/pngs/theta_cont0}}\\
        \subfloat[]{\label{main:b}      
                \includegraphics[scale=.36]{/tera/phil/nchaparr/python/Plotting/Dec252013/pngs/theta_cont1}}\\ 
        \subfloat[]{\label{main:c}      
                \includegraphics[scale=.36]{/tera/phil/nchaparr/python/Plotting/Dec252013/pngs/theta_cont2}} 
 \end{minipage}             
\quad
\begin{minipage}[b]{0.5\linewidth}
        \subfloat[]{\label{main:d}
                \includegraphics[scale=.36]{/tera/phil/nchaparr/python/Plotting/Dec252013/pngs/wvel_cont0}}\\
       
       \subfloat[]{\label{main:e}
                \includegraphics[scale=.36]{/tera/phil/nchaparr/python/Plotting/Dec252013/pngs/wvel_cont1}}\\
        
       \subfloat[]{\label{main:f}
                \includegraphics[scale=.36]{/tera/phil/nchaparr/python/Plotting/Dec252013/pngs/wvel_cont2}}                 
\end{minipage}
        \caption{$\theta^{'}$ (left) and $w^{'}$ (right) at 2 hours at $h_{0}$ (a,d), $h$(b,e) and $h_{1}$(c,f)}
        \label{fig:conts602point5run}
\end{figure}


\section{h and  $\Delta h$ based on Average Profiles}
\label{sec:hdeltahavprofs}     
\FloatBarrier
\begin{figure}[htbp]
    \centering
    %plot_height.py[master 1573b9d] h vs time plot
    \includegraphics[scale=.5]{/tera/phil/nchaparr/python/Plotting/Dec252013/pngs/hvstime}
    \caption{}
    \label{fig:hvstime}   % label should change
\end{figure}


\begin{figure}[htbp]
    \centering
    %plot_height.py [master 01d5f21] he ves scaled time plot
    \includegraphics[scale=.5]{/tera/phil/nchaparr/python/Plotting/Dec252013/pngs/hvsscaledtime}
    \caption{}
    \label{fig:hvsscaledtime}   % label should change
\end{figure}

\begin{figure}[htbp]
    \centering
    %plot_height.py [master 9cd73aa] delta h vs time  
    \includegraphics[scale=.5]{/tera/phil/nchaparr/python/Plotting/Dec252013/pngs/deltahstime}
    \caption{}
    \label{fig:deltahvstime}   % label should change
\end{figure}

\begin{figure}[htbp]
    \centering
    %plot_height.py [master fd5c6b1] delta h vs time1  
    \includegraphics[scale=.5]{/tera/phil/nchaparr/python/Plotting/Dec252013/pngs/deltahstime1}
    \caption{}
    \label{fig:deltahvstime1}   % label should change
\end{figure}

\begin{figure}[htbp]
    \centering
    %plot_height.py [master 00908a6] sacleddeltastime 
    \includegraphics[scale=.5]{/tera/phil/nchaparr/python/Plotting/Dec252013/pngs/scaleddeltahstime}
    \caption{}
    \label{fig:scaleddeltahstime}   % label should change
\end{figure}

\begin{figure}[htbp]

\begin{minipage}[b]{0.5\linewidth}
         
        \subfloat[]{\label{main:a}
                \includegraphics[scale=.36]{/tera/phil/nchaparr/python/Plotting/Dec252013/pngs/invristime}}\\
        \end{minipage}             
\quad
\begin{minipage}[b]{0.5\linewidth}
        \subfloat[]{\label{main:d}
          %plot_height [master fbd2dfd] invristime1
                \includegraphics[scale=.36]{/tera/phil/nchaparr/python/Plotting/Dec252013/pngs/invristime1}}\\
       
       \end{minipage}
        \caption{}
        \label{fig:invristime}
\end{figure}


\begin{figure}[htbp]
\begin{minipage}[b]{0.5\linewidth}
        %plot_height [master fbd2dfd] invristime1
        \subfloat[]{\label{main:a}
                \includegraphics[scale=.36]{/tera/phil/nchaparr/python/Plotting/Dec252013/pngs/scaledweinvri}}\\
        \end{minipage}             
\quad
\begin{minipage}[b]{0.5\linewidth}
        \subfloat[]{\label{main:d}          
          %plot_height [master 01c3721] scaledweinvri1
                \includegraphics[scale=.36]{/tera/phil/nchaparr/python/Plotting/Dec252013/pngs/scaledweinvri1}}\\
       
       \end{minipage}
        \caption{}
        \label{fig:scaledweinvri}
\end{figure}

\begin{figure}[htbp]
\begin{minipage}[b]{0.5\linewidth}
        %plot_height [master c7af4de] scaleddeltahinvri
        \subfloat[]{\label{main:a}
                \includegraphics[scale=.36]{/tera/phil/nchaparr/python/Plotting/Dec252013/pngs/scaleddeltahinvri}}\\
        \end{minipage}             
\quad
\begin{minipage}[b]{0.5\linewidth}
        \subfloat[]{\label{main:d}          
          %plot_height [master b9c30ad] scaleddeltahinvri1
                \includegraphics[scale=.36]{/tera/phil/nchaparr/python/Plotting/Dec252013/pngs/scaleddeltahinvri1}}\\
       
       \end{minipage}
        \caption{}
        \label{fig:scaledweinvri}
\end{figure}


%\subsection{$\theta$ and Flux}

%Profiles of horizonatally averaged ensemble averaged $\theta$.\\

%We expect to see formation of a measureable mixed layer with uniform temperature topped by an entrainment region.\\

%Get $\theta_{t+1} -  \theta_{t}$ profiles to predict flux profiles: $\frac{d\theta}{dz} = \overline{w^{'}\theta^{'}}$ so points at where sucessive $\theta$ profiles intersect should more or less correspond to points of zero flux crossing.  Points at which $\theta$s decrease most from one time to the next, should correspond to negative peak in fluxes.\\

%Compare Flux profiles to predicted flux profiles to $\theta$ profiles.\\

%Some type of quadrant analysis : plots of horizontally averaged ensemble averaged upwarm, downwarm, upcold, downcold alongside average flux.\\

%\subsection{2d FFTs}

%\subsection{Root Mean Squared Velocity Profiles}

%\section{h, and EL Limits based on Average Profiles}

%\subsection{Definitions}

%\subsection{Plots}

%\section{Scaling Relationships of $W_{e}$ and $\Delta h$}
%\label{sec:Scaling Relationships of $W_{e}$ and $\Delta h$}

%\section{Local Mixed Layer Height Distriubutions}

%\section{Flux Quadrant Analysis}

\endinput

Any text after an \endinput is ignored.
You could put scraps here or things in progress.
