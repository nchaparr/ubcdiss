%% The following is a directive for TeXShop to indicate the main file
%%!TEX root = diss.tex

\chapter{Results}
\label{ch:results}
\setlength{\parindent}{0cm}

\section{Description of Runs}
\FloatBarrier

All 10 member cases of the ensemble were carried out on a 3.2 x 4.8 Km horizontal 
domain ($\Delta x = \Delta y = 25m$, $nx=128$, $ny=192$).  
$nx$, $ny$ were chosen based on the optimal distribution across processor nodes.  
The vertical grid ($nz=312$) was of higher resolution around the 
entrainment layer (\acs{EL}) ($\Delta z = 5m$), and lower below and above it 
($\Delta z = 10 \ to \ 100 m$). Grid size was chosen so that
a full spectrum of turbulence would be resolved within the \acs{EL} 
in line with the findings of \citeauthor{SullPat} (\citeyear{SullPat}).  The 7 runs
vary depending on surface heat flux ($\overline{w^{'}\theta^{'}_{s}}$) 
and initial lapse rate ($\gamma$).

%description of runs i.e. 10 member ensembles each had delta x, delta y=25 and a region of delta z=5m enclosing the 
%Entrainment zone, z=25 below  and stretched to 100 above. 

\label{sec:Runs}

\begin{table}[!ht]
    \begin{center}
    \begin{tabular}{ | l | l | l | l |}
    \hline
    $\overline{w^{'}\theta^{'}_{s}}$ / $\gamma$ & 10 (K/Km) & 5 (K/Km) & 2.5 (K/Km) \\ \hline
     150 (W/m2)& \hspace{5mm} \ding{51} &\hspace{5mm} \ding{51}\footnotemark &  \\ \hline
     100 (W/m2)& \hspace{5mm} \ding{51} & \hspace{5mm} \ding{51} & \\ \hline
     60 (W/m2) & \hspace{5mm} \ding{51} & \hspace{5mm} \ding{51} & \hspace{5mm} \ding{51}\\ \hline
     
%\end

   
\end{tabular}
\caption{Runs in terms of $\overline{w^{'} \theta^{'}_{s}}$ and initial lapse rate $\gamma$}
\label{fig:tableofruns}   
\end{center}    
\end{table}
\footnotetext{Incomplete run: EL exceeded high resolution vertical grid after 7 hours}

\clearpage

\section{Relevant Definitions}
\FloatBarrier

Here, the \acs{CBL} height and \acs{EL} limits are defined based on the vertical  $\frac{\partial \overline{\theta}}{\partial z}$
profile.  Namely, the \acs{CBL} height $h$ is the point where  $\frac{\partial \overline{\theta}}{\partial z}$ is maximum, the 
lower \acs{EL} limit is the point at which  $\frac{\partial \overline{\theta}}{\partial z}$ first increases
significantly from zero, i.e. exceeds a threshold value, above the surface layer, and the upper \acs{EL} limit $h_{1}$ is the point where
$\frac{\partial \overline{\theta}}{\partial z}$ resumes $\gamma$ as shown in Figure \ref{fig:hdefs}.\\

As \citeauthor{BrooksFowler2} (\citeyear{BrooksFowler2}) point out, when using an average vertical 
tracer profile there is no universal criterion for a significant gradient.  So a threshold value for the
lower \acs{EL} limit ($h_{0}$) was chosen such that it was positive, small i.e. an order of magnitude 
less than $\gamma$ and the same for all runs.  For the sake of rigor, the main corresponding result 
was calculated based on two additional threshold values in Section \ref{subsec:deltahri}.\\

The temperature jump is defined here in two ways.  $\Delta \theta$ is he difference in $\overline{\theta}$ across the \acs{EL} and is larger than that used by  \citeauthor{FedConzMir04} in \citeyear{FedConzMir04} to verify their zero order bulk model. So, for the sake of comparison and to investigate the resulting effects on $a$ in Equation \ref{eq:ervsri} we also define a zero-order type temperature jump, $\delta \theta$ , as the difference between the mixed layer (\acs{ML}) average potential temperature $\overline{\theta}_{ML}$ less ($\overline{\theta}$) at $h$ on the initial profile as in Figure \ref{fig:hdefs}.  Analogous definitions are based on the vertical average heat flux profile for comparison with the results of \citeauthor{FedConzMir04} (\citeyear{FedConzMir04}) and \citeauthor{BrooksFowler2} (\citeyear{BrooksFowler2}). 
\\ 

As will be shown in Section \ref{subsec:deltahri} the relationship of scaled \acs{EL}  depth to \acs{Ri} varies depending on whether the heights outlined above are based on the $\frac{\partial \overline{\theta}}{\partial z}$ profile or its scaled version, $\frac{\frac{\partial \overline{\theta}}{\partial z}}{\gamma}$.\\  
 
%table outlining my definitions, and other comparable ones, ie brooks and fowlers aren't comparable
% re justifying my arbitrary choice of threshold for the lower limit:  seee page 250 Brooks and Fowler. 
%there must be other places in my key papers where the 'arbitraryness' is referred to.

\clearpage

\section{Verifying the Model Output}
\label{sec:CheckingtheModel}
\subsection{Time till well-mixed}%Spin Up Time according to the Convective Time Scale $\tau$}
\FloatBarrier

Time must be allowed to establish statistically steady turbulent flow.  \citeauthor{SullMoengStev} 
(\citeyear{SullMoengStev}) recommended 10 eddy turnover times based on the convective time scale 
$\tau = \frac{h}{w^{*}} = \frac{h}{ \left( \frac{gh}{\overline{\theta}_{ML}}(\overline{w^{'} \theta^{'}_{s}}) \right)^{\frac{1}{3}} } $, 
and \citeauthor{BrooksFowler2} (\citeyear{BrooksFowler2}) chose a simulated time of 2 hours.  For all of 
the runs, at least 10 eddy turnover times were completed by 2 simulated hours (Figure \ref{fig:ScaledTimevsTime}).  
Although each run has a distinct convective velocity scale, $w^{*}$, that increases with time, 
dividing boundary layer height, $h$, by it to obtain $\tau$ results in a collapse from 7 to 3 curves, 
one for each $\gamma$.\\

Figure \ref{fig:tempgradfluxprofs1005} shows a measurable well mixed layer (\acs{ML}) and \acs{EL} based on the horizontally averaged, ensemble averaged potential temperature ($\overline{\theta}$) profile, by 2 hours.  After 2 or 3 hours the \acs{EL} is fully contained within the vertical region of high resolution.\\

Figures \ref{fig:tempgradfluxprofs1005} and \ref{fig:rms} (a) show that the averaged heat fluxes ($\overline{w^{'}\theta^{'}}$) and root mean squared vertical velocity perturbations ($\sqrt{w^{'2}}$) become self similar and are scaled well by the surface heat flux $\overline{w^{'}\theta^{'}}_{s}$ and $w^{*}$ respectively after 2 hours.  In Figure \ref{fig:rms} (b) vertical $\sqrt{u^{'2}}$ profiles become self similar later in the run and exhibit a peak near the \acs{CBL} top, where vertical motions become inhibited by $\gamma$. \\

\begin{figure}[!h]
    \centering
    % plot_height.py [master 03d2835] Round 1 of Plots in Results 
    \includegraphics[scale=.5]{/newtera/tera/phil/nchaparr/python/Plotting/Dec252013/pngs/scaledtimevstime}
    \caption{Plots of scaled time vs time for all runs.  Scaled time is based on the convective time scale 
    and can be thought of as the number of times an eddy has reached the top of the CBL.}
    \label{fig:ScaledTimevsTime}   
\end{figure}

\begin{figure}[htbp]
    \centering
    % plot_dthetaflux.py [master 03d2835] Round 1 of Plots in Results
    \includegraphics[scale=.5]{/newtera/tera/phil/nchaparr/python/Plotting/Nov302013/pngs/theta_flux_profs}
    \caption{Vertical profiles of the ensemble and horizontally averaged potential temperature ($\overline{\theta}$), its vertical gradient ($\frac{\partial \overline{\theta}}{\partial z}$)  
     and heat flux ($\overline{w^{'}\theta^{'}}$) for the 100/5 run}
    \label{fig:tempgradfluxprofs1005}   % label should change
\end{figure}

\begin{figure}[htbp]
    \centering
    % plot_dthetaflux.py [master 9883fda] Round 2 of Plots in Results
    %\includegraphics[scale=.5]{/newtera/tera/phil/nchaparr/python/Plotting/Mar52014/pngs/scaled_theta_flux_profs}
    %
    \includegraphics[scale=.5]{/newtera/tera/phil/nchaparr/python/Plotting/Nov302013/pngs/scaled_flux_profs}
    \caption{$\overline{w^{'}\theta^{'}}$ and scaled $\overline{w^{'}\theta^{'}}$  vs scaled height for the 100/5 run}
    \label{fig:scaledfluxprofs15010}   % label should change
\end{figure}

\begin{figure}[htbp]
%Pcolor_Peaks.py [master 61491be] rss_fit plots
\begin{minipage}[b]{0.5\linewidth}
        %
        \subfloat[]{\label{main:a}
                \includegraphics[scale=.36]{/newtera/tera/phil/nchaparr/python/Plotting/Mar52014/pngs/rmswvels}}\\
        \end{minipage}             
\quad
\begin{minipage}[b]{0.5\linewidth}
        \subfloat[]{\label{main:b}          
          
                \includegraphics[scale=.36]{/newtera/tera/phil/nchaparr/python/Plotting/Mar52014/pngs/rmsuvels}}\\
       
       \end{minipage}
        \caption{(a)$\sqrt{w^{'2}}$ and scaled $\sqrt{w^{'2}}$ vs scaled height for the 150/10 run, (b) $\sqrt{u^{'2}}$ and scaled $\sqrt{u^{'2}}$ vs scaled height for the 150/10 run}
        \label{fig:rms}
\end{figure}

\clearpage

\subsection{FFT Energy Spectra}
\FloatBarrier

In Figure \ref{fig:scalarfftw602point5}, two dimensional \acs{FFT} power spectra taken of horizontal slices of $w^{\prime}$ at three different levels ($h_{0}$, $h$ and $h_{1}$) are collapsed to one dimension by integrating around a semi-circle of positive wave-numbers.
Isotropy in all radial directions is assumed and $k = \sqrt{k_{x}^{2} + k_{y}^{2}}$.  The resulting scalar density spectra show peaks in energy at the larger scales, cascading to the lower scales roughly according to a $\frac{-5}{3}$ slope lower in the \acs{EL}.  At
the top of the \acs{EL} where turbulence is suppressed by stability, the slope is steeper.  The peak in energy occurs at smaller scales
at $h$ as compared to at $h_{0}$, indicating a change in the size of the dominant turbulent structures. The spectra for the horizontal
turbulent velocity perturbations were analogous but show lower energy as expected.\\

\begin{figure}[htbp]
    \centering
    % fft_chap.py [master 03d2835] Round 1 of Plots in Results
    \includegraphics[scale=.5]{/newtera/tera/phil/nchaparr/python/Plotting/Dec252013/pngs/scalarfftpow_w}
    \caption{Scalar FFT  energy vs wavenumber ($k = \sqrt{k_{x}^{2}+k_{y}^{2}}$) for the 60/2.5 run
at 2 hours.  $E(k)$ is $E(k_{x}, k_{y})$ integrated around circles of radius $k$.  
   $E(k_{x}, k_{y})$ is the total integrated energy over the 2D domain.  
   $k_{x}$ and $k_{y}$ are number of waves per domain length.}
   \label{fig:scalarfftw602point5}   % label should change
\end{figure}

\clearpage

\subsection{Ensemble and horizontally averaged vertical Potential Temperature $\overline{\theta}$ 
and Heat Flux $\overline{w^{'}\theta^{'}}$ profiles}
%Average Potential Temperature, Heat Flux and Kinetic Energy}
\FloatBarrier

The $\overline{\theta}$ profiles exhibit an \acs{ML} above which  $\frac{\partial\overline{\theta}}{\partial z}>0$ 
and reaches a maximum value at $h$ before resuming $\gamma$  at $h_{1}$ (Figures \ref{fig:tempgradfluxprofs1005} and \ref{fig:pottempprofs2hrs}).  Convective boundary layer \acs{CBL} growth is stimulated by $\overline{w^{'}\theta^{'}}_{s}$ and inhibited by $\gamma$.\\

The horizontally averaged, ensemble averaged heat flux ($\overline{w^{'}\theta^{'}}$) profiles decrease from the surface value ($\overline{w^{'}\theta^{'}_{s}}$) passing through zero to a minimum before increasing to zero (Figures \ref{fig:tempgradfluxprofs1005} and  \ref{fig:fluxprofs2hrs}).  All minima are less  in magnitude than the zero order approximation ($-.2 \times \overline{w^{'}\theta^{'}_{s}}$) but seem to increase with increased $\gamma$.\\


\begin{figure}[htbp]
    \centering
    % plot_theta_profs.py [master 9883fda] Round 2 of Plots in Results
    \includegraphics[scale=.5]{/newtera/tera/phil/nchaparr/python/Plotting/Dec252013/pngs/theta_profs2hrs}
    \caption{$\overline{\theta}$ profiles at 2 hours}
    \label{fig:pottempprofs2hrs}   % label should change
\end{figure}

\begin{figure}[htbp]
    \centering
    %plot_theta_profs.py [master 03d2835] Round 1 of Plots in Results
    \includegraphics[scale=.5]{/newtera/tera/phil/nchaparr/python/Plotting/Dec252013/pngs/flux_profs2hrs}
    \caption{Scaled $\overline{w^{'}\theta^{'}}_{s}$ profiles at 2 hours}
    \label{fig:fluxprofs2hrs}   % label should change
\end{figure}

\clearpage

\subsection{Visualization of Structures Within the Entrainment Layer}
\FloatBarrier

Horizontal slices, at $h_{0}$, $h$ and $h_{1}$ as showing in Figure \ref{fig:hdefs} of the potential temperature 
and vertical velocity perturbations are plotted to see the turbulent structures.  At the bottom of the \acs{EL} ($h_{0}$) 
in the 150/10 run (Figure \ref{fig:conts} (a) and (d)) coherent areas of positive and negative temperature perturbations 
correspond to areas of upward and downward moving air.\\  

In Figure \ref{fig:conts} (b) and (e) the individual plumes of relatively cool air are more evident at the inversion ($h$) and their 
locations correspond to areas of upward motion.  Most of the upward moving cool areas are adjacent to and even 
encircled by smaller areas of downward moving warm air.  At $h_{1}$ ((c) and (f)) peaks of cool air are associated 
with both up and down-welling.\\  

In the 60/2.5 run (Figure \ref{fig:conts1}) a similar progression is evident. %but the impinging, cool upward moving plumes are more defined.  This is to be expected since a weaker $\gamma$ allows more deformation of the inversion interface. Not True! Range of temp perturbations alot wider, ie more negative in \\   

\begin{figure}[htbp]
\caption{$\theta^{'}$ (left) and $w^{'}$ (right) at 2 hours at $h_{0}$ (a,d), $h$ (c,e) and $h_{1}$ (d,f) for the 150/10 run.} 
\begin{minipage}[b]{0.5\linewidth}
  
        %Flux_Quads.py [master 03d2835] Round 1 of Plots in Results
        \subfloat[]{\label{main:a}
                \includegraphics[scale=.36]{/newtera/tera/phil/nchaparr/python/Plotting/Mar52014/pngs/theta_cont0}}\\
        \subfloat[]{\label{main:b}      
                \includegraphics[scale=.36]{/newtera/tera/phil/nchaparr/python/Plotting/Mar52014/pngs/theta_cont1}}\\ 
        \subfloat[]{\label{main:c}      
                \includegraphics[scale=.36]{/newtera/tera/phil/nchaparr/python/Plotting/Mar52014/pngs/theta_cont2}} 
 \end{minipage}             
\quad
\begin{minipage}[b]{0.5\linewidth}
        %Flux_Quads.py [master 9883fda] Round 2 of Plots in Results
        \subfloat[]{\label{main:d}
                \includegraphics[scale=.36]{/newtera/tera/phil/nchaparr/python/Plotting/Mar52014/pngs/wvel_cont0}}\\
       
       \subfloat[]{\label{main:e}
                \includegraphics[scale=.36]{/newtera/tera/phil/nchaparr/python/Plotting/Mar52014/pngs/wvel_cont1}}\\
        
       \subfloat[]{\label{main:f}
                \includegraphics[scale=.36]{/newtera/tera/phil/nchaparr/python/Plotting/Mar52014/pngs/wvel_cont2}}                 
\end{minipage}
        
        \label{fig:conts}
\end{figure}

\begin{figure}[htbp]
\caption{$\theta^{'}$ (left) and $w^{'}$ (right) at 2 hours at $h_{0}$ (a,d), $h$(b,e) and $h_{1}$(c,f) for the 60/2.5 run.}
\begin{minipage}[b]{0.5\linewidth} 
        
        \subfloat[]{\label{main:a}
                \includegraphics[scale=.36]{/newtera/tera/phil/nchaparr/python/Plotting/Dec252013/pngs/theta_cont0}}\\
        \subfloat[]{\label{main:b}      
                \includegraphics[scale=.36]{/newtera/tera/phil/nchaparr/python/Plotting/Dec252013/pngs/theta_cont1}}\\ 
        \subfloat[]{\label{main:c}      
                \includegraphics[scale=.36]{/newtera/tera/phil/nchaparr/python/Plotting/Dec252013/pngs/theta_cont2}} 
 \end{minipage}             
\quad
\begin{minipage}[b]{0.5\linewidth}
        \subfloat[]{\label{main:d}
                \includegraphics[scale=.36]{/newtera/tera/phil/nchaparr/python/Plotting/Dec252013/pngs/wvel_cont0}}\\
       
       \subfloat[]{\label{main:e}
                \includegraphics[scale=.36]{/newtera/tera/phil/nchaparr/python/Plotting/Dec252013/pngs/wvel_cont1}}\\
        
       \subfloat[]{\label{main:f}
                \includegraphics[scale=.36]{/newtera/tera/phil/nchaparr/python/Plotting/Dec252013/pngs/wvel_cont2}}                 
\end{minipage}
        
        \label{fig:conts1}
\end{figure}

\clearpage

\section{Local Mixed Layer Heights ($h_{0}^{l}$)}
\label{sec:locmlh}     
\FloatBarrier

Local $\theta$ profiles (Figures \ref{fig:rssfitshigh} and \ref{fig:rssfitslow}) exhibit a distinct 
\acs{ML} before resuming $\gamma$ but 
not always a clearly defined \acs{EL}.  There are sharp changes in the profile well into the free 
atmosphere, due possibly to waves, which render the gradient method for determining $h^{l}$ 
unusable.  Instead a linear regression method is used, whereby three lines representing: the
 \acs{ML}, the \acs{EL} and the upper lapse rate ($\gamma$), are fit to the profile according 
to the minimum residual sum of squares (RSS).  Determining local \acs{ML} height ($h_{0}^{l}$) was 
more straight forward than the local height of maximum potential temperature gradient 
($h^{l}$) for the reasons stated above.\\  

Figure \ref{fig:rssfitshigh} shows two local $\theta$ profiles where $h_{0}^{l}$ is relatively high.  
A sharp interface is evident indicating that this is within an active plume impinging on the stable layer.
In Figure \ref{fig:rssfitslow} where $h_{0}^{l}$ is relatively low a less defined interface indicates 
a point now outside a rising plume.  When $\gamma$ is lower as in Figure \ref{fig:rssfitslow} (a), these inactive locations show a larger vertical region that could be called a local \acs{EL}.  The Contour plots in Figure \ref{fig:conts2} show regions of high 
$h_{0}^{l}$ corresponding to regions of upward moving relatively cool air at $h$.\\

The distribution of $h_{0}^{l}$ is related to the depth of the entrainment layer (\acs{EL}).
Spread increases with increasing $\overline{w^{'}\theta^{'}_{s}}$ and decreases with increasing $\gamma$
(Figure \ref{fig:localhhist}).  When scaled by $h$  (Figure \ref{fig:localhpdf}), the local \acs{ML} height distribution 
has spread that narrows with increased $\gamma$ and seems relatively uninfluenced by change in $\overline{w^{'}\theta^{'}}_{s}$.  
The upper limit seems to be constant at about 1.1($\times h$) , whereas the lower limit varies 
depending on $\gamma$.   Runs with lower $h$ and narrower $\Delta h$ have relatively 
larger spacing between bins and so higher numbers in each bin.  The above supports the results outlined in
Section \ref{subsec:deltahri}.\\

\begin{figure}[htbp]
%Pcolor_Peaks.py [master 61491be] rss_fit plots
\begin{minipage}[b]{0.5\linewidth}
        %
        \subfloat[]{\label{main:a}
                \includegraphics[scale=.36]{/newtera/tera/phil/nchaparr/python/Plotting/Dec252013/pngs/rss_fit_high}}\\
        \end{minipage}             
\quad
\begin{minipage}[b]{0.5\linewidth}
        \subfloat[]{\label{main:b}          
          
                \includegraphics[scale=.36]{/newtera/tera/phil/nchaparr/python/Plotting/Mar52014/pngs/rss_fit_high}}\\
       
       \end{minipage}
        \caption{Local vertical $\theta$ profiles with 3-line fit for the 60/2.5 (a) and 150/10 (b) runs at 
points where $h^{l}_{0}$ is high.}
        \label{fig:rssfitshigh}
\end{figure}

\begin{figure}[htbp]
%Pcolor_Peaks.py [master 61491be] rss_fit plots
\begin{minipage}[b]{0.5\linewidth}
        %
        \subfloat[]{\label{main:a}
                \includegraphics[scale=.36]{/newtera/tera/phil/nchaparr/python/Plotting/Dec252013/pngs/rss_fit_low}}\\
        \end{minipage}             
\quad
\begin{minipage}[b]{0.5\linewidth}
        \subfloat[]{\label{main:b}          
          
                \includegraphics[scale=.36]{/newtera/tera/phil/nchaparr/python/Plotting/Mar52014/pngs/rss_fit_low}}\\
       
       \end{minipage}
        \caption{Local vertical $\theta$ profiles with 3-line fit for the 60/2.5 (a) and 150/10 (b) runs at 
points where $h^{l}_{0}$ is low.}
        \label{fig:rssfitslow}
\end{figure}

\begin{figure}[htbp]
\caption{$\theta^{'}$ (a,d), $w^{'}$(b,e) at $h_{1}$(c,f) and local ML height $h^{l}_{0}$ at 2 hours for 60/2.5 (left) and 150/10 (right) runs}
\begin{minipage}[b]{0.5\linewidth} 
        
        \subfloat[]{\label{main:a}
                \includegraphics[scale=.36]{/newtera/tera/phil/nchaparr/python/Plotting/Dec252013/pngs/theta_cont1}}\\
        \subfloat[]{\label{main:b}      
                \includegraphics[scale=.36]{/newtera/tera/phil/nchaparr/python/Plotting/Dec252013/pngs/wvel_cont1}}\\ 
        \subfloat[]{\label{main:c}
          %Get_ML_Heights.py [master b7c3e5b] h_cont
                \includegraphics[scale=.36]{/newtera/tera/phil/nchaparr/python/Plotting/Dec252013/pngs/h_cont}} 
 \end{minipage}             
\quad
\begin{minipage}[b]{0.5\linewidth}
        \subfloat[]{\label{main:d}
                \includegraphics[scale=.36]{/newtera/tera/phil/nchaparr/python/Plotting/Mar52014/pngs/theta_cont1}}\\
       
       \subfloat[]{\label{main:e}
                \includegraphics[scale=.36]{/newtera/tera/phil/nchaparr/python/Plotting/Mar52014/pngs/wvel_cont1}}\\
        
       \subfloat[]{\label{main:f}
                \includegraphics[scale=.36]{/newtera/tera/phil/nchaparr/python/Plotting/Mar52014/pngs/h_cont}}                 
\end{minipage}
        
        \label{fig:conts2}
\end{figure}

%could put these sideways.  could do with checking the 2.5/60 low hs

%could put these sideways.  could do with checking the 2.5/60 low hs

\begin{figure}[htbp]
\caption{Histograms of $h^{l}_{0}$ for $\overline{w^{'}\theta^{'}_{s}} = 150$ to $60 (W / m^{2})$ (a to c) and $ \gamma = 10$ to $2.5 (K/Km)$ (c to g) at 5 hours}
%ML_Height_hist.py(sp?) master f992942ade
\begin{minipage}[b]{0.32\linewidth} 
        
        \subfloat[]{\label{main:a}
                \includegraphics[scale=.22]{/newtera/tera/phil/nchaparr/python/Plotting/Mar52014/pngs/ML_Height_hist}}\\
        \subfloat[]{\label{main:b}      
                \includegraphics[scale=.22]{/newtera/tera/phil/nchaparr/python/Plotting/Dec142013/pngs/ML_Height_hist}}\\ 
        \subfloat[]{\label{main:c}          
                \includegraphics[scale=.22]{/newtera/tera/phil/nchaparr/python/Plotting/Mar12014/pngs/ML_Height_hist}} 
 \end{minipage}             
%\quad
\begin{minipage}[b]{0.32\linewidth}
        \subfloat[]{\label{main:d}
                \includegraphics[scale=.22]{/newtera/tera/phil/nchaparr/python/Plotting/Jan152014_1/pngs/ML_Height_hist}}\\
       \subfloat[]{\label{main:e}
                \includegraphics[scale=.22]{/newtera/tera/phil/nchaparr/python/Plotting/Nov302013/pngs/ML_Height_hist}}\\
       \subfloat[]{\label{main:f}
                \includegraphics[scale=.22]{/newtera/tera/phil/nchaparr/python/Plotting/Dec202013/pngs/ML_Height_hist}}                 
\end{minipage}
\begin{minipage}[b]{0.32\linewidth}
        %\subfloat[]{\label{main:d}
        %        \includegraphics[scale=.18]{/newtera/tera/phil/nchaparr/python/Plotting/Mar52014/pngs/ML_Height_hist}}\\
       
       %\subfloat[]{\label{main:e}
       %         \includegraphics[scale=.18]{/newtera/tera/phil/nchaparr/python/Plotting/Mar52014/pngs/ML_Height_hist}}\\
       \vspace{10mm} 
       \subfloat[]{\label{main:f}
                \includegraphics[scale=.22]{/newtera/tera/phil/nchaparr/python/Plotting/Dec252013/pngs/ML_Height_hist}}                 
\end{minipage}

        
        \label{fig:localhhist}
\end{figure}

\begin{figure}[htbp]
\caption{PDFs of $\frac{h^{l}_{0}}{h}$ for $\overline{w^{'}\theta^{'}_{s}} = 150$ to $60 (W / m^{2})$ (a to c) and $ \gamma = 10$ to $2.5 (K/Km)$ (c to g) at 5 hours}
%ML_Height_hist.py(sp?) master f992942ade
\begin{minipage}[b]{0.32\linewidth} 
        
        \subfloat[]{\label{main:a}
                \includegraphics[scale=.22]{/newtera/tera/phil/nchaparr/python/Plotting/Mar52014/pngs/Scaled_ML_Height_hist}}\\
        \subfloat[]{\label{main:b}      
                \includegraphics[scale=.22]{/newtera/tera/phil/nchaparr/python/Plotting/Dec142013/pngs/Scaled_ML_Height_hist}}\\ 
        \subfloat[]{\label{main:c}          
                \includegraphics[scale=.22]{/newtera/tera/phil/nchaparr/python/Plotting/Mar12014/pngs/Scaled_ML_Height_hist}} 
 \end{minipage}             
%\quad
\begin{minipage}[b]{0.32\linewidth}
        \subfloat[]{\label{main:d}
                \includegraphics[scale=.22]{/newtera/tera/phil/nchaparr/python/Plotting/Jan152014_1/pngs/Scaled_ML_Height_hist}}\\
       \subfloat[]{\label{main:e}
                \includegraphics[scale=.22]{/newtera/tera/phil/nchaparr/python/Plotting/Nov302013/pngs/Scaled_ML_Height_hist}}\\
       \subfloat[]{\label{main:f}
                \includegraphics[scale=.22]{/newtera/tera/phil/nchaparr/python/Plotting/Dec202013/pngs/Scaled_ML_Height_hist}}                 
\end{minipage}
\begin{minipage}[b]{0.32\linewidth}
        %\subfloat[]{\label{main:d}
        %        \includegraphics[scale=.18]{/newtera/tera/phil/nchaparr/python/Plotting/Mar52014/pngs/ML_Height_hist}}\\
       
       %\subfloat[]{\label{main:e}
       %         \includegraphics[scale=.18]{/newtera/tera/phil/nchaparr/python/Plotting/Mar52014/pngs/ML_Height_hist}}\\
       \vspace{10mm} 
       \subfloat[]{\label{main:f}
                \includegraphics[scale=.22]{/newtera/tera/phil/nchaparr/python/Plotting/Dec252013/pngs/Scaled_ML_Height_hist}}                 
\end{minipage}        
        \label{fig:localhpdf}
\end{figure}

%\begin{figure}[htbp]
 %   \centering
    %plot_height.py [master 199de9a7cf]  
  %  \includegraphics[scale=.5]{/newtera/tera/phil/nchaparr/python/Plotting/Dec252013/pngs/varvsinvri}
  %  \caption{Variance vs \acs{Ri}$^{-1}$ at 5 hours}
   % \label{fig:varsvsinvri}   % label should change
%\end{figure}

\clearpage

\section{Flux Quadrants}
\label{sec:fluxquadrants}     
\FloatBarrier

\subsection{Vertical Average Profiles}

As \citeauthor{SullMoengStev} (\citeyear{SullMoengStev}) point out and Figure \ref{fig:fluxqadprofs} shows, in  when broken out into four quadrants the $\overline{w^{'}\theta^{'}}$ profiles have upper extrema above that of the total average profile ($z_{f}$).  2D histograms of the four quadrants are plotted at $h_{0}$, $h$ 
and $h_{1}$ to see how the distributions are influenced by changes in $\overline{w^{'} \theta^{'}}$ 
and $\gamma$.\\
 
\begin{figure}[htbp]
\begin{minipage}[b]{0.5\linewidth}        %
        \subfloat[]{\label{main:a}
                \includegraphics[scale=.36]{/newtera/tera/phil/nchaparr/python/Plotting/Dec252013/pngs/fluxquadprofs}}\\
        \end{minipage}             
\quad
\begin{minipage}[b]{0.5\linewidth}
        \subfloat[]{\label{main:d}          
          %Flux_Quads.py [master 0eaf94e] fluxquadprofs
                \includegraphics[scale=.36]{/newtera/tera/phil/nchaparr/python/Plotting/Mar52014/pngs/fluxquadprofs}}\\
       \end{minipage}
        \caption{Scaled $\overline{w^{'} \theta^{'}}$ quadrant profiles at 5 hours for the 60/2.5 (a) and 150/10 (b) run}
        \label{fig:fluxqadprofs}
\end{figure}

\clearpage
\subsection{2D Histograms of $\theta^{'}$ and $w^{'}$}

In Figures \ref{fig:fluxquadsh_0}, \ref{fig:fluxquadsh1} and \ref{fig:fluxquadsh2} 2D Histograms of the four quadrants are plotted at $h_{0}$, $h$ and $h_{1}$ to visualize how the distributions are influenced by changes in $\overline{w^{,} \theta^{,}}_{s}$ and $\gamma$.  In order to isolate the effects of $\gamma$,  $w^{'}$ and $\theta^{'}$ are scaled by $w^{*}$ and $\theta^{*}$ respectively and plotted in Figures \ref{fig:scaled_fluxquadsh_0}, \ref{fig:scaled_fluxquadsh1} and \ref{fig:scaled_fluxquadsh2}.\\ 

At $h_{0}$ (Figure \ref{fig:fluxquadsh_0}) fast updraughts are relatively warm and are inhibited by increasing upper stability.  The spread in velocity increases with increasing $\overline{w^{,}\theta^{,}}_{s}$. Figure \ref{fig:scaled_fluxquadsh_0} shows that the distributions are scaled well by the convective scales, but that there is a slight damping of $w^{'}$ with increased $\gamma$.\\

At $h$ (Figure \ref{fig:fluxquadsh1}) the faster updraughts are now relatively cool and movement (both up and down) of warmer air from aloft becomes more prominent.  The spread of $w^{'}$ and $\theta^{'}$ both increase with increasing $\overline{w^{'}\theta^{'}}$ whereas that of $\theta^{'}$ increases only slightly with increased stability.  As expected, stability inhibits both upward and downward $w^{'}$. The scaled version in Figure \ref{fig:scaled_fluxquadsh1} shows a damping of the velocity perturbations and a positive shift in temperature perturbations with increased $\gamma$.\\ 

Although the quadrant of overall largest magnitude is that of
upward moving cool air ($w^{'+}\theta^{'-}$), \citeauthor{SullMoengStev}'s (\citeyear{SullMoengStev}) assertion that in the \acs{EL} the net heat flux is downward moving warm ($w^{'-}\theta^{'+}$) air because the other three quadrants cancel, is found to be approximately true.  At the top of the \acs{EL} (Figure \ref{fig:fluxquadsh2}) velocities are damped and the distributions approach symmetry apart from some slow, cool, impinging up and down-draughts as in Figure \ref{fig:conts2}. Applying the convective scales in Figure \ref{fig:scaled_fluxquads2} distinctly shows a narrowing in the spread of $w^{'}$ and a stretching in the spread of $\theta^{'}$ with increased $\gamma$.\\

%redo these with one colorbar see 
%http://stackoverflow.com/questions/13784201/matplotlib-2-subplots-1-colorbar  

\begin{figure}[htbp]
\centering
 \includegraphics[scale=.8]{/newtera/tera/phil/nchaparr/python/Plotting/Dec252013/pngs/fluxquadhists0}                 
\caption{ $\overline{w^{'}\theta^{'}}$ quadrants at $h_{0}$ for $w^{'}\theta^{'} = 150 - 60 (W/m^{2}$) (top-bottom) and $\gamma = 10 - 2.5 (K/Km)$ (left-right) at 5 hours}
\label{fig:fluxquadsh_0}
\end{figure}

\begin{figure}[htbp]
\centering
 \includegraphics[scale=.8]{/newtera/tera/phil/nchaparr/python/Plotting/Dec252013/pngs/scaled_fluxquadhists0}                 
\caption{ $\overline{w^{'}\theta^{'}}$ quadrants at $h_{0}$ for $w^{'}\theta^{'} = 150 - 60 (W/m^{2}$) (top-bottom) and $\gamma = 10 - 2.5 (K/Km)$ (left-right) at 5 hours}
\label{fig:scaled_fluxquadsh_0}
\end{figure}


\begin{figure}[htbp]
\centering
 \includegraphics[scale=.8]{/newtera/tera/phil/nchaparr/python/Plotting/Dec252013/pngs/fluxquadhists1}                 
\caption{ $\overline{w^{'}\theta^{'}}$ quadrants at $h$ for $w^{'}\theta^{'} = 150 \ - \ 60$(W/$m^{2}$) (top - bottom) and $\gamma = 10 \ - \  2.5$(K/Km) (left - right) at 5 hours}
\label{fig:fluxquadsh1}
\end{figure}

\begin{figure}[htbp]
\centering
 \includegraphics[scale=.8]{/newtera/tera/phil/nchaparr/python/Plotting/Dec252013/pngs/scaled_fluxquadhists1}                 
\caption{ $\overline{w^{'}\theta^{'}}$ quadrants at $h$ for $w^{'}\theta^{'} = 150 \ - \ 60$(W/$m^{2}$) (top - bottom) and $\gamma = 10 \ - \  2.5$(K/Km) (left - right) at 5 hours}
\label{fig:scaled_fluxquadsh1}
\end{figure}


\begin{figure}[htbp]
\caption{ $\overline{w^{'}\theta^{'}}$ quadrants at $h_{1}$ for $w^{'}\theta^{'} = 150 \ to \ 60$(W/$m^{2}$) (top to bottom) and $\gamma = 10 \ to \ 2.5$(K/Km) (left to right) at 5 hours}
\centering
 \includegraphics[scale=.8]{/newtera/tera/phil/nchaparr/python/Plotting/Dec252013/pngs/fluxquadhists2}                 
\label{fig:fluxquadsh2}
\end{figure}

\begin{figure}[htbp]
\caption{ $\overline{w^{'}\theta^{'}}$ quadrants at $h_{1}$ for $w^{'}\theta^{'} = 150 \ to \ 60$(W/$m^{2}$) (top to bottom) and $\gamma = 10 \ to \ 2.5$(K/Km) (left to right) at 5 hours}
\centering
 \includegraphics[scale=.8]{/newtera/tera/phil/nchaparr/python/Plotting/Dec252013/pngs/scaled_fluxquadhists2}                 
\label{fig:scaled_fluxquadsh2}
\end{figure}

\clearpage

\subsection{Downward Moving Warm Air at $h$}
\label{sec:downwarm}

The magnitude of the average downward moving quadrant ($\overline{w^{'-}\theta^{'+}}$) at $h$ can be taken as a measure of entrainment.  Figure \ref{fig:downwarm} shows that this increases in time as well as with increased $\overline{w^{'}\theta^{'}}_{s}$.  Grouping according to $\overline{w^{'}\theta^{'}}_{s}$ is evident and there is further collapse when this is applied as scale in Figure \ref{fig:downwarm} (b).

\begin{figure}[htbp]
\begin{minipage}[b]{0.5\linewidth}
        %plot_height [master c7af4de] scaleddeltahinvri
        \subfloat[]{\label{main:a}
                \includegraphics[scale=.36]{/newtera/tera/phil/nchaparr/python/Plotting/Dec252013/pngs/downwarm.pdf}}\\
        \end{minipage}             
\quad
\begin{minipage}[b]{0.5\linewidth}
        \subfloat[]{\label{main:d}          
          %plot_height [master b9c30ad] scaleddeltahinvri1
                \includegraphics[scale=.36]{/newtera/tera/phil/nchaparr/python/Plotting/Dec252013/pngs/scaled_downwarm.pdf}}\\
     
       \end{minipage}
        \caption{Plots of (a) the average downward moving warm air at $h$ ($\overline{w^{\prime-}\theta^{\prime+}}_{h}$) and (b) $\overline{w^{\prime-}\theta^{\prime+}}_{h}$ scaled by the average vertical turbulent heat flux $\overline{w^{\prime}\theta^{\prime}}_{s}$ vs time}
        \label{fig:downwarm}
\end{figure}

Further partitioning $\overline{w^{'-}\theta^{'+}}_{h}$ into its velocity and temperature components reveals additional scaling.  Figure \ref{fig:downwarm_wvel} shows that the velocity component $\overline{w^{'-}}_{h} \ where \ \overline{\theta^{'}}_{h}>0$, is effectively scaled by $w^{*}$.\\   

\begin{figure}[htbp]
\begin{minipage}[b]{0.5\linewidth}
        %plot_height [master c7af4de] scaleddeltahinvri
        \subfloat[]{\label{main:a}
                \includegraphics[scale=.36]{/newtera/tera/phil/nchaparr/python/Plotting/Dec252013/pngs/downwarm_wvel.pdf}}\\
        \end{minipage}             
\quad
\begin{minipage}[b]{0.5\linewidth}
        \subfloat[]{\label{main:d}          
          %plot_height [master b9c30ad] scaleddeltahinvri1
                \includegraphics[scale=.36]{/newtera/tera/phil/nchaparr/python/Plotting/Dec252013/pngs/scaled_downwarm_wvel.pdf}}\\       
       \end{minipage}
        \caption{Plots of (a) the average negative vertical turbulent velocity perturbation $w^{\prime-}$ at points where the turbulent $\theta$ perturbation is positive ($\theta^{\prime}>0$) and (b) $w^{\prime-}$ where $\theta^{\prime}>0$}, scaled by $w^{*}$.}
        \label{fig:downwarm_wvel}
\end{figure}

The curves representing $\overline{\theta^{'+}}_{h} \ where \ \overline{w^{'}}_{h}>0$ vs time do collapse when scaled by $\theta^{*}$ in Figure \ref{fig:downwarm_theta}.  However it seems this component approaches a constant proportion of $\gamma \Delta h$ in Figure \ref{fig:downwarm_theta1} indicating that the effects of $\gamma$ on the positive temperature perturbations at $h$ may be more important than $\overline{w^{'}\theta^{'}}_{s}$. 

\begin{figure}[htbp]
\begin{minipage}[b]{0.5\linewidth}
        %plot_height [master c7af4de] scaleddeltahinvri
        \subfloat[]{\label{main:a}
                \includegraphics[scale=.36]{/newtera/tera/phil/nchaparr/python/Plotting/Dec252013/pngs/downwarm_theta.pdf}}\\
        \end{minipage}             
\quad
\begin{minipage}[b]{0.5\linewidth}
        \subfloat[]{\label{main:d}          
          %plot_height [master b9c30ad] scaleddeltahinvri1
                \includegraphics[scale=.36]{/newtera/tera/phil/nchaparr/python/Plotting/Dec252013/pngs/scaled_downwarm_theta.pdf}}\\      
       \end{minipage}
        \caption{Plots of (a) the average positive potential temperature perturbation $\theta^{\prime+}$ at points where the turbulent vertical velocity perturbation is negative ($w^{\prime}<0$) and (b) $\theta^{\prime+}$ where $w^{\prime}<0$} scaled by $\theta^{*}$.}
        \label{fig:downwarm_theta}
\end{figure}

\begin{figure}[htbp]
\begin{minipage}[b]{0.5\linewidth}
        %plot_height [master c7af4de] scaleddeltahinvri
        \subfloat[]{\label{main:a}
                \includegraphics[scale=.36]{/newtera/tera/phil/nchaparr/python/Plotting/Dec252013/pngs/downwarm_theta.pdf}}\\
        \end{minipage}             
\quad
\begin{minipage}[b]{0.5\linewidth}
        \subfloat[]{\label{main:d}          
          %plot_height [master b9c30ad] scaleddeltahinvri1
                \includegraphics[scale=.36]{/newtera/tera/phil/nchaparr/python/Plotting/Dec252013/pngs/scaled_downwarm_theta1.pdf}}\\      
       \end{minipage}
        \caption{Plots of (a) the average positive potential temperature perturbation $\theta^{\prime+}$ at points where the turbulent vertical velocity perturbation is negative ($w^{\prime}<0$) (b) $\theta^{\prime+}$ where $w^{\prime}<0$} scaled by $\gamma \Delta h$.}
        \label{fig:downwarm_theta1}
\end{figure}

\clearpage

\section{$h$ and  $\Delta h$ based on Average Profiles}
\label{sec:hdeltahavprofs}

\FloatBarrier

\subsection{Relationship of Entrainment Layer Depth to Richardson Number}
\label{subsec:deltahri}
\FloatBarrier

The scaled upper EL limits ($\frac{h_{1}}{h}$) collapse well in Figure \ref{fig:scaledELlims} 
to an initial value of approximately 1.15, decreasing to about 1.1.  $\frac{h_{0}}{h}$s appear 
grouped according to $\gamma$ and increase with respect to time.  So overall the scaled \acs{EL} appears
to narrow with time.   The scaled flux based \acs{EL} ($\frac{z_{f0}}{z_{f}}$ and $\frac{z_{f1}}{z_{f}}$) appears to remain constant 
with respect to time in Figure \ref{fig:scaledELlims1}.\\

The lower entrainment layer limit $h_{0}$ is the point at which the vertical 
$\frac{\partial \overline{\theta}}{\partial z}$ exceeds a threshold (.0002) is chosen such that
it is positive, and at least an order of magnitude smaller than $\gamma$.   Although the resulting 
scaled \acs{EL} depth decreases with increasing \acs{Ri} grouping according to $\gamma$ is evident 
in Figure \ref{fig:scaledeltahinvri}.\\

To explore how varying the threshold value affects the relationship between scaled \acs{EL} depth
and Richardson number (\acs{Ri}), 

\begin{equation}\label{eq:dhvsri}
\frac{\Delta h}{h} \propto Ri ^{b}
\end{equation}

plots analogous to Figure \ref{fig:scaledeltahinvri} were produced at two 
additional thresholds.  A higher threshold value (.0004) results in a higher $h_{0}$ (Figure \ref{fig:thresh1})   
and so a narrower \acs{EL} but a similar grouping according to $\gamma$ (Figure\ref{fig:scaledeltahinvri1}).
A lower threshold value (.0001) results in a lower $h_{0}$ (Figure \ref{fig:thresh2})
but also similar grouping according to $\gamma$ (Figure \ref{fig:scaledeltahinvri2}).\\


There is a collapsing effect on the scaled $\Delta h$ vs \acs{Ri} relationship when
the heights are defined based on the scaled vertical potential temperature gradient 
$\frac{\partial \overline{\theta}}{\partial z} / \gamma$ profile in Figure \ref{fig:deltahinvri_scaled}.  The magnitude of $\frac{\partial \overline{\theta}}{\partial z}$ in the upper \acs{ML} is scaled by $\gamma$ which compliments the finding in Section \ref{sec:downwarm}.  Furthermore the scaled magnitude of $\Delta h$ decreases with increasing \acs{Ri} as supported by Figure \ref{fig:localhpdf}.  The plot in Figure \ref{fig:loglogdeltahinvri} seems to support an exponent $b = -\frac{1}{2}$ and Figure \ref{fig:deltahinvri_scaled} (b) shows little or no \acs{Ri} dependence when $\Delta h$ and so $\Delta \theta$ are based on the $\overline{w^{'}\theta^{'}}$ profile.

\begin{figure}[htbp]
    \centering
    %plot_height.py [master fd5c6b1] delta h vs time1  
    \includegraphics[scale=.5]{/newtera/tera/phil/nchaparr/python/Plotting/Dec252013/pngs/scaleddeltahstime}
    \caption{Scaled Entrainment Layer limits ($\frac{h_{1}}{h}$ and $\frac{h_{0}}{h}$) vs time}
    \label{fig:scaledELlims}   % label should change
\end{figure}

%need flux based el limits scaled by either h or hf
\begin{figure}[htbp]
    \centering
    %plot_height.py [master fd5c6b1] delta h vs time1  
    \includegraphics[scale=.5]{/newtera/tera/phil/nchaparr/python/Plotting/Dec252013/pngs/scaled_h_f0_time}
    \caption{Scaled Entrainment Layer limits ($z_{f1}$ and $z_{f0}$) vs time}
    \label{fig:scaledELlims1}   % label should change
\end{figure}

\begin{figure}[htbp]
    \centering
    %plot_height.py [master fd5c6b1] delta h vs time1  
    \includegraphics[scale=.5]{/newtera/tera/phil/nchaparr/python/Plotting/Dec252013/pngs/theta_grad_profs}
    \caption{Vertical $\frac{\partial \overline{\theta}}{\partial z}$ profiles with threshold at .0002}
    \label{fig:thresh}   % label should change
\end{figure}

\begin{figure}[htbp]
\centering
%\begin{minipage}[b]{0.5\linewidth}
        %plot_height [master c7af4de] scaleddeltahinvri
 %       \subfloat[]{\label{main:a}
 \includegraphics[scale=.5]{/newtera/tera/phil/nchaparr/python/Plotting/Dec252013/pngs/scaleddeltahinvri}
  %      \end{minipage}             
%\quad
%\begin{minipage}[b]{0.5\linewidth}
 %       \subfloat[]{\label{main:d}          
          %plot_height [master b9c30ad] scaleddeltahinvri1
  %              \includegraphics[scale=.36]{/newtera/tera/phil/nchaparr/python/Plotting/Dec252013/pngs/scaleddeltahinvri1}}\\
       
   %    \end{minipage}
        \caption{Scaled EL depth vs inverse bulk Richardson Number with threshold at .0002}
         \label{fig:scaledeltahinvri}
\end{figure}

\begin{figure}[htbp]
    \centering
    %plot_height.py [master fd5c6b1] delta h vs time1  
    \includegraphics[scale=.5]{/newtera/tera/phil/nchaparr/python/Plotting/Dec252013/pngs/theta_grad_profs5}
    \caption{Vertical $\frac{\partial \overline{\theta}}{\partial z}$ profiles with threshold at .0004}
    \label{fig:thresh1}   % label should change
\end{figure}

\begin{figure}[htbp]
    \centering
    %plot_height.py [master fd5c6b1] delta h vs time1  
    \includegraphics[scale=.5]{/newtera/tera/phil/nchaparr/python/Plotting/Dec252013/pngs/scaleddeltahinvri_5}
    \caption{Scaled EL depth vs inverse Richardson Number with threshold at .0004}
    \label{fig:scaledeltahinvri1}   % label should change
\end{figure}

\begin{figure}[htbp]
    \centering
    %plot_height.py [master fd5c6b1] delta h vs time1  
    \includegraphics[scale=.5]{/newtera/tera/phil/nchaparr/python/Plotting/Dec252013/pngs/theta_grad_profs6}
    \caption{Vertical $\frac{\partial \overline{\theta}}{\partial z}$ profiles with threshold at .0001}
    \label{fig:thresh2}   % label should change
\end{figure}

\begin{figure}[htbp]
    \centering
    %plot_height.py [master fd5c6b1] delta h vs time1  
    \includegraphics[scale=.5]{/newtera/tera/phil/nchaparr/python/Plotting/Dec252013/pngs/scaleddeltahinvri_6}
    \caption{Scaled EL depth vs inverse bulk Richardson Number with threshold at .0001}
    \label{fig:scaledeltahinvri2}   % label should change
\end{figure}

\begin{figure}[htbp]
    \centering
    %plot_height.py [master fd5c6b1] delta h vs time1  
    \includegraphics[scale=.5]{/newtera/tera/phil/nchaparr/python/Plotting/Dec252013/pngs/scaled_theta_grad_profs}
    \caption{Scaled vertical $\frac{\partial \overline{\theta}}{\partial z}$ profiles with threshold at .03}
    \label{fig:thresh3}   % label should change
\end{figure}

\begin{figure}[htbp]
\begin{minipage}[b]{0.5\linewidth}
        %
        \subfloat[]{\label{main:a}
                \includegraphics[scale=.36]{/newtera/tera/phil/nchaparr/python/Plotting/Dec252013/pngs/scaleddeltahinvri_4}}\\
        \end{minipage}             
\quad
\begin{minipage}[b]{0.5\linewidth}
        \subfloat[]{\label{main:d}          
          %
                \includegraphics[scale=.36]{/newtera/tera/phil/nchaparr/python/Plotting/Dec252013/pngs/scaleddeltahinvri_f}}\\      
       \end{minipage}
        \caption{Plots of scaled \acs{EL} depth vs \acs{Ri}$^{-1}$ with \acs{CBL} height. \acs{EL} limits and so $\Delta \theta$ based on the $\frac{\frac{\partial \overline{\theta}}{\partial z}}{\gamma}$ in (a) and the $\frac{\overline{w^{'}\theta^{'}}}{\overline{w^{'}\theta^{'}}_{s}}$ profile in (b).}
        \label{fig:deltahinvri_scaled}
\end{figure}


\begin{figure}[htbp]
\centering
\includegraphics[scale=.5]{/newtera/tera/phil/nchaparr/python/Plotting/Dec252013/pngs/loglog_scaleddeltahinvri_4}\\
\caption{Scaled \acs{EL} depth vs \acs{Ri}$^{-1}$ based on the $\frac{\frac{\partial \overline{\theta}}{\partial z}}{\gamma}$ profile in log-log coordinates to see likely values of the exponent $b$}
\label{fig:loglogdeltahinvri}
\end{figure}

\clearpage

\subsection{Relationship of Entrainment Rate to Richardson Number}
\FloatBarrier

Convective Boundary Layer (CBL) height ($h$) (Figure \ref{fig:hvstime}) grows rapidly initially with a steadily decreasing rate
and relates to the square-root of time (Figure \ref{fig:loghvstime}). The height of minimum average heat flux $z_{f}$ is a constant proportion of $h$ in Figure \ref{fig:zfvstime} indicating that this point advances more slowly than $h$.\\
  
\begin{figure}[htbp]
    \centering
    %plot_height.py[master 1573b9d] h vs time plot
    \includegraphics[scale=.5]{/newtera/tera/phil/nchaparr/python/Plotting/Dec252013/pngs/hvstime}
    \caption{$h$ vs time for all runs}
    \label{fig:hvstime}   % label should change
\end{figure}

\begin{figure}[htbp]
    \centering
    %plot_height.py[master 1573b9d] h vs time plot
    \includegraphics[scale=.5]{/newtera/tera/phil/nchaparr/python/Plotting/Dec252013/pngs/hstimelog}
    \caption{$h$ vs time for all runs on log-log coordinates}
    \label{fig:loghvstime}   % label should change
\end{figure}

%need flux height scaled by h plot

\begin{figure}[htbp]
    \centering
    %plot_height.py[master 1573b9d] h vs time plot
    \includegraphics[scale=.5]{/newtera/tera/phil/nchaparr/python/Plotting/Dec252013/pngs/scaled_h_f_time}
    \caption{$\frac{z_{f}}{h}$ vs Time}
    \label{fig:zfvstime}   % label should change
\end{figure}

The inverse Richardson numbers in Figure \ref{fig:invristime_f}, \acs{Ri}$_{\Delta}$ and \acs{Ri}$_{\delta}$, based on the scaled $\frac{\partial \overline{\theta}}{\partial z}$ profile decrease in time and group according to $\gamma$. There is an overall difference in magnitude since $\Delta \theta > \delta \theta$. Something similar can be said when $\Delta \theta$ and $\delta \theta$ are based on the $\overline{w^{'}\theta^{'}}$ profile although \acs{Ri}$_{\Delta}$ shows more scatter in Figure \ref{fig:invristime_f} (a).\\  

\begin{figure}[htbp]

\begin{minipage}[b]{0.5\linewidth}
         
        \subfloat[]{\label{main:a}
                \includegraphics[scale=.36]{/newtera/tera/phil/nchaparr/python/Plotting/Dec252013/pngs/invristime_Delta}}\\
        \end{minipage}             
\quad
\begin{minipage}[b]{0.5\linewidth}
        \subfloat[]{\label{main:d}
          %plot_height [master fbd2dfd] invristime1, see branches named accordingly
                \includegraphics[scale=.36]{/newtera/tera/phil/nchaparr/python/Plotting/Dec252013/pngs/invristime_delta}}\\
       
       \end{minipage}
        \caption{Inverse Richardson number vs time based on the $\frac{\frac{\partial \overline{\theta}}{\partial z}}{\gamma}$
profile using $\Delta h$ across the \acs{EL} in (a) and $\delta \theta$ at $h$ in (b).  See Table \ref{table:reldefs}.}
        \label{fig:invristime}
\end{figure}

\begin{figure}[htbp]
\begin{minipage}[b]{0.5\linewidth}
         
        \subfloat[]{\label{main:a}
                \includegraphics[scale=.36]{/newtera/tera/phil/nchaparr/python/Plotting/Dec252013/pngs/invristime_Delta_f}}\\
        \end{minipage}             
\quad
\begin{minipage}[b]{0.5\linewidth}
        \subfloat[]{\label{main:d}
          %plot_height [master fbd2dfd] invristime1, see branches named accordingly
                \includegraphics[scale=.36]{/newtera/tera/phil/nchaparr/python/Plotting/Dec252013/pngs/invristime_delta_f}}\\
       
       \end{minipage}
        \caption{Inverse Richardson number vs time based on the $\frac{\overline{w^{'}\theta^{'}}}{\overline{w^{'}\theta^{'}}_{s}}$
profile using $\Delta h$ across the \acs{EL} in (a) and $\delta \theta$ at $z_{f}$ in (b).  See Figure \ref{fig:hdefs}.}
        \label{fig:invristime_f}
\end{figure}

The entrainment rate ($w_{e}= \frac{dh}{dt}$) is determined from the slope of a second order polynomial fit to $h(time)$ in Figure \ref{fig:hvstime}. All heights are now based on the $\frac{\partial \overline{\theta}}{\partial z}$ profile scaled by $\gamma$. When $w_{e}$ is scaled by $w^{*}$, the resulting relationship to \acs{Ri}$_{\Delta}$ 

\begin{equation}\label{eq:ervsri}
\frac{w_{e}}{w^{*}} \propto Ri^{a}
\end{equation}


in Figure \ref{fig:weinvri} (a) seems to have exponent $a = -1$ at lower \acs{Ri}$_{\Delta}$ and $a = -\frac{3}{2}$ at higher \acs{Ri}$_{\Delta}$.  Whereas in (b) it is not clear which, if any, values of $a$ fit the relationship to \acs{Ri}$_{\delta}$ best. \\    

\begin{figure}[htbp]
\begin{minipage}[b]{0.5\linewidth}
        %plot_height [master fbd2dfd] invristime1
        \subfloat[]{\label{main:a}
                \includegraphics[scale=.36]{/newtera/tera/phil/nchaparr/python/Plotting/Dec252013/pngs/scaledweinvri_Delta}}\\
        \end{minipage}             
\quad
\begin{minipage}[b]{0.5\linewidth}
        \subfloat[]{\label{main:d}          
          %plot_height [master 01c3721] scaledweinvri1
                \includegraphics[scale=.36]{/newtera/tera/phil/nchaparr/python/Plotting/Dec252013/pngs/scaledweinvri_delta}}\\       
       \end{minipage}
        \caption{Scaled entrainment rate vs inverse Richardson number (\acs{Ri}$^{-1}$), where \acs{Ri} is based on the $\frac{\frac{\partial \overline{\theta}}{\partial z}}{\gamma}$ profile using $\Delta h$ across the \acs{EL} in (a) and $\delta \theta$ at $h$ in (b). See Figure \ref{fig:hdefs}.}
        \label{fig:weinvri}
\end{figure}

Figure \ref{fig:weinvri_f} is as Figure \ref{fig:weinvri} except all heights are based on the scaled $\overline{w^{'}\theta^{'}}$ profile. The relationship of scaled entrainment rate to \acs{Ri}$_{\Delta}$ in (a) shows scatter and either value of $a$ or a value in between could fit.  Whereas the exponent in the relationship to \acs{Ri}$_{\delta}$ in (b) seems to change throughout the run(s) and a value less (in magnitude) than $-1$ might fit better. \\    

\begin{figure}[htbp]
\begin{minipage}[b]{0.5\linewidth}
        %plot_height [master fbd2dfd] invristime1
        \subfloat[]{\label{main:a}
                \includegraphics[scale=.36]{/newtera/tera/phil/nchaparr/python/Plotting/Dec252013/pngs/scaledweinvri_Delta_f}}\\
        \end{minipage}             
\quad
\begin{minipage}[b]{0.5\linewidth}
        \subfloat[]{\label{main:d}          
          %plot_height [master 01c3721] scaledweinvri1
                \includegraphics[scale=.36]{/newtera/tera/phil/nchaparr/python/Plotting/Dec252013/pngs/scaledweinvri_delta_f}}\\
       
       \end{minipage}
        \caption{Scaled entrainment rate vs inverse Richardson number (\acs{Ri}$^{-1}$), where \acs{Ri} is based on the $\frac{\overline{w^{'}\theta^{'}}}{\overline{w^{'}\theta^{'}}_{s}}$
profile using $\Delta h$ across the \acs{EL} in (a) and $\delta \theta$ at $z_{f}$ in (b).  See Figure \ref{fig:hdefs}.}
        \label{fig:weinvri_f}
\end{figure}

In conclusion the relationship of scaled entrainment rate to \acs{Ri}$_{\Delta}$ based on the $\frac{\partial \overline{\theta}}{\partial z}$ profile scaled by $\gamma$ shows the least scatter over time and between runs.  Here the exponent seems to start at a value close to $-1$ increasing, with higher \acs{Ri}, to close to $-\frac{3}{2}$.  Overall the definition of the temperature jump certainly has an effect, $\Delta \theta$ yielding a higher value of $a$ than $\delta \theta$.

\endinput

Any text after an \endinput is ignored.
You could put scraps here or things in progress.
