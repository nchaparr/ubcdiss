%% The following is a directive for TeXShop to indicate the main file
%%!TEX root = diss.tex

\chapter{Results}
\label{ch:results}

\section{Runs}
\label{sec:Runs}
\begin{center}
    \begin{tabular}{ | l | l | l | p{5cm} |}
    \hline
    $\overline{w^{'}\theta^{'}_{s}}$ / $\gamma$ & 10 (K/Km) & Max Temp & Summary \\ \hline
    Monday & 11C & 22C & A clear day with lots of sunshine.  
    However, the strong breeze will bring down the temperatures. \\ \hline
    Tuesday & 9C & 19C & Cloudy with rain, across many northern regions. Clear spells
    across most of Scotland and Northern Ireland,
    but rain reaching the far northwest. \\ \hline
    Wednesday & 10C & 21C & Rain will still linger for the morning.
    Conditions will improve by early afternoon and continue
    throughout the evening. \\
    \hline
    \end{tabular}
\end{center}

%\section{Checking the Model}
%\label{sec:CheckingtheModel}


%\subsection{$\theta$ and Flux}

%Profiles of horizonatally averaged ensemble averaged $\theta$.\\

%We expect to see formation of a measureable mixed layer with uniform temperature topped by an entrainment region.\\

%Get $\theta_{t+1} -  \theta_{t}$ profiles to predict flux profiles: $\frac{d\theta}{dz} = \overline{w^{'}\theta^{'}}$ so points at where sucessive $\theta$ profiles intersect should more or less correspond to points of zero flux crossing.  Points at which $\theta$s decrease most from one time to the next, should correspond to negative peak in fluxes.\\

%Compare Flux profiles to predicted flux profiles to $\theta$ profiles.\\

%Some type of quadrant analysis : plots of horizontally averaged ensemble averaged upwarm, downwarm, upcold, downcold alongside average flux.\\

%\subsection{2d FFTs}

%\subsection{Root Mean Squared Velocity Profiles}

%\section{h, and EL Limits based on Average Profiles}

%\subsection{Definitions}

%\subsection{Plots}

%\section{Scaling Relationships of $W_{e}$ and $\Delta h$}
%\label{sec:Scaling Relationships of $W_{e}$ and $\Delta h$}

%\section{Local Mixed Layer Height Distriubutions}

%\section{Flux Quadrant Analysis}

\endinput

Any text after an \endinput is ignored.
You could put scraps here or things in progress.
