%% The following is a directive for TeXShop to indicate the main file
%%!TEX root = diss.tex

\chapter{Results}
\label{ch:results}
\setlength{\parindent}{0cm}

\section{Runs}

All 10 member cases of the ensemble were carried out on a 3.2 x 4.8 Km horizontal domain ($\Delta x = \Delta y = 25m$, $nx=128$, $ny=192$)
$nx$, $nx$ were chosen based on the optimal distribution accross processor nodes.  The vertical grid ($nz=312$) was of higher resolution around the 
Entrainment Layer (\acs{EL}) ($\Delta z = 5m$), and lower below and above it ($\Delta z = 10 \ to \ 100 m$). Grid size was chosen so that
a full spectrum of turbulence would be resolved within the \acs{EL} in line with the findings in \cite{SullPat}.  The 7 runs
vary depending on surface heat flux ($\overline{w^{,}\theta^{,}_{s}}$) and initial lapse rate ($\gamma$).

%description of runs ie 10 member ensembles each had delta x, delta y=25 and a region of delta z=5m enclosing the 
%Entrainment zone, z=25 below  and streched to 100 above. 

\label{sec:Runs}

\begin{table}[!ht]
    \begin{center}
    \begin{tabular}{ | l | l | l | l |}
    \hline
    $\overline{w^{'}\theta^{'}_{s}}$ / $\gamma$ & 10 (K/Km) & 5 (K/Km) & 2.5 (K/Km) \\ \hline
     150 (W/m2)& \hspace{5mm} \ding{51} &\hspace{5mm} \ding{51}\footnotemark &  \\ \hline
     100 (W/m2)& \hspace{5mm} \ding{51} & \hspace{5mm} \ding{51} & \\ \hline
     60 (W/m2) & \hspace{5mm} \ding{51} & \hspace{5mm} \ding{51} & \hspace{5mm} \ding{51}\\ \hline
     %\end

   
\end{tabular}
\caption{Runs in terms of $\overline{w^{,} \theta^{,}_{s}}$ and initial lapse rate $\gamma$}
\label{fig:tableofruns}   
\end{center}    
\end{table}
\footnotetext{Incomplete run: EL exceded high resolution vertical grid after 7 hours}


\section{Relevant Definitions}
%table outlining my definitions, and other comparable ones, ie brooks and fowlers aren't comparable
% re justifying my arbitrary choice of threshold for the lower limit:  seee page 250 Brooks and Fowler. 
%there must be other places in my key papers where the 'arbitraryness' is referred to.


\section{Verifying Model Output}
\label{sec:CheckingtheModel}
\subsection{}%Spin Up Time according to the Convective Time Scale $\tau$}
\FloatBarrier

Time must be allowed to establish statistically steady turbulent flow.  \citeauthor{SullMoengStev} in \cite{SullMoengStev}
recommended 10 eddie turnover times based on the convective time scale $\tau = \frac{h}{w^{*}} = \frac{h}{ \left( \frac{gh}{\overline{\theta}_{ML}}(\overline{w^{,} \theta^{,}_{s}}) \right)^{\frac{1}{3}} } $, 
and \citeauthor{BrooksFowler2} in \cite{BrooksFowler2} chose a simulated time of 2 hours.  For all of the runs, 10 eddie 
turnover times were completed by 2 hours (Figure \ref{fig:ScaledTimevsTime}).  Although each run has a distinct
convective velocity scale, that increases with time ($w^{*}(time)$), dividing boundary layer height ($h$) by it
to obtain $\tau$ results in a collapse from 7 to 3 curves, one for each $\gamma$.\\


\begin{figure}[!h]
    \centering
    % plot_height.py [master 03d2835] Round 1 of Plots in Results 
    \includegraphics[scale=.5]{/tera/phil/nchaparr/python/Plotting/Dec252013/pngs/scaledtimevstime}
    \caption{Plots of scaled time vs time for all runs}
    \label{fig:ScaledTimevsTime}   
\end{figure}

A Measureable well Mixed Layer (\acs{ML}) and \acs{EL} based on the horizontaly averaged, ensemble averaged
potential temperature ($\overline{\theta}$) profile develops after 2 hours (Figure \ref{fig:tempgradfluxprofs15010}).\\

\begin{figure}[htbp]
    \centering
    % plot_dthetaflux.py [master 03d2835] Round 1 of Plots in Results
    \includegraphics[scale=.5]{/tera/phil/nchaparr/python/Plotting/Mar52014/pngs/theta_flux_profs}
    \caption{Ensemble and horizontally averaged potential temperature ($\overline{\theta}$), its vertical gradient ($\frac{\partial \overline{\theta}}{\partial z}$)  and heat flux ($\overline{w^{,}\theta^{,}}$) vertical profiles for the 150/10 run}
    \label{fig:tempgradfluxprofs15010}   % label should change
\end{figure}

Averaged heat fluxes ($\overline{w^{,}\theta^{,}}$) and root mean squared vertical velocity perturbations ($\sqrt{w^{,2}}$) (Figure \ref{fig:rmswvelprofs15010}) are scaled 
well by $\overline{w^{,}\theta^{,}}_{s}$ and the convective velocity scale ($w^{*}$) respectively after 2 hours 
(Figure \ref{fig:scaledtempgradfluxprofs15010}).\\


\begin{figure}[htbp]
    \centering
    % plot_dthetaflux.py [master 9883fda] Round 2 of Plots in Results
    \includegraphics[scale=.5]{/tera/phil/nchaparr/python/Plotting/Mar52014/pngs/scaled_theta_flux_profs}
    \caption{$\overline{\theta}$, scaled $\frac{\partial \overline{\theta}}{\partial z}$ and scaled $\overline{w^{,}\theta^{,}}$  vs scaled height for the 150/10 run}
    \label{fig:scaledtempgradfluxprofs15010}   % label should change
\end{figure}

\begin{figure}[htbp]
    \centering
    %w_analysis.py [master 03d2835] Round 1 of Plots in Results
    \includegraphics[scale=.5]{/tera/phil/nchaparr/python/Plotting/Mar52014/pngs/rmswvels}
    \caption{$\sqrt{w^{,2}}$ vertical profiles for the 150/10 run}
    \label{fig:rmswvelprofs15010}   % label should change
\end{figure}

\clearpage
\subsection{}
\FloatBarrier

Two dimensional \acs{FFT} power spectra taken of horizontal slices of $w^{,}$ (\ref{fig:scalardfftw602point5}) at three
different levels ($h_{0}$, $h$ and $h_{1}$) are collapsed to one dimension by integrating around a semi-circle of positive wave-numbers.
Isotropy in all radial directions is assumed and $k = \sqrt{k_{x}^{2} + k_{y}^{2}}$.  The resulting scalar density spectra show peaks in 
energy at the larger scales, cascading to the lower scales roughly according to a $\frac{-5}{3}$ slope lower in the \acs{EL}.  At
the top of the \acs{EL} where turbulence is supressed by stability, the slope is steeper.  The peak in energy occur at smaller scales
at $h$ as compared to at $h_{0}$, indicating a change in the size of the dominant turbulent structures.\\

\begin{figure}[htbp]
    \centering
    %fft_nchap.py [master 03d2835] Round 1 of Plots in Results
    \includegraphics[scale=.5]{/tera/phil/nchaparr/python/Plotting/Dec252013/pngs/2dfftpow}
    \caption{2D FFT energy densities of $w^{,}$ at $h_{0}$ vs wavenumber for the 60/2.5 run}
    \label{fig:scalardfftw602point5}   % label should change
\end{figure}

\begin{figure}[htbp]
    \centering
    % fft_chap.py [master 03d2835] Round 1 of Plots in Results
    \includegraphics[scale=.5]{/tera/phil/nchaparr/python/Plotting/Dec252013/pngs/scalarfftpow}
    \caption{Scalar FFT  energy densities vs wavenumber ($k = \sqrt{k_{x}^{2}+k_{y}^{2}}$) for the 60/2.5 run}
    \label{fig:2fftw602point5}   % label should change
\end{figure}

\clearpage

\subsection{}%Average Potential Temperature, Heat Flux and Kinetic Energy}
\FloatBarrier

The $\overline{\theta}$ profiles exhibit an \acs{ML} of height $h_{0}$ (the point at which $\frac{\partial \overline{\theta}}{\partial z}$ first exceeds 0), 
topped by a region where $\frac{\partial\overline{\theta}}{\partial z}>0$ within which there is a maximum value of $\frac{\partial \overline{\theta}}{\partial z}$ 
at $h$ before resuming $\gamma$  at $h_{1}$ (Figures \ref{fig:tempgradfluxprofs15010} and \ref{fig:pottempprofs2hrs}).\\

\begin{figure}[htbp]
    \centering
    % plot_theta_profs.py [master 9883fda] Round 2 of Plots in Results
    \includegraphics[scale=.5]{/tera/phil/nchaparr/python/Plotting/Dec252013/pngs/theta_profs2hrs}
    \caption{$\overline{\theta}$ profiles at 2 hours}
    \label{fig:pottempprofs2hrs}   % label should change
\end{figure}


The horizonally averaged, ensemble averaged heat flux ($\overline{w^{,}\theta^{,}}$) profiles decrease from the surface value
($\overline{w^{,}\theta^{,}_{s}}$) passing through zero to a minumum before increasing to zero.(Figures \ref{fig:tempgradfluxprofs15010} and
 \ref{fig:fluxprofs2hrs})\\

\begin{figure}[htbp]
    \centering
    %plot_theta_profs.py [master 03d2835] Round 1 of Plots in Results
    \includegraphics[scale=.5]{/tera/phil/nchaparr/python/Plotting/Dec252013/pngs/flux_profs2hrs}
    \caption{$\overline{w^{,}\theta^{,}}_{s}$ profiles at 2 hours}
    \label{fig:fluxprofs2hrs}   % label should change
\end{figure}

Each of the $\overline{\theta}$ and $\overline{w^{,}\theta^{,}}$ profiles has a region that can be defined as an \acs{EL}.
Here we use the former definition.  The point of minimum $\overline{w^{,}\theta^{,}}$ is lower than $h$.  
\citeauthor{SullMoengStev} in \cite{SullMoengStev} noted that the upper extrema of the individual flux quadrant profiles 
are closer or even coincide with $h$.\\

($\sqrt{u^{,2}}$) profiles show a dominance of vertical velocity perturbation ($w^{'}$) in the \acs{ML}, with a peak in horizontal velocity
($u^{'}$ and $v^{'}$) within the \acs{EL} where the buoyancy driven $w^{'}$ is inhibited by stability (Figure \ref{fig:rmsvel150102hrs}). \\

\begin{figure}[htbp]
    \centering
    %plot_vars.py [master 03d2835] Round 1 of Plots in Results
    \includegraphics[scale=.5]{/tera/phil/nchaparr/python/Plotting/Mar52014/pngs/rmsvel2}
    \caption{Vertical $\frac{\sqrt[]{u^{,2}}}{w^{*}}$ profiles at 2 hours for the 150/10 run}
    \label{fig:rmsvel150102hrs}   % label should change
\end{figure}

\clearpage

\subsection{Visualization of Structures Within the Entrainment Layer}
\FloatBarrier
At the bottom of the \acs{EL} ($h_{0}$) in the 150/10 run (Figure \ref{fig:conts} (a) and (d)) coherent
areas of positive and negative temperature perturbations correspond to areas of upward and downward moving air.  
The individual plumes of relatively cool air are more evident at the inversion ($h$) ((b) and (e)) and their 
locations correspond to areas of upward motion.  Most of the upward moving cool areas are adjacent to and even 
encircled by smaller areas of downward moving warm air.  At $h_{1}$ ((c) and (f)) peaks of cool air are associated 
with both up and down-welling.\\  

In the 60/2.5 run (Figure \ref{fig:conts1}) a similar progression is evident but the impinging, cool upward moving
plumes are more defined.\\   

\begin{figure}[htbp]
\caption{$\theta^{'}$ (left) and $w^{'}$ (right) at 2 hours at $h_{0}$ (a,d), $h$ (c,e) and $h_{1}$ (d,f)}
\begin{minipage}[b]{0.5\linewidth}
  
        %Flux_Quads.py [master 03d2835] Round 1 of Plots in Results
        \subfloat[]{\label{main:a}
                \includegraphics[scale=.36]{/tera/phil/nchaparr/python/Plotting/Mar52014/pngs/theta_cont0}}\\
        \subfloat[]{\label{main:b}      
                \includegraphics[scale=.36]{/tera/phil/nchaparr/python/Plotting/Mar52014/pngs/theta_cont1}}\\ 
        \subfloat[]{\label{main:c}      
                \includegraphics[scale=.36]{/tera/phil/nchaparr/python/Plotting/Mar52014/pngs/theta_cont2}} 
 \end{minipage}             
\quad
\begin{minipage}[b]{0.5\linewidth}
        %Flux_Quads.py [master 9883fda] Round 2 of Plots in Results
        \subfloat[]{\label{main:d}
                \includegraphics[scale=.36]{/tera/phil/nchaparr/python/Plotting/Mar52014/pngs/wvel_cont0}}\\
       
       \subfloat[]{\label{main:e}
                \includegraphics[scale=.36]{/tera/phil/nchaparr/python/Plotting/Mar52014/pngs/wvel_cont1}}\\
        
       \subfloat[]{\label{main:f}
                \includegraphics[scale=.36]{/tera/phil/nchaparr/python/Plotting/Mar52014/pngs/wvel_cont2}}                 
\end{minipage}
        
        \label{fig:conts}
\end{figure}

\begin{figure}[htbp]
\caption{$\theta^{'}$ (left) and $w^{'}$ (right) at 2 hours at $h_{0}$ (a,d), $h$(b,e) and $h_{1}$(c,f)}
\begin{minipage}[b]{0.5\linewidth} 
        
        \subfloat[]{\label{main:a}
                \includegraphics[scale=.36]{/tera/phil/nchaparr/python/Plotting/Dec252013/pngs/theta_cont0}}\\
        \subfloat[]{\label{main:b}      
                \includegraphics[scale=.36]{/tera/phil/nchaparr/python/Plotting/Dec252013/pngs/theta_cont1}}\\ 
        \subfloat[]{\label{main:c}      
                \includegraphics[scale=.36]{/tera/phil/nchaparr/python/Plotting/Dec252013/pngs/theta_cont2}} 
 \end{minipage}             
\quad
\begin{minipage}[b]{0.5\linewidth}
        \subfloat[]{\label{main:d}
                \includegraphics[scale=.36]{/tera/phil/nchaparr/python/Plotting/Dec252013/pngs/wvel_cont0}}\\
       
       \subfloat[]{\label{main:e}
                \includegraphics[scale=.36]{/tera/phil/nchaparr/python/Plotting/Dec252013/pngs/wvel_cont1}}\\
        
       \subfloat[]{\label{main:f}
                \includegraphics[scale=.36]{/tera/phil/nchaparr/python/Plotting/Dec252013/pngs/wvel_cont2}}                 
\end{minipage}
        
        \label{fig:conts1}
\end{figure}

\clearpage


\section{Local Mixed Layer Heights ($h_{0}^{l}$)}
\label{sec:locmlh}     
\FloatBarrier

Local $\theta$ profiles (Fig. \ref{fig:rssfits}) exhibit a distinct \acs{ML} before resuming $\gamma$ but not 
a clearly defined \acs{EL}. There are sharp changes in the profile well into the free atmosphere, due possibly to 
waves, which render the gradient method for determining $h^{l}$ unusable.  Instead a linear regression method is used, 
whereby three lines representing: the mixed layer (ML), the EL and the upper lapse rate ($\gamma$) are fit to the 
profile according to the minimum residual sum of squares (RSS).  Determining local \acs{ML} height ($h_{0}^{l}$) was 
more straight forward than the local height of maximum potential temperature gradient $h^{l}$ for the reasons stated 
above.\\  

\begin{figure}[htbp]
%Pcolor_Peaks.py [master 61491be] rss_fit plots
\begin{minipage}[b]{0.5\linewidth}
        %
        \subfloat[]{\label{main:a}
                \includegraphics[scale=.36]{/tera/phil/nchaparr/python/Plotting/Dec252013/pngs/rss_fit}}\\
        \end{minipage}             
\quad
\begin{minipage}[b]{0.5\linewidth}
        \subfloat[]{\label{main:b}          
          
                \includegraphics[scale=.36]{/tera/phil/nchaparr/python/Plotting/Mar52014/pngs/rss_fit}}\\
       
       \end{minipage}
        \caption{Local vertical $\theta$ profiles with 3-line fit for the 60/2.5 (a) and 150/10 (b) runs}
        \label{fig:rssfits}
\end{figure}

Contour plots (Fig \ref{fig:conts2}) show regions of high $h_{0}^{l}$ corresponding to regions of 
upward moving relatively cool air at $h_{1}$.\\

\begin{figure}[htbp]
\caption{$\theta^{'}$ (a,d), $w^{'}$(b,e) at $h_{1}$(c,f) and local $h_{0}$ at 2 hours for 60/2.5 (left) and 150/10 (right) runs}
\begin{minipage}[b]{0.5\linewidth} 
        
        \subfloat[]{\label{main:a}
                \includegraphics[scale=.36]{/tera/phil/nchaparr/python/Plotting/Dec252013/pngs/theta_cont2}}\\
        \subfloat[]{\label{main:b}      
                \includegraphics[scale=.36]{/tera/phil/nchaparr/python/Plotting/Dec252013/pngs/wvel_cont2}}\\ 
        \subfloat[]{\label{main:c}
          %Get_ML_Heights.py [master b7c3e5b] h_cont
                \includegraphics[scale=.36]{/tera/phil/nchaparr/python/Plotting/Dec252013/pngs/h_cont}} 
 \end{minipage}             
\quad
\begin{minipage}[b]{0.5\linewidth}
        \subfloat[]{\label{main:d}
                \includegraphics[scale=.36]{/tera/phil/nchaparr/python/Plotting/Mar52014/pngs/theta_cont2}}\\
       
       \subfloat[]{\label{main:e}
                \includegraphics[scale=.36]{/tera/phil/nchaparr/python/Plotting/Mar52014/pngs/wvel_cont2}}\\
        
       \subfloat[]{\label{main:f}
                \includegraphics[scale=.36]{/tera/phil/nchaparr/python/Plotting/Mar52014/pngs/h_cont}}                 
\end{minipage}
        
        \label{fig:conts2}
\end{figure}

Distributions of $h_{0}^{l}$ at 5 hours (Fig \ref{fig:localhhist}) show increased variance with increasing $\overline{w^{,}\theta^{,}_{s}}$
and decreasing $\gamma$.  Due to the clustering according to $\gamma$ this progression is less effectively represented by plotting 
variance against $Ri^{-1}$ (Fig. \ref{fig:varsvsinvri})

%could put these sideways.  could do with checking the 2.5/60 low hs

\begin{figure}[htbp]
\caption{Histograms of $h^{l}_{0}$ for $\overline{w^{,}\theta^{,}_{s}} = 150 \ to \ 60$ (W \ $m^{2}$) (top to bottom) and $ \gamma = 10 to 2.5$ (K/Km) (left to right) at 5 hours}
%ML_Height_hist.py(sp?) master f992942ade
\begin{minipage}[b]{0.32\linewidth} 
        
        \subfloat[]{\label{main:a}
                \includegraphics[scale=.22]{/tera/phil/nchaparr/python/Plotting/Mar52014/pngs/ML_Height_hist}}\\
        \subfloat[]{\label{main:b}      
                \includegraphics[scale=.22]{/tera/phil/nchaparr/python/Plotting/Dec142013/pngs/ML_Height_hist}}\\ 
        \subfloat[]{\label{main:c}          
                \includegraphics[scale=.22]{/tera/phil/nchaparr/python/Plotting/Mar12014/pngs/ML_Height_hist}} 
 \end{minipage}             
%\quad
\begin{minipage}[b]{0.32\linewidth}
        \subfloat[]{\label{main:d}
                \includegraphics[scale=.22]{/tera/phil/nchaparr/python/Plotting/Jan152014_1/pngs/ML_Height_hist}}\\
       \subfloat[]{\label{main:e}
                \includegraphics[scale=.22]{/tera/phil/nchaparr/python/Plotting/Nov302013/pngs/ML_Height_hist}}\\
       \subfloat[]{\label{main:f}
                \includegraphics[scale=.22]{/tera/phil/nchaparr/python/Plotting/Dec202013/pngs/ML_Height_hist}}                 
\end{minipage}
\begin{minipage}[b]{0.32\linewidth}
        %\subfloat[]{\label{main:d}
        %        \includegraphics[scale=.18]{/tera/phil/nchaparr/python/Plotting/Mar52014/pngs/ML_Height_hist}}\\
       
       %\subfloat[]{\label{main:e}
       %         \includegraphics[scale=.18]{/tera/phil/nchaparr/python/Plotting/Mar52014/pngs/ML_Height_hist}}\\
       \vspace{10mm} 
       \subfloat[]{\label{main:f}
                \includegraphics[scale=.22]{/tera/phil/nchaparr/python/Plotting/Dec252013/pngs/ML_Height_hist}}                 
\end{minipage}

        
        \label{fig:localhhist}
\end{figure}

\begin{figure}[htbp]
    \centering
    %plot_height.py [master 199de9a7cf]  
    \includegraphics[scale=.5]{/tera/phil/nchaparr/python/Plotting/Dec252013/pngs/varvsinvri}
    \caption{Variance vs \acs{Ri}$^{-1}$ at 5 hours}
    \label{fig:varsvsinvri}   % label should change
\end{figure}

\clearpage

\section{Flux Quadrants}
\label{sec:fluxquadrants}     
\FloatBarrier

As in \cite{SullMoengStev} when broken out into four quadrants (Fig. \ref{fig:fluxqadprofs}) the $\overline{w^{,}\theta^{,}}$ profiles have upper extrema above that of the
total average profile, which is located between them and the upward moving relatively warm (upwarm) minimum.  Increased stability aloft inhibits the extent of both 
$h$ and $\Delta h$.\\  

\begin{figure}[htbp]
\begin{minipage}[b]{0.5\linewidth}
        %
        \subfloat[]{\label{main:a}
                \includegraphics[scale=.36]{/tera/phil/nchaparr/python/Plotting/Dec252013/pngs/fluxquadprofs}}\\
        \end{minipage}             
\quad
\begin{minipage}[b]{0.5\linewidth}
        \subfloat[]{\label{main:d}          
          %Flux_Quads.py [master 0eaf94e] fluxquadprofs
                \includegraphics[scale=.36]{/tera/phil/nchaparr/python/Plotting/Mar52014/pngs/fluxquadprofs}}\\
       \end{minipage}
        \caption{$\overline{w^{,} \theta^{,}}$ quadrant profiles at 5 hours for the 60/2.5 (a) and 150/10 (b) run}
        \label{fig:fluxqadprofs}
\end{figure}

2D Histograms of the four quadrants are plotted at $h_{0}$, $h$ and $h_{1}$ (based on the averaged $\theta$ profiles)
to visualize how the distributions are influenced by changes in $\overline{w^{,} \theta^{,}}$ and $\gamma$.  At $h_{0}$
(Fig. \ref{fig:fluxquadsh0}) fast updraughts are relatively warm and are inhibited by increasing upper stability.  The spread
in velocity increases with increasing $\overline{w^{,}\theta^{,}}_{s}$\\

%redo these with one colorbar see 
%http://stackoverflow.com/questions/13784201/matplotlib-2-subplots-1-colorbar  

\begin{figure}[htbp]
%
\begin{minipage}[b]{0.32\linewidth} 
        
        \subfloat[]{\label{main:a}
                \includegraphics[scale=.22]{/tera/phil/nchaparr/python/Plotting/Mar52014/pngs/fluxquadhist0}}\\
        \subfloat[]{\label{main:b}      
                \includegraphics[scale=.22]{/tera/phil/nchaparr/python/Plotting/Dec142013/pngs/fluxquadhist0}}\\ 
        \subfloat[]{\label{main:c}          
                \includegraphics[scale=.22]{/tera/phil/nchaparr/python/Plotting/Mar12014/pngs/fluxquadhist0}} 
 \end{minipage}             
%
\begin{minipage}[b]{0.32\linewidth}
        \subfloat[]{\label{main:d}
                \includegraphics[scale=.22]{/tera/phil/nchaparr/python/Plotting/Jan152014_1/pngs/fluxquadhist0}}\\
       \subfloat[]{\label{main:e}
                \includegraphics[scale=.22]{/tera/phil/nchaparr/python/Plotting/Nov302013/pngs/fluxquadhist0}}\\
       \subfloat[]{\label{main:f}
                \includegraphics[scale=.22]{/tera/phil/nchaparr/python/Plotting/Dec202013/pngs/fluxquadhist0}}                 
\end{minipage}
\begin{minipage}[b]{0.32\linewidth}
        %\subfloat[]{\label{main:d}
        %        \includegraphics[scale=.18]{/tera/phil/nchaparr/python/Plotting/Mar52014/pngs/ML_Height_hist}}\\
       
       %\subfloat[]{\label{main:e}
       %         \includegraphics[scale=.18]{/tera/phil/nchaparr/python/Plotting/Mar52014/pngs/ML_Height_hist}}\\
       \vspace{10mm} 
       \subfloat[]{\label{main:f}
                \includegraphics[scale=.22]{/tera/phil/nchaparr/python/Plotting/Dec252013/pngs/fluxquadhist0}}                 

\end{minipage}
\label{fig:fluxquadsh0}
\caption{ $\overline{w^{,}\theta^{,}}$ quadrants at $h_{0}$ for $w^{,}\theta^{,} = 150 - 60$ (W/$m^{2}$) (a-c) and $\gamma = 10 - 2.5$ (K/Km) (c-f-g) at 5 hours}
\end{figure}

At $h$ (Fig \ref{fig:fluxquadsh}) the faster updraughts are now relatively cool and movement (both up and down) of warmer air from aloft
becomes more prominent.  Damping of cool updraught speed and increase in $\theta^{,}$ with increasing $\gamma$
is evident.\\

\begin{figure}[htbp]
%
\caption{ $\overline{w^{,}\theta^{,}}$ quadrants at $h$ for $w^{,}\theta^{,} = 150 \ to \ 60$(W/$m^{2}$) (top to bottom) and $\gamma = 10 \ to \  2.5$(K/Km) (left to right) at 5 hours}
\begin{minipage}[b]{0.32\linewidth} 
        
        \subfloat[]{\label{main:a}
                \includegraphics[scale=.22]{/tera/phil/nchaparr/python/Plotting/Mar52014/pngs/fluxquadhist1}}\\
        \subfloat[]{\label{main:b}      
                \includegraphics[scale=.22]{/tera/phil/nchaparr/python/Plotting/Dec142013/pngs/fluxquadhist1}}\\ 
        \subfloat[]{\label{main:c}          
                \includegraphics[scale=.22]{/tera/phil/nchaparr/python/Plotting/Mar12014/pngs/fluxquadhist1}} 
 \end{minipage}             
%
\begin{minipage}[b]{0.32\linewidth}
        \subfloat[]{\label{main:d}
                \includegraphics[scale=.22]{/tera/phil/nchaparr/python/Plotting/Jan152014_1/pngs/fluxquadhist1}}\\
       \subfloat[]{\label{main:e}
                \includegraphics[scale=.22]{/tera/phil/nchaparr/python/Plotting/Nov302013/pngs/fluxquadhist1}}\\
       \subfloat[]{\label{main:f}
                \includegraphics[scale=.22]{/tera/phil/nchaparr/python/Plotting/Dec202013/pngs/fluxquadhist1}}                 
\end{minipage}
\begin{minipage}[b]{0.32\linewidth}
        %\subfloat[]{\label{main:d}
        %        \includegraphics[scale=.18]{/tera/phil/nchaparr/python/Plotting/Mar52014/pngs/ML_Height_hist}}\\
       
       %\subfloat[]{\label{main:e}
       %         \includegraphics[scale=.18]{/tera/phil/nchaparr/python/Plotting/Mar52014/pngs/ML_Height_hist}}\\
       \vspace{10mm} 
       \subfloat[]{\label{main:f}
                \includegraphics[scale=.22]{/tera/phil/nchaparr/python/Plotting/Dec252013/pngs/fluxquadhist1}}                 
\end{minipage}
\label{fig:fluxquadsh}

\end{figure}

At the top of the EL (Fig \ref{fig:fluxquadsh}) velocities are damped and the distributions approach symmetry appart from some slow, cool updraughts
as in Fig \ref{fig:conts2}. \\


\begin{figure}[htbp]
%
\caption{ $\overline{w^{,}\theta^{,}}$ quadrants at $h_{1}$ for $w^{,}\theta^{,} = 150 \ to \ 60$(W/$m^{2}$) (top to bottom) and $\gamma = 10 \ to \ 2.5$(K/Km) (left to right) at 5 hours}
\begin{minipage}[b]{0.32\linewidth} 
        
        \subfloat[]{\label{main:a}
                \includegraphics[scale=.22]{/tera/phil/nchaparr/python/Plotting/Mar52014/pngs/fluxquadhist2}}\\
        \subfloat[]{\label{main:b}      
                \includegraphics[scale=.22]{/tera/phil/nchaparr/python/Plotting/Dec142013/pngs/fluxquadhist2}}\\ 
        \subfloat[]{\label{main:c}          
                \includegraphics[scale=.22]{/tera/phil/nchaparr/python/Plotting/Mar12014/pngs/fluxquadhist2}} 
 \end{minipage}             
%
\begin{minipage}[b]{0.32\linewidth}
        \subfloat[]{\label{main:d}
                \includegraphics[scale=.22]{/tera/phil/nchaparr/python/Plotting/Jan152014_1/pngs/fluxquadhist2}}\\
       \subfloat[]{\label{main:e}
                \includegraphics[scale=.22]{/tera/phil/nchaparr/python/Plotting/Nov302013/pngs/fluxquadhist2}}\\
       \subfloat[]{\label{main:f}
                \includegraphics[scale=.22]{/tera/phil/nchaparr/python/Plotting/Dec202013/pngs/fluxquadhist2}}                 
\end{minipage}
\begin{minipage}[b]{0.32\linewidth}
        %\subfloat[]{\label{main:d}
        %        \includegraphics[scale=.18]{/tera/phil/nchaparr/python/Plotting/Mar52014/pngs/ML_Height_hist}}\\
       
       %\subfloat[]{\label{main:e}
       %         \includegraphics[scale=.18]{/tera/phil/nchaparr/python/Plotting/Mar52014/pngs/ML_Height_hist}}\\
       \vspace{10mm} 
       \subfloat[]{\label{main:f}
                \includegraphics[scale=.22]{/tera/phil/nchaparr/python/Plotting/Dec252013/pngs/fluxquadhist2}}                 
\end{minipage}

\label{fig:fluxquadsh1}
\end{figure}

\clearpage

\section{$h$ and  $\Delta h$ based on Average Profiles}
\label{sec:hdeltahavprofs}

\FloatBarrier

\subsection{Reminder of Relevant Definitions}
\FloatBarrier
%definitions
\begin{figure}[htbp]
    \centering
    %plot_height.py[master 1573b9d] h vs time plot
    \includegraphics[scale=.5]{/tera/phil/nchaparr/python/Plotting/Dec252013/pngs/height_defs}
    \caption{$h$ vs time for all runs}
    \label{fig:hvstime}   % label should change
\end{figure}

\begin{table}[!ht]
    \begin{center}
    \begin{tabular}{ | p{3cm} | p{3cm} | p{3cm} |}
    \hline
     Sullivan, Moeng & Fedorovich, Conzemius &  Garcia, Mellado\\ \hline
     $h$ & $z_{f}$& $z_{enc} \approx z_{f0}$ \\ \hline
     $\Delta \theta = \overline{\theta}(z_{f1})-\overline{\theta}(z_{f})$ & $\Delta b = b_{0}(z_{f}) -b(z_{f})$ & $\Delta b = b_{0}(z_{f}) - b(z_{f}) $\\ \hline
      & $\delta b = b(z_{f1})$ - $b(z_{f0})$ & $\delta b = b(z_{f1})$ - $b(z_{f0})$\\ \hline
      $w_{*}= (h B_{s})^{\frac{1}{3}}$, $B_{s} = \frac{g}{\overline{\theta_{ML}}}\overline{w^{'}\theta^{'}}_{s}$ & $w_{*}= (z_{f} B_{s})^{\frac{1}{3}}$ & $w_{*}= (z_{f0} B_{s})^{\frac{1}{3}}$\\ \hline
      $\tau = \frac{h}{w^{*}}$ & $\tau=N^{-1}$ & $\frac{t}{\tau}=\frac{z_{enc}}{L_{0}}$, $L_{0}=\left(\frac{B_{s}}{N^{3}}\right)^{\frac{1}{2}}$\\ \hline
 
   
\end{tabular}
\caption{Relevant definitions from three key publications}
\label{fig:}   
\end{center}    
\end{table}
%\footnotetext[1]{Fedorovich et al and Garcia et al use buoancy instead of $\theta$, but they are interchangeable for this purpose}



\subsection{$\frac{w_{e}}{w^{*}}$ vs $Ri^{-1}$}
\FloatBarrier
Covective Boundary Layer (CBL) height ($h$) (Fig. \ref{fig:hvstime}) grows rapidly initially with a steadily decreasing rate
and is a linear function of scaled time (Fig \ref{fig:hvsscaledtime}).\\
  
\begin{figure}[htbp]
    \centering
    %plot_height.py[master 1573b9d] h vs time plot
    \includegraphics[scale=.5]{/tera/phil/nchaparr/python/Plotting/Dec252013/pngs/hvstime}
    \caption{$h$ vs time for all runs}
    \label{fig:hvstime}   % label should change
\end{figure}

%need flux height scaled by h plot

\begin{figure}[htbp]
    \centering
    %plot_height.py[master 1573b9d] h vs time plot
    \includegraphics[scale=.5]{/tera/phil/nchaparr/python/Plotting/Dec252013/pngs/scaled_h_f_time}
    \caption{$\frac{z_{f}}{h}$ vs Time}
    \label{fig:hvstime}   % label should change
\end{figure}


%redo Ri with Delta theta (?)

%Do a log log plot of the below? to confirm the -1 slope? 

%get slope to compare with other work.

Inverse Richardson Numbers (\acs{Ri}$^{-1}$) (bulk \acs{Ri}: $=\frac{gh}{\overline{\theta_{ML}}} \frac{\Delta \theta}{w^{*2}}$, $\Delta \theta = \overline{\theta}(h_{1})-\overline{\theta}(h_{0})$ 
and gradient \acs{Ri}: $=\frac{g}{\overline{\theta_{ML}}} \frac{\gamma h^{2} }{w^{*2}}$) decrease with respect to time, 
clustering according to $\gamma$. (Fig. \ref{fig:invristime})\\

\begin{figure}[htbp]

\begin{minipage}[b]{0.5\linewidth}
         
        \subfloat[]{\label{main:a}
                \includegraphics[scale=.36]{/tera/phil/nchaparr/python/Plotting/Dec252013/pngs/invristime}}\\
        \end{minipage}             
\quad
\begin{minipage}[b]{0.5\linewidth}
        \subfloat[]{\label{main:d}
          %plot_height [master fbd2dfd] invristime1, see branches named accordingly
                \includegraphics[scale=.36]{/tera/phil/nchaparr/python/Plotting/Dec252013/pngs/invristime1}}\\
       
       \end{minipage}
        \caption{Inverse bulk (a)  and gradient (b) Richardson Number vs time}
        \label{fig:invristime}
\end{figure}

The entrainment rate ($w_{e}= \frac{dh}{dt}$) is determined from the slope of a second order polynomial fit to $h(time)$ (Fig. \ref{fig:hvstime}).  When scaled by ($w^{*}$) it is a roughly linear function 
of the bulk $Ri^{-1}$, whereas it is asymptotic to a linear function of the gradient $Ri^{-1}$. (Fig. \ref{fig:scaledweinvri})\\    

\begin{figure}[htbp]
\begin{minipage}[b]{0.5\linewidth}
        %plot_height [master fbd2dfd] invristime1
        \subfloat[]{\label{main:a}
                \includegraphics[scale=.36]{/tera/phil/nchaparr/python/Plotting/Dec252013/pngs/scaledweinvri}}\\
        \end{minipage}             
\quad
\begin{minipage}[b]{0.5\linewidth}
        \subfloat[]{\label{main:d}          
          %plot_height [master 01c3721] scaledweinvri1
                \includegraphics[scale=.36]{/tera/phil/nchaparr/python/Plotting/Dec252013/pngs/scaledweinvri1}}\\
       
       \end{minipage}
        \caption{Scaled Entrainment rate vs inverse bulk (a)  and gradient (b) Richardson Number}
        \label{fig:scaledweinvri}
\end{figure}

\subsection{$\frac{\Delta h}{h}$ vs $Ri^{-1}$}
\FloatBarrier

EL depth ($h_{1}-h_{0}=\Delta h$) (Fig. \ref{fig:deltahvstime1}) increases with time
although not as rapidly as $h$.  It deepens with increasing $\overline{w^{,}\theta^{,}_{s}}$ 
and narrows with increasing $\gamma$. \\

\begin{figure}[htbp]
    \centering
    %plot_height.py [master fd5c6b1] delta h vs time1  
    \includegraphics[scale=.5]{/tera/phil/nchaparr/python/Plotting/Dec252013/pngs/deltahstime1}
    \caption{$\Delta h$ vs time}
    \label{fig:deltahvstime1}   % label should change
\end{figure}

The scaled upper EL limits (Fig. \ref{fig:scaleddeltahstime}) ($\frac{h_{1}}{h}$) collapse well 
to an initial value of approximately 1.75, decreasing to about 1.5 .  $\frac{h_{0}}{h}$s appear 
grouped according to $\gamma$ and increase with respect to time.\\  

%need flux based el limits scaled by either h or hf

Although the scaled \acs{EL} depth decreases with increasing \acs{Ri} (gradient and bulk Fig \ref{fig:scaledeltahinvri})
grouping according to $\gamma$ is evident.\\

\begin{figure}[htbp]
\begin{minipage}[b]{0.5\linewidth}
        %plot_height [master c7af4de] scaleddeltahinvri
        \subfloat[]{\label{main:a}
                \includegraphics[scale=.36]{/tera/phil/nchaparr/python/Plotting/Dec252013/pngs/scaleddeltahinvri}}\\
        \end{minipage}             
\quad
\begin{minipage}[b]{0.5\linewidth}
        \subfloat[]{\label{main:d}          
          %plot_height [master b9c30ad] scaleddeltahinvri1
                \includegraphics[scale=.36]{/tera/phil/nchaparr/python/Plotting/Dec252013/pngs/scaleddeltahinvri1}}\\
       
       \end{minipage}
        \caption{Scaled EL depth vs inverse bulk (a)  and gradient (b) Richardson Number}
        \label{fig:scaledeltahinvri}
\end{figure}


%need bit about new definitions

\begin{figure}[htbp]
\begin{minipage}[b]{0.5\linewidth}
        %plot_height [master c7af4de] scaleddeltahinvri
        \subfloat[]{\label{main:a}
                \includegraphics[scale=.36]{/tera/phil/nchaparr/python/Plotting/Dec252013/pngs/scaleddeltahinvri_4}}\\
        \end{minipage}             
\quad
\begin{minipage}[b]{0.5\linewidth}
        \subfloat[]{\label{main:d}          
          %plot_height [master b9c30ad] scaleddeltahinvri1
                \includegraphics[scale=.36]{/tera/phil/nchaparr/python/Plotting/Dec252013/pngs/scaleddeltahinvri1_1}}\\
       
       \end{minipage}
        \caption{}
        \label{fig:}
\end{figure}



\endinput

Any text after an \endinput is ignored.
You could put scraps here or things in progress.
