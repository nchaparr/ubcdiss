%% The following is a directive for TeXShop to indicate the main file
%%!TEX root = diss.tex

\chapter{Results}
\label{ch:results}
\setlength{\parindent}{0cm}

\section{Description of Runs}
\FloatBarrier

All 10 member cases of the ensemble were carried out on a 3.2 x 4.8 Km horizontal 
domain ($\Delta x = \Delta y = 25m$, $nx=128$, $ny=192$).  
$nx$, $ny$ were chosen based on the optimal distribution accross processor nodes.  
The vertical grid ($nz=312$) was of higher resolution around the 
entrainment layer (\acs{EL}) ($\Delta z = 5m$), and lower below and above it 
($\Delta z = 10 \ to \ 100 m$). Grid size was chosen so that
a full spectrum of turbulence would be resolved within the \acs{EL} 
in line with the findings of \citeauthor{SullPat} in \cite{SullPat}.  The 7 runs
vary depending on surface heat flux ($\overline{w^{'}\theta^{'}_{s}}$) 
and initial lapse rate ($\gamma$).

%description of runs ie 10 member ensembles each had delta x, delta y=25 and a region of delta z=5m enclosing the 
%Entrainment zone, z=25 below  and streched to 100 above. 

\label{sec:Runs}

\begin{table}[!ht]
    \begin{center}
    \begin{tabular}{ | l | l | l | l |}
    \hline
    $\overline{w^{'}\theta^{'}_{s}}$ / $\gamma$ & 10 (K/Km) & 5 (K/Km) & 2.5 (K/Km) \\ \hline
     150 (W/m2)& \hspace{5mm} \ding{51} &\hspace{5mm} \ding{51}\footnotemark &  \\ \hline
     100 (W/m2)& \hspace{5mm} \ding{51} & \hspace{5mm} \ding{51} & \\ \hline
     60 (W/m2) & \hspace{5mm} \ding{51} & \hspace{5mm} \ding{51} & \hspace{5mm} \ding{51}\\ \hline
     
%\end

   
\end{tabular}
\caption{Runs in terms of $\overline{w^{'} \theta^{'}_{s}}$ and initial lapse rate $\gamma$}
\label{fig:tableofruns}   
\end{center}    
\end{table}
\footnotetext{Incomplete run: EL exceded high resolution vertical grid after 7 hours}

<<<<<<< HEAD
\clearpage

\section{Relevant Definitions}
\FloatBarrier

Here, the \acs{CBL} height and \acs{EL} limits are defined based on the vertical  $\frac{\partial \overline{\theta}}{\partial z}$
profile.  Namely, the \acs{CBL} height $h$ is the point where  $\frac{\partial \overline{\theta}}{\partial z}$ is maximum, the 
lower \acs{EL} limit is the point at which  $\frac{\partial \overline{\theta}}{\partial z}$ first increases
significantly from zero i.e. exceeds a threshold value above the surface layer, and the upper \acs{EL} limit $h_{1}$ is the point where
$\frac{\partial \overline{\theta}}{\partial z}$ resumes $\gamma$. (Figure \ref{fig:hdefs})\\

As \citeauthor{BrooksFowler2} point out in \cite{BrooksFowler2}, when using an average vertical 
tracer profile there is no universal critereon for a significant gradient.  So a threshold value for the
lower \acs{EL} limit ($h_{0}$) was chosen such that it was positive, small i.e. an order of magnitude 
less than $\gamma$ and the same for all runs.  For the sake of rigor, the main corresponding result 
was calculated based on two additional threshold values in Section \ref{subsec:deltahri}.\\

The temperature jump is defined here as the difference in $\overline{\theta}$ accross the \acs{EL}.
So, it is larger than those used by  \citeauthor{FedConzMir04} in \cite{FedConzMir04} to verify 
their zero order model and \citeauthor{SullMoengStev} in \cite{SullMoengStev} (Table \ref{table:reldefs}).
We also define a zero-order type temperature jump as the difference between the mixed layer (\acs{ML}) 
average potential temperature ($\overline{\theta}$) less ($\overline{\theta}$) at $h$ on the initial profile.
\\ 
 
%table outlining my definitions, and other comparable ones, ie brooks and fowlers aren't comparable
% re justifying my arbitrary choice of threshold for the lower limit:  seee page 250 Brooks and Fowler. 
%there must be other places in my key papers where the 'arbitraryness' is referred to.
\begin{figure}[htbp]
    \centering
    %
    \includegraphics[scale=.5]{/tera/phil/nchaparr/python/Plotting/Dec252013/pngs/height_defs}
    \caption{Height Definitions}
    \label{fig:hdefs}   % label should change
\end{figure}

\begin{table}[htbp]
    %\begin{center}
%\centerline{
    \begin{tabular}{p{4cm}p{4cm}p{4cm}}
    \hline
      CBL Height & $\theta$ Jump & Richardson Number   \\ \hline 
      $h$ & $\Delta \theta = \overline{\theta}(h_{1}) - \overline{\theta}(h_{0})$ & \acs{Ri}$=\frac{\frac{g}{\overline{\theta}} \delta \theta h}{w^{*2}}$\\
          & $\delta \theta = \overline{\theta}_{ML}-\overline{\theta}_{0}(h)$ & \acs{Ri}$_{\delta}=\frac{\frac{g}{\overline{\theta}} \Delta \theta h}{w^{*2}}$\\ \hline      
    \end{tabular}
%}
\caption{Relevant Definitions used in this Study}
\label{table:reldefs}   
%\end{center}    
\end{table}

\clearpage

\section{Verifying the Model Output}
\label{sec:CheckingtheModel}
\subsection{Time till well-mixed}%Spin Up Time according to the Convective Time Scale $\tau$}
\FloatBarrier

Time must be allowed to establish statistically steady turbulent flow.  \citeauthor{SullMoengStev} in 
\cite{SullMoengStev} recommended 10 eddie turnover times based on the convective time scale 
$\tau = \frac{h}{w^{*}} = \frac{h}{ \left( \frac{gh}{\overline{\theta}_{ML}}(\overline{w^{'} \theta^{'}_{s}}) \right)^{\frac{1}{3}} } $, 
and \citeauthor{BrooksFowler2} in \cite{BrooksFowler2} chose a simulated time of 2 hours.  For all of 
the runs, at least 10 eddie turnover times were completed by 2 simulated hours (Figure \ref{fig:ScaledTimevsTime}).  
Although each run has a distinct convective velocity scale that increases with time ($w^{*}(time)$), 
dividing boundary layer height ($h$) by it to obtain $\tau$ results in a collapse from 7 to 3 curves, 
one for each $\gamma$.\\

A measureable well mixed layer (\acs{ML}) and \acs{EL} based on the horizontaly averaged, ensemble averaged
potential temperature ($\overline{\theta}$) profile develops after 2 hours 
(Figure \ref{fig:tempgradfluxprofs15010}).  After 2 or 3 hours the \acs{EL} is fully contained within the vertical 
region of high resolution.\\

Averaged heat fluxes ($\overline{w^{'}\theta^{'}}$) (Figure \ref{fig:scaledfluxprofs15010}) and 
root mean squared vertical velocity perturbations ($\sqrt{w^{'2}}$) (Figure \ref{fig:rmswvelprofs15010})
become self similar and are scaled well by the surface heat flux ($\overline{w^{'}\theta^{'}}_{s}$) 
and the convective velocity scale ($w^{*}$) respectively after 2 hours.\\
=======

\section{Relevant Definitions}
%table outlining my definitions, and other comparable ones, ie brooks and fowlers aren't comparable
% re justifying my arbitrary choice of threshold for the lower limit:  seee page 250 Brooks and Fowler. 
%there must be other places in my key papers where the 'arbitraryness' is referred to.


\section{Verifying Model Output}
\label{sec:CheckingtheModel}
\subsection{}%Spin Up Time according to the Convective Time Scale $\tau$}
\FloatBarrier

Time must be allowed to establish statistically steady turbulent flow.  \citeauthor{SullMoengStev} in \cite{SullMoengStev}
recommended 10 eddie turnover times based on the convective time scale $\tau = \frac{h}{w^{*}} = \frac{h}{ \left( \frac{gh}{\overline{\theta}_{ML}}(\overline{w^{,} \theta^{,}_{s}}) \right)^{\frac{1}{3}} } $, 
and \citeauthor{BrooksFowler2} in \cite{BrooksFowler2} chose a simulated time of 2 hours.  For all of the runs, 10 eddie 
turnover times were completed by 2 hours (Figure \ref{fig:ScaledTimevsTime}).  Although each run has a distinct
convective velocity scale, that increases with time ($w^{*}(time)$), dividing boundary layer height ($h$) by it
to obtain $\tau$ results in a collapse from 7 to 3 curves, one for each $\gamma$.\\
>>>>>>> added discussion


\begin{figure}[!h]
    \centering
    % plot_height.py [master 03d2835] Round 1 of Plots in Results 
    \includegraphics[scale=.5]{/tera/phil/nchaparr/python/Plotting/Dec252013/pngs/scaledtimevstime}
    \caption{Plots of scaled time vs time for all runs.  Scaled time is based on the convective time scale 
    and can be thought of as the number of times an eddie has reached the top of the CBL. }
    \label{fig:ScaledTimevsTime}   
\end{figure}

\begin{figure}[htbp]
    \centering
    % plot_dthetaflux.py [master 03d2835] Round 1 of Plots in Results
    \includegraphics[scale=.5]{/tera/phil/nchaparr/python/Plotting/Mar52014/pngs/theta_flux_profs}
    \caption{Vertical profiles of the ensemble and horizontally averaged potential temperature ($\overline{\theta}$), its vertical gradient ($\frac{\partial \overline{\theta}}{\partial z}$)  
     and heat flux ($\overline{w^{'}\theta^{'}}$) for the 150/10 run}
    \label{fig:tempgradfluxprofs15010}   % label should change
\end{figure}

\begin{figure}[htbp]
    \centering
    % plot_dthetaflux.py [master 9883fda] Round 2 of Plots in Results
    %\includegraphics[scale=.5]{/tera/phil/nchaparr/python/Plotting/Mar52014/pngs/scaled_theta_flux_profs}
    %[master 41d073a] Mar52014 scaled_flux_profs by plot_dthetaflux.py
    \includegraphics[scale=.5]{/tera/phil/nchaparr/python/Plotting/Mar52014/pngs/scaled_flux_profs}
    \caption{$\overline{w^{'}\theta^{'}}$ and scaled $\overline{w^{'}\theta^{'}}$  vs scaled height for the 150/10 run}
    \label{fig:scaledfluxprofs15010}   % label should change
\end{figure}

\begin{figure}[htbp]
    \centering
    %w_analysis.py [master 03d2835] Round 1 of Plots in Results
    \includegraphics[scale=.5]{/tera/phil/nchaparr/python/Plotting/Mar52014/pngs/rmswvels}
    \caption{$\sqrt{w^{,2}}$ vs scaled height for the 150/10 run}
    \label{fig:rmswvelprofs15010}   % label should change
\end{figure}

\clearpage
<<<<<<< HEAD
\subsection{FFT Energy Spectra}
\FloatBarrier

Two dimensional \acs{FFT} power spectra taken of horizontal slices of $w^{'}$ 
(Figure \ref{fig:2fftw602point5}) at three different levels ($h_{0}$, $h$ and $h_{1}$) are collapsed to 
one dimension by integrating around a circle of wave-number radius $k$.  Isotropy in all radial 
directions is assumed and $k = \sqrt{k_{x}^{2} + k_{y}^{2}}$.  \\

The resulting scalar density spectra show peaks in  energy at the larger scales, cascading to the lower 
scales roughly according to a $\frac{-5}{3}$ slope, lower in the \acs{EL}.  At the top of the \acs{EL} 
where turbulence is supressed by stability, the slope is steeper.  The peak in energy occurs at smaller 
scales at the inversion ($h$) as compared to at the bottom of the \acs{EL} ($h_{0}$), indicating a 
change in the size of the dominant turbulent structures further into the entrainment layer (\acs{EL}).\\

\begin{figure}[htbp]
    \centering
    % fft_chap.py [master 03d2835] Round 1 of Plots in Results
    \includegraphics[scale=.5]{/tera/phil/nchaparr/python/Plotting/Dec252013/pngs/scalarfftpow}
    \caption{Scalar FFT  energy vs wavenumber ($k = \sqrt{k_{x}^{2}+k_{y}^{2}}$) for the 60/2.5 run
at 2 hours.  $E(k)$ is $E(k_{x}, k_{y})$ integrated around circles of radius $k$.  
   $E(k_{x}, k_{y})$ is the total integrated energy over the 2D domain.  
   $k_{x}$ and $k_{y}$ are number of waves per domain length.}
    \label{fig:2fftw602point5}   % label should change
\end{figure}

\clearpage

\subsection{Ensemble and horizontally averaged vertical Potential Temperature $\overline{\theta}$ 
and Heat Flux profiles $\overline{w^{'}\theta^{'}}$}
%Average Potential Temperature, Heat Flux and Kinetic Energy}
=======
\subsection{}
\FloatBarrier

Two dimensional \acs{FFT} power spectra taken of horizontal slices of $w^{,}$ (\ref{fig:scalardfftw602point5}) at three
different levels ($h_{0}$, $h$ and $h_{1}$) are collapsed to one dimension by integrating around a semi-circle of positive wave-numbers.
Isotropy in all radial directions is assumed and $k = \sqrt{k_{x}^{2} + k_{y}^{2}}$.  The resulting scalar density spectra show peaks in 
energy at the larger scales, cascading to the lower scales roughly according to a $\frac{-5}{3}$ slope lower in the \acs{EL}.  At
the top of the \acs{EL} where turbulence is supressed by stability, the slope is steeper.  The peak in energy occur at smaller scales
at $h$ as compared to at $h_{0}$, indicating a change in the size of the dominant turbulent structures.\\

\begin{figure}[htbp]
    \centering
    %fft_nchap.py [master 03d2835] Round 1 of Plots in Results
    \includegraphics[scale=.5]{/tera/phil/nchaparr/python/Plotting/Dec252013/pngs/2dfftpow}
    \caption{2D FFT energy densities of $w^{,}$ at $h_{0}$ vs wavenumber for the 60/2.5 run}
    \label{fig:scalardfftw602point5}   % label should change
\end{figure}

\begin{figure}[htbp]
    \centering
    % fft_chap.py [master 03d2835] Round 1 of Plots in Results
    \includegraphics[scale=.5]{/tera/phil/nchaparr/python/Plotting/Dec252013/pngs/scalarfftpow}
    \caption{Scalar FFT  energy densities vs wavenumber ($k = \sqrt{k_{x}^{2}+k_{y}^{2}}$) for the 60/2.5 run}
    \label{fig:2fftw602point5}   % label should change
\end{figure}

\clearpage

\subsection{}%Average Potential Temperature, Heat Flux and Kinetic Energy}
>>>>>>> added discussion
\FloatBarrier

The $\overline{\theta}$ profiles exhibit an \acs{ML} above which  $\frac{\partial\overline{\theta}}{\partial z}>0$ 
and reaches a maximum value at $h$ before resuming $\gamma$  at $h_{1}$ 
(Figures \ref{fig:tempgradfluxprofs15010} and \ref{fig:pottempprofs2hrs}).  Convective boundary layer \acs{CBL} growth is stimulated 
by $\overline{w^{'}\theta^{'}}_{s}$ and inhibited by $\gamma$.\\

The horizonally averaged, ensemble averaged heat flux ($\overline{w^{'}\theta^{'}}$) profiles decrease 
from the surface value ($\overline{w^{'}\theta^{'}_{s}}$) passing through zero to a minumum before 
increasing to zero (Figures \ref{fig:tempgradfluxprofs15010} and  \ref{fig:fluxprofs2hrs}).  All minima are 
less  in magnitude than the zero order approximation ($-.2 \times \overline{w^{'}\theta^{'}_{s}}$).\\


\begin{figure}[htbp]
    \centering
    % plot_theta_profs.py [master 9883fda] Round 2 of Plots in Results
    \includegraphics[scale=.5]{/tera/phil/nchaparr/python/Plotting/Dec252013/pngs/theta_profs2hrs}
    \caption{$\overline{\theta}$ profiles at 2 hours}
    \label{fig:pottempprofs2hrs}   % label should change
\end{figure}

\begin{figure}[htbp]
    \centering
    %plot_theta_profs.py [master 03d2835] Round 1 of Plots in Results
    \includegraphics[scale=.5]{/tera/phil/nchaparr/python/Plotting/Dec252013/pngs/flux_profs2hrs}
    \caption{Scaled $\overline{w^{'}\theta^{'}}_{s}$ profiles at 2 hours}
    \label{fig:fluxprofs2hrs}   % label should change
\end{figure}

%Each of the $\overline{\theta}$ and $\overline{w^{'}\theta^{'}}$ profiles has a region that can be defined as an \acs{EL}.
%Here we use the former definition.  The point of minimum $\overline{w^{'}\theta^{'}}$ is lower than $h$.  
%\citeauthor{SullMoengStev} in \cite{SullMoengStev} noted that the upper extrema of the individual flux quadrant profiles 
%are closer or even coincide with $h$.\\

<<<<<<< HEAD
%($\sqrt{u^{,2}}$) profiles show a dominance of vertical velocity perturbation ($w^{'}$) in the \acs{ML}, with a peak in horizontal velocity
%($u^{'}$ and $v^{'}$) within the \acs{EL} where the buoyancy driven $w^{'}$ is inhibited by stability (Figure \ref{fig:rmsvel150102hrs}). \\

%\begin{figure}[htbp]
%    \centering
    %plot_vars.py [master 03d2835] Round 1 of Plots in Results
%    \includegraphics[scale=.5]{/tera/phil/nchaparr/python/Plotting/Mar52014/pngs/rmsvel2}
%    \caption{Vertical $\frac{\sqrt[]{u^{,2}}}{w^{*}}$ profiles at 2 hours for the 150/10 run}
%    \label{fig:rmsvel150102hrs}   % label should change
%\end{figure}
=======
\begin{figure}[htbp]
    \centering
    %plot_vars.py [master 03d2835] Round 1 of Plots in Results
    \includegraphics[scale=.5]{/tera/phil/nchaparr/python/Plotting/Mar52014/pngs/rmsvel2}
    \caption{Vertical $\frac{\sqrt[]{u^{,2}}}{w^{*}}$ profiles at 2 hours for the 150/10 run}
    \label{fig:rmsvel150102hrs}   % label should change
\end{figure}
>>>>>>> added discussion

\clearpage

\subsection{Visualization of Structures Within the Entrainment Layer}
\FloatBarrier

Horizontal slices, at the three entrainment layer (\acs{EL}) levels, of the potential temperature 
and vertical velocity perturbations are plotted to see the turbulent structures.  At the bottom of the \acs{EL} ($h_{0}$) 
in the 150/10 run (Figure \ref{fig:conts} (a) and (d)) coherent areas of positive and negative temperature perturbations 
correspond to areas of upward and downward moving air.\\

The individual plumes of relatively cool air are more evident at the inversion ($h$) and their 
locations correspond to areas of upward motion ((b) and (e)).  Most of the upward moving cool areas are adjacent to and even 
encircled by smaller areas of downward moving warm air.  At $h_{1}$ ((c) and (f)) peaks of cool air are associated 
with both up and down-welling.\\  

In the 60/2.5 run (Figure \ref{fig:conts1}) a similar progression is evident but the impinging, cool upward moving
plumes are more defined.  This is to be expected since stronger stability inhibits deformation of the 
inversion interface.\\   

\begin{figure}[htbp]
\caption{$\theta^{'}$ (left) and $w^{'}$ (right) at 2 hours at $h_{0}$ (a,d), $h$ (c,e) and $h_{1}$ (d,f)}
\begin{minipage}[b]{0.5\linewidth}
  
        %Flux_Quads.py [master 03d2835] Round 1 of Plots in Results
        \subfloat[]{\label{main:a}
                \includegraphics[scale=.36]{/tera/phil/nchaparr/python/Plotting/Mar52014/pngs/theta_cont0}}\\
        \subfloat[]{\label{main:b}      
                \includegraphics[scale=.36]{/tera/phil/nchaparr/python/Plotting/Mar52014/pngs/theta_cont1}}\\ 
        \subfloat[]{\label{main:c}      
                \includegraphics[scale=.36]{/tera/phil/nchaparr/python/Plotting/Mar52014/pngs/theta_cont2}} 
 \end{minipage}             
\quad
\begin{minipage}[b]{0.5\linewidth}
        %Flux_Quads.py [master 9883fda] Round 2 of Plots in Results
        \subfloat[]{\label{main:d}
                \includegraphics[scale=.36]{/tera/phil/nchaparr/python/Plotting/Mar52014/pngs/wvel_cont0}}\\
       
       \subfloat[]{\label{main:e}
                \includegraphics[scale=.36]{/tera/phil/nchaparr/python/Plotting/Mar52014/pngs/wvel_cont1}}\\
        
       \subfloat[]{\label{main:f}
                \includegraphics[scale=.36]{/tera/phil/nchaparr/python/Plotting/Mar52014/pngs/wvel_cont2}}                 
\end{minipage}
        
        \label{fig:conts}
\end{figure}

\begin{figure}[htbp]
\caption{$\theta^{'}$ (left) and $w^{'}$ (right) at 2 hours at $h_{0}$ (a,d), $h$(b,e) and $h_{1}$(c,f)}
\begin{minipage}[b]{0.5\linewidth} 
        
        \subfloat[]{\label{main:a}
                \includegraphics[scale=.36]{/tera/phil/nchaparr/python/Plotting/Dec252013/pngs/theta_cont0}}\\
        \subfloat[]{\label{main:b}      
                \includegraphics[scale=.36]{/tera/phil/nchaparr/python/Plotting/Dec252013/pngs/theta_cont1}}\\ 
        \subfloat[]{\label{main:c}      
                \includegraphics[scale=.36]{/tera/phil/nchaparr/python/Plotting/Dec252013/pngs/theta_cont2}} 
 \end{minipage}             
\quad
\begin{minipage}[b]{0.5\linewidth}
        \subfloat[]{\label{main:d}
                \includegraphics[scale=.36]{/tera/phil/nchaparr/python/Plotting/Dec252013/pngs/wvel_cont0}}\\
       
       \subfloat[]{\label{main:e}
                \includegraphics[scale=.36]{/tera/phil/nchaparr/python/Plotting/Dec252013/pngs/wvel_cont1}}\\
        
       \subfloat[]{\label{main:f}
                \includegraphics[scale=.36]{/tera/phil/nchaparr/python/Plotting/Dec252013/pngs/wvel_cont2}}                 
\end{minipage}
        
        \label{fig:conts1}
\end{figure}

\clearpage


\section{Local Mixed Layer Heights ($h_{0}^{l}$)}
\label{sec:locmlh}     
\FloatBarrier

<<<<<<< HEAD
Local $\theta$ profiles (Figures \ref{fig:rssfitshigh} and \ref{fig:rssfitslow}) exhibit a distinct 
\acs{ML} before resuming $\gamma$ but 
not always a clearly defined \acs{EL}.  There are sharp changes in the profile well into the free 
atmosphere, due possibly to waves, which render the gradient method for determining $h^{l}$ 
unusable.  Instead a linear regression method is used, whereby three lines representing: the
 \acs{ML}, the \acs{EL} and the upper lapse rate ($\gamma$), are fit to the profile according 
to the minimum residual sum of squares (RSS).  Determining local \acs{ML} height ($h_{0}^{l}$) was 
more straight forward than the local height of maximum potential temperature gradient 
($h^{l}$) for the reasons stated above.\\  

Figure \ref{fig:rssfitshigh} shows two local $\theta$ profiles where $h_{0}^{l}$ is relatively high.  
A sharp interface is evident indicating that this is within an active plume impinging on the stable layer.
In Figure \ref{fig:rssfitslow} where $h_{0}^{l}$ is relatively low a less defined interface indicates 
a point now outside a rising plume.  Contour plots (Figure \ref{fig:conts2}) show regions of high 
$h_{0}^{l}$ corresponding to regions of upward moving relatively cool air at $h$.\\

The distribution of $h_{0}^{l}$ is related to the depth of the entrainment layer (\acs{EL}).
Spread increases with increasing $\overline{w^{'}\theta^{'}_{s}}$ and decreases with increasing $\gamma$
(Figure \ref{fig:localhhist}).  When scaled by $h$  (Figure \ref{fig:localhpdf}), the local \acs{ML} height distribution 
has spread that narrows with increased $\gamma$ and seems relatively uninfluenced by change in $\overline{w^{'}\theta^{'}}_{s}$.  
The upper limit seems to be constant at about 1.1($\times h$) , whereas the lower limit varies 
depending on $\gamma$.   Runs with lower $h$ and narrower $\Delta h$ have relativiely 
larger spacing between bins and so higher numbers in each bin.  The above supports the results outlined in
Section \ref{subsec:deltahri}.\\


\begin{figure}[htbp]
%Pcolor_Peaks.py [master 61491be] rss_fit plots
\begin{minipage}[b]{0.5\linewidth}
        %
        \subfloat[]{\label{main:a}
                \includegraphics[scale=.36]{/tera/phil/nchaparr/python/Plotting/Dec252013/pngs/rss_fit_high}}\\
        \end{minipage}             
\quad
\begin{minipage}[b]{0.5\linewidth}
        \subfloat[]{\label{main:b}          
          
                \includegraphics[scale=.36]{/tera/phil/nchaparr/python/Plotting/Mar52014/pngs/rss_fit_high}}\\
       
       \end{minipage}
        \caption{Local vertical $\theta$ profiles with 3-line fit for the 60/2.5 (a) and 150/10 (b) runs at 
points where $h^{l}_{0}$ is high.}
        \label{fig:rssfitshigh}
\end{figure}
=======
Local $\theta$ profiles (Fig. \ref{fig:rssfits}) exhibit a distinct \acs{ML} before resuming $\gamma$ but not 
a clearly defined \acs{EL}. There are sharp changes in the profile well into the free atmosphere, due possibly to 
waves, which render the gradient method for determining $h^{l}$ unusable.  Instead a linear regression method is used, 
whereby three lines representing: the mixed layer (ML), the EL and the upper lapse rate ($\gamma$) are fit to the 
profile according to the minimum residual sum of squares (RSS).  Determining local \acs{ML} height ($h_{0}^{l}$) was 
more straight forward than the local height of maximum potential temperature gradient $h^{l}$ for the reasons stated 
above.\\  
>>>>>>> added discussion

\begin{figure}[htbp]
%Pcolor_Peaks.py [master 61491be] rss_fit plots
\begin{minipage}[b]{0.5\linewidth}
        %
        \subfloat[]{\label{main:a}
                \includegraphics[scale=.36]{/tera/phil/nchaparr/python/Plotting/Dec252013/pngs/rss_fit_low}}\\
        \end{minipage}             
\quad
\begin{minipage}[b]{0.5\linewidth}
        \subfloat[]{\label{main:b}          
          
                \includegraphics[scale=.36]{/tera/phil/nchaparr/python/Plotting/Mar52014/pngs/rss_fit_low}}\\
       
       \end{minipage}
        \caption{Local vertical $\theta$ profiles with 3-line fit for the 60/2.5 (a) and 150/10 (b) runs at 
points where $h^{l}_{0}$ is low.}
        \label{fig:rssfitslow}
\end{figure}

\begin{figure}[htbp]
\caption{$\theta^{'}$ (a,d), $w^{'}$(b,e) at $h_{1}$(c,f) and local ML height $h^{l}_{0}$ at 2 hours for 60/2.5 (left) and 150/10 (right) runs}
\begin{minipage}[b]{0.5\linewidth} 
        
        \subfloat[]{\label{main:a}
                \includegraphics[scale=.36]{/tera/phil/nchaparr/python/Plotting/Dec252013/pngs/theta_cont1}}\\
        \subfloat[]{\label{main:b}      
                \includegraphics[scale=.36]{/tera/phil/nchaparr/python/Plotting/Dec252013/pngs/wvel_cont1}}\\ 
        \subfloat[]{\label{main:c}
          %Get_ML_Heights.py [master b7c3e5b] h_cont
                \includegraphics[scale=.36]{/tera/phil/nchaparr/python/Plotting/Dec252013/pngs/h_cont}} 
 \end{minipage}             
\quad
\begin{minipage}[b]{0.5\linewidth}
        \subfloat[]{\label{main:d}
                \includegraphics[scale=.36]{/tera/phil/nchaparr/python/Plotting/Mar52014/pngs/theta_cont1}}\\
       
       \subfloat[]{\label{main:e}
                \includegraphics[scale=.36]{/tera/phil/nchaparr/python/Plotting/Mar52014/pngs/wvel_cont1}}\\
        
       \subfloat[]{\label{main:f}
                \includegraphics[scale=.36]{/tera/phil/nchaparr/python/Plotting/Mar52014/pngs/h_cont}}                 
\end{minipage}
        
        \label{fig:conts2}
\end{figure}

%could put these sideways.  could do with checking the 2.5/60 low hs

%could put these sideways.  could do with checking the 2.5/60 low hs

\begin{figure}[htbp]
\caption{Histograms of $h^{l}_{0}$ for $\overline{w^{'}\theta^{'}_{s}} = 150$ to $60 (W / m^{2})$ (a to c) and $ \gamma = 10$ to $2.5 (K/Km)$ (c to g) at 5 hours}
%ML_Height_hist.py(sp?) master f992942ade
\begin{minipage}[b]{0.32\linewidth} 
        
        \subfloat[]{\label{main:a}
                \includegraphics[scale=.22]{/tera/phil/nchaparr/python/Plotting/Mar52014/pngs/ML_Height_hist}}\\
        \subfloat[]{\label{main:b}      
                \includegraphics[scale=.22]{/tera/phil/nchaparr/python/Plotting/Dec142013/pngs/ML_Height_hist}}\\ 
        \subfloat[]{\label{main:c}          
                \includegraphics[scale=.22]{/tera/phil/nchaparr/python/Plotting/Mar12014/pngs/ML_Height_hist}} 
 \end{minipage}             
%\quad
\begin{minipage}[b]{0.32\linewidth}
        \subfloat[]{\label{main:d}
                \includegraphics[scale=.22]{/tera/phil/nchaparr/python/Plotting/Jan152014_1/pngs/ML_Height_hist}}\\
       \subfloat[]{\label{main:e}
                \includegraphics[scale=.22]{/tera/phil/nchaparr/python/Plotting/Nov302013/pngs/ML_Height_hist}}\\
       \subfloat[]{\label{main:f}
                \includegraphics[scale=.22]{/tera/phil/nchaparr/python/Plotting/Dec202013/pngs/ML_Height_hist}}                 
\end{minipage}
\begin{minipage}[b]{0.32\linewidth}
        %\subfloat[]{\label{main:d}
        %        \includegraphics[scale=.18]{/tera/phil/nchaparr/python/Plotting/Mar52014/pngs/ML_Height_hist}}\\
       
       %\subfloat[]{\label{main:e}
       %         \includegraphics[scale=.18]{/tera/phil/nchaparr/python/Plotting/Mar52014/pngs/ML_Height_hist}}\\
       \vspace{10mm} 
       \subfloat[]{\label{main:f}
                \includegraphics[scale=.22]{/tera/phil/nchaparr/python/Plotting/Dec252013/pngs/ML_Height_hist}}                 
\end{minipage}

        
        \label{fig:localhhist}
\end{figure}

\begin{figure}[htbp]
\caption{PDFs of $\frac{h^{l}_{0}}{h}$ for $\overline{w^{'}\theta^{'}_{s}} = 150$ to $60 (W / m^{2})$ (a to c) and $ \gamma = 10$ to $2.5 (K/Km)$ (c to g) at 5 hours}
%ML_Height_hist.py(sp?) master f992942ade
\begin{minipage}[b]{0.32\linewidth} 
        
        \subfloat[]{\label{main:a}
                \includegraphics[scale=.22]{/tera/phil/nchaparr/python/Plotting/Mar52014/pngs/scaled_ML_Height_hist}}\\
        \subfloat[]{\label{main:b}      
                \includegraphics[scale=.22]{/tera/phil/nchaparr/python/Plotting/Dec142013/pngs/scaled_ML_Height_hist}}\\ 
        \subfloat[]{\label{main:c}          
                \includegraphics[scale=.22]{/tera/phil/nchaparr/python/Plotting/Mar12014/pngs/scaled_ML_Height_hist}} 
 \end{minipage}             
%\quad
\begin{minipage}[b]{0.32\linewidth}
        \subfloat[]{\label{main:d}
                \includegraphics[scale=.22]{/tera/phil/nchaparr/python/Plotting/Jan152014_1/pngs/scaled_ML_Height_hist}}\\
       \subfloat[]{\label{main:e}
                \includegraphics[scale=.22]{/tera/phil/nchaparr/python/Plotting/Nov302013/pngs/scaled_ML_Height_hist}}\\
       \subfloat[]{\label{main:f}
                \includegraphics[scale=.22]{/tera/phil/nchaparr/python/Plotting/Dec202013/pngs/scaled_ML_Height_hist}}                 
\end{minipage}
\begin{minipage}[b]{0.32\linewidth}
        %\subfloat[]{\label{main:d}
        %        \includegraphics[scale=.18]{/tera/phil/nchaparr/python/Plotting/Mar52014/pngs/ML_Height_hist}}\\
       
       %\subfloat[]{\label{main:e}
       %         \includegraphics[scale=.18]{/tera/phil/nchaparr/python/Plotting/Mar52014/pngs/ML_Height_hist}}\\
       \vspace{10mm} 
       \subfloat[]{\label{main:f}
                \includegraphics[scale=.22]{/tera/phil/nchaparr/python/Plotting/Dec252013/pngs/scaled_ML_Height_hist}}                 
\end{minipage}

        
        \label{fig:localhpdf}
\end{figure}

%\begin{figure}[htbp]
 %   \centering
    %plot_height.py [master 199de9a7cf]  
  %  \includegraphics[scale=.5]{/tera/phil/nchaparr/python/Plotting/Dec252013/pngs/varvsinvri}
  %  \caption{Variance vs \acs{Ri}$^{-1}$ at 5 hours}
   % \label{fig:varsvsinvri}   % label should change
%\end{figure}

\clearpage

\section{Flux Quadrants}
\label{sec:fluxquadrants}     
\FloatBarrier

As \citeauthor{SullMoengStev} point out in \cite{SullMoengStev} when broken out into four quadrants 
(Figure \ref{fig:fluxqadprofs}) the $\overline{w^{'}\theta^{'}}$ profiles have upper extrema above 
that of the total average profile ($z_{f}$).  2D histograms of the four quadrants are plotted at $h_{0}$, $h$ 
and $h_{1}$ to see how the distributions are influenced by changes in $\overline{w^{'} \theta^{'}}$ 
and $\gamma$.\\

 At $h_{0}$ (Figure \ref{fig:fluxquadsh0}) fast updraughts are relatively warm.  The spread in $w^{'}$ 
increases with increasing $\overline{w^{'}\theta^{'}}_{s}$ and decreases with increased $\gamma$.
At $h$ (Figure \ref{fig:fluxqadprofs}) the faster updraughts are now relatively cool and movement 
(both up and down) of warmer air from aloft becomes more prominent.  The spread of $w^{'}$ 
and $\theta^{'}$ both increase with increasing $\overline{w^{'}\theta^{'}}$ whereas that of
$\theta^{'}$ increases only slightly with increased stability.  As expected stability inhibits both
upward and downward $w^{'}$. \\ 

Although the quadrant of overall largest magnitude is that of
upward moving cool air, \citeauthor{SullMoengStev}'s assertion in \cite{SullMoengStev}
that in the \acs{EL} the heat flux is effectively due to downward moving warm air because
the other three quadrants cancel, is found to be approximately true.  At the top of the EL (Figure \ref{fig:fluxquadsh1}) 
velocities are damped and the distributions approach symmetry appart from some slow, cool, impinging up- 
and down-draughts as in Figure \ref{fig:conts2}. \\

\begin{figure}[htbp]
\begin{minipage}[b]{0.5\linewidth}
        %
        \subfloat[]{\label{main:a}
                \includegraphics[scale=.36]{/tera/phil/nchaparr/python/Plotting/Dec252013/pngs/fluxquadprofs}}\\
        \end{minipage}             
\quad
\begin{minipage}[b]{0.5\linewidth}
        \subfloat[]{\label{main:d}          
          %Flux_Quads.py [master 0eaf94e] fluxquadprofs
                \includegraphics[scale=.36]{/tera/phil/nchaparr/python/Plotting/Mar52014/pngs/fluxquadprofs}}\\
       \end{minipage}
        \caption{Scaled $\overline{w^{'} \theta^{'}}$ quadrant profiles at 5 hours for the 60/2.5 (a) and 150/10 (b) run}
        \label{fig:fluxqadprofs}
\end{figure}

<<<<<<< HEAD
=======
2D Histograms of the four quadrants are plotted at $h_{0}$, $h$ and $h_{1}$ (based on the averaged $\theta$ profiles)
to visualize how the distributions are influenced by changes in $\overline{w^{,} \theta^{,}}$ and $\gamma$.  At $h_{0}$
(Fig. \ref{fig:fluxquadsh0}) fast updraughts are relatively warm and are inhibited by increasing upper stability.  The spread
in velocity increases with increasing $\overline{w^{,}\theta^{,}}_{s}$\\

>>>>>>> added discussion
%redo these with one colorbar see 
%http://stackoverflow.com/questions/13784201/matplotlib-2-subplots-1-colorbar  

\begin{figure}[htbp]
\centering
 \includegraphics[scale=.8]{/tera/phil/nchaparr/python/Plotting/Dec252013/pngs/fluxquadhists0}                 
%\end{minipage}
\label{fig:fluxquadsh0}
\caption{ $\overline{w^{'}\theta^{'}}$ quadrants at $h_{0}$ for $w^{'}\theta^{'} = 150 - 60 (W/m^{2}$) (top-bottom) and $\gamma = 10 - 2.5 (K/Km)$ (left-right) at 5 hours}
\end{figure}

\begin{figure}[htbp]
\centering
 \includegraphics[scale=.8]{/tera/phil/nchaparr/python/Plotting/Dec252013/pngs/scaled_fluxquadhist0}                 
%\end{minipage}
\label{fig:scaled_fluxquadsh0}
\caption{ $\overline{w^{'}\theta^{'}}$ quadrants at $h_{0}$ for $w^{'}\theta^{'} = 150 - 60 (W/m^{2}$) (top-bottom) and $\gamma = 10 - 2.5 (K/Km)$ (left-right) at 5 hours}
\end{figure}


\begin{figure}[htbp]
%
\centering
 \includegraphics[scale=.8]{/tera/phil/nchaparr/python/Plotting/Dec252013/pngs/fluxquadhists1}                 

\caption{ $\overline{w^{'}\theta^{'}}$ quadrants at $h$ for $w^{'}\theta^{'} = 150 \ - \ 60$(W/$m^{2}$) (top - bottom) and $\gamma = 10 \ - \  2.5$(K/Km) (left - right) at 5 hours}

\label{fig:fluxquadsh}

\end{figure}

\begin{figure}[htbp]
%
\centering
 \includegraphics[scale=.8]{/tera/phil/nchaparr/python/Plotting/Dec252013/pngs/scaled_fluxquadhist1}                 

\caption{ $\overline{w^{'}\theta^{'}}$ quadrants at $h$ for $w^{'}\theta^{'} = 150 \ - \ 60$(W/$m^{2}$) (top - bottom) and $\gamma = 10 \ - \  2.5$(K/Km) (left - right) at 5 hours}

\label{fig:fluxquadsh}

\end{figure}


\begin{figure}[htbp]
%
\caption{ $\overline{w^{'}\theta^{'}}$ quadrants at $h_{1}$ for $w^{'}\theta^{'} = 150 \ to \ 60$(W/$m^{2}$) (top to bottom) and $\gamma = 10 \ to \ 2.5$(K/Km) (left to right) at 5 hours}
\centering
 \includegraphics[scale=.8]{/tera/phil/nchaparr/python/Plotting/Dec252013/pngs/fluxquadhists2}                 


\label{fig:fluxquadsh1}
\end{figure}

\begin{figure}[htbp]
%
\caption{ $\overline{w^{'}\theta^{'}}$ quadrants at $h_{1}$ for $w^{'}\theta^{'} = 150 \ to \ 60$(W/$m^{2}$) (top to bottom) and $\gamma = 10 \ to \ 2.5$(K/Km) (left to right) at 5 hours}
\centering
 \includegraphics[scale=.8]{/tera/phil/nchaparr/python/Plotting/Dec252013/pngs/scaled_fluxquadhist2}                 


\label{fig:fluxquadsh1}
\end{figure}

\clearpage

\section{$h$ and  $\Delta h$ based on Average Profiles}
\label{sec:hdeltahavprofs}

\FloatBarrier
\subsection{Reminder of Relevant Definitions}
\FloatBarrier
Here we define \acs{CBL} height $h$  as the point at which 
$\frac{\partial \overline{\theta}}{\partial z}$ is maximum and the \acs{EL} limits: $h_{0}$
the point at which $\frac{\partial \overline{\theta}}{\partial z}$ first exceeds a threshold
and $h_{1}$ the point at which $\frac{\partial \overline{\theta}}{\partial z}$ resumes $\gamma$.
The temperature jump $\Delta \theta$ is the difference accross the \acs{EL}.\\

%definitions
\begin{figure}[htbp]
    \centering
    %plot_height.py[master 1573b9d] h vs time plot
    \includegraphics[scale=.5]{/tera/phil/nchaparr/python/Plotting/Dec252013/pngs/height_defs}
    \caption{Height Definitions}
    \label{fig:hdefs1}   % label should change
\end{figure}

\begin{table}[htbp]
    %\begin{center}
\centerline{
    \begin{tabular}{p{3cm} p{3cm}  p{3cm}  p{3cm}  p{3cm} }
    \hline
     Description & This Study & Sullivan et al. \citeyear{SullMoengStev} & Fedorovich et al.\cite{FedConzMir04} \\ \hline %&  Garcia and Mellado \cite{GarciaMellado}
     CBL Height&$h$&$h$ & $z_{f}$ \\ \hline %& $z_{enc} \approx z_{f0}$
     Temperature Jump&$\Delta \theta = \overline{\theta}(h_{1})-\overline{\theta}(h_{0})$&$\Delta \theta = \overline{\theta}(z_{f1})-\overline{\theta}(z_{f})$ & $\Delta b = b_{0}(z_{f}) -b(z_{f})$ \\ %\hline %& $\Delta b = b_{0}(z_{f}) - b(z_{f}) $
     && &$\delta b = b(z_{f1})$ - $b(z_{f0})$\\ %\hline %& $\delta b = b(z_{f1})$ - $b(z_{f0})$&
     Convective Velocity Scale&$w_{*}= (h B_{s})^{\frac{1}{3}}$, $B_{s} = \frac{g}{\overline{\theta_{ML}}}\overline{w^{'}\theta^{'}}_{s}$&$w_{*}= (h B_{s})^{\frac{1}{3}}$, $B_{s} = \frac{g}{\overline{\theta_{ML}}}\overline{w^{'}\theta^{'}}_{s}$ & $w_{*}= (z_{f} B_{s})^{\frac{1}{3}}$\\ %\hline %& $w_{*}= (z_{f0} B_{s})^{\frac{1}{3}}$
     Richardson Number&$Ri = \frac{\Delta \theta h}{{w^{*2}}}$ &$Ri = \frac{\Delta \theta h}{w^{*2}}$ & $Ri_{\Delta b} = \frac{\Delta b z_{f}}{w^{*2}}$, $Ri_{\delta b} = \frac{\delta b_{i}z_{f}}{w^{*2}}$\\ \hline %& $Ri_{i, f} = \frac{\Delta b z_{enc}}{w^{*2}}$, $Ri_{*} = \frac{\delta b_{i} z_{enc}}{w^{*2}}$
  \end{tabular}
}
\caption{Comparison of relevant definitions with those from key publications}
\label{fig:}   
%\end{center}    
\end{table}

%\footnotetext[1]{Fedorovich et al and Garcia et al use buoancy instead of $\theta$, but they are interchangeable for this purpose}
\clearpage
\subsection{$\frac{w_{e}}{w^{*}}$ vs $Ri^{-1}$}
\FloatBarrier
Covective Boundary Layer (CBL) height ($h$) (Figure \ref{fig:hvstime}) grows rapidly initially with a 
steadily decreasing rate and relates to the square root of time (Figure\ref{fig:hvstimeloglog}).  
It is found to be proportionate to the height of minimum flux ($z_{f}$) (Figure \ref{fig:hvstime1}).\\
% thus 
%indicating it is in the quasi-steady state regime outlined by \citeauthor{FedConzMir04} in \cite{FedConzMir04} 
%in which the zero order relationship of the scaled entrainment rate to Richardson Number
%(\acs{Ri}) is expected.\\

Inverse Richardson Number (\acs{Ri}$^{-1}$) decreases with respect to time 
and clusters according to $\gamma$. (Figure \ref{fig:invristime}).  The entrainment rate ($w_{e}= \frac{dh}{dt}$) 
is determined from the slope of a second order polynomial fit to $h(time)$ (Figure \ref{fig:hvstime}).  
When scaled by ($w^{*}$) it is a roughly linear function of  \acs{Ri}$^{-1}$ (Figure \ref{fig:scaledweinvri}).\\    
  
\begin{figure}[htbp]
    \centering
    %plot_height.py[master 1573b9d] h vs time plot
    \includegraphics[scale=.5]{/tera/phil/nchaparr/python/Plotting/Dec252013/pngs/hvstime}
    \caption{$h$ vs time for all runs}
    \label{fig:hvstime}   % label should change
\end{figure}

\begin{figure}[htbp]
    \centering
    %plot_height.py[master 1573b9d] h vs time plot
    \includegraphics[scale=.5]{/tera/phil/nchaparr/python/Plotting/Dec252013/pngs/hstimelog}
    \caption{Log-Log plot of $h$ vs time for all runs}
    \label{fig:hvstimeloglog}   % label should change
\end{figure}

\begin{figure}[htbp]
    \centering
    %plot_height.py[master 1573b9d] h vs time plot
    \includegraphics[scale=.5]{/tera/phil/nchaparr/python/Plotting/Dec252013/pngs/scaled_h_f_time}
    \caption{$\frac{z_{f}}{h}$ vs Time}
    \label{fig:hvstime1}   % label should change
\end{figure}

\begin{figure}[htbp]

\begin{minipage}[b]{0.5\linewidth}
 %\centering        
        \subfloat[]{\label{main:a}
                \includegraphics[scale=.36]{/tera/phil/nchaparr/python/Plotting/Dec252013/pngs/invristime}}\\
        \end{minipage}             
\quad
\begin{minipage}[b]{0.5\linewidth}
        \subfloat[]{\label{main:d}
          %plot_height [master fbd2dfd] invristime1, see branches named accordingly
                \includegraphics[scale=.36]{/tera/phil/nchaparr/python/Plotting/Dec252013/pngs/invristime_zero_order}}\\
       
       \end{minipage}
        \caption{Inverse bulk Richardson Number vs time}
        \label{fig:invristime}
\end{figure}

\begin{figure}[htbp]
%\centering
\begin{minipage}[b]{0.5\linewidth}
        %plot_height [master fbd2dfd] invristime1
        \subfloat[]{\label{main:a}
       \includegraphics[scale=.36]{/tera/phil/nchaparr/python/Plotting/Dec252013/pngs/loglog_scaledweinvri}}\\
        \end{minipage}             
\quad
\begin{minipage}[b]{0.5\linewidth}
        \subfloat[]{\label{main:b}          
          %plot_height [master 01c3721] scaledweinvri1
                \includegraphics[scale=.36]{/tera/phil/nchaparr/python/Plotting/Dec252013/pngs/loglog_scaledweinvri_zero_order}}\\
       
       \end{minipage}
        \caption{Scaled Entrainment rate vs inverse Richardson Number (\acs{Ri})}
        \label{fig:scaledweinvri}
\end{figure}

\clearpage
<<<<<<< HEAD
=======

\section{$h$ and  $\Delta h$ based on Average Profiles}
\label{sec:hdeltahavprofs}

\FloatBarrier

\subsection{Reminder of Relevant Definitions}
\FloatBarrier
%definitions
\begin{figure}[htbp]
    \centering
    %plot_height.py[master 1573b9d] h vs time plot
    \includegraphics[scale=.5]{/tera/phil/nchaparr/python/Plotting/Dec252013/pngs/height_defs}
    \caption{$h$ vs time for all runs}
    \label{fig:hvstime}   % label should change
\end{figure}

\begin{table}[!ht]
    \begin{center}
    \begin{tabular}{ | p{3cm} | p{3cm} | p{3cm} |}
    \hline
     Sullivan, Moeng & Fedorovich, Conzemius &  Garcia, Mellado\\ \hline
     $h$ & $z_{f}$& $z_{enc} \approx z_{f0}$ \\ \hline
     $\Delta \theta = \overline{\theta}(z_{f1})-\overline{\theta}(z_{f})$ & $\Delta b = b_{0}(z_{f}) -b(z_{f})$ & $\Delta b = b_{0}(z_{f}) - b(z_{f}) $\\ \hline
      & $\delta b = b(z_{f1})$ - $b(z_{f0})$ & $\delta b = b(z_{f1})$ - $b(z_{f0})$\\ \hline
      $w_{*}= (h B_{s})^{\frac{1}{3}}$, $B_{s} = \frac{g}{\overline{\theta_{ML}}}\overline{w^{'}\theta^{'}}_{s}$ & $w_{*}= (z_{f} B_{s})^{\frac{1}{3}}$ & $w_{*}= (z_{f0} B_{s})^{\frac{1}{3}}$\\ \hline
      $\tau = \frac{h}{w^{*}}$ & $\tau=N^{-1}$ & $\frac{t}{\tau}=\frac{z_{enc}}{L_{0}}$, $L_{0}=\left(\frac{B_{s}}{N^{3}}\right)^{\frac{1}{2}}$\\ \hline
 
   
\end{tabular}
\caption{Relevant definitions from three key publications}
\label{fig:}   
\end{center}    
\end{table}
%\footnotetext[1]{Fedorovich et al and Garcia et al use buoancy instead of $\theta$, but they are interchangeable for this purpose}
>>>>>>> added discussion

\subsection{$\frac{\Delta h}{h}$ vs $Ri^{-1}$}
\label{subsec:deltahri}
\FloatBarrier

<<<<<<< HEAD
The scaled upper EL limits ($\frac{h_{1}}{h}$) collapse well in Figure \ref{fig:scaledELlims} 
to an initial value of approximately 1.15, decreasing to about 1.1.  $\frac{h_{0}}{h}$s appear 
grouped according to $\gamma$ and increase with respect to time.  So overall the scaled \acs{EL} appears
to narrow with time.   The scaled flux based \acs{EL} ($z_{f0}$ and $z_{f1}$) appears to remain constant 
with respect to time in Figure \ref{fig:scaledELlims1}.\\

The lower entrainment layer limit $h_{0}$ is the point at which the vertical 
$\frac{\partial \overline{\theta}}{\partial z}$ exceeds a threshold (.0002) chosen such that
it is positive, and at least an order of magnitude smaller than $\gamma$.   Although the resulting 
scaled \acs{EL} depth decreases with increasing \acs{Ri} grouping according to $\gamma$ is evident 
in Figure \ref{fig:scaledeltahinvri}.\\

To explore how varying the threshold value affects the relationship between scaled \acs{EL} depth
and Richardson number (\acs{Ri}), plots analogous to Figure \ref{fig:scaledeltahinvri} were produced at two 
additional thresholds.  A higher threshold value (.0004) results in a higher $h_{0}$ (Figure \ref{fig:thresh1})   
and so a narrower \acs{EL} but a similar grouping according to $\gamma$ (Figure\ref{fig:scaledeltahinvri1}).
A lower threshold value (.0001) results in a lower $h_{0}$ (Figure \ref{fig:thresh2})
but also similar grouping according to $\gamma$ (Figure \ref{fig:scaledeltahinvri2}.\\

When the height definitions are based on the scaled vertical $\frac{\partial \overline{\theta}}{\partial z}$
i.e. $\frac{\partial \overline{\theta}}{\partial z} / \gamma$ profile, only $h_{0}$ changes and for clarity we 
call this \acs{EL} depth $\Delta h^{*}$ and the revised Richardson number Ri$^{*}$.   The curves now collapse and 
scaled \acs{EL} depth is seen to decrease with increasing \acs{Ri}$^{*}$ (Figures \ref{fig:thresh3} to \ref{fig:ELvsri}).\\

There is a slight collapsing effect on the scaled entrainment rate vs \acs{Ri} relationship when
the heights are defined based on the scaled vertical potential temperature gradient 
$\frac{\partial \overline{\theta}}{\partial z} / \gamma$ profile in Figure \ref{fig:scaledweinvri2}.\\

\begin{figure}[htbp]
    \centering
    %plot_height.py [master fd5c6b1] delta h vs time1  
    \includegraphics[scale=.5]{/tera/phil/nchaparr/python/Plotting/Dec252013/pngs/deltahstime1}
    \caption{Scaled Entrainment Layer limits ($\frac{h_{1}}{h}$ and $\frac{h_{0}}{h}$) vs time}
    \label{fig:scaledELlims}   % label should change
\end{figure}

%need flux based el limits scaled by either h or hf
\begin{figure}[htbp]
    \centering
    %plot_height.py [master fd5c6b1] delta h vs time1  
    \includegraphics[scale=.5]{/tera/phil/nchaparr/python/Plotting/Dec252013/pngs/scaled_h_f0_time}
    \caption{Scaled Entrainment Layer limits ($z_{f1}$ and $z_{f0}$) vs time}
    \label{fig:scaledELlims1}   % label should change
\end{figure}

\begin{figure}[htbp]
    \centering
    %plot_height.py [master fd5c6b1] delta h vs time1  
    \includegraphics[scale=.5]{/tera/phil/nchaparr/python/Plotting/Dec252013/pngs/theta_grad_profs.pdf}
    \caption{Vertical $\frac{\partial \overline{\theta}}{\partial z}$ profiles with threshold at .0002}
    \label{fig:thresh}   % label should change
\end{figure}

\begin{figure}[htbp]
\centering
%\begin{minipage}[b]{0.5\linewidth}
        %plot_height [master c7af4de] scaleddeltahinvri
 %       \subfloat[]{\label{main:a}
 \includegraphics[scale=.5]{/tera/phil/nchaparr/python/Plotting/Dec252013/pngs/scaleddeltahinvri}
  %      \end{minipage}             
%\quad
%\begin{minipage}[b]{0.5\linewidth}
 %       \subfloat[]{\label{main:d}          
          %plot_height [master b9c30ad] scaleddeltahinvri1
  %              \includegraphics[scale=.36]{/tera/phil/nchaparr/python/Plotting/Dec252013/pngs/scaleddeltahinvri1}}\\
       
   %    \end{minipage}
        \caption{Scaled EL depth vs inverse bulk Richardson Number with threshold at .0002}
         \label{fig:scaledeltahinvri}
\end{figure}

\begin{figure}[htbp]
    \centering
    %plot_height.py [master fd5c6b1] delta h vs time1  
    \includegraphics[scale=.5]{/tera/phil/nchaparr/python/Plotting/Dec252013/pngs/theta_grad_profs5.pdf}
    \caption{Vertical $\frac{\partial \overline{\theta}}{\partial z}$ profiles with threshold at .0004}
    \label{fig:thresh1}   % label should change
\end{figure}

\begin{figure}[htbp]
    \centering
    %plot_height.py [master fd5c6b1] delta h vs time1  
    \includegraphics[scale=.5]{/tera/phil/nchaparr/python/Plotting/Dec252013/pngs/scaleddeltahinvri_5}
    \caption{Scaled EL depth vs inverse Richardson Number with threshold at .0004}
    \label{fig:scaledeltahinvri1}   % label should change
\end{figure}

\begin{figure}[htbp]
    \centering
    %plot_height.py [master fd5c6b1] delta h vs time1  
    \includegraphics[scale=.5]{/tera/phil/nchaparr/python/Plotting/Dec252013/pngs/theta_grad_profs6.pdf}
    \caption{Vertical $\frac{\partial \overline{\theta}}{\partial z}$ profiles with threshold at .0001}
    \label{fig:thresh2}   % label should change
\end{figure}

\begin{figure}[htbp]
    \centering
    %plot_height.py [master fd5c6b1] delta h vs time1  
    \includegraphics[scale=.5]{/tera/phil/nchaparr/python/Plotting/Dec252013/pngs/scaleddeltahinvri_6}
    \caption{Scaled EL depth vs inverse bulk Richardson Number with threshold at .0001}
    \label{fig:scaledeltahinvri2}   % label should change
\end{figure}

\begin{figure}[htbp]
    \centering
    %plot_height.py [master fd5c6b1] delta h vs time1  
    \includegraphics[scale=.5]{/tera/phil/nchaparr/python/Plotting/Dec252013/pngs/scaled_theta_grad_profs}
    \caption{Scaled vertical $\frac{\partial \overline{\theta}}{\partial z}$ profiles with threshold at .03}
    \label{fig:thresh3}   % label should change
\end{figure}

\begin{figure}[htbp]
    \centering
    %plot_height.py [master fd5c6b1] delta h vs time1  
    \includegraphics[scale=.5]{/tera/phil/nchaparr/python/Plotting/Dec252013/pngs/scaleddeltahinvri_4}
    \caption{Revised height definitions based on scaled $\frac{\partial \overline{\theta}}{\partial z}$ profiles with threshold at .03}
    \label{fig:heightdefs1}   % label should change
\end{figure}


\begin{figure}[htbp]
%\begin{minipage}[b]{0.5\linewidth}
        %plot_height [master c7af4de] scaleddeltahinvri
 %       \subfloat[]{\label{main:a}
                \includegraphics[scale=.5]{/tera/phil/nchaparr/python/Plotting/Dec252013/pngs/loglog_scaleddeltahinvri_4}\\
  %      \end{minipage}             
%\quad
%\begin{minipage}[b]{0.5\linewidth}
 %       \subfloat[]{\label{main:d}          
          %plot_height [master b9c30ad] scaleddeltahinvri1
   %             \includegraphics[scale=.36]{/tera/phil/nchaparr/python/Plotting/Dec252013/pngs/scaleddeltahinvri}}\\
       
  %     \end{minipage}
        \caption{Scaled EL Depths vs inverse bulk Richardson number based on scaled $\frac{\partial \overline{\theta}}{\partial z}$ (a) and $\frac{\partial \overline{\theta}}{\partial z}$ (b)}
        \label{fig:ELvsri}
\end{figure}

\begin{figure}[htbp]
\begin{minipage}[b]{0.5\linewidth}
        %plot_height [master c7af4de] scaleddeltahinvri
        \subfloat[]{\label{main:a}
                \includegraphics[scale=.36]{/tera/phil/nchaparr/python/Plotting/Dec252013/pngs/loglog_scaledweinvri1.pdf}}\\
        \end{minipage}             
\quad
\begin{minipage}[b]{0.5\linewidth}
        \subfloat[]{\label{main:d}          
          %plot_height [master b9c30ad] scaleddeltahinvri1
                \includegraphics[scale=.36]{/tera/phil/nchaparr/python/Plotting/Dec252013/pngs/loglog_scaledweinvri}}\\
       
       \end{minipage}
        \caption{Scaled Entrainment Rate vs inverse bulk Richardson number based on scaled $\frac{\partial \overline{\theta}}{\partial z}$ (a) and $\frac{\partial \overline{\theta}}{\partial z}$ (b)}
        \label{fig:scaledweinvri2}
\end{figure}
=======

\subsection{$\frac{w_{e}}{w^{*}}$ vs $Ri^{-1}$}
\FloatBarrier
Covective Boundary Layer (CBL) height ($h$) (Fig. \ref{fig:hvstime}) grows rapidly initially with a steadily decreasing rate
and is a linear function of scaled time (Fig \ref{fig:hvsscaledtime}).\\
  
\begin{figure}[htbp]
    \centering
    %plot_height.py[master 1573b9d] h vs time plot
    \includegraphics[scale=.5]{/tera/phil/nchaparr/python/Plotting/Dec252013/pngs/hvstime}
    \caption{$h$ vs time for all runs}
    \label{fig:hvstime}   % label should change
\end{figure}

%need flux height scaled by h plot

\begin{figure}[htbp]
    \centering
    %plot_height.py[master 1573b9d] h vs time plot
    \includegraphics[scale=.5]{/tera/phil/nchaparr/python/Plotting/Dec252013/pngs/scaled_h_f_time}
    \caption{$\frac{z_{f}}{h}$ vs Time}
    \label{fig:hvstime}   % label should change
\end{figure}


%redo Ri with Delta theta (?)

%Do a log log plot of the below? to confirm the -1 slope? 

%get slope to compare with other work.

Inverse Richardson Numbers (\acs{Ri}$^{-1}$) (bulk \acs{Ri}: $=\frac{gh}{\overline{\theta_{ML}}} \frac{\Delta \theta}{w^{*2}}$, $\Delta \theta = \overline{\theta}(h_{1})-\overline{\theta}(h_{0})$ 
and gradient \acs{Ri}: $=\frac{g}{\overline{\theta_{ML}}} \frac{\gamma h^{2} }{w^{*2}}$) decrease with respect to time, 
clustering according to $\gamma$. (Fig. \ref{fig:invristime})\\

\begin{figure}[htbp]

\begin{minipage}[b]{0.5\linewidth}
         
        \subfloat[]{\label{main:a}
                \includegraphics[scale=.36]{/tera/phil/nchaparr/python/Plotting/Dec252013/pngs/invristime}}\\
        \end{minipage}             
\quad
\begin{minipage}[b]{0.5\linewidth}
        \subfloat[]{\label{main:d}
          %plot_height [master fbd2dfd] invristime1, see branches named accordingly
                \includegraphics[scale=.36]{/tera/phil/nchaparr/python/Plotting/Dec252013/pngs/invristime1}}\\
       
       \end{minipage}
        \caption{Inverse bulk (a)  and gradient (b) Richardson Number vs time}
        \label{fig:invristime}
\end{figure}

The entrainment rate ($w_{e}= \frac{dh}{dt}$) is determined from the slope of a second order polynomial fit to $h(time)$ (Fig. \ref{fig:hvstime}).  When scaled by ($w^{*}$) it is a roughly linear function 
of the bulk $Ri^{-1}$, whereas it is asymptotic to a linear function of the gradient $Ri^{-1}$. (Fig. \ref{fig:scaledweinvri})\\    

\begin{figure}[htbp]
\begin{minipage}[b]{0.5\linewidth}
        %plot_height [master fbd2dfd] invristime1
        \subfloat[]{\label{main:a}
                \includegraphics[scale=.36]{/tera/phil/nchaparr/python/Plotting/Dec252013/pngs/scaledweinvri}}\\
        \end{minipage}             
\quad
\begin{minipage}[b]{0.5\linewidth}
        \subfloat[]{\label{main:d}          
          %plot_height [master 01c3721] scaledweinvri1
                \includegraphics[scale=.36]{/tera/phil/nchaparr/python/Plotting/Dec252013/pngs/scaledweinvri1}}\\
       
       \end{minipage}
        \caption{Scaled Entrainment rate vs inverse bulk (a)  and gradient (b) Richardson Number}
        \label{fig:scaledweinvri}
\end{figure}

\subsection{$\frac{\Delta h}{h}$ vs $Ri^{-1}$}
\FloatBarrier

EL depth ($h_{1}-h_{0}=\Delta h$) (Fig. \ref{fig:deltahvstime1}) increases with time
although not as rapidly as $h$.  It deepens with increasing $\overline{w^{,}\theta^{,}_{s}}$ 
and narrows with increasing $\gamma$. \\

\begin{figure}[htbp]
    \centering
    %plot_height.py [master fd5c6b1] delta h vs time1  
    \includegraphics[scale=.5]{/tera/phil/nchaparr/python/Plotting/Dec252013/pngs/deltahstime1}
    \caption{$\Delta h$ vs time}
    \label{fig:deltahvstime1}   % label should change
\end{figure}

The scaled upper EL limits (Fig. \ref{fig:scaleddeltahstime}) ($\frac{h_{1}}{h}$) collapse well 
to an initial value of approximately 1.75, decreasing to about 1.5 .  $\frac{h_{0}}{h}$s appear 
grouped according to $\gamma$ and increase with respect to time.\\  

%need flux based el limits scaled by either h or hf

Although the scaled \acs{EL} depth decreases with increasing \acs{Ri} (gradient and bulk Fig \ref{fig:scaledeltahinvri})
grouping according to $\gamma$ is evident.\\

\begin{figure}[htbp]
\begin{minipage}[b]{0.5\linewidth}
        %plot_height [master c7af4de] scaleddeltahinvri
        \subfloat[]{\label{main:a}
                \includegraphics[scale=.36]{/tera/phil/nchaparr/python/Plotting/Dec252013/pngs/scaleddeltahinvri}}\\
        \end{minipage}             
\quad
\begin{minipage}[b]{0.5\linewidth}
        \subfloat[]{\label{main:d}          
          %plot_height [master b9c30ad] scaleddeltahinvri1
                \includegraphics[scale=.36]{/tera/phil/nchaparr/python/Plotting/Dec252013/pngs/scaleddeltahinvri1}}\\
       
       \end{minipage}
        \caption{Scaled EL depth vs inverse bulk (a)  and gradient (b) Richardson Number}
        \label{fig:scaledeltahinvri}
\end{figure}


%need bit about new definitions

\begin{figure}[htbp]
\begin{minipage}[b]{0.5\linewidth}
        %plot_height [master c7af4de] scaleddeltahinvri
        \subfloat[]{\label{main:a}
                \includegraphics[scale=.36]{/tera/phil/nchaparr/python/Plotting/Dec252013/pngs/scaleddeltahinvri_4}}\\
        \end{minipage}             
\quad
\begin{minipage}[b]{0.5\linewidth}
        \subfloat[]{\label{main:d}          
          %plot_height [master b9c30ad] scaleddeltahinvri1
                \includegraphics[scale=.36]{/tera/phil/nchaparr/python/Plotting/Dec252013/pngs/scaleddeltahinvri1_1}}\\
       
       \end{minipage}
        \caption{}
        \label{fig:}
\end{figure}


>>>>>>> added discussion

\endinput

Any text after an \endinput is ignored.
You could put scraps here or things in progress.
