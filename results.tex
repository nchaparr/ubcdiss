%% The following is a directive for TeXShop to indicate the main file
%%!TEX root = diss.tex

\chapter{Results}
\label{ch:results}
\setlength{\parindent}{0cm}

Each 10 member ensemble run was allowed a period of time to develop the three layer structure (\acs{ML}, \acs{EL}
and \acs{FA}) as seen from the vertical average potential temperature ($\overline{\theta}$) and vertical 
turbulent heat flux ($\overline{w^{'}\theta^{'}}$) profiles in Section \ref{sec:CheckingtheModel}.
The convective time scale ($\tau$) for a thermal to reach the \acs{CBL} top ($h$) was seen to depend
on $\gamma$, signalling the importance of this external parameter. \acs{FFT} spectra of turbulent 
velocity perturbations the \acs{ML} showed a satisfactory inertial subrange and several coherent 
impinging thermals were observed in the \acs{EL} at any given time after 2-3 hours, indicating that 
realistic turbulence was being simulated.\\

Using a multi-linear regression method in Section \ref{subsec:locmlh} , the local \acs{ML} heights ($h^{l}_{0}$) 
were determined.  Their distributions approached similarity when scaled by $h$, showing variation in the lower 
limit with variation in $\gamma$.  The tops of the local vertical $\theta$ profiles differ from each other as well 
as from the average profile, but a uniform \acs{ML} is evident. 2D distributions of the local turbulent perturbations,
$w^{'}$ and $\theta^{'}$ show some variation with $\gamma$ when scaled by the convective scales $w^{*}$ and $\theta^{*}$
in Section \ref{subsec:fluxquadrants}.  Focus is then narrowed to the average downward moving warm quadrant at $h$ ($\overline{w^{'-}\theta^{'+}}_{h}$, $\overline{w^{'-}}_{h} \ where \ \theta^{'} >0$ and $\overline{\theta^{'+}}_{h} \ where \ w^{'} < 0 $) to examine the direct influence of $\gamma$ on entrainment.\\    

Initially, in Section \ref{sec:deltahri}, the \acs{CBL} height and \acs{EL} limits are defined based on the vertical  $\frac{\partial \overline{\theta}}{\partial z}$ profile.  For comparison with results from other studies these heights also based on the vertical $\overline{w^{'}\theta^{'}}$ profiles as shown in Figure \ref{fig:hdefs}.  As \citeauthor{BrooksFowler2} (\citeyear{BrooksFowler2}) point out, when using an average vertical tracer profile there is no universal criterion for a significant gradient.  So a threshold value for the lower \acs{EL} limit ($h_{0}$) was chosen such that it was positive, small i.e. an order of magnitude less than $\gamma$ and the same for all runs.  For the sake of rigor, plots of the relationship

\begin{equation}
\frac{\Delta h}{h} \propto Ri ^{b} \tag{\ref{eq:dhvsri}}
\end{equation}

were produced based on two additional threshold values. The importance of $\gamma$ is revealed again as the curves representing
equation \ref{eq:dhvsri} become self-similar when heights are based on the scaled $\frac{\partial \overline{\theta}}{\partial z}$
profile, $\frac{\frac{\partial \overline{\theta}}{\partial z}}{\gamma}$.  This prompts the use of the scaled profiles for the heights
($h_{0}$, $h$, $h_{1}$ and $z_{f0}$, $z_{f}$, $z_{f1}$) in the subsequent section.\\


In Section \ref{sec:weri} the temperature jump is defined in four ways to examine the resulting effects on the relationship

\begin{equation}
\frac{w_{e}}{w^{*}} \propto Ri^{a} \tag{\ref{eq:ervsri}} 
\end{equation}

$\Delta \theta$ is the difference in $\overline{\theta}$ across the \acs{EL}.  For the sake of comparison and to investigate the resulting effects on $a$ in equation \ref{eq:ervsri} we also define a zero-order type temperature jump, $\delta \theta$ , as the difference between the mixed layer average potential temperature $\overline{\theta}_{ML}$ less $\overline{\theta}$ at $h$ on the initial profile as in Figure \ref{fig:hdefs}.  Analogous definitions are based on the vertical average heat flux profile ($\overline{w^{'}\theta^{'}}$) for comparison with the results of other studies. 
\\ 

\clearpage

\section{Verifying the Model Output}
\label{sec:CheckingtheModel}
\subsection{Initialization and Spin-Up Time}%Spin Up Time according to the Convective Time Scale $\tau$}
\FloatBarrier

All 10 member cases of the ensemble were carried out on a 3.2 x 4.8 Km horizontal 
domain ($\Delta x = \Delta y = 25m$, $nx=128$, $ny=192$).  
$nx$, $ny$ were chosen based on the optimal distribution across processor nodes.  
The vertical grid ($nz=312$) was of higher resolution around the 
entrainment layer ($\Delta z = 5m$), and lower below and above it 
($\Delta z = 10 \ to \ 100 m$). Grid size was chosen so that
a full spectrum of turbulence would be resolved within the \acs{EL} 
in line with the findings of \citeauthor{SullPat} (\citeyear{SullPat}).  The 7 runs
are all initialized with a constant surface heat flux ($\overline{w^{'}\theta^{'}_{s}}$) 
acting against a uniform initial lapse rate ($\gamma$), and differ from each other
based on these two external forcing parameters.\\

\begin{table}[!ht]
    \begin{center}
    \begin{tabular}{ | l | l | l | l |}
    \hline
    $\overline{w^{'}\theta^{'}_{s}}$ / $\gamma$ & 10 (K/Km) & 5 (K/Km) & 2.5 (K/Km) \\ \hline
     150 (W/m2)& \hspace{5mm} \ding{51} &\hspace{5mm} \ding{51}\footnotemark &  \\ \hline
     100 (W/m2)& \hspace{5mm} \ding{51} & \hspace{5mm} \ding{51} & \\ \hline
     60 (W/m2) & \hspace{5mm} \ding{51} & \hspace{5mm} \ding{51} & \hspace{5mm} \ding{51}\\ \hline
     
%\end

   
\end{tabular}
\caption{Runs in terms of $\overline{w^{'} \theta^{'}_{s}}$ and initial lapse rate $\gamma$}
\label{fig:tableofruns}   
\end{center}    
\end{table}
\footnotetext{Incomplete run: EL exceeded high resolution vertical grid after 7 hours}

Time must be allowed to establish statistically steady turbulent flow.  \citeauthor{SullMoengStev} 
(\citeyear{SullMoengStev}) recommended 10 eddy turnover times based on the convective time scale 
$\tau = \frac{h}{w^{*}} = \frac{h}{ \left( \frac{gh}{\overline{\theta}_{ML}}(\overline{w^{'} \theta^{'}_{s}}) \right)^{\frac{1}{3}} } $, 
and \citeauthor{BrooksFowler2} (\citeyear{BrooksFowler2}) chose a simulated time of 2 hours.  For all of 
the runs, at least 10 eddy turnover times were completed by 2 simulated hours (Figure \ref{fig:ScaledTimevsTime}).  
Although each run has a distinct convective velocity scale $w^{*}$, that increases with time, 
dividing boundary layer height, $h$, by it to obtain $\tau$ results in a collapse from 7 to 3 curves, 
one for each $\gamma$.\\

\begin{figure}[!h]
    \centering
    % plot_height.py [master 03d2835] Round 1 of Plots in Results 
    \includegraphics[scale=.5]{/newtera/tera/phil/nchaparr/python/Plotting/Dec252013/pngs/scaledtimevstime}
    \caption[Scaled time vs Time]{Plots of scaled time vs time for all runs.  Scaled time is based on the convective time scale $\tau$ 
    and can be thought of as the number of times an eddy has reached the top of the CBL.}
    \label{fig:ScaledTimevsTime}   
\end{figure}

Figure \ref{fig:tempgradfluxprofs1005} shows a measurable well mixed layer (\acs{ML}) and \acs{EL} based on the horizontally averaged, ensemble averaged potential temperature ($\overline{\theta}$) profile, by 2 hours.  After 2 or 3 hours the \acs{EL} is fully contained within the vertical region of high resolution.  Figure \ref{fig:tempgradfluxprofs1005} shows that the averaged heat fluxes ($\overline{w^{'}\theta^{'}}$) become self similar and are scaled well by the surface heat flux $(\overline{w^{'}\theta^{'}})_{s}$ by 2 hours.\\


\begin{figure}[htbp]
    \centering
    % plot_dthetaflux.py [master 03d2835] Round 1 of Plots in Results
    \includegraphics[scale=.5]{/newtera/tera/phil/nchaparr/python/Plotting/Nov302013/pngs/theta_flux_profs}
    \caption[Vertical profiles of $\overline{\theta}$, $\frac{\partial \overline{\theta}}{\partial z}$ and $\overline{w^{'}\theta^{'}}$]{Vertical profiles of the ensemble and horizontally averaged potential temperature ($\overline{\theta}$), its vertical gradient ($\frac{\partial \overline{\theta}}{\partial z}$)  
     and heat flux ($\overline{w^{'}\theta^{'}}$) for the 100/5 run}
    \label{fig:tempgradfluxprofs1005}   % label should change
\end{figure}

\begin{figure}[htbp]
    \centering
    % plot_dthetaflux.py [master 9883fda] Round 2 of Plots in Results
    %\includegraphics[scale=.5]{/newtera/tera/phil/nchaparr/python/Plotting/Mar52014/pngs/scaled_theta_flux_profs}
    %
    \includegraphics[scale=.5]{/newtera/tera/phil/nchaparr/python/Plotting/Nov302013/pngs/scaled_flux_profs}
    \caption[$\overline{w^{'}\theta^{'}}$ scaled by $(\overline{w^{'}\theta^{'}})_{s}$]{$\overline{w^{'}\theta^{'}}$ and scaled $\overline{w^{'}\theta^{'}}$  vs scaled height for the 100/5 run}
    \label{fig:scaledfluxprofs15010}   % label should change
\end{figure}

\clearpage

\subsection{Ensemble and horizontally averaged vertical Potential Temperature $\overline{\theta}$ 
and Heat Flux $\overline{w^{'}\theta^{'}}$ Profiles}
%Average Potential Temperature, Heat Flux and Kinetic Energy}
\FloatBarrier

The $\overline{\theta}$ profiles exhibit an \acs{ML} above which  $\frac{\partial\overline{\theta}}{\partial z}>0$ 
and reaches a maximum value at $h$ before resuming $\gamma$  at $h_{1}$ (Figures \ref{fig:tempgradfluxprofs1005} and \ref{fig:pottempprofs2hrs}).  Convective boundary layer \acs{CBL} growth is stimulated by $(\overline{w^{'}\theta^{'}})_{s}$ and inhibited by $\gamma$.\\

The horizontally averaged, ensemble averaged heat flux ($\overline{w^{'}\theta^{'}}$) profiles decrease from the surface value ($(\overline{w^{'}\theta^{'})_{s}}$) passing through zero to a minimum before increasing to zero (Figures \ref{fig:tempgradfluxprofs1005} and  \ref{fig:fluxprofs2hrs}).  They are almost self-similar across runs when scaled by $(\overline{w^{'}\theta^{'}})_{s}$. All minima are less  in magnitude than the zero order approximation ($-.2 \times (\overline{w^{'}\theta^{'})_{s}}$) but seem to increase with increased $\gamma$.\\


\begin{figure}[htbp]
    \centering
    % plot_theta_profs.py [master 9883fda] Round 2 of Plots in Results
    \includegraphics[scale=.5]{/newtera/tera/phil/nchaparr/python/Plotting/Dec252013/pngs/theta_profs2hrs}
    \caption{$\overline{\theta}$ profiles at 2 hours for all runs}
    \label{fig:pottempprofs2hrs}   % label should change
\end{figure}

\begin{figure}[htbp]
    \centering
    %plot_theta_profs.py [master 03d2835] Round 1 of Plots in Results
    \includegraphics[scale=.5]{/newtera/tera/phil/nchaparr/python/Plotting/Dec252013/pngs/flux_profs2hrs}
    \caption{Scaled $(\overline{w^{'}\theta^{'}})_{s}$ profiles at 2 hours for all runs}
    \label{fig:fluxprofs2hrs}   % label should change
\end{figure}

\clearpage

\subsection{FFT Energy Spectra}
\FloatBarrier

In Figure \ref{fig:scalarfftw602point5}, two dimensional \acs{FFT} power spectra taken of horizontal slices of $w^{\prime}$ at three different levels ($h_{0}$, $h$ and $h_{1}$) are collapsed to one dimension by integrating around a semi-circle of positive wave-numbers.
Isotropy in all radial directions is assumed and $k = \sqrt{k_{x}^{2} + k_{y}^{2}}$.  The resulting scalar density spectra show peaks in energy at the larger scales, cascading to the lower scales roughly according to a $\frac{-5}{3}$ slope lower in the \acs{EL}.  At
the top of the \acs{EL} where turbulence is suppressed by stability, the slope is steeper.  The peak in energy occurs at smaller scales
at $h$ as compared to at $h_{0}$, indicating a change in the size of the dominant turbulent structures. The spectra for the horizontal
turbulent velocity perturbations were analogous but show lower energy as expected.\\

\begin{figure}[htbp]
    \centering
    % fft_chap.py [master 03d2835] Round 1 of Plots in Results
    \includegraphics[scale=.5]{/newtera/tera/phil/nchaparr/python/Plotting/Dec252013/pngs/scalarfftpow_w}
    \caption[\acs{FFT} energy spectra of $w^{'}$]{Scalar FFT  energy vs wavenumber ($k = \sqrt{k_{x}^{2}+k_{y}^{2}}$) for the 60/2.5 run
at 2 hours.  $E(k)$ is $E(k_{x}, k_{y})$ integrated around circles of radius $k$.  
   $E(k_{x}, k_{y})$ is the total integrated energy over the 2D domain.  
   $k_{x}$ and $k_{y}$ are number of waves per domain length.}
   \label{fig:scalarfftw602point5}   % label should change
\end{figure}

\clearpage

\subsection{Visualization of Structures Within the Entrainment Layer}
\FloatBarrier

Horizontal slices, at $h_{0}$, $h$ and $h_{1}$ as shown in Figure \ref{fig:hdefs} of the potential temperature 
and vertical velocity perturbations are plotted to see the turbulent structures.  Figure \ref{fig:conts} shows the bottom of the \acs{EL} ($h_{0}$) for the 150/10 run where coherent areas of positive and negative temperature perturbations 
correspond to areas of upward and downward moving air.  In (b) and (e) the individual thermals of relatively cool air are more evident at the inversion ($h$) and their locations correspond to areas of upward motion.  Most of the upward moving cool areas are adjacent to and even 
encircled by smaller areas of downward moving warm air.  At $h_{1}$ ((c) and (f)) peaks of cool air are associated 
with both up and down-welling.\\  

\begin{figure}[htbp]
\caption[2D horizontal slices of $\theta^{'}$ and $w^{'}$]{$\theta^{'}$ (left) and $w^{'}$ (right) at 2 hours at $h_{0}$ (a,d), $h$ (c,e) and $h_{1}$ (d,f) for the 150/10 run.} 
\begin{minipage}[b]{0.5\linewidth}
  
        %Flux_Quads.py [master 03d2835] Round 1 of Plots in Results
        \subfloat[]{\label{main:a}
                \includegraphics[scale=.36]{/newtera/tera/phil/nchaparr/python/Plotting/Mar52014/pngs/theta_cont0}}\\
        \subfloat[]{\label{main:b}      
                \includegraphics[scale=.36]{/newtera/tera/phil/nchaparr/python/Plotting/Mar52014/pngs/theta_cont1}}\\ 
        \subfloat[]{\label{main:c}      
                \includegraphics[scale=.36]{/newtera/tera/phil/nchaparr/python/Plotting/Mar52014/pngs/theta_cont2}} 
 \end{minipage}             
\quad
\begin{minipage}[b]{0.5\linewidth}
        %Flux_Quads.py [master 9883fda] Round 2 of Plots in Results
        \subfloat[]{\label{main:d}
                \includegraphics[scale=.36]{/newtera/tera/phil/nchaparr/python/Plotting/Mar52014/pngs/wvel_cont0}}\\
       
       \subfloat[]{\label{main:e}
                \includegraphics[scale=.36]{/newtera/tera/phil/nchaparr/python/Plotting/Mar52014/pngs/wvel_cont1}}\\
        
       \subfloat[]{\label{main:f}
                \includegraphics[scale=.36]{/newtera/tera/phil/nchaparr/python/Plotting/Mar52014/pngs/wvel_cont2}}                 
\end{minipage}        
        \label{fig:conts}
\end{figure}

\clearpage

\section{Local Statistics (Q1)}

\textbf{Q1: How do the distributions of local \acs{CBL} height, and the joint distributions of $w^{'}$ and $\theta^{'}$ within the \acs{EL}, vary with $(\overline{w^{'}\theta^{'}})_{s}$ and $\gamma$?}\\

\subsection{Local Mixed Layer Heights ($h_{0}^{l}$)}
\label{subsec:locmlh}     
\FloatBarrier

Local $\theta$ profiles (Figures \ref{fig:rssfitshigh} and \ref{fig:rssfitslow}) exhibit a distinct 
\acs{ML} before resuming $\gamma$ but 
not always a clearly defined \acs{EL}.  There are sharp changes in the profile well into the free 
atmosphere, due possibly to waves, which render the gradient method for determining $h^{l}$ 
unusable.  Instead a linear regression method is used, whereby three lines representing: the
 \acs{ML}, the \acs{EL} and the upper lapse rate ($\gamma$), are fit to the profile according 
to the minimum residual sum of squares (RSS).  Determining local \acs{ML} height ($h_{0}^{l}$) was 
more straight forward than the local height of maximum potential temperature gradient 
($h^{l}$) for the reasons stated above.\\  

Figure \ref{fig:rssfitshigh} shows two local $\theta$ profiles where $h_{0}^{l}$ is relatively high.  
A sharp interface is evident indicating that this is within an active thermal impinging on the stable layer.
In Figure \ref{fig:rssfitslow} where $h_{0}^{l}$ is relatively low a less defined interface indicates 
a point now outside a rising thermal.  When $\gamma$ is lower as in Figure \ref{fig:rssfitslow} (a), these inactive locations show a larger vertical region that could be called a local \acs{EL}.  In 2d horizontal plots, not shown here, regions of high 
$h_{0}^{l}$ corresponded to regions of upward moving relatively cool air at $h$.\\

The distribution of $h_{0}^{l}$ is related to the depth of the entrainment layer (\acs{EL}).
Spread increases with increasing $(\overline{w^{'}\theta^{'}})_{s}$ and decreases with increasing $\gamma$
(Figure \ref{fig:localhhist}).  When scaled by $h$  (Figure \ref{fig:localhpdf}), the local \acs{ML} height distribution 
narrows with increased $\gamma$ and seems relatively uninfluenced by change in $(\overline{w^{'}\theta^{'}})_{s}$.  
The upper limit seems to be constant at about 1.1($\times h$) , whereas the lower limit varies 
depending on $\gamma$.   Runs with lower $h$ and narrower $\Delta h$ have relatively 
larger spacing between bins and so higher numbers in each bin.  The above supports the results of
Section \ref{sec:deltahri}.\\

\begin{figure}[htbp]
%Pcolor_Peaks.py [master 61491be] rss_fit plots
\begin{minipage}[b]{0.5\linewidth}
        %
        \subfloat[]{\label{main:a}
                \includegraphics[scale=.36]{/newtera/tera/phil/nchaparr/python/Plotting/Dec252013/pngs/rss_fit_high}}\\
        \end{minipage}             
\quad
\begin{minipage}[b]{0.5\linewidth}
        \subfloat[]{\label{main:b}          
          
                \includegraphics[scale=.36]{/newtera/tera/phil/nchaparr/python/Plotting/Mar52014/pngs/rss_fit_high}}\\
       
       \end{minipage}
        \caption[High local \acs{ML}]{Local vertical $\theta$ profiles with 3-line fit for the 60/2.5 (a) and 150/10 (b) runs at 
points where $h^{l}_{0}$ is high.}
        \label{fig:rssfitshigh}
\end{figure}

\begin{figure}[htbp]
%Pcolor_Peaks.py [master 61491be] rss_fit plots
\begin{minipage}[b]{0.5\linewidth}
        %
        \subfloat[]{\label{main:a}
                \includegraphics[scale=.36]{/newtera/tera/phil/nchaparr/python/Plotting/Dec252013/pngs/rss_fit_low}}\\
        \end{minipage}             
\quad
\begin{minipage}[b]{0.5\linewidth}
        \subfloat[]{\label{main:b}          
          
                \includegraphics[scale=.36]{/newtera/tera/phil/nchaparr/python/Plotting/Mar52014/pngs/rss_fit_low}}\\
       
       \end{minipage}
        \caption[Low local \acs{ML}]{Local vertical $\theta$ profiles with 3-line fit for the 60/2.5 (a) and 150/10 (b) runs at 
points where $h^{l}_{0}$ is low.}
        \label{fig:rssfitslow}
\end{figure}

\begin{figure}[htbp]
\caption[Local \acs{ML} height distributions]{Histograms of $h^{l}_{0}$ for $(\overline{w^{'}\theta^{'}})_{s} = 150$ to $60 (W / m^{2})$ (a to c) and $ \gamma = 10$ to $2.5 (K/Km)$ (c to g) at 5 hours}
%ML_Height_hist.py(sp?) master f992942ade
\begin{minipage}[b]{0.32\linewidth} 
        
        \subfloat[]{\label{main:a}
                \includegraphics[scale=.22]{/newtera/tera/phil/nchaparr/python/Plotting/Mar52014/pngs/ML_Height_hist}}\\
        \subfloat[]{\label{main:b}      
                \includegraphics[scale=.22]{/newtera/tera/phil/nchaparr/python/Plotting/Dec142013/pngs/ML_Height_hist}}\\ 
        \subfloat[]{\label{main:c}          
                \includegraphics[scale=.22]{/newtera/tera/phil/nchaparr/python/Plotting/Mar12014/pngs/ML_Height_hist}} 
 \end{minipage}             
%\quad
\begin{minipage}[b]{0.32\linewidth}
        \subfloat[]{\label{main:d}
                \includegraphics[scale=.22]{/newtera/tera/phil/nchaparr/python/Plotting/Jan152014_1/pngs/ML_Height_hist}}\\
       \subfloat[]{\label{main:e}
                \includegraphics[scale=.22]{/newtera/tera/phil/nchaparr/python/Plotting/Nov302013/pngs/ML_Height_hist}}\\
       \subfloat[]{\label{main:f}
                \includegraphics[scale=.22]{/newtera/tera/phil/nchaparr/python/Plotting/Dec202013/pngs/ML_Height_hist}}                 
\end{minipage}
\begin{minipage}[b]{0.32\linewidth}
        %\subfloat[]{\label{main:d}
        %        \includegraphics[scale=.18]{/newtera/tera/phil/nchaparr/python/Plotting/Mar52014/pngs/ML_Height_hist}}\\
       
       %\subfloat[]{\label{main:e}
       %         \includegraphics[scale=.18]{/newtera/tera/phil/nchaparr/python/Plotting/Mar52014/pngs/ML_Height_hist}}\\
       \vspace{10mm} 
       \subfloat[]{\label{main:f}
                \includegraphics[scale=.22]{/newtera/tera/phil/nchaparr/python/Plotting/Dec252013/pngs/ML_Height_hist}}                 
\end{minipage}
       
        \label{fig:localhhist}
\end{figure}

\begin{figure}[htbp]
\caption[Scaled local \acs{ML} height PDFs]{PDFs of $\frac{h^{l}_{0}}{h}$ for $(\overline{w^{'}\theta^{'}})_{s} = 150$ to $60 (W / m^{2})$ (a to c) and $ \gamma = 10$ to $2.5 (K/Km)$ (c to g) at 5 hours}
%ML_Height_hist.py(sp?) master f992942ade
\begin{minipage}[b]{0.32\linewidth} 
        
        \subfloat[]{\label{main:a}
                \includegraphics[scale=.22]{/newtera/tera/phil/nchaparr/python/Plotting/Mar52014/pngs/Scaled_ML_Height_hist}}\\
        \subfloat[]{\label{main:b}      
                \includegraphics[scale=.22]{/newtera/tera/phil/nchaparr/python/Plotting/Dec142013/pngs/Scaled_ML_Height_hist}}\\ 
        \subfloat[]{\label{main:c}          
                \includegraphics[scale=.22]{/newtera/tera/phil/nchaparr/python/Plotting/Mar12014/pngs/Scaled_ML_Height_hist}} 
 \end{minipage}             
%\quad
\begin{minipage}[b]{0.32\linewidth}
        \subfloat[]{\label{main:d}
                \includegraphics[scale=.22]{/newtera/tera/phil/nchaparr/python/Plotting/Jan152014_1/pngs/Scaled_ML_Height_hist}}\\
       \subfloat[]{\label{main:e}
                \includegraphics[scale=.22]{/newtera/tera/phil/nchaparr/python/Plotting/Nov302013/pngs/Scaled_ML_Height_hist}}\\
       \subfloat[]{\label{main:f}
                \includegraphics[scale=.22]{/newtera/tera/phil/nchaparr/python/Plotting/Dec202013/pngs/Scaled_ML_Height_hist}}                 
\end{minipage}
\begin{minipage}[b]{0.32\linewidth}
        %\subfloat[]{\label{main:d}
        %        \includegraphics[scale=.18]{/newtera/tera/phil/nchaparr/python/Plotting/Mar52014/pngs/ML_Height_hist}}\\
       
       %\subfloat[]{\label{main:e}
       %         \includegraphics[scale=.18]{/newtera/tera/phil/nchaparr/python/Plotting/Mar52014/pngs/ML_Height_hist}}\\
       \vspace{10mm} 
       \subfloat[]{\label{main:f}
                \includegraphics[scale=.22]{/newtera/tera/phil/nchaparr/python/Plotting/Dec252013/pngs/Scaled_ML_Height_hist}}                 
\end{minipage}        
        \label{fig:localhpdf}
\end{figure}

%\begin{figure}[htbp]
 %   \centering
    %plot_height.py [master 199de9a7cf]  
  %  \includegraphics[scale=.5]{/newtera/tera/phil/nchaparr/python/Plotting/Dec252013/pngs/varvsinvri}
  %  \caption{Variance vs \acs{Ri}$^{-1}$ at 5 hours}
   % \label{fig:varsvsinvri}   % label should change
%\end{figure}

\clearpage

\subsection{Flux Quadrants and the local turbulent Velocity and Potential Temperature Perturbations}
\label{subsec:fluxquadrants}     
\FloatBarrier
\subsubsection{2D Histograms of $\theta^{'}$ and $w^{'}$}

In Figure \ref{fig:flqdhsts} 2D histograms of the four quadrants are plotted at $h_{0}$ (a), $h$ (b) and $h_{1}$ (c). At $h_{0}$ fast updraughts are relatively warm.  At $h$ the faster updraughts are now relatively cool and movement (both up and down) of warmer air from aloft becomes more prominent. At the top of the \acs{EL} (c) velocities are damped and the distributions approach symmetry apart from some slow, cool, impinging up and down-draughts.\\

\begin{figure}[htbp]
\makebox[\textwidth][c]{
%ML_Height_hist.py(sp?) master f992942ade
\begin{minipage}[b]{0.4\textwidth}
 
\subfloat[]{\label{main:c}          
          %plot_height [master b9c30ad] scaleddeltahinvri1
                \includegraphics[scale=.3]{/newtera/tera/phil/nchaparr/python/Plotting/Dec252013/pngs/fluxquadhist0}}\\      
             
\end{minipage}             
%\quad
\begin{minipage}[b]{0.4\textwidth}
\subfloat[]{\label{main:a}
                \includegraphics[scale=.3]{/newtera/tera/phil/nchaparr/python/Plotting/Dec252013/pngs/fluxquadhist1}}\\
               
\end{minipage}
\begin{minipage}[b]{0.4\textwidth}
\subfloat[]{\label{main:b}          
          %plot_height [master b9c30ad] scaleddeltahinvri1
                \includegraphics[scale=.3]{/newtera/tera/phil/nchaparr/python/Plotting/Dec252013/pngs/fluxquadhist2}}\\        
\end{minipage}
}
\caption[2D distributions of $w^{'}$ and $\theta^{'}$]{2D Histograms of $w^{'}$ and $\theta^{'}$ at $h_{0}$, $h$ and $h_{1}$ for the 60/2.5 run at 5 hours}        
        \label{fig:flqdhsts}
\end{figure}

\clearpage

The 2D histograms for all runs at $h$ are plotted in Figure \ref{fig:fluxquadsh1} to visualize how the distributions are influenced by changes in $\overline{w^{'} \theta^{'}}_{s}$ and $\gamma$.  In order to isolate the effects of $\gamma$,  $w^{'}$ and $\theta^{'}$ are scaled by $w^{*}$ and $\theta^{*}$ respectively and plotted in Figure \ref{fig:scaled_fluxquadsh1}.  The spread of $w^{'}$ and $\theta^{'}$ both increase with increasing $\overline{w^{'}\theta^{'}}$ whereas that of $\theta^{'}$ increases only slightly with increased stability in Figure \ref{fig:fluxquadsh1}.  As expected, stability inhibits both upward and downward $w^{'}$. The scaled version in Figure \ref{fig:scaled_fluxquadsh1} shows a damping of the velocity perturbations and a positive shift in temperature perturbations with increased $\gamma$.\\ 

Although the quadrant of overall largest magnitude is that of upward moving cool air ($w^{'+}\theta^{'-}$), \citeauthor{SullMoengStev}'s (\citeyear{SullMoengStev}) assertion that in the \acs{EL} the net heat flux is downward moving warm ($w^{'-}\theta^{'+}$) air because the other three quadrants cancel, is found to be approximately true.\\


\begin{figure}[htbp]
\centering
 \includegraphics[scale=.8]{/newtera/tera/phil/nchaparr/python/Plotting/Dec252013/pngs/fluxquadhists1}                 
\caption[2D distributions of $w^{'}$ and $\theta^{'}$ for all runs]{$\overline{w^{'}\theta^{'}}$ quadrants at $h$ for $w^{'}\theta^{'} = 150 \ - \ 60$(W/$m^{2}$) (top - bottom) and $\gamma = 10 \ - \  2.5$(K/Km) (left - right) at 5 hours}
\label{fig:fluxquadsh1}
\end{figure}

\begin{figure}[htbp]
\centering
 \includegraphics[scale=.8]{/newtera/tera/phil/nchaparr/python/Plotting/Dec252013/pngs/scaled_fluxquadhists1}                 
\caption[Scaled 2D distributions of $w^{'}$ and $\theta^{'}$ for all runs]{$\overline{w^{'}\theta^{'}}$ quadrants at $h$ for $(\overline{w^{'}\theta^{'}})_{s} = 150 \ - \ 60$(W/$m^{2}$) (top - bottom) and $\gamma = 10 \ - \  2.5$(K/Km) (left - right) at 5 hours}
\label{fig:scaled_fluxquadsh1}
\end{figure}


\clearpage

\subsubsection{Downward Moving Warm Air at $h$}
\label{sec:downwarm}

The magnitude of the average downward moving quadrant ($\overline{w^{'-}\theta^{'+}}$) at $h$ can be taken as a measure of entrainment.  Figure \ref{fig:downwarm} shows that this increases in time as well as with increased $(\overline{w^{'}\theta^{'}})_{s}$.  Grouping according to $(\overline{w^{'}\theta^{'}})_{s}$ is evident and there is further collapse when this is applied as scale in Figure \ref{fig:downwarm} (b).\\

\begin{figure}[htbp]
\begin{minipage}[b]{0.5\linewidth}
        %plot_height [master c7af4de] scaleddeltahinvri
        \subfloat[]{\label{main:a}
                \includegraphics[scale=.36]{/newtera/tera/phil/nchaparr/python/Plotting/Dec252013/pngs/downwarm.pdf}}\\
        \end{minipage}             
\quad
\begin{minipage}[b]{0.5\linewidth}
        \subfloat[]{\label{main:d}          
          %plot_height [master b9c30ad] scaleddeltahinvri1
                \includegraphics[scale=.36]{/newtera/tera/phil/nchaparr/python/Plotting/Dec252013/pngs/scaled_downwarm.pdf}}\\
     
       \end{minipage}
        \caption[Downward moving warm air at $h$]{Plots of (a) the average downward moving warm air at $h$ ($\overline{w^{\prime-}\theta^{\prime+}}_{h}$) and (b) $\overline{w^{\prime-}\theta^{\prime+}}_{h}$ scaled by the average vertical turbulent heat flux $(\overline{w^{\prime}\theta^{\prime}})_{s}$ vs time}
        \label{fig:downwarm}
\end{figure}\\

Further partitioning $\overline{w^{'-}\theta^{'+}}_{h}$ into its velocity and temperature components reveals additional scaling.  Figure \ref{fig:downwarm_wvel} shows that the velocity component $\overline{w^{'-}}_{h} \ where \ \overline{\theta^{'}}_{h}>0$, is effectively scaled by $w^{*}$.\\   

\begin{figure}[htbp]
\begin{minipage}[b]{0.5\linewidth}
        %plot_height [master c7af4de] scaleddeltahinvri
        \subfloat[]{\label{main:a}
                \includegraphics[scale=.36]{/newtera/tera/phil/nchaparr/python/Plotting/Dec252013/pngs/downwarm_wvel.pdf}}\\
        \end{minipage}             
\quad
\begin{minipage}[b]{0.5\linewidth}
        \subfloat[]{\label{main:d}          
          %plot_height [master b9c30ad] scaleddeltahinvri1
                \includegraphics[scale=.36]{/newtera/tera/phil/nchaparr/python/Plotting/Dec252013/pngs/scaled_downwarm_wvel.pdf}}\\       
       \end{minipage}
        \caption[Downward turbulent velocity perturbation at $h$]{(a) Average negative vertical turbulent velocity perturbation at $h$ $\overline{w^{\prime-}}_{h}$ at points where $\theta^{\prime}>0$ and (b) $\overline{w^{\prime-}}_{h}$ where $\theta^{\prime}>0$ scaled by $w^{*}$.}
        \label{fig:downwarm_wvel}
\end{figure}

The curves representing $\overline{\theta^{'+}}_{h} \ where \ \overline{w^{'}}_{h}>0$ vs time do collapse when scaled by $\theta^{*}$ in Figure \ref{fig:downwarm_theta}.  However it seems this component approaches a constant proportion of $\gamma \Delta h$ in Figure \ref{fig:downwarm_theta1} indicating that the effects of $\gamma$ on the positive temperature perturbations at $h$ may be more important than $(\overline{w^{'}\theta^{'}})_{s}$. 

\begin{figure}[htbp]
\begin{minipage}[b]{0.5\linewidth}
        %plot_height [master c7af4de] scaleddeltahinvri
        \subfloat[]{\label{main:a}
                \includegraphics[scale=.36]{/newtera/tera/phil/nchaparr/python/Plotting/Dec252013/pngs/downwarm_theta.pdf}}\\
        \end{minipage}             
\quad
\begin{minipage}[b]{0.5\linewidth}
        \subfloat[]{\label{main:d}          
          %plot_height [master b9c30ad] scaleddeltahinvri1
                \includegraphics[scale=.36]{/newtera/tera/phil/nchaparr/python/Plotting/Dec252013/pngs/scaled_downwarm_theta.pdf}}\\      
       \end{minipage}
        \caption[Positive potential temperature perturbation at $h$ (i)]{(a) Average positive potential temperature perturbation $\overline{\theta^{\prime+}}_{h}$ at points where $w^{\prime}<0$ and (b) $\overline{\theta^{\prime+}}_{h}$ where $w^{\prime}<0$ scaled by $\theta^{*}$.}
        \label{fig:downwarm_theta}
\end{figure}

\begin{figure}[htbp]
\begin{minipage}[b]{0.5\linewidth}
        %plot_height [master c7af4de] scaleddeltahinvri
        \subfloat[]{\label{main:a}
                \includegraphics[scale=.36]{/newtera/tera/phil/nchaparr/python/Plotting/Dec252013/pngs/downwarm_theta.pdf}}\\
        \end{minipage}             
\quad
\begin{minipage}[b]{0.5\linewidth}
        \subfloat[]{\label{main:d}          
          %plot_height [master b9c30ad] scaleddeltahinvri1
                \includegraphics[scale=.36]{/newtera/tera/phil/nchaparr/python/Plotting/Dec252013/pngs/scaled_downwarm_theta1.pdf}}\\      
       \end{minipage}
        \caption[Positive potential temperature perturbation at $h$ (ii)]{(a) Average positive potential temperature perturbation at $h$ $\theta^{\prime+}_{h}$ at points where $w^{\prime}<0$ and (b) $\overline{\theta^{\prime+}}_{h}$ where $w^{\prime}<0$ scaled by $\gamma \Delta h$.}
        \label{fig:downwarm_theta1}
\end{figure}

\clearpage

%\section{$h$ and  $\Delta h$ based on Average Profiles}
%\label{sec:hdeltahavprofs}

%\FloatBarrier

\section{Relationship of Entrainment Layer Depth to Richardson Number (Q2)}
\label{sec:deltahri}
\FloatBarrier
\textbf{Q2: Can the \acs{EL} limits be defined based on the $\overline{\theta}$ profile and what is the relationship} 


\begin{equation}
\frac{\Delta h}{h} \propto Ri ^{b}\tag{\ref{eq:dhvsri}}
\end{equation}


\textbf{of the resulting depth ($\Delta h$) to \acs{Ri}?}\\

The scaled upper EL limits ($\frac{h_{1}}{h}$) collapse well in Figure \ref{fig:scaledELlims} (a) 
to an initial value of approximately 1.15, decreasing to about 1.1.  $\frac{h_{0}}{h}$s appear 
grouped according to $\gamma$ and increase with respect to time.  So overall the scaled \acs{EL} appears
to narrow with time.   The scaled flux based \acs{EL} ($\frac{z_{f0}}{z_{f}}$ and $\frac{z_{f1}}{z_{f}}$) appears to remain constant 
with respect to time in Figure \ref{fig:scaledELlims} (b).\\

The lower entrainment layer limit $h_{0}$ is the point at which the vertical 
$\frac{\partial \overline{\theta}}{\partial z}$ exceeds a threshold (.0002), chosen such that
it is positive, and at least an order of magnitude smaller than $\gamma$.   Although the resulting 
scaled \acs{EL} depth decreases with increasing \acs{Ri} grouping according to $\gamma$ is evident 
in Figure \ref{fig:scaledeltahinvri}.\\


\begin{figure}[htbp]
\begin{minipage}[b]{0.5\linewidth}
        %plot_height [master fbd2dfd] invristime1
        \subfloat[]{\label{main:a}
\includegraphics[scale=.36]{/newtera/tera/phil/nchaparr/python/Plotting/Dec252013/pngs/scaleddeltahstime}}
 
    %            \includegraphics[scale=.36]{/newtera/tera/phil/nchaparr/python/Plotting/Dec252013/pngs/scaledweinvri_Delta}}\\
        \end{minipage}             
\quad
\begin{minipage}[b]{0.5\linewidth}
        \subfloat[]{\label{main:d}
\includegraphics[scale=.36]{/newtera/tera/phil/nchaparr/python/Plotting/Dec252013/pngs/scaled_h_f0_time}}              
          %plot_height [master 01c3721] scaledweinvri1
                %\includegraphics[scale=.36]{/newtera/tera/phil/nchaparr/python/Plotting/Dec252013/pngs/scaledweinvri_delta}}\\       
       \end{minipage}
        \caption[Scaled \acs{EL} limits]{(a)Scaled entrainment layer limits ($\frac{h_{1}}{h}$ and $\frac{h_{0}}{h}$) vs time and (b) Scaled entrainment layer limits ($\frac{z_{f1}}{z_{f}}$ and $\frac{z_{f0}}{z_{f}}$) vs time}
        \label{fig:scaledELlims}
\end{figure}


\begin{figure}[htbp]
    \centering
    %plot_height.py [master fd5c6b1] delta h vs time1  
    \includegraphics[scale=.5]{/newtera/tera/phil/nchaparr/python/Plotting/Dec252013/pngs/theta_grad_profs}
    \caption{Vertical $\frac{\partial \overline{\theta}}{\partial z}$ profiles with threshold at .0002}
    \label{fig:thresh}   % label should change
\end{figure}

\begin{figure}[htbp]
\centering
%\begin{minipage}[b]{0.5\linewidth}
        %plot_height [master c7af4de] scaleddeltahinvri
 %       \subfloat[]{\label{main:a}
 \includegraphics[scale=.5]{/newtera/tera/phil/nchaparr/python/Plotting/Dec252013/pngs/scaleddeltahinvri}
  %      \end{minipage}             
%\quad
%\begin{minipage}[b]{0.5\linewidth}
 %       \subfloat[]{\label{main:d}          
          %plot_height [master b9c30ad] scaleddeltahinvri1
  %              \includegraphics[scale=.36]{/newtera/tera/phil/nchaparr/python/Plotting/Dec252013/pngs/scaleddeltahinvri1}}\\
       
   %    \end{minipage}
        \caption{Scaled EL depth ($\frac{h_{1}}{h}$ and $\frac{h_{0}}{h}$) vs inverse bulk Richardson Number with threshold at .0002}
         \label{fig:scaledeltahinvri}
\end{figure}

\clearpage
\subsubsection{Threshold Test for lower \acs{EL} Limit, $h_{0}$}
To explore how varying the threshold value affects the relationship between scaled \acs{EL} depth
and Richardson number (\acs{Ri}), 

plots analogous to Figure \ref{fig:scaledeltahinvri} were produced at two 
additional thresholds.  A higher value (.0004) results in a higher $h_{0}$   
and so a narrower \acs{EL} but a similar grouping according to $\gamma$ (Figure \ref{fig:scaledeltahinvri1}).
A lower threshold value (.0001) results in a lower $h_{0}$
but also similar grouping according to $\gamma$ (Figure \ref{fig:scaledeltahinvri2}).\\

\begin{figure}[htbp]
    \centering
    %plot_height.py [master fd5c6b1] delta h vs time1  
    \includegraphics[scale=.5]{/newtera/tera/phil/nchaparr/python/Plotting/Dec252013/pngs/scaleddeltahinvri_5}
    \caption{Scaled EL depth vs inverse Richardson Number with threshold at .0004}
    \label{fig:scaledeltahinvri1}   % label should change
\end{figure}

\begin{figure}[htbp]
    \centering
    %plot_height.py [master fd5c6b1] delta h vs time1  
    \includegraphics[scale=.5]{/newtera/tera/phil/nchaparr/python/Plotting/Dec252013/pngs/scaleddeltahinvri_6}
    \caption{Scaled EL depth vs inverse bulk Richardson Number with threshold at .0001}
    \label{fig:scaledeltahinvri2}   % label should change
\end{figure}

\clearpage
\subsubsection{\acs{EL} Limits based on scaled vertical Profiles: $\frac{\frac{\partial \overline{\theta}}{\partial z}}{\gamma}$ and $\frac{\overline{w^{'}\theta^{'}}}{(\overline{w^{'}\theta^{'}})_{s}}$}
There is a collapsing effect on the scaled $\Delta h$ vs \acs{Ri} relationship when
the heights are defined based on the scaled vertical potential temperature gradient 
$\frac{\frac{\partial \overline{\theta}}{\partial z}}{\gamma}$ profile in Figure \ref{fig:deltahinvri_scaled}.  So from here on all heights will be defined based on the scaled average profiles.  The magnitude of $\frac{\partial \overline{\theta}}{\partial z}$ in the upper \acs{ML} is scaled by $\gamma$ which compliments the finding in Section \ref{sec:downwarm}.  Furthermore the scaled magnitude of $\Delta h$ decreases with increasing \acs{Ri} as supported by Figure \ref{fig:localhpdf}.  Figure \ref{fig:deltahinvri_scaled} (b) shows little or no \acs{Ri} dependence when $\Delta h$, and so $\Delta \theta$, are based on the $\overline{w^{'}\theta^{'}}$ profile.  The plot in Figure \ref{fig:loglogdeltahinvri} seems to support an exponent $b = -\frac{1}{2}$.

\begin{figure}[htbp]
    \centering
    %plot_height.py [master fd5c6b1] delta h vs time1  
    \includegraphics[scale=.5]{/newtera/tera/phil/nchaparr/python/Plotting/Dec252013/pngs/scaled_theta_grad_profs}
    \caption{Scaled vertical $\frac{\partial \overline{\theta}}{\partial z}$ profiles with threshold at .03}
    \label{fig:thresh3}   % label should change
\end{figure}

\begin{figure}[htbp]
\begin{minipage}[b]{0.5\linewidth}
        %
        \subfloat[]{\label{main:a}
                \includegraphics[scale=.36]{/newtera/tera/phil/nchaparr/python/Plotting/Dec252013/pngs/scaleddeltahinvri_4}}\\
        \end{minipage}             
\quad
\begin{minipage}[b]{0.5\linewidth}
        \subfloat[]{\label{main:d}          
          %
                \includegraphics[scale=.36]{/newtera/tera/phil/nchaparr/python/Plotting/Dec252013/pngs/scaleddeltahinvri_f}}\\      
       \end{minipage}
        \caption[scaled \acs{EL} depth vs \acs{Ri}$^{-1}$]{Plots of scaled \acs{EL} depth vs \acs{Ri}$^{-1}$. \acs{EL} limits and so $\Delta \theta$ are based on the $\frac{\frac{\partial \overline{\theta}}{\partial z}}{\gamma}$ in (a) and the $\frac{\overline{w^{'}\theta^{'}}}{(\overline{w^{'}\theta^{'}})_{s}}$ profile in (b).}
        \label{fig:deltahinvri_scaled}
\end{figure}


\begin{figure}[htbp]
\centering
\includegraphics[scale=.5]{/newtera/tera/phil/nchaparr/python/Plotting/Dec252013/pngs/loglog_scaleddeltahinvri_4}\\
\caption[Log-log plot of scaled \acs{EL} depth vs \acs{Ri}$^{-1}$]{Scaled \acs{EL} depth vs \acs{Ri}$^{-1}$ based on the $\frac{\frac{\partial \overline{\theta}}{\partial z}}{\gamma}$ profile in log-log coordinates to see likely values of the exponent $b$}
\label{fig:loglogdeltahinvri}
\end{figure}

\clearpage

\section{Relationship of Entrainment Rate to Richardson Number (Q3)}
\FloatBarrier
\label{sec:weri}
\textbf{Q3: How does defining the $\theta$ jump based on the vertical $\overline{\theta}$ profile across the \acs{EL} as in Figure \ref{fig:1storder} vs at the inversion ($h$) as in Figure \ref{fig:0order}, affect the entrainment relation 
\begin{equation}
\frac{w_{e}}{w^{*}} \propto Ri^{a} \tag{\ref{eq:ervsri}}
\end{equation}
and in particular $a$?}\\

Convective boundary layer height ($h$) in Figure \ref{fig:hvstime} grows rapidly initially with a steadily decreasing rate
and relates to the square-root of time in Figure \ref{fig:loghvstime} so I can assume a quasi-steady state. The height of minimum average heat flux $z_{f}$ is a constant proportion of $h$ in Figure \ref{fig:zfvstime} indicating that this point advances more slowly than $h$.\\
  
\begin{figure}[htbp]
    \centering
    %plot_height.py[master 1573b9d] h vs time plot
    \includegraphics[scale=.5]{/newtera/tera/phil/nchaparr/python/Plotting/Dec252013/pngs/hvstime}
    \caption{$h$ vs time for all runs}
    \label{fig:hvstime}   % label should change
\end{figure}

\begin{figure}[htbp]
    \centering
    %plot_height.py[master 1573b9d] h vs time plot
    \includegraphics[scale=.5]{/newtera/tera/phil/nchaparr/python/Plotting/Dec252013/pngs/hstimelog}
    \caption{$h$ vs time for all runs on log-log coordinates}
    \label{fig:loghvstime}   % label should change
\end{figure}

%need flux height scaled by h plot

\begin{figure}[htbp]
    \centering
    %plot_height.py[master 1573b9d] h vs time plot
    \includegraphics[scale=.5]{/newtera/tera/phil/nchaparr/python/Plotting/Dec252013/pngs/scaled_h_f_time}
    \caption{$\frac{z_{f}}{h}$ vs Time}
    \label{fig:zfvstime}   % label should change
\end{figure}

\clearpage

\subsubsection{\acs{CBL} Height and $\theta$ Jumps based on $\frac{\frac{\partial \overline{\theta}}{\partial z}}{\gamma}$}

The inverse Richardson numbers in Figure \ref{fig:invristime}, \acs{Ri}$_{\Delta}$ and \acs{Ri}$_{\delta}$, based on the scaled $\frac{\partial \overline{\theta}}{\partial z}$ profile decrease in time and group according to $\gamma$. There is an overall difference in magnitude since $\Delta \theta > \delta \theta$.\\  

\begin{figure}[htbp]

\begin{minipage}[b]{0.5\linewidth}
         
        \subfloat[]{\label{main:a}
                \includegraphics[scale=.36]{/newtera/tera/phil/nchaparr/python/Plotting/Dec252013/pngs/invristime_Delta}}\\
        \end{minipage}             
\quad
\begin{minipage}[b]{0.5\linewidth}
        \subfloat[]{\label{main:d}
          %plot_height [master fbd2dfd] invristime1, see branches named accordingly
                \includegraphics[scale=.36]{/newtera/tera/phil/nchaparr/python/Plotting/Dec252013/pngs/invristime_delta}}\\
       
       \end{minipage}
        \caption[Richardson numbers based on $\frac{\frac{\partial \overline{\theta}}{\partial z}}{\gamma}$]{Inverse Richardson number vs time based on the $\frac{\frac{\partial \overline{\theta}}{\partial z}}{\gamma}$
profile using $\Delta h$ across the \acs{EL} in (a) and $\delta \theta$ at $h$ in (b).  See Table \ref{table:reldefs}.}
        \label{fig:invristime}
\end{figure}

\clearpage

The entrainment rate ($w_{e}= \frac{dh}{dt}$) is determined from the slope of a second order polynomial fit to $h(time)$ in Figure \ref{fig:hvstime}.  When $w_{e}$ is scaled by $w^{*}$, the resulting relationship to \acs{Ri}$_{\Delta}$ 
in Figure \ref{fig:weinvri} (a) seems to have exponent $a = -1$ at lower \acs{Ri}$_{\Delta}$ and $a = -\frac{3}{2}$ at higher \acs{Ri}$_{\Delta}$.  Whereas in (b) it seems to approach a value of $a = -1$ at higher \acs{Ri} but a value of lower in magnitude might fit better overall. \\    

\begin{figure}[htbp]
\begin{minipage}[b]{0.5\linewidth}
        %plot_height [master fbd2dfd] invristime1
        \subfloat[]{\label{main:a}
                \includegraphics[scale=.36]{/newtera/tera/phil/nchaparr/python/Plotting/Dec252013/pngs/scaledweinvri_Delta}}\\
        \end{minipage}             
\quad
\begin{minipage}[b]{0.5\linewidth}
        \subfloat[]{\label{main:d}          
          %plot_height [master 01c3721] scaledweinvri1
                \includegraphics[scale=.36]{/newtera/tera/phil/nchaparr/python/Plotting/Dec252013/pngs/scaledweinvri_delta}}\\       
       \end{minipage}
        \caption[Scaled entrainment rate vs inverse Richardson number (i)]{Scaled entrainment rate vs inverse Richardson number (\acs{Ri}$^{-1}$), where \acs{Ri} is based on the $\frac{\frac{\partial \overline{\theta}}{\partial z}}{\gamma}$ profile using $\Delta h$ across the \acs{EL} in (a) and $\delta \theta$ at $h$ in (b). See Figure \ref{fig:hdefs}.}
        \label{fig:weinvri}
\end{figure}

\clearpage

\subsubsection{\acs{CBL} Height and $\theta$ Jumps based on $\frac{\overline{w^{'}\theta^{'}}}{(\overline{w^{'}\theta^{'}})_{s}}$}

Something similar can be said when $\Delta \theta$ and $\delta \theta$ are based on the $\overline{w^{'}\theta^{'}}$ profile although \acs{Ri}$_{\Delta}$ shows more scatter in Figure \ref{fig:invristime_f} (a).

\begin{figure}[htbp]
\begin{minipage}[b]{0.5\linewidth}
         
        \subfloat[]{\label{main:a}
                \includegraphics[scale=.36]{/newtera/tera/phil/nchaparr/python/Plotting/Dec252013/pngs/invristime_Delta_f}}\\
        \end{minipage}             
\quad
\begin{minipage}[b]{0.5\linewidth}
        \subfloat[]{\label{main:d}
          %plot_height [master fbd2dfd] invristime1, see branches named accordingly
                \includegraphics[scale=.36]{/newtera/tera/phil/nchaparr/python/Plotting/Dec252013/pngs/invristime_delta_f}}\\
       
       \end{minipage}
        \caption[Richardson numbers based on $\frac{\overline{w^{'}\theta^{'}}}{\overline{w^{'}\theta^{'}}_{s}}$]{Inverse Richardson number vs time based on the $\frac{\overline{w^{'}\theta^{'}}}{\overline{w^{'}\theta^{'}}_{s}}$
profile using $\Delta \theta$ across the \acs{EL} in (a) and $\delta \theta$ at $z_{f}$ in (b).  See Figure \ref{fig:hdefs}.}
        \label{fig:invristime_f}
\end{figure}

Figure \ref{fig:weinvri_f} is as Figure \ref{fig:weinvri} except all heights are based on the scaled $\overline{w^{'}\theta^{'}}$ profile. The relationship of scaled entrainment rate to \acs{Ri}$_{\Delta}$ in (a) shows scatter and either value of $a$ or a value in between could fit.  Whereas the exponent in the relationship to \acs{Ri}$_{\delta}$ in (b) seems to change throughout the run(s) and a value less (in magnitude) than $-1$ might fit better. \\    

\begin{figure}[htbp]
\begin{minipage}[b]{0.5\linewidth}
        %plot_height [master fbd2dfd] invristime1
        \subfloat[]{\label{main:a}
                \includegraphics[scale=.36]{/newtera/tera/phil/nchaparr/python/Plotting/Dec252013/pngs/scaledweinvri_Delta_f}}\\
        \end{minipage}             
\quad
\begin{minipage}[b]{0.5\linewidth}
        \subfloat[]{\label{main:d}          
          %plot_height [master 01c3721] scaledweinvri1
                \includegraphics[scale=.36]{/newtera/tera/phil/nchaparr/python/Plotting/Dec252013/pngs/scaledweinvri_delta_f}}\\
       
       \end{minipage}
        \caption[Scaled entrainment rate vs inverse Richardson number (ii)]{Scaled entrainment rate vs inverse Richardson number (\acs{Ri}$^{-1}$), where \acs{Ri} is based on the $\frac{\overline{w^{'}\theta^{'}}}{(\overline{w^{'}\theta^{'}})_{s}}$
profile using $\Delta h$ across the \acs{EL} in (a) and $\delta \theta$ at $z_{f}$ in (b).  See Figure \ref{fig:hdefs}.}
        \label{fig:weinvri_f}
\end{figure}

\clearpage

In conclusion the relationship of scaled entrainment rate to \acs{Ri}$_{\Delta}$ based on the $\frac{\frac{\partial \overline{\theta}}{\partial z}}{\gamma}$ profile shows the least scatter over time and between runs (\ref{fig:weinvri}).  Here the exponent seems to start at a value close to $-1$ increasing, with higher \acs{Ri}, to close to $-\frac{3}{2}$.  Overall the definition of the temperature jump certainly has an effect, $\Delta \theta$ yielding a higher value of $a$ than $\delta \theta$.

\endinput

Any text after an \endinput is ignored.
You could put scraps here or things in progress.
