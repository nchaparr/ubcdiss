%%%%%%%%%%%%%%%%%%%%%%%%%%%%%%%%%%%%%%%%%%%%%%%%%%%%%%%%%%%%%%%%%%%%%%
% Template for a UBC-compliant dissertation
% At the minimum, you will need to change the information found
% after the "Document meta-data"
%
%!TEX TS-program = pdflatex
%!TEX encoding = UTF-8 Unicode

%% The ubcdiss class provides several options:
%%   fogscopy
%%       set parameters to exactly how FoGS specifies
%%         * single-sided
%%         * page-numbering starts from title page
%%         * the lists of figures and tables have each entry prefixed
%%           with 'Figure' or 'Table'
%%       This can be tested by `\iffogscopy ... \else ... \fi'
%%   10pt, 11pt, 12pt
%%       set default font size
%%   oneside, twoside
%%       whether to format for single-sided or double-sided printing
%%   balanced
%%       when double-sided, ensure page content is centred
%%       rather than slightly offset (the default)
%%   singlespacing, onehalfspacing, doublespacing
%%       set default inter-line text spacing; the ubcdiss class
%%       provides \textspacing to revert to this configured spacing
%%   draft
%%       disable more intensive processing, such as including
%%       graphics, etc.
%%

% For submission to FoGS
\documentclass[fogscopy,onehalfspacing,11pt]{ubcdiss}

% For your own copies (looks nicer)
% \documentclass[balanced,twoside,11pt]{ubcdiss}
\usepackage{wrapfig}
\usepackage{flafter}
%%%%%%%%%%%%%%%%%%%%%%%%%%%%%%%%%%%%%%%%%%%%%%%%%%%%%%%%%%%%%%%%%%%%%%
%%%%%%%%%%%%%%%%%%%%%%%%%%%%%%%%%%%%%%%%%%%%%%%%%%%%%%%%%%%%%%%%%%%%%%
%%
%% FONTS:
%% 
%% The defaults below configures Times Roman for the serif font,
%% Helvetica for the sans serif font, and Courier for the
%% typewriter-style font.  Configuring fonts can be time
%% consuming; we recommend skipping to END FONTS!
%% 
%% If you're feeling brave, have lots of time, and wish to use one
%% your platform's native fonts, see the commented out bits below for
%% XeTeX/XeLaTeX.  This is not for the faint at heart. 
%% (And shouldn't you be writing? :-)
%%

%% NFSS font specification (New Font Selection Scheme)
%\usepackage{times,mathptmx,courier}
%\usepackage[scaled=.92]{helvet}

%% Math or theory people may want to include the handy AMS macros
%\usepackage{amssymb}
\usepackage{amsmath}
%\usepackage{amsfonts}

%% The pifont package provides access to the elements in the dingbat font.   
%% Use \ding{##} for a particular dingbat (see p7 of psnfss2e.pdf)
%%   Useful:
%%     51,52 different forms of a checkmark
%%     54,55,56 different forms of a cross (saltyre)
%%     172-181 are 1-10 in open circle (serif)
%%     182-191 are 1-10 black circle (serif)
%%     192-201 are 1-10 in open circle (sans serif)
%%     202-211 are 1-10 in black circle (sans serif)
%% \begin{dinglist}{##}\item... or dingautolist (which auto-increments)
%% to create a bullet list with the provided character.
\usepackage{pifont}

%%%%%%%%%%%%%%%%%%%%%%%%%%%%%%%%%%%%%%%%%%%%%%%%%%%%%%%%%%%%%%%%%%%%%%
%% Configure fonts for XeTeX / XeLaTeX using the fontspec package.
%% Be sure to check out the fontspec documentation.
%\usepackage{fontspec,xltxtra,xunicode}	% required
%\defaultfontfeatures{Mapping=tex-text}	% recommended
%% Minion Pro and Myriad Pro are shipped with some versions of
%% Adobe Reader.  Adobe representatives have commented that these
%% fonts can be used outside of Adobe Reader.
%\setromanfont[Numbers=OldStyle]{Minion Pro}
%\setsansfont[Numbers=OldStyle,Scale=MatchLowercase]{Myriad Pro}
%\setmonofont[Scale=MatchLowercase]{Andale Mono}

%% Other alternatives:
%\setromanfont[Mapping=tex-text]{Adobe Caslon}
%\setsansfont[Scale=MatchLowercase]{Gill Sans}
%\setsansfont[Scale=MatchLowercase,Mapping=tex-text]{Futura}
%\setmonofont[Scale=MatchLowercase]{Andale Mono}
%\newfontfamily{\SYM}[Scale=0.9]{Zapf Dingbats}
%% END FONTS
%%%%%%%%%%%%%%%%%%%%%%%%%%%%%%%%%%%%%%%%%%%%%%%%%%%%%%%%%%%%%%%%%%%%%%
%%%%%%%%%%%%%%%%%%%%%%%%%%%%%%%%%%%%%%%%%%%%%%%%%%%%%%%%%%%%%%%%%%%%%%



%%%%%%%%%%%%%%%%%%%%%%%%%%%%%%%%%%%%%%%%%%%%%%%%%%%%%%%%%%%%%%%%%%%%%%
%%%%%%%%%%%%%%%%%%%%%%%%%%%%%%%%%%%%%%%%%%%%%%%%%%%%%%%%%%%%%%%%%%%%%%
%%
%% Recommended packages
%%
\usepackage{checkend}	% better error messages on left-open environments
\usepackage{graphicx}	% for incorporating external images

%% booktabs: provides some special commands for typesetting tables as used
%% in excellent journals.  Ignore the examples in the Lamport book!
\usepackage{booktabs}

%% listings: useful support for including source code listings, with
%% optional special keyword formatting.  The \lstset{} causes
%% the text to be typeset in a smaller sans serif font, with
%% proportional spacing.
\usepackage{listings}
\lstset{basicstyle=\sffamily\scriptsize,showstringspaces=false,fontadjust}

%% The acronym package provides support for defining acronyms, providing
%% their expansion when first used, and building glossaries.  See the
%% example in glossary.tex and the example usage throughout the example
%% document.
%% NOTE: to use \MakeTextLowercase in the \acsfont command below,
%%   we *must* use the `nohyperlinks' option -- it causes errors with
%%   hyperref otherwise.  See Section 5.2 in the ``LaTeX 2e for Class
%%   and Package Writers Guide'' (clsguide.pdf) for details.
%\usepackage[printonlyused,nohyperlinks]{acronym}
\usepackage[printonlyused]{acronym}
%%\usepackage{glossaries}
%% The ubcdiss.cls loads the `textcase' package which provides commands
%% for upper-casing and lower-casing text.  The following causes
%% the acronym package to typeset acronyms in small-caps
%% as recommended by Bringhurst.
%\renewcommand{\acsfont}[1]{{\scshape \MakeTextLowercase{#1}}}

%% color: add support for expressing colour models.  Grey can be used
%% to great effect to emphasize other parts of a graphic or text.
%% For an excellent set of examples, see Tufte's "Visual Display of
%% Quantitative Information" or "Envisioning Information".
\usepackage{color}
\definecolor{greytext}{gray}{0.5}
\definecolor{offyellow}{cmyk}{0, 0, 1, .2}

%% comment: provides a new {comment} environment: all text inside the
%% environment is ignored.
%%   \begin{comment} ignored text ... \end{comment}
\usepackage{comment}

%% The natbib package provides more sophisticated citing commands
%% such as \citeauthor{} to provide the author names of a work,
%% \citet{} to produce an author-and-reference citation,
%% \citep{} to produce a parenthetical citation.
%% We use \citeeg{} to provide examples
\usepackage[numbers,sort&compress]{natbib}
\newcommand{\citeeg}[1]{\citep[e.g.,][]{#1}}
%\usepackage[nottoc,notlof,notlot]{tocbibind} 
\renewcommand\bibname{References}
%% The titlesec package provides commands to vary how chapter and
%% section titles are typeset.  The following uses more compact
%% spacings above and below the title.  The titleformat that follow
%% ensure chapter/section titles are set in singlespace.
\usepackage[compact]{titlesec}
\titleformat*{\section}{\singlespacing\raggedright\bfseries\Large}
\titleformat*{\subsection}{\singlespacing\raggedright\bfseries\large}
\titleformat*{\subsubsection}{\singlespacing\raggedright\bfseries}
\titleformat*{\paragraph}{\singlespacing\raggedright\itshape}
\titleformat{\chapter}{\normalfont\huge}{\thechapter.}{20pt}{\huge\bfseries}

%% The caption package provides support for varying how table and
%% figure captions are typeset.
\usepackage[format=hang,indention=-1cm,labelfont={bf},margin=1em]{caption}

%% url: for typesetting URLs and smart(er) hyphenation.
%% \url{http://...} 
\usepackage{url}
\urlstyle{sf}	% typeset urls in sans-serif


%%%%%%%%%%%%%%%%%%%%%%%%%%%%%%%%%%%%%%%%%%%%%%%%%%%%%%%%%%%%%%%%%%%%%%
%%%%%%%%%%%%%%%%%%%%%%%%%%%%%%%%%%%%%%%%%%%%%%%%%%%%%%%%%%%%%%%%%%%%%%
%%
%% Possibly useful packages: you may need to explicitly install
%% these from CTAN if they aren't part of your distribution;
%% teTeX seems to ship with a smaller base than MikTeX and MacTeX.
%%
%\usepackage{pdfpages}	% insert pages from other PDF files
%\usepackage{longtable}	% provide tables spanning multiple pages
%\usepackage{chngpage}	% support changing the page widths on demand
%\usepackage{tabularx}	% an enhanced tabular environment

%% enumitem: support pausing and resuming enumerate environments.
%\usepackage{enumitem}

%% rotating: provides two environments, sidewaystable and sidewaysfigure,
%% for typesetting tables and figures in landscape mode.  
%\usepackage{rotating}

%% subfig: provides for including subfigures within a figure,
%% and includes being able to separately reference the subfigures.
%%may be deprecated (nchap May182014)
\usepackage{subfig}
\usepackage{multicol}
%\usepackage{caption}
%\usepackage{subcaption}
\usepackage{array}
%% ragged2e: provides several new new commands \Centering, \RaggedLeft,
%% \RaggedRight and \justifying and new environments Center, FlushLeft,
%% FlushRight and justify, which set ragged text and are easily
%% configurable to allow hyphenation.
%\usepackage{ragged2e}

%% The ulem package provides a \sout{} for striking out text and
%% \xout for crossing out text.  The normalem and normalbf are
%% necessary as the package messes with the emphasis and bold fonts
%% otherwise.
%\usepackage[normalem,normalbf]{ulem}    % for \sout

%%%%%%%%%%%%%%%%%%%%%%%%%%%%%%%%%%%%%%%%%%%%%%%%%%%%%%%%%%%%%%%%%%%%%%
%% HYPERREF:
%% The hyperref package provides for embedding hyperlinks into your
%% document.  By default the table of contents, references, citations,
%% and footnotes are hyperlinked.
%%
%% Hyperref provides a very handy command for doing cross-references:
%% \autoref{}.  This is similar to \ref{} and \pageref{} except that
%% it automagically puts in the *type* of reference.  For example,
%% referencing a figure's label will put the text `Figure 3.4'.
%% And the text will be hyperlinked to the appropriate place in the
%% document.
%%
%% Generally hyperref should appear after most other packages

%% The following puts hyperlinks in very faint grey boxes.
%% The `pagebackref' causes the references in the bibliography to have
%% back-references to the citing page; `backref' puts the citing section
%% number.  See further below for other examples of using hyperref.
%% 2009/12/09: now use `linktocpage' (Jacek Kisynski): FoGS now prefers
%%   that the ToC, LoF, LoT place the hyperlink on the page number,
%%   rather than the entry text.
\usepackage[bookmarks,bookmarksnumbered,%
    citebordercolor={0.8 0.8 0.8},filebordercolor={0.8 0.8 0.8},%
    linkbordercolor={0.8 0.8 0.8},pagebordercolor={0.8 0.8 0.8},%
    urlbordercolor={0.8 0.8 0.8},%
    linktocpage%pagebackref,
    ]{hyperref}
%% The following change how the the back-references text is typeset in a
%% bibliography when `backref' or `pagebackref' are used
%\renewcommand\backrefpagesname{\(\rightarrow\) pages}
%\renewcommand\backref{\textcolor{greytext} \backrefpagesname\ }

%% The following uses most defaults, which causes hyperlinks to be
%% surrounded by colourful boxes; the colours are only visible in
%% PDFs and don't show up when printed:
%\usepackage[bookmarks,bookmarksnumbered]{hyperref}

%% The following disables the colourful boxes around hyperlinks.
%\usepackage[bookmarks,bookmarksnumbered,pdfborder={0 0 0}]{hyperref}

%% The following disables all hyperlinking, but still enabled use of
%% \autoref{}
%\usepackage[draft]{hyperref}

%% The following commands causes chapter and section references to
%% uppercase the part name.
\renewcommand{\chapterautorefname}{Chapter}
\renewcommand{\sectionautorefname}{Section}
\renewcommand{\subsectionautorefname}{Section}
\renewcommand{\subsubsectionautorefname}{Section}

%% If you have long page numbers (e.g., roman numbers in the 
%% preliminary pages for page 28 = xxviii), you might need to
%% uncomment the following and tweak the \@pnumwidth length
%% (default: 1.55em).  See the tocloft documentation at
%% http://www.ctan.org/tex-archive/macros/latex/contrib/tocloft/
% \makeatletter
% \renewcommand{\@pnumwidth}{3em}
% \makeatother

%%%%%%%%%%%%%%%%%%%%%%%%%%%%%%%%%%%%%%%%%%%%%%%%%%%%%%%%%%%%%%%%%%%%%%
%%%%%%%%%%%%%%%%%%%%%%%%%%%%%%%%%%%%%%%%%%%%%%%%%%%%%%%%%%%%%%%%%%%%%%
%%
%% Some special settings that controls how text is typeset
%%
% \raggedbottom		% pages don't have to line up nicely on the last line
% \sloppy		% be a bit more relaxed in inter-word spacing
% \clubpenalty=10000	% try harder to avoid orphans
% \widowpenalty=10000	% try harder to avoid widows
% \tolerance=1000

%% And include some of our own useful macros
\input{macros}

%%%%%%%%%%%%%%%%%%%%%%%%%%%%%%%%%%%%%%%%%%%%%%%%%%%%%%%%%%%%%%%%%%%%%%
%%%%%%%%%%%%%%%%%%%%%%%%%%%%%%%%%%%%%%%%%%%%%%%%%%%%%%%%%%%%%%%%%%%%%%
%%
%% Document meta-data: be sure to also change the \hypersetup information
%%

\title{Investigating the Effects of Upper Lapse Rate and Surface Heat Flux on an idealized Convective Atmospheric Boundary Layer Entrainment Layer using Large Eddy Simulation}
%\subtitle{If you want a subtitle}

\author{Niamh Chaparro}
%\previousdegree{B. Basket Weaving, University of Illustrious Arts, 1991}
%\previousdegree{M. Silly Walks, Another University, 1994}

% What is this dissertation for?
\degreetitle{MSc}

\institution{The University Of British Columbia}
\campus{Vancouver}

\faculty{The Faculty of Graduate Studies}
\department{Earth, Ocean and Atmospheric Sciences}
%\submissionmonth{April}
%\submissionyear{2192}

%% hyperref package provides support for embedding meta-data in .PDF
%% files
%\hypersetup{
 % pdftitle={Change this title!  (DRAFT: \today)},
 % pdfauthor={Johnny Canuck},
 % pdfkeywords={Your keywords here}
%}


\usepackage{placeins}
%%%%%%%%%%%%%%%%%%%%%%%%%%%%%%%%%%%%%%%%%%%%%%%%%%%%%%%%%%%%%%%%%%%%%%
%%%%%%%%%%%%%%%%%%%%%%%%%%%%%%%%%%%%%%%%%%%%%%%%%%%%%%%%%%%%%%%%%%%%%%
%% 
%% The document content
%%

%% LaTeX's \includeonly commands causes any uses of \include{} to only
%% include files that are in the list.  This is helpful to produce
%% subsets of your thesis (e.g., for committee members who want to see
%% the dissertation chapter by chapter).  It also saves time by 
%% avoiding reprocessing the entire file.
%\includeonly{intro,conclusions}
%\includeonly{discussion}

\begin{document}

%%%%%%%%%%%%%%%%%%%%%%%%%%%%%%%%%%%%%%%%%%%%%%%%%%
%% From Thesis Components: Tradtional Thesis
%% <http://www.grad.ubc.ca/current-students/dissertation-thesis-preparation/order-components>

% Preliminary Pages (numbered in lower case Roman numerals)
%    1. Title page (mandatory)
\maketitle

%    2. Abstract (mandatory - maximum 350 words)
%%% The following is a directive for TeXShop to indicate the main file
%%!TEX root = diss.tex

\chapter{Abstract}
\setlength{\parindent}{0cm}
The Atmospheric convective boundary layer has been studied for over thirty years in order to understand the dynamics and scaling behaviour of its growth by entrainment.  This enables prediction of its entrainment rate and entrainment zone depth, and so parameterizations thereof for use in global circulation models.\\

Fundamentals, such as the dependence of the entrainment rate and entrainment zone depth on the convective Richardson number, have been established but there is still unresolved discussion about the form of these relationships.  Details regarding the structure of the entrainment zone continue to emerge.  The variety of convective boundary layer height and entrainment zone depth definitions adds further complexity.  The study described in this thesis aims to join this ongoing discussion.\\

A dry, shear-free, idealized convective boundary layer was modeled using a large eddy simulation.  The use of ten ensemble cases enabled calculation of true ensemble averages and potential temperature fluctuations as well as providing smooth average profiles.  A range of convective Richardson numbers was achieved by varying the two principle external parameters: surface heat flux and stable upper lapse rate.\\

The gradient method for determining local convective boundary layer height was found to be unreliable so a multi-linear regression method was used instead.  Distributions of the local heights thus determined were found to narrow with increased upper stability.  Height and entrainment zone depth were then defined based on the ensemble and horizontally averaged potential temperature profile.  The resulting relationships of entrainment rate and entrainment zone depth to Richardson number showed behaviour in general agreement with theory and the results of other studies.  The potential temperature gradient in the upper convective boundary layer and entrainment zone was seen to depend on the upper lapse rate, as was the positive downward moving temperature fluctuations at the \acs{CBL} top.  Overall, once the surface heat flux was accounted for by applying the \acs{CBL} height as a scale, the upper lapse rate emerged as the dominant parameter influencing scaled entrainment zone depth, and potential temperature variance and gradients in the entrainment zone and upper convective boundary layer.   
%This document provides brief instructions for using the \class{ubcdiss}
%class to write a \acs{UBC}-conformant dissertation in \LaTeX.  This
%document is itself written using the \class{ubcdiss} class and is
%intended to serve as an example of writing a dissertation in \LaTeX.
%This document has embedded \acp{URL} and is intended to be viewed
%using a computer-based \ac{PDF} reader.

%Note: Abstracts should generally try to avoid using acronyms.

%Note: at \ac{UBC}, both the \ac{FoGS} Ph.D. defence programme and the
%Library's online submission system restricts abstracts to 350
%words.

% Consider placing version information if you circulate multiple drafts
%\vfill
%\begin{center}
%\begin{sf}
%\fbox{Revision: \today}
%\end{sf}
%\end{center}

%\cleardoublepage

%    3. Preface
%%% The following is a directive for TeXShop to indicate the main file
%%!TEX root = diss.tex

\chapter{Preface}

The study described in this thesis was identified and designed by myself, Niamh Chaparro.
I carried out all of the model runs and analyses.

%\cleardoublepage

%    4. Table of contents (mandatory - list all items in the preliminary pages
%    starting with the abstract, followed by chapter headings and
%    subheadings, bibliographies and appendices)
\tableofcontents
%\cleardoublepage	% required by tocloft package

%    5. List of tables (mandatory if thesis has tables)
%\listoftables
%\cleardoublepage	% required by tocloft package

%    6. List of figures (mandatory if thesis has figures)
%\listoffigures
%\cleardoublepage	% required by tocloft package

%    7. List of illustrations (mandatory if thesis has illustrations)
%    8. Lists of symbols, abbreviations or other (optional)

%    9. Glossary (optional)
%%% The following is a directive for TeXShop to indicate the main file
%%!TEX root = diss.tex

\chapter{Glossary}

%%\makeglossaries


% use \acrodef to define an acronym, but no listing
\acrodef{UI}{user interface}
\acrodef{UBC}{University of British Columbia}

% The acronym environment will typeset only those acronyms that were
% *actually used* in the course of the document
\begin{acronym}[ANOVA]
\acro{ANOVA}[ANOVA]{Analysis of Variance, statistical 
techniques to identify sources of variability between groups}

% \acro{ $\overline{w^{,} \theta^{,}_{s}}$ }{Surface heat flux in $\frac{Watts}{meters^{2}}$}

% \acro{$\gamma$}{Initial and upper potential temperature lapse rate in Kelvin}

\acro{EL}[EL]{Entrainment Layer}
\acro{ML}[ML]{Mixed Layer}
\acro{CBL}[CBL]{Convective Boundary Layer}
\acro{LES}{LES}{Large Eddy Simulation}
\acro{DNS}{DNS}{Direct Numerical Simulation}
\acro{GCM}{GCM}{Global Circulation Model}
\acro{TKE}{TKE}{Turbulence Kinetic Energy}
\acro{Ri}[Ri]{Richardson Number 
\acroextra{, the bulk Richardson Number is 
$\frac{gh}{\overline{\theta}_{ML}} \frac{\Delta \theta}{w^{*2}}$, 
$\Delta \theta = \overline{\theta}(h_{1})-\overline{\theta}(h_{0})$ 
and the gradient Richardson Number is 
$\frac{g}{\overline{\theta}_{ML}} \frac{\gamma h^{2} }{w^{*2}}$ 
}}

\acro{FFT}[FFT]{Fast Fourrier Transform}

%\acro{FoGS}[FoGS]{The Faculty of Graduate Studies}
%\acro{PDF}{Portable Document Format}
%\acro{RCS}[RCS]{Revision control system\acroextra{, a software
%    tool for tracking changes to a set of files}}
%\acro{TLX}[TLX]{Task Load Index\acroextra{, an instrument for gauging
%  the subjective mental workload experienced by a human in performing
%  a task}}
%\acro{UML}{Unified Modelling Language\acroextra{, a visual language
%    for modelling the structure of software artefacts}}
%\acro{URL}{Unique Resource Locator\acroextra{, used to describe a
%    means for obtaining some resource on the world wide web}}
%\acro{W3C}[W3C]{\acroextra{the }World Wide Web Consortium\acroextra{,
%    the standards body for web technologies}}
%\acro{XML}{Extensible Markup Language}

\end{acronym}

% You can also use \newacro{}{} to only define acronyms
% but without explictly creating a glossary
% 
% \newacro{ANOVA}[ANOVA]{Analysis of Variance\acroextra{, a set of
%   statistical techniques to identify sources of variability between groups.}}
% \newacro{API}[API]{application programming interface}
% \newacro{GOMS}[GOMS]{Goals, Operators, Methods, and Selection\acroextra{,
%   a framework for usability analysis.}}
% \newacro{TLX}[TLX]{Task Load Index\acroextra{, an instrument for gauging
%   the subjective mental workload experienced by a human in performing
%   a task.}}
% \newacro{UI}[UI]{user interface}
% \newacro{UML}[UML]{Unified Modelling Language}
% \newacro{W3C}[W3C]{World Wide Web Consortium}
% \newacro{XML}[XML]{Extensible Markup Language}
	% always input, since other macros may rely on it

\textspacing		% begin one-half or double spacing

%   10. Acknowledgements (optional)
%\include{ack}

%   11. Dedication (optional)

% Body of Thesis (not all sections may apply)
\mainmatter

\acresetall	% reset all acronyms used so far

%    1. Introduction
%%% The following is a directive for TeXShop to indicate the main file
%%!TEX root = diss.tex
\chapter{Introduction} 
\label{ch:Introduction}
\setlength{\parindent}{0cm}

\section{Motivation}
\label{sec:Mot}

The daytime convective atmospheric boundary layer (\acs{CBL}) over land starts to grow after sunrise when the surface becomes warmer than the air above it.  Coherent turbulent structures (thermals) begin to form and rise, since their relative warmth causes them to be less dense than their surroundings, and so buoyant.  The temperature profile of the residual boundary layer is stable; i.e., potential temperature ($\theta$, see Section \ref{sec:pottemp}) increases with height.  The thermals rise to their neutral buoyancy level, overshoot, and then overturn or recoil.  Concurrently, warm stable air from the free atmosphere (FA) above is trapped or enveloped and subsequently mixed into the growing turbulent mixed layer (\acs{ML}) (\citeauthor{Stull-BLMetIntro} \citeyear{Stull-BLMetIntro}).  This mixing at the top of the \acs{CBL} is known as entrainment and the region over which it occurs, the entrainment zone (\acs{EZ}, \citeauthor{DearWill80} \citeyear{DearWill80}). A common, simplified conceptual model of this case is the dry shear free \acs{CBL} (\citeauthor{SullMoengStev} \citeyear{SullMoengStev}, \citeauthor{FedConzMir04} \citeyear{FedConzMir04} \citeauthor{BrooksFowler2} \citeyear{BrooksFowler2}). This model serves as an intellectually accessible way to understand the dynamic and complex \acs{CBL} and its \acs{EZ}.\\

\acs{CBL} height (h) and the prediction thereof are important for calculating the concentration of any atmospheric species within the \acs{ML} as well as the sizes of the turbulent structures.  In combination with the level at which clouds condense (lifting condensation level) knowledge of \acs{EZ} depth facilitates predictions pertaining to the formation of cumulus clouds.  For example cloud cover increases as more thermals rise above their lifting condensation level (\citeauthor{WilStu} \citeyear{WilStu}).  Parameterizations for both \acs{CBL} growth and \acs{EZ} depth are required in mesoscale and general circulation models (\acs{GCM}s).  Furthermore it is an attractive goal to develop a robust set of scales for this region analogous to Monin-Obvukov Theory (\citeauthor{Stull-BLMetIntro} \citeyear{Stull-BLMetIntro}, \citeauthor{Traum11} \citeyear{Traum11}, \citeauthor{SteynBaldHoff} \citeyear{SteynBaldHoff}, \citeauthor{StullNelEl} \citeyear{StullNelEl}, \citeauthor{Sorbjan} \citeyear{Sorbjan}).\\

Atmospheric \acs{CBL} entrainment has been studied as a separate phenomenon (\citeauthor{StullNelEl} \citeyear{StullNelEl}, \citeauthor{SullMoengStev} \citeyear{SullMoengStev}, \citeauthor{FedConzMir04} \citeyear{FedConzMir04}, \citeauthor{BrooksFowler2} \citeyear{BrooksFowler2}) as well as within the wider topic of entrainment in geophysical flows (\citeauthor{Turner86} \citeyear{Turner86}). There is broad agreement as to the fundamental scaling parameters and relationships involved.  However, the discussion as to how the parameters are defined and measured (\citeauthor{BrooksFowler2} \citeyear{BrooksFowler2}, \citeauthor{Traum11} \citeyear{Traum11}) and the exact forms of the resulting relationships continues (\citeauthor{SullMoengStev} \citeyear{SullMoengStev}, \citeauthor{FedConzMir04} \citeyear{FedConzMir04} \citeauthor{BrooksFowler2} \citeyear{BrooksFowler2}).  This prompts me to ask the research questions I build up to in Section \ref{sec:relback} and outline in Section \ref{sec:resquest}.

%%%%%%%%%%%%%%%%%%%%%%%%%%%%%%%%%%%%%%%%%%%%%%%%%%%%%%%%%%%%%%%%%%%%%%

\section{Relevant Background}
\label{sec:relback}
\subsection{The Convective Boundary Layer (CBL)}

\acs{CBL} grows in three stages: (i) slowly after sunrise as the nightime boundary layer is burned off, (ii) rapidly in the late morning as the top rises through the residual layer and (ii) slowly, when the previous day's capping inversion is reached.  Convective turbulence and the dominant upward vertical motions then begin to subside as the surface cools.  While the surface is warm, buoyancy driven thermals of somewhat uniform potential temperature ($\theta$) and tracer concentration at their cores form and entrain surrounding air laterally as they rise, as well as trapping and mixing in stable warm from above (\citeauthor{Stull-BLMetIntro} \citeyear{Stull-BLMetIntro}, \citeauthor{CrumStullEl} \citeyear{CrumStullEl}).  Under conditions of strong convection and weak winds, buoyantly driven turbulence dominates and shear-driven turbulence is insignificant (\citeauthor{DirLEddy} \citeyear{DirLEddy}). Thermal overshoot relative to their neutral buoyancy level, and subsequent entrainment of the warmer air from aloft augments the warming caused by the surface turbulent heat flux $(\overline{w^{'}\theta^{'}})_{s}$ (see Section \ref{sec:pottemp}) and results in a $\theta$ jump or inversion at the \acs{CBL} top (\citeauthor{SchmidtSchu} \citeyear{SchmidtSchu}, \citeauthor{Turner86} \citeyear{Turner86}).  There may also be a residual inversion from the day before, possibly strengthened by subsidence (\citeauthor{Stull-BLMetIntro} \citeyear{Stull-BLMetIntro}, \citeauthor{SullMoengStev} \citeyear{SullMoengStev}).\\  

Lidar images such as Figure \ref{fig:DuPeFla} show the overall structure of the \acs{CBL} with rising thermals, impinging on the air above (\citeauthor{CrumStullEl} \citeyear{CrumStullEl}, \citeauthor{Traum11} \citeyear{Traum11}).  

\begin{figure}[htbp]
    \centering
    %plot_height.py[master 1573b9d] h vs time plot
    \includegraphics[scale=1]{/newtera/tera/phil/nchaparr/python/Plotting/Dec252013/pngs/DuPeFla}
    \caption[Lidar backscatter image of the \acs{CBL}]{Lidar backscatter image of the \acs{CBL} from \citeauthor{DuPeFla} \citeyear{DuPeFla}.  The horizontal distance between the two peaks corresponds to approximately 1700 meters.}
    \label{fig:DuPeFla}   % label should change
\end{figure}

This has been effectively modelled using large eddy simulation (\acs{LES}) by \citeauthor{SchmidtSchu} (\citeyear{SchmidtSchu}) who used horizontal slices of turbulent potential temperature and vertical velocity fluctuations ($\theta^{'}$, $w^{'}$) at various vertical levels to show how the thermals form, merge and impinge at the \acs{CBL} top with concurrent peripheral downward motions.  The latter is supported in the \acs{LES} visualizations of \citeauthor{SullMoengStev} (\citeyear{SullMoengStev}).  The vertical cross section within the \acs{EZ} in Figure \ref{fig:SullMoeng} shows the relatively cooler thermals and trapped warmer air as well as the closely associated upward motion of cooler air and downward motion of warmer air.\\ 

\begin{figure}[htbp]
    \centering
    %plot_height.py[master 1573b9d] h vs time plot
    \includegraphics[scale=1.5]{/newtera/tera/phil/nchaparr/python/Plotting/Dec252013/pngs/SullMoeng}
    \caption[Visualization of entrainment from an \acs{LES}]{Flow visualization from \citeauthor{SullMoengStev} \citeyear{SullMoengStev} showing a modelled \acs{CBL} thermal enveloping \acs{FA} air.}
    \label{fig:SullMoeng}   % label should change
\end{figure}

On average these convective turbulent structures create a fully turbulent mixed layer (\acs{ML}) with eddy sizes cascading through an inertial subrange to the molecular scales at which energy is lost via viscous dissipation (\citeauthor{Stull-BLMetIntro} \citeyear{Stull-BLMetIntro}).  Here, as represented in Figure \ref{fig:1storder}, $\overline{\theta}$ is close to uniform and increases with respect to time due to $(\overline{w^{'}\theta^{'}})_{s}$ and the downward flux of entrained stable air at the inversion $(\overline{w^{'}\theta^{'}})_{h}$.  \acs{ML} turbulence is dominated by warm updraughts and cool downdraughts.  With proximity to the top the updraughts become relatively cool and warmer \acs{FA} air from above is drawn downward, so in the \acs{ML} $\overline{w^{'}\theta^{'}}$ is positive and decreasing.  Directly above the \acs{ML} the air is stable with intermittent turbulence and, on average, transitions from a uniform \acs{ML} potential temperature ($\frac{\partial \overline{\theta}}{\partial z} \approx 0$) to a stable lapse rate ($\gamma$).  A peak in the average vertical gradient ($\frac{\partial \overline{\theta}}{\partial z}$) at the inversion represents regions where thermals have exceeded their neutral buoyancy level (see Figure \ref{fig:1storder}).\\

\begin{figure}[htbp]
    \centering
    %plot_height.py[master 1573b9d] h vs time plot
    \includegraphics[scale=.55]{/newtera/tera/phil/nchaparr/python/Plotting/Dec252013/pngs/first_order.pdf}
    \caption[Idealized vertical average profiles for a dry \acs{CBL}]{Idealized vertical average profiles for a dry \acs{CBL} in the absence of large scale winds or subsidence. (a) $\overline{\theta}_{ML}$ is the average mixed layer potential temperature. $h$ is the height of maximum gradient in the $\overline{\theta}$ profile. $\overline{\theta}_{0}$ (dotted line) is the initial $\overline{\theta}$ profile which has a slope $\gamma$. The mixed layer, entrainment layer and free atmosphere are denoted \acs{ML}, \acs{EZ} and \acs{FA} respectively.  (b) $\overline{w^{'}\theta^{'}}$ is the average surface turbulent heat flux.  The EZ boundaries (dashed lines) enclose the region of negative $\overline{w^{'}\theta^{'}}$.}
    \label{fig:1storder}   % label should change
\end{figure}

\citeauthor{StullNelEl} (\citeyear{StullNelEl}) outline the stages of \acs{CBL} growth from when the sub-layers of the nocturnal boundary layer are entrained, until the previous day's capping inversion is reached and a quasi-steady growth is attained.  The \acs{EZ} depth relative to \acs{CBL} height varies throughout these stages and its relationship to scaled entrainment is hysteresial.  Numerical studies typically represent this last quasi-steady phase involving a constant $(\overline{w^{'}\theta^{'}})_{s}$ working against an inversion and or a stable $\gamma$ (\citeauthor{SchmidtSchu} \citeyear{SchmidtSchu}, \citeauthor{Sorbjan} \citeyear{Sorbjan}, \citeauthor{SullMoengStev} \citeyear{SullMoengStev}, \citeauthor{FedConzMir04} \citeyear{FedConzMir04}, \citeauthor{BrooksFowler2} \citeyear{BrooksFowler2}, \citeauthor{GarciaMellado} \citeyear{GarciaMellado}).  

\subsection{CBL Height ($h$)}
\label{subsec:cblh}

The \acs{ML} is fully turbulent with a uniform average potential temperature ($\overline{\theta}$) which increases sharply over the \acs{EZ} . Aerosol and water vapour concentrations decrease dramatically with transition to the stable upper \acs{FA}.  So any of these characteristics can support a definition of \acs{CBL} height ($h$).  \citeauthor{StullNelEl} (\citeyear{StullNelEl}) defined $h$ in terms of the percentage of \acs{ML} air and identified it by eye from Lidar back-scatter images.  \citeauthor{Traum11} (\citeyear{Traum11}) compared four automated methods applied to Lidar images:
 
\begin{itemize}
\item{a suitable threshold value above which the air is categorized as \acs{ML} air,}  
\item{the point of minimum (largest negative) vertical gradient,}
\item{the point of minimum vertical gradient based on a fitted idealized curve,}
 \item{and the maximum wavelet covariance.}  
\end{itemize}

\acs{CBL} height detection is a wide and varied field.  Both \citeauthor{BrooksFowler2} (\citeyear{BrooksFowler2}) and \citeauthor{Traum11} \citeyear{Traum11} provide more thorough reviews.\\

Numerical models produce hundreds of local horizontal points from which smooth averaged vertical profiles are obtained, and statistically robust relationships inferred. \citeauthor{BrooksFowler2} (\citeyear{BrooksFowler2}) applied a wavelet technique to identify the height of maximum covariance in local vertical tracer profiles in their \acs{LES} study.  They compared this method to the gradient method i.e. locating the height of most negative vertical gradient, as well as the height of minimum $\overline{w^{'}\theta^{'}}$ as shown later in Figure \ref{fig:hdefs}.  This last definition is common among \acs{LES} and laboratory studies where it has been referred to as the inversion height (\citeauthor{DearWill80} \citeyear{DearWill80}, \citeauthor{Sorbjan1} \citeyear{Sorbjan1}, \citeauthor{FedConzMir04} \citeyear{FedConzMir04}).  \citeauthor{SullMoengStev} (\citeyear{SullMoengStev}) clarified that the extrema of the four $\overline{w^{'}\theta^{'}}$ quadrants (upward warm: $\overline{w^{'+}\theta^{'+}}$, downward warm: $\overline{w^{'-}\theta^{'+}}$, upward cool: $\overline{w^{'+}\theta^{'-}}$, downward cool: $\overline{w^{'-}\theta^{'-}}$) in the \acs{EZ} more or less correspond to the average point of maximum $\frac{\partial \overline{\theta}}{\partial z}$ (see $h$ in Figure \ref{fig:1storder}), whereas the point of minimum $\overline{w^{'}\theta^{'}}$ was consistently lower. They defined \acs{CBL} height based on local $\frac{\partial \theta}{\partial z}$ and applied horizontal averaging as well as two methods based on $\overline{w^{'}\theta^{'}}$
for comparison.\\

%So far, no published \acs{LES} study defines the height in terms of the $\overline{\theta}$ or $\frac{\partial \overline{\theta}}{\partial z}$ profile even though analytical models, from which \acs{CBL} growth parameterizations stem, rely on an idealized version thereof. \citeauthor{GarciaMellado} \citeyear{GarciaMellado} do include it as one of their measures of \acs{CBL} height in their direct numerical simulation (\acs{DNS}) study.       

\subsection{CBL Growth by Entrainment}
\label{subsec:cblgrowth}

The \acs{CBL} grows by trapping pockets of warm stable air between
or adjacent to impinging thermal plumes.  \citeauthor{Traum11} (\citeyear{Traum11}) summarize two categories of \acs{CBL} entrainment:\\

\begin{itemize}

\item{Non turbulent fluid can be engulfed between or in the overturning of thermal plumes. This kind of event has been supported by the visualizations in \citeauthor{SullMoengStev}'s (\citeyear{SullMoengStev}) \acs{LES} study as well as in \citeauthor{Traum11}'s (\citeyear{Traum11}) observations. In both it appeared to occur under a weak inversion or upper lapse rate ($\gamma$)}

\item{
Impinging thermal plumes distort the inversion interface dragging wisps of warm stable air down at their edges or during recoil under a strong inversion or lapse rate. This type of event is supported by the findings  of both \citeauthor{SullMoengStev} (\citeyear{SullMoengStev}) and \citeauthor{Traum11} (\citeyear{Traum11}).}

\end{itemize}

Shear induced instabilities do occur at the top of the atmospheric boundary layer and in some laboratory studies of turbulent boundary layers, under conditions of very high stability, breaking of internal waves have been observed.  Entrainment via the former is relatively insignificant in strong convection, and the latter has not been directly observed in real or modeled atmospheric \acs{CBL}s over the range of conditions considered here (\citeauthor{Traum11} \citeyear{Traum11}, \citeauthor{SullMoengStev} \citeyear{SullMoengStev}).

\subsection{The CBL Entrainment Layer (EZ)}
\label{subsec:cblel}

The \acs{ML} is fully turbulent but the top is characterized by stable air with intermittent turbulence due to the higher reaching thermals. \citeauthor{GarciaMellado} (\citeyear{GarciaMellado}) demonstrate that the \acs{EZ} is subdivided in terms of length and buoyancy scales.  That is, the lower region is comprised of mostly turbulent air with pockets of stable warmer air that are quickly mixed, and so scales with the convective scales (see section \ref{subsec:scales}). Whereas the upper region is mostly stable apart from the impinging thermals so scaling here is more influenced by the lapse rate ($\gamma$).  In the \acs{EZ} the average vertical heat flux, $\overline{w^{'}\theta^{'}}$, switches sign relative to that in the \acs{ML}.  The fast updraughts are now relatively cool $\overline{w^{'+}\theta^{'-}}$.  In their analysis of the four $\overline{w^{'}\theta^{'}}$ quadrants \citeauthor{SullMoengStev} (\citeyear{SullMoengStev}) concluded that the net dynamic in this region is downward motion of warm air ($\overline{w^{'-}\theta^{'+}}$) from the free atmosphere (\acs{FA}) since the other three quadrants effectively cancel.\\

In terms of tracer concentration, and for example based on a Lidar backscatter profile, there are two ways to conceptually define the \acs{EZ}.  It can be thought of as the range in space (or time) over which local \acs{CBL} height varies (\citeauthor{CrumStullEl} \citeyear{CrumStullEl}) or a local region over which the concentration (or back-scatter intensity) transitions from \acs{ML} to \acs{FA} values (\citeauthor{Traum11} \citeyear{Traum11}).  The latter can be estimated using either curve-fitting or wavelet techniques (\citeauthor{Traum11} \citeyear{Traum11}, \citeauthor{SteynBaldHoff} \citeyear{SteynBaldHoff}, \citeauthor{BrooksFowler2} \citeyear{BrooksFowler2}).\\

\citeauthor{BrooksFowler2} apply a wavelet technique to tracer profiles for the determination of \acs{EZ} boundaries, in their \citeyear{BrooksFowler2} \acs{LES} study.  However, it is more common in numerical modelling and laboratory studies for the \acs{EZ} boundaries to be defined based on the average vertical turbulent heat flux ($\overline{w^{'}\theta^{'}}$) i.e. the points enclosing the negative region as shown in Figure \ref{fig:1storder} (\citeauthor{DearWill80} \citeyear{DearWill80}, \citeauthor{FedConzMir04} \citeyear{FedConzMir04}, \citeauthor{GarciaMellado} \citeyear{GarciaMellado}).  Bulk models based on the representation in Figure \ref{fig:1storder} assume the region of negative $\overline{w^{'}\theta^{'}}$ coincides with the region where $\overline{\theta}$ transitions from the \acs{ML} value to the \acs{FA} value (\citeauthor{Deardorff79} \citeyear{Deardorff79}, \citeauthor{FedConzMir04} \citeyear{FedConzMir04}) but no modelling studies use the vertical $\overline{\theta}$ profile to define the \acs{EZ}.\\

Since $\overline{\theta}$ modeled by an \acs{LES} is not strictly constant with respect to height in the \acs{ML} (\citeauthor{FedConzMir04} \citeyear{FedConzMir04}), a threshold value for $\overline{\theta}$ or its vertical gradient must be chosen to identify the lower \acs{EZ} boundary.  In their \citeyear{BrooksFowler2} \acs{LES} study \citeauthor{BrooksFowler2} encountered inconsistencies when determining the \acs{EZ} boundaries from the average tracer profile.  Although their tracer profile was quite different to a simulated \acs{CBL} $\overline{\theta}$ profile, this could serve as cautionary note.\\             

Our understanding of the the characteristics and dynamics of the atmospheric \acs{CBL} entrainment layer evolves with the increasing body of measurement (\citeauthor{Traum11} \citeyear{Traum11}, \citeauthor{StullNelEl} \citeyear{StullNelEl}), laboratory (\citeauthor{DearWill80} \citeyear{DearWill80}) and numerical studies (\citeauthor{Deardorff72} \citeyear{Deardorff72}, \citeauthor{Sorbjan} \citeyear{Sorbjan}, \citeauthor{SullMoengStev} \citeyear{SullMoengStev}, \citeauthor{FedConzMir04} \citeyear{FedConzMir04}, \citeauthor{BrooksFowler2} \citeyear{BrooksFowler2}, \citeauthor{GarciaMellado} \citeyear{GarciaMellado}). Parameterizations for \acs{CBL} growth and \acs{EZ} depth are derived based on bulk models and are evaluated using \acs{LES} output and measurements (\citeauthor{FedConzMir04} \citeyear{FedConzMir04}, \citeauthor{Boers89} \citeyear{Boers89}).  So the relationship between theory, numerical simulation and measurement is inextricable and any study based on one must refer to at least one of the others.\\  

%%%%%%%%%%%%%%%%%%%%%%%%%%%%%%%%%%%%%%%%%%%%%%%%%%%%%%%%%%%%%%%%%%%%%%
\subsection{Modelling the CBL and EZ}
\label{subsec:}

\subsubsection{Bulk Models}
\label{subsubsec:}
Bulk  models for the Convective Boundary layer (\acs{CBL}) based on average, vertical profiles of \acs{ML} quantities can be subdivided into: (i) zero order as represented in Figure \ref{fig:0order} and (ii) first (and higher) order bulk models as represented in Figure \ref{fig:1storder}. Order refers to the number of prognostic variables, and increased order corresponds to increasing complexity in the shape of the  $\overline{\theta}$ and $\overline{w^{'}\theta^{'}}$ profiles at the top of the \acs{ML}.\\

\begin{figure}[htbp]
    \centering
    %plot_height.py[master 1573b9d] h vs time plot
    \includegraphics[scale=.55]{/newtera/tera/phil/nchaparr/python/Plotting/Dec252013/pngs/zero_order.pdf}
    \caption[Zero order \acs{CBL}]{Simplified version of Figure is \ref{fig:1storder} such that the \acs{EZ} is infinitesimely thin. (a) $h$ is the height of the  inversion and $\delta \theta$ the corresponding temperature jump, that is, the difference between $\overline{\theta}_{ML}$ and $\overline{\theta}_{0}(h)$. This is different, although related, to the jump across the \acs{EZ} in Figure \ref{fig:1storder} $\Delta \theta$. (b) The $\overline{w^{'}\theta^{'}}$ profile is linear and decreasing until it reaches a maximum negative value at $h$ of $-.2(\overline{w^{'}\theta^{'}})_{s}$. Here there is a discontinuity as it jumps to zero.}
    \label{fig:0order}   % label should change
\end{figure}

Zero order bulk models assume an \acs{ML} of uniform potential temperature ($\overline{\theta}_{ML}$) topped by an infinitesimally thin layer across which there is a temperature jump ($\delta \theta$) and above which is a constant lapse rate ($\gamma$).  The assumed average vertical turbulent heat flux, $\overline{w^{'}\theta^{'}}$, decreases linearly from the surface up, reaching a maximum negative value $(\overline{w^{'}\theta^{'}})_{h}$ .  This is a constant proportion of the surface value, usually $-.2(\overline{w^{'}\theta^{'}})_{s}$ (see Section 4 in \citeauthor{Tennekes73} \citeyear{Tennekes73} for a discussion). At the temperature inversion $\overline{w^{'}\theta^{'}}$ jumps to zero across the infinitesimally thin layer.  Equations for the evolution of \acs{CBL} height, $\overline{\theta}_{ML}$ and $\delta \theta$ are derived on this basis (\citeauthor{Tennekes73} \citeyear{Tennekes73}).\\

If the \acs{CBL} height ($h$) is rising, air is being drawn in from the stable free atmosphere (FA) layer above and cooled i.e. it is decreasing in enthalpy.  The rate of decrease in enthalpy with respect to time is $c_{p}\rho \delta \theta \frac{dh}{dt}$ (see Section \ref{sec:pottemp}) per unit of horizontal area where $\frac{dh}{dt}$ is the entrainment rate ($w_{e}$).  Since the lapse rate above the inversion is stable \citeauthor{Tennekes73} (\citeyear{Tennekes73}) equates this enthalpy loss to the average vertical turbulent heat flux at the inversion

\begin{equation}
\delta \theta \frac{dh}{dt} = -(\overline{w^{'}\theta^{'}})_{h}. 
\end{equation}  

The \acs{ML} warming rate is arrived at via the simplified Reynolds averaged conservation of enthalpy, for which the full derivation is shown in Section \ref{sec:rdent}.

\begin{equation}
\label{eq:warming}
\frac{\partial \overline{\theta}_{ML}}{\partial t} = -\frac{\partial}{\partial z}\overline{w^{'}\theta^{'}}.
\end{equation}

Assuming $\overline{w^{'}\theta^{'}}$ has a constant slope this becomes

\begin{equation}
\frac{\partial \overline{\theta}_{ML}}{\partial t} = \frac{(\overline{w^{'}\theta^{'}})_{s}-(\overline{w^{'}\theta^{'}})_{h}}{h}
\end{equation}

and the evolution of the temperature jump ($\delta \theta$) depends on the rate of \acs{CBL} height ($h$) increase, the upper lapse rate $\gamma$ and the \acs{ML} warming rate
  
\begin{equation}
\frac{d\delta \theta}{dt} = \gamma\frac{dh}{dt} - \frac{d\overline{\theta}_{ML}}{dt}.
\end{equation}

An assumption about the vertical heat flux at the inversion ($h$), such as the entrainment ratio, closes this set

\begin{equation}
\frac{(\overline{w^{'}\theta^{'}})_{h}}{(\overline{w^{'}\theta^{'}})_{s}} = -.2 \ .
\end{equation}\\

The relevant quantities in equations 2.2 through 2.5 are idealized, ensemble averages. There is some variation within this class of model.  For example the rate equation for $h$ (entrainment relation) can alternatively be derived based on the turbulent kinetic energy budget (\citeauthor{FedConzMir04} \citeyear{FedConzMir04}) but they are all based on the simplified $\overline{\theta}$ and $\overline{w^{'}\theta^{'}}$ profiles outlined above.\\  

First (and higher) order models assume an \acs{EZ} of finite depth at the top of the ML, defined by two heights:  the top of the ML ($h_{0}$) and the point where \acs{FA} characteristics are resumed ($h_{1}$).  The derivations are more complex and involve assumptions about the \acs{EZ} i.e.: 

\begin{itemize}
\item{$\Delta h = h_{1} - h_{0}$ = Constant (\citeauthor{Betts74} \citeyear{Betts74})}

\item{$\Delta h = h_{1} - h_{0}$ is related to the zero-order jump at $h$ by two right angled triangles with opposite sides
of lengths $h_{1} - h$ and $h - h_{0}$ (\citeauthor{BatchGryn} \citeyear{BatchGryn})}

\item{$\Delta h$ or maximum overshoot distance $d \propto \frac{w^{*}}{N}$ where $w^{*}$ is the convective vertical velocity scale and $N = \sqrt{\frac{g}{\overline{\theta}} \frac{\partial \overline{\theta}}{\partial z}}$ is the Brunt-Vaisala frequency (\citeauthor{Stull73} \citeyear{Stull73})}
 
\item{For $h_{0}<z<h_{1}$ $\overline{\theta} = \overline{\theta}_{ML} + f(z,t) \Delta \theta$ where $f(z,t)$ is a dimensionless shape factor (\citeauthor{Deardorff79} \citeyear{Deardorff79}, \citeauthor{FedConzMir04} \citeyear{FedConzMir04})}
\end{itemize}
 \\

Although development of these models is beyond the scope of this thesis, they are mentioned to give context to the parameterizations considered in Sections \ref{subsec:erri} and \ref{subsec:scales}. \\         

\subsubsection{Numerical Simulations}
\label{subsec:}

Numerical simulation of the \acs{CBL} is carried out by solving the Navier Stokes equations, simplified according to a suitable approximation, on a discrete grid.  Types of simulations can be grouped according to the scales of motion they resolve.  In direct numerical simulations (\acs{DNS}) the full range of spatial and temporal turbulence are resolved from the size of the domain down to the smallest dissipative scales i.e. the Kolmogorov micro-scales (\citeauthor{Kolmog} \citeyear{Kolmog}).  This requires a dense numerical grid and so can be computationally prohibitive.\\

In an \acs{LES} motion on scales smaller than the grid spacing are filtered out and parameterized by a sub-grid scale closure model. General circulation models (\acs{GCM}) solve the Navier Stokes equations on a spherical grid and parameterize smaller scale processes including convection and cloud cover.  \acs{LES} has increasingly been used to better understand the \acs{CBL} since \citeauthor{Deardorff72} (\citeyear{Deardorff72}) applied this relatively new method for this purpose.  \citeauthor{SullMoengStev} (\citeyear{SullMoengStev}), \citeauthor{FedConzMir04} (\citeyear{FedConzMir04}) and \citeauthor{BrooksFowler2} in (\citeyear{BrooksFowler2}) used it to study the structure and scaling behaviour of the \acs{EZ}.\\

%%%%%%%%%%%%%%%%%%%%%%%%%%%%%%%%%%%%%%%%%%%%%%%%%%%%%%%%%%%%%%%%%%%%%%
\subsection{Scales and Scaling Relations of the \acs{CBL} and \acs{EZ}}
\label{subsec:scales}

\subsubsection{Length Scale ($h$)}
\label{subsubsec:}

\citeauthor{Deardorff72} (\citeyear{Deardorff72}) demonstrated that dominant turbulent structures in penetrative convection scale with \acs{CBL} height, which he referred to as the inversion height but measured as the height of minimum average vertical heat flux: $z_{f}$ as shown later in Figure \ref{fig:hdefs} (\citeauthor{DearWill80} \citeyear{DearWill80}).  Since then, the distinction between the two has been clarified (see Section \ref{subsec:cblh}) and here $h$ refers strictly to the height of maximum average potential temperature gradient. There are alternatives. For example turbulence based definitions, such as the velocity variance and the distance over which velocity is correlated with itself, represent the current turbulent dynamics rather than the recent turbulence history as does $h$ (\citeauthor{Traum11} \citeyear{Traum11}).\\

\subsubsection{Convective Velocity Scale ($w^{*}$)}
\label{subsubsec:convel}

Given an average surface vertical heat flux $(\overline{w^{'}\theta^{'}})_{s}$ a surface buoyancy flux can be defined as $\frac{g}{\overline{\theta}}(\overline{w^{'}\theta^{'}})_{s}$ which gives the convective velocity scale when multiplied by the appropriate length scale.  Since the result has units $\frac{m^{3}}{s^{3}}$ a cube root is applied\\

\begin{equation}
w^{*} = \left( \frac{gh}{\overline{\theta}}(\overline{w^{'}\theta^{'}})_{s} \right)^{\frac{1}{3}}.
\end{equation}\\

\citeauthor{Deardorff70} (\citeyear{Deardorff70}) confirmed that this effectively scaled the local vertical turbulent velocity fluctuations ($w^{'}$) in the \acs{CBL}.  \citeauthor{Sorbjan}'s (\citeyear{Sorbjan}) work supports this, even at the \acs{CBL} top.  The \acs{CBL} entrainment rate ($w_{e} = \frac{dh}{dt}$) depends on the magnitude of $w^{'}$ which is driven by $(\overline{w^{'}\theta^{'}})_{s}$. Stability aloft suppresses $\frac{dh}{dt}$ so the influence of $\gamma$ is indirectly accounted for via $h$ in $w^{*}$.\\

\subsubsection{Convective Time Scale ($\tau$)}
\label{subsubsec:}

It follows that the time a thermal, travelling at velocity scaled by $w^{*}$, takes to reach the top of the \acs{CBL} i.e. travel a distance $h$ is scaled by

\begin{equation}
\tau = \frac{h}{\left( \frac{gh}{\overline{\theta}}(\overline{w^{'}\theta^{'}})_{s} \right)^{\frac{1}{3}}}.
\end{equation}

 This is also referred to as the convective overturn time scale.  \citeauthor{SullMoengStev} (\citeyear{SullMoengStev}) 
showed a linear relationship between $h$ and time scaled by $\tau$ . An alternative is the Brunt-Vaisala frequency i.e. the time scale
 associated with the buoyant thermals overshooting and sinking (\citeauthor{FedConzMir04} \citeyear{FedConzMir04}).  The ratio of these two time-scales forms a parameter which characterizes this system (see \citeauthor{Sorbjan}\citeyear{Sorbjan} and \citeauthor{Deardorff79} \citeyear{Deardorff79}). 

\subsubsection{Temperature Scale ($\theta^{*}$)}
\label{subsubsec:}

The \acs{CBL} temperature fluctuations $\theta^{'}$ are influenced by $\overline{w^{'}\theta^{'}}$ from both the surface and the \acs{CBL} top.
\citeauthor {Deardorff70} (\citeyear{Deardorff70}) showed that an effective scale based on the convective velocity scale is

\begin{equation}
\theta^{*} = \frac{(\overline{w^{'}\theta^{'}})_{s}}{w^{*}}.
\end{equation} 

Whereas \citeauthor{Sorbjan} (\citeyear{Sorbjan}) showed that with proximity to the \acs{CBL} top the effects of \acs{FA} stability $\gamma$ become more important.
 
\subsubsection{Buoyancy Richardson Number (\acs{Ri})}
\label{subsubsec:}

The flux Richardson ($R_{f}$) number expresses the balance between mechanical and buoyant production of turbulent kinetic energy (\acs{TKE}) and is obtained from the ratio of these two terms in the \acs{TKE} budget equation (See Appendix, \citeauthor{Stull-BLMetIntro} \citeyear{Stull-BLMetIntro}):

\begin{equation}
R_{f} = \frac{\frac{g}{\overline{\theta}}( \overline{w^{'}\theta^{'}})_{s}}{\overline{u_{i}^{'}u_{j}^{'}}\frac{\partial \overline{U}_{i}}{\partial x_{j}}}.
\end{equation}
 
Assuming horizontal homogeneity and vertically constant subsidence yields
  
\begin{equation}
R_{f} = \frac{\frac{g}{\overline{\theta}} \left( \overline{w^{'}\theta^{'}} \right)}{\overline{u^{'}w^{'}}\frac{\partial \overline{U}}{\partial z} + \overline{v^{'}w^{'}}\frac{\partial \overline{V}}{\partial z}}.
\end{equation}

Applying first order closure to the flux terms, i.e. assuming they are proportional to the vertical gradients, gives the gradient Richardson number ($R_{g}$)

\begin{equation}
R_{g} = \frac{ \frac{g}{\overline{\theta}} \frac{\partial \overline{\theta}}{\partial z}}{\left( \frac{ \partial \overline{U}}{\partial z} \right)^{2} + \left( \frac{\partial \overline{V}}{\partial z} \right)^{2}}, 
\end{equation}

However, in the \acs{EZ} buoyancy acts to suppress buoyant production of \acs{TKE}.  Applying a bulk approximation to the denominator, and expressing it in terms of scales yields a squared ratio of two time scales

\begin{equation}
\label{eq:gradri}
R_{g} = \frac{\frac{g}{\overline{\theta}} \frac{\partial \overline{\theta}}{\partial z}}{\frac{U^{*2}}{L^{2}}} = N^{2}\frac{L^{2}}{U^{*2}},
\end{equation}

where $U^{*}$ and $L^{*}$ are appropriate velocity and length scales.  Applying the bulk approximation to both the numerator and denominator yields the bulk Richardson number:

\begin{equation}
R_{b} = \frac{\frac{g}{\overline{\theta}} \Delta \theta L^{*}}{U^{*2}}.
\end{equation}

A natural choice of length and velocity scales for the \acs{CBL} are $h$ and $w^{*}$ giving the convective or buoyancy Richardson number:

\begin{equation}
Ri = \frac{\frac{g}{\overline{\theta}} \Delta \theta h}{w^{*2}}.
\end{equation}

Where $\Delta \theta$ can be replaced by $\delta \theta$ as in \citeauthor{FedConzMir04} (\citeyear{FedConzMir04}) and \citeauthor{GarciaMellado} (\citeyear{GarciaMellado}).  \acs{Ri} can also be arrived at by considering the principal forcings of the system, or from non-dimensionalizing the entrainment relation  derived analytically (\citeauthor{Tennekes73}  \citeyear{Tennekes73}, \citeauthor{Deardorff72} \citeyear{Deardorff72}). It is central to any study on \acs{CBL} entrainment (\citeauthor{SullMoengStev} \citeyear{SullMoengStev}, \citeauthor{FedConzMir04} \citeyear{FedConzMir04}, \citeauthor{Traum11} \citeyear{Traum11}, \citeauthor{BrooksFowler2} \citeyear{BrooksFowler2}).

%TODO: switch order to match results

\subsubsection{Relationship of Entrainment Layer Depth to Richardson Number}

A relationship of the scaled entrainment layer \acs{EZ} depth to \acs{Ri}
\begin{equation}\label{eq:dhvsri}
\frac{\Delta h}{h} \propto Ri ^{b}
\end{equation}

 is arrived at by considering the deceleration of a thermal
as it overshoots its neutral buoyancy level (\citeauthor{StullNelEl} \citeyear{StullNelEl}).  If the velocity of the thermal is assumed to be
proportional to $w^{*}$ and the decelerating force is due to the buoyancy difference, or $\theta$ jump, then the distance the thermal overshoots
($d$) can be approximated by

\begin{equation}
d \propto \frac{w^{*2}}{\frac{g}{\overline{\theta}_{ML}} \Delta \theta}. 
\end{equation} 

If the \acs{EZ} depth is proportional to the overshoot distance ($d$) then

\begin{equation}
\frac{\Delta h}{h} \propto \frac{w^{*2}}{\frac{g}{\overline{\theta}_{ML}} \Delta \theta h} = Ri^{-1}. 
\end{equation} 

Alternatively, \citeauthor{Boers89} \citeyear{Boers89} integrated the internal ($U$), potential ($P$) and kinetic ($K$) energy over a hydrostatic atmosphere

\begin{equation}
U = \frac{c_{v}}{g}\int^{p_{0}}_{0}Tdp.
\end{equation}

\begin{equation}
P = \frac{R}{c_{v}}U,
\end{equation}

and

\begin{equation}
K = \frac{1}{2} \int^{p_{0}}_{0}\frac{w^{2}}{g}dp.
\end{equation}

$p_{0}$ is the surface pressure, $R$ and $c_{v}$ are the gas constant and heat capacity of dry air at constant volume.
$T$ is temperature.  Initially there is a flat infinitesimally thin inversion interface  which is distorted by an
impinging thermal.  The resulting height difference is assumed sinusoidal and an average $\Delta h$ is obtained by integrating 
over a wavelength.  At this point, no entrainment is assumed to have occurred and all of the initial kinetic energy ($K_{i}$) has been transferred to the change in potential energy ($\Delta P$).

\begin{equation}
K_{i} = P_{f} - P_{i} = \Delta P
\end{equation}

Assuming a dry adiabatic atmosphere and that the vertical velocity in the layer below the inversion can be approximated by the convective velocity scale ($w^{*}$), the following expression is reached

\begin{equation}
\left(\frac{\Delta h}{h}\right)^{2} \propto \frac{T_{0} w^{*2}}{g \Delta \theta h}.
\end{equation}

The reference temperature, $T_{0}$, can be replaced by $\overline{\theta}_{ML}$ to give

\begin{equation}
\frac{\Delta h}{h} \propto Ri^{-\frac{1}{2}}
\end{equation}
\subsubsection{Relationship of Entrainment Rate to Richardson Number}
\label{subsec:erri}
The relationship between scaled entrainment rate and the buoyancy Richardson number (\acs{Ri})

\begin{equation}\label{eq:ervsri}
\frac{w_{e}}{w^{*}} \propto Ri^{a}
\end{equation}

is arrived at according to the zero order bulk model through thermodynamic arguments, or by integration of the conservation of enthalpy or turbulent kinetic energy equations over the growing \acs{CBL}. (\citeauthor{Tennekes73} \citeyear{Tennekes73}, \citeauthor{Deardorff79} \citeyear{Deardorff79}, \citeauthor{FedConzMir04} \citeyear{FedConzMir04}). It has been verified in numerous laboratory and numerical studies (\citeauthor{DearWill80} \citeyear{DearWill80}, \citeauthor{SullMoengStev} \citeyear{SullMoengStev}, \citeauthor{FedConzMir04} \citeyear{FedConzMir04}, \citeauthor{BrooksFowler2} \citeyear{BrooksFowler2}), but there is still some 
unresolved discussion as the the exact value of a.  It seems there are two possible values, $-\frac{3}{2}$ and $-1$, the first of which \citeauthor{Turner86} (\citeyear{Turner86}) suggested occurs at high stability when buoyant recoil of impinging thermals becomes more important than their convective overturning. Assume that an impinging thermal supplies kinetic energy ($K$) per unit time and per unit area for entrainment, in terms of appropriate length and time scales $L^{*}$ and $t^{*}$ as follows 

\begin{equation}
K \propto \frac{\overline{\rho} L^{*3} U^{*2}}{L^{*2} t^{*}},
\end{equation}

and that the corresponding change in potential energy per unit time and area of the rising \acs{CBL} is

\begin{equation}
\Delta P \propto g \Delta \theta h \frac{d h}{ dt}  
\end{equation}

where $\Delta \theta$ can be replaced with $\delta \theta$.  If $L^{*}$ is the penetration depth of the thermals travelling at velocity scaled by $w^{*}$ against a decelerating force
$g \frac{\Delta \theta}{\overline{\theta}}$

\begin{equation}
L^{*} = \frac{w^{*2} \overline{\theta}}{\Delta \theta}.  
\end{equation}

and $t^{*}$ is the response time of the inversion layer to a thermal of length $h$

\begin{equation}
t^{*} = \sqrt{h \frac{\overline{\theta}}{g \Delta \theta}}  
\end{equation}

then assuming all of $K$ is transferred to the change in potential energy ($\Delta P$) and using the convective velocity scaled, yields

\begin{equation}
\frac{\frac{dh}{dt}}{w^{*}} \propto \frac{\overline{\theta} w^{*2}}{g \Delta \theta h} \sqrt{\frac{\overline{\theta} w^{*2}}{g \Delta \theta h}},
\end{equation}

i.e.

\begin{equation}
\frac{w_{e}}{w^{*}} \propto Ri^{-\frac{3}{2}}.
\end{equation}

Adding further complexity to this discussion, \citeauthor{FedConzMir04} (\citeyear{FedConzMir04}) suggest that this power law relationship ($a = -\frac{3}{2}$) can be arrived at through defining the $\theta$ jump across the \acs{EZ} rather than at $h$ (see Figure \ref{fig:1storder}).\\

%%%%%%%%%%%%%%%%%%%%%%%%%%%%%%%%%%%%%%%%%%%%%%%%%%%%%%%%%%%%%%%%%%%%%%

\section{Research Questions}
\label{sec:resquest}

A simplified conceptual model of the dry, shear-free \acs{CBL} in the absence of large scale winds is represented in Figure \ref{fig:1storder}.  The two principal external parameters in this case, are the average vertical turbulent surface heat flux $(\overline{w^{'}\theta^{'}})_{s}$ and the upper lapse rate ($\gamma$) (\citeauthor{FedConzMir04} \citeyear{FedConzMir04},\citeauthor{Sorbjan} \citeyear{Sorbjan}).  They have opposing effects, that is to say $(\overline{w^{'}\theta^{'}})_{s}$ drives upward turbulent velocity ($w^{'+}$) and so \acs{CBL} growth ($w_{e}$) whereas $\gamma$  suppresses it.  Conversely they both cause positive turbulent potential temperature fluctuations ($\theta^{'+}$) and so warming of the \acs{CBL}.  In the \acs{EZ} the thermals from the surface are now relatively cool.  They turn downwards as they interact with the stable \acs{FA} concurrently bringing down warmer air.  \citeauthor{SullMoengStev} (\citeyear{SullMoengStev}) demonstrated these dynamics by partitioning $\overline{w^{'} \theta^{'}}$ into four quadrants.  \citeauthor{Sorbjan} (\citeyear{Sorbjan}) asserted and showed that in this region the turbulent potential temperature fluctuations ($\theta^{'}$) are strongly influenced by $\gamma$ whereas the turbulent vertical velocity fluctuations ($w^{'}$) are almost independent thereof. Inspired by these two studies and to gain some insight into the dynamics of this idealized \acs{CBL} I ask \textbf{Q1: How do the distributions of local \acs{CBL} height, and the joint distributions of $w^{'}$ and $\theta^{'}$ within the \acs{EZ}, vary with $(\overline{w^{'}\theta^{'}})_{s}$ and $\gamma$?}\\

The relationship between scaled \acs{EZ} depth and \acs{Ri} 

\begin{equation} 
\frac{\Delta h}{h} \propto  \acs{Ri}^{b} \tag{\ref{eq:dhvsri}}
\end{equation}

has been explored and justified in field measurement, laboratory and numerical studies.  There is disagreement with respect to its exact form, in part stemming from variation in height and $\theta$ jump definitions, but in general its magnitude relative to h decreases with increasing \acs{Ri}. Although referred to in most relevant studies and relied upon in analytical models, the vertical average potential temperature profile has not been used to define the \acs{EZ} (\citeauthor{DearWill80} \citeyear{DearWill80}, \citeauthor{StullNelEl} \citeyear{StullNelEl}, \citeauthor{FedConzMir04} \citeyear{FedConzMir04}, \citeauthor{Boers89} \citeyear{Boers89}, \citeauthor{BrooksFowler2} \citeyear{BrooksFowler2}). This leads me to ask \textbf{Q2: Can the \acs{EZ} boundaries be defined based on the $\overline{\theta}$ profile and what is the relationship of the resulting depth ($\Delta h$) to \acs{Ri}?}\\


A further simplification to the dry, shear-free, \acs{CBL} model without large scale velocities, is to regard the \acs{EZ} depth as infinitesimely small as in Figure \ref{fig:0order}.  The relationship of the scaled, time rate of change of $h$ (entrainment rate: $w_{e}$) to \acs{Ri} can be derived based on this model (\citeauthor{Tennekes73} \citeyear{Tennekes73}, \citeauthor{Deardorff79} \citeyear{Deardorff79}, \citeauthor{FedConzMir04} \citeyear{FedConzMir04})

\begin{equation}
\frac{w_{e}}{w^{*}} \propto  \acs{Ri}^{a}.\tag{\ref{eq:ervsri}} 
\end{equation}
 
This will be referred to as the entrainment relation.  Although that there is such a relationship is well established, discussion as to the power exponent of \acs{Ri} is unresolved and results from studies justify values of both $-\frac{3}{2}$ and $-1$. See \citeauthor{Traum11} (\citeyear{Traum11}) for a summary and review.  \citeauthor{Turner86} (\citeyear{Turner86}) explains this disparity in terms of entrainment mechanism such that the higher value occurs when thermals recoil rather than overturn in response to a stronger $\theta$ jump (or inversion).  Whereas \citeauthor{SullMoengStev} (\citeyear{SullMoengStev}) notice a deviation from the lower power ($-1$) at lower \acs{Ri} and attribute it to the effect of a the shape of $\overline{\theta}$ within a thicker \acs{EZ}.  Both \citeauthor{FedConzMir04} (\citeyear{FedConzMir04}) and \citeauthor{GarciaMellado} (\citeyear{GarciaMellado}) show how the definition of the $\theta$ jump influences the time rate  of change of \acs{Ri} and so effects $a$. \textbf{Q3: How does defining the $\theta$ jump based on the vertical $\overline{\theta}$ profile across the \acs{EZ} as in Figure \ref{fig:1storder} vs at the inversion ($h$) as in Figure \ref{fig:0order}, affect the entrainment relation and in particular $a$?}\\


\section{Approach to Research Questions}
\label{sec:Approach}
\subsubsection{General Setup}

I modelled the dry shear free \acs{CBL} and \acs{EZ} using \acs{LES}, specifically the cloud resolving model System for Atmospheric Modelling (SAM) to be outlined in Chapter 3.  An ensemble of 10 cases was run to obtain true ensemble averages and turbulent potential temperature variances ($\theta^{'}$), each case had a domain of area 3.2 x 4.8 Km$^{2}$. Grid spacing was influenced by the resolution study of \citeauthor{SullPat} (\citeyear{SullPat}) and the vertical grid within the \acs{EZ} was of higher resolution than that applied in other comparable work.  The runs were initialized with a constant $(\overline{w^{'}\theta^{'}})_{s}$ acting a against a uniform $\gamma$.  So, the  $\theta$ jump arose from the overshoot of the thermals, rather than being initially imposed as in \citeauthor{SullMoengStev} (\citeyear{SullMoengStev}) and \citeauthor{BrooksFowler2} (\citeyear{BrooksFowler2}).\\

\subsubsection{Verifying Output}

Before addressing the questions stated in Section \ref{sec:resquest} I will examine the modeled output to make sure it represents a realistic turbulent \acs{CBL} in Chapter 3 section 2. I will verify that the averaged vertical profiles are as expected and coherent thermals are being produced.  FFT energy density spectra will show if there is adequate scale separation between the structures of greatest energy and the grid spacing, and that realistic, isotropic turbulence is being modelled.  

\subsubsection{Q1 Entrainment Zone Structure}     
The \acs{EZ} can be thought of in terms of the distribution of individual thermal heights, or local heights. \citeauthor{SullMoengStev} (\citeyear{SullMoengStev}) measured local height by locating the vertical point of maximum $\theta$ gradient, and observed the effects of varying \acs{Ri} on the resulting distributions. However this method is problematic when gradients in the upper profile exceed that at the inversion (\citeauthor{BrooksFowler2} \citeyear{BrooksFowler2}).  \citeauthor{SteynBaldHoff} (\citeyear{SteynBaldHoff}) fitted an idealized curve to a Lidar backscatter profile.  This method produces a smooth curve based on the full original profile on which a maximum can easily be located.  I will apply a multi-linear regression method outlined in \citeauthor{Vieth} (\citeyear{Vieth}) to the local $\theta$ profile, representing the \acs{ML}, \acs{EZ} and \acs{FA} each with a separate line segment. From this fit, I will locate the \acs{ML} top ($h^{l}_{0}$).  I'll observe how the resulting distributions are effected by changes in $(\overline{w^{'}\theta^{'}})_{s}$ and $\gamma$ using histograms in Chapter 3 Section 3.\\

\citeauthor{SullMoengStev} (\citeyear{SullMoengStev}) broke the turbulent vertical heat flux $w^{'}\theta^{'}$ into four quadrants and used this combined with local flow visualizations to show how \acs{CBL} thermals impinge and draw down warm air from above. \citeauthor{MahrtPaum} (\citeyear{MahrtPaum}) used 2 dimensional contour plots of local $w^{'}$ and $\theta^{'}$ measurements to analyze their joint distributions.  In his \citeyear{Sorbjan} \acs{LES} study \citeauthor{Sorbjan} showed that in the \acs{EZ}, $\theta^{'}$ is strongly influenced by $\gamma$  whereas $w^{'}$ is independent thereof.  Influenced by these three studies, I will use 2 dimensional histograms at $h$ and so within the \acs{EZ} to look at how the distributions of local $w^{'}$ and $\theta^{'}$ are effected by changes in $\gamma$ and $(\overline{w^{'}\theta^{'}})_{s}$ .  I will magnify the effects of $\gamma$, by applying the convective scales, $\theta^{*}$ and $w^{*}$ and hone in specifically on the entrained air at $h$ in Chapter 3 Section 4.\\    

\subsubsection{Q2 Entrainment Zone Boundaries}
       
Here I define the \acs{CBL} height as the location of maximum vertical $\overline{\theta}$ gradient as in Figure \ref{fig:hdefs}.  The lower and upper \acs{EZ} boundaries are then the points at which $\frac{\partial \overline{\theta}}{\partial z}$ significantly exceeds zero and where it resumes $\gamma$.  The lower boundary requires choice of a threshold value which should be small, positive and less than $\gamma$. Since it is somewhat arbitrary I will compare results based on three different threshold values in Chapter 3 section 5.  \citeauthor{FedConzMir04} (\citeyear{FedConzMir04}) and \citeauthor{BrooksFowler2} (\citeyear{BrooksFowler2}) defined the \acs{EZ} in terms of the vertical $\overline{w^{'}\theta^{'}}$ profiles as in Figure \ref{fig:hdefs} but disagreed on the shape of the relationship of scaled \acs{EZ} depth to \acs{Ri} (equation 2.1).  As well as observing this relationship using the height definitions based on the $\overline{\theta}$ profile, I will apply the definitions based on the $\overline{w^{'}\theta^{'}}$ profile for comparison with \citeauthor{BrooksFowler2} (\citeyear{BrooksFowler2}) and \citeauthor{FedConzMir04} (\citeyear{FedConzMir04}) in Chapter 3 section 4.\\  

\begin{figure}[htbp]
    \centering
    %plot_height.py[master 1573b9d] h vs time plot
    \includegraphics[scale=.5]{/newtera/tera/phil/nchaparr/python/Plotting/Dec252013/pngs/height_defs.pdf}
    \caption[Height definitions]{Height definitions based on the average vertical profiles. $\theta_{0}$ is the initial potential temperature.}
    \label{fig:hdefs}   % label should change
\end{figure}

\begin{table}[htbp]
    \begin{center}
%\centerline{
    \begin{tabular}{ p{2cm} p{4cm}  p{3cm}  p{3cm} p{3cm} }
    %\hline
      \acs{CBL} Height & \acs{ML} $\overline{\theta}$ & $\theta$ Jump & \acs{Ri} \\ \hline 
       $h$ & $\overline{\theta}_{ML} = \frac{1}{h}\int^{h}_{0}\overline{\theta}(z)dz$ & $\Delta \theta=\overline{\theta}(h_{1})-\overline{\theta}(h_{0})$ & \acs{Ri}$_{\Delta}=\frac{\frac{g}{\overline{\theta}_{ML}}\Delta \theta h}{w^{*2}}$  \\ [.3cm] %\hline
        
       & &$\delta \theta = \overline{\theta}_{0}(h)- \overline{\theta}_{ML}$ & \acs{Ri}$_{\delta}=\frac{\frac{g}{\overline{\theta}_{ML}} \delta \theta h}{w^{*2}}$ \\ \hline
      \end{tabular}
%}
\caption[Height definitions]{Definitions based on the vertical $\overline{\theta}$ profile in Figure \ref{fig:hdefs}.  To obtain those based on the $\overline{w^{'}\theta^{'}}$ profile, replace $h_{0}$, $h$ and $h_{0}$ with $z_{f0}$, $z_{f}$ and $z_{f1}$}
\label{table:reldefs}   
\end{center}    
\end{table}

\subsubsection{Q3 Entrainment Rate Parameterization}
As discussed in see Section \ref{subsec:erri} the form of the entrainment relation is thought to vary based on the mechanism that initiates entrainment, which in turn depends on the magnitude of \acs{Ri}.  Furthermore the ways in which the height and $\theta$ jump are defined have an effect. I will vary the definition of the $\theta$ jump as outlined in table \ref{table:reldefs} in order to discern between how this, and variation in initial conditions, influence the entrainment relation and in particular $a$. I will reproduce this analysis using height definitions based on $\overline{w^{'}\theta^{'}}$ for comparison with the results of \citeauthor{FedConzMir04} (\citeyear{FedConzMir04}).

\endinput

Any text after an \endinput is ignored.
You could put scraps here or things in progress.
%\begin{epigraph}
 %   \emph{If I have seen farther it is by standing on the shoulders of
  %  Giants.} ---~Sir Isaac Newton (1855)
%\end{epigraph}

%This document provides a quick set of instructions for using the
%\class{ubcdiss} class to write a dissertation in \LaTeX. 
%Unfortunately this document cannot provide an introduction to using
%\LaTeX.  The classic reference for learning \LaTeX\ is
%\citeauthor{lamport-1994-ladps}'s
%book~\cite{lamport-1994-ladps}.  There are also many freely-available
%tutorials online;
%\webref{http://www.andy-roberts.net/misc/latex/}{Andy Roberts' online
 %   \LaTeX\ tutorials}
%seems to be excellent.
%The source code for this docment, however, is intended to serve as
%an example for creating a \LaTeX\ version of your dissertation.

%We start by discussing organizational issues, such as splitting
%your dissertation into multiple files, in
%\autoref{sec:SuggestedThesisOrganization}.
%We then cover the ease of managing cross-references in \LaTeX\ in
%\autoref{sec:CrossReferences}.
%We cover managing and using bibliographies with \BibTeX\ in
%\autoref{sec:BibTeX}. 
%We briefly describe typesetting attractive tables in
%\autoref{sec:TypesettingTables}.
%We briefly describe including external figures in
%\autoref{sec:Graphics}, and using special characters and symbols
%in \autoref{sec:SpecialSymbols}.
%As it is often useful to track different versions of your dissertation,
%we discuss revision control further in
%\autoref{sec:DissertationRevisionControl}. 
%We conclude with pointers to additional sources of informat%ion in
%\autoref{sec:Conclusions}.

  
%The \acs{UBC} \acf{FoGS} specifies a particular arrangement of the
%components forming a thesis.\footnote{See
 %   \url{http://www.grad.ubc.ca/current-students/dissertation-thesis-preparation/order-components}}
%This template reflects that arrangement.

%In terms of writing your thesis, the recommended best practice for
%organizing large documents in \LaTeX\ is to place each chapter in
%a separate file.  These chapters are then included from the main
%file through the use of \verb+\include{file}+.  A thesis might
%be described as six files such as \file{intro.tex},
%\file{relwork.tex}, \file{model.tex}, \file{eval.tex},
%\file{discuss.tex}, and \file{concl.tex}.

%We also encourage you to use macros for separating how something
%will be typeset (\eg bold, or italics) from the meaning of that
%something. 
%For example, if you look at \file{intro.tex}, you will see repeated
%uses of a macro \verb+\file{}+ to indicate file names.
%The \verb+\file{}+ macro is defined in the file \file{macros.tex}.
%The consistent use of \verb+\file{}+ throughout the text not only
%indicates that the argument to the macro represents a file (providing
%meaning or semantics), but also allows easily changing how
%file names are typeset simply by changing the definition of the
%\verb+\file{}+ macro.
%\file{macros.tex} contains other useful macros for properly typesetting
%things like the proper uses of the latinate \emph{exempli grati\={a}}
%and \emph{id est} (\ie \verb+\eg+ and \verb+\ie+), 
%web references with a footnoted \acs{URL} (\verb+\webref{url}{text}+),
%as well as definitions specific to this documentation
%(\verb+\latexpackage{}+).

 
%\LaTeX\ make managing cross-references easy, and the \latexpackage{hyperref}
%package's\ \verb+\autoref{}+ command\footnote{%
 %   The \latexpackage{hyperref} package is included by default in this
 
%   template.}
%makes it easier still. 

%A thing to be cross-referenced, such as a section, figure, or equation,
%is \emph{labelled} using a unique, user-provided identifier, defined
%using the \verb+\label{}+ command.  
%The thing is referenced elsewhere using the \verb+\autoref{}+ command.
%For example, this section was defined using:
%\begin{lstlisting}
%    \section{Making Cross-References}
 %   \label{sec:CrossReferences}
%\end{lstlisting}
%References to this section are made as follows:
%\begin{lstlisting}
 %   We then cover the ease of managing cross-references in \LaTeX\
  %  in \autoref{sec:CrossReferences}.
%\end{lstlisting}
%\verb+\autoref{}+ takes care of determining the \emph{type} of the 
%thing being referenced, so the example above is rendered as
%\begin{quote}
%    We then cover the ease of managing cross-references in \LaTeX\
%    in \autoref{sec:CrossReferences}.
%\end{quote}

%The label is any simple sequence of characters, numbers, digits,
%and some punctuation marks such as ``:'' and ``--''; there should
%be no spaces.  Try to use a consistent key format: this simplifies
%remembering how to make references.  This document uses a prefix
%to indicate the type of the thing being referenced, such as \texttt{sec}
%for sections, \texttt{fig} for figures, \texttt{tbl} for tables,
%and \texttt{eqn} for equations.

%For details on defining the text used to describe the type
%of \emph{thing}, search \file{diss.tex} and the \latexpackage{hyperref}
%documentation for \texttt{autorefname}.


%    2. Main body
% Generally recommended to put each chapter into a separate file
%\include{relatedwork}
%%% The following is a directive for TeXShop to indicate the main file
%%!TEX root = diss.tex

\chapter{Concepts}
\label{ch:Conceptss}
\setlength{\parindent}{0cm}

\section{Modelling the Convective Boundary Layer}
\label{CBLModels}

\subsection{Analytical Models}
The most common analytical models for the Convective Boundary layer (\acs{CBL}) can be subdivided into: (i) zero order
and (ii) first order bulk models.\\

Zero order bulk models assume a Mixed Layer (\acs{ML}) of constant potential temperature topped
by an infinitessimally thin layer accross with there is a temperature jump.  The assumed vertical heat flux ($w^{,}\theta^{,}$) profiles are
linearly decreasing from the surface up, reaching a maximum negative value proportional to the surface value (usually -.2)
at the temperature inversion and going to zero over an infinitessimally small layer.  Equations for the evolution of ML or inversion height (Entrainment Relation),
and average mixed layer temperature and temperature jump at the inversion are derived from the reynolds average conservation of enthalpy
euqation integrated over the ML, simplified and closed according to the entrainment relation and assumptions such as those mentioned above.  
So relevant quantities are ensemble averaged.\\

\cite{Deardorff79}

First order models assume an Entrainment layer (\acs{EL}) of finite depth at the top of the ML, defined by two heights:
the top of the ML ($h_{0}$) and the point where free atmospheric characteristics are resumed ($h_{1}$).  The derivations are more complex and 
examples of implifying assumptions about the \acs{EL} are: $\Delta h = h_{1} - h_{0} = 0$ (Betts, see \cite{Deardorff79}), $\Delta h$ or maximum 
overshoot distance $d \propto \frac{w^{*}}{N}$ where $w^{*}$ is the relevant vertical velocity scale and $N = \sqrt{\frac{g}{\overline{\theta}} \frac{\partial \overline{\theta}}{\partial z}}$
is the Brunt-Vaisalla frequency (Stull 1972 see dropbox folder) and that between $h_{0}$ and $h_{1}$ $\overline{\theta} = \overline{\theta}_{ML} + f(x,t) \Delta \overline{\theta}$ where
$f(x,t) \Delta \overline{\theta}$ is a dimensionless shape factor \cite{Deardorff79}.\\

Although validation or development of these models is beyond the scope of thisthesis, mention of them is necessary to give context to the scaling 
parameters and relationships to be considered. \\         

\subsection{Numerical Models using the Navier Stokes Equations: Large Eddie Simulation}

Direct Numerical Simulataions (\acs{DNS}) solve the Navier Stokes Equations without use of sub grid scale closure, i.e. all range of spatial
and temportal turbulence must be resolved.  A Large Eddy Simulation (\acs{LES}) filters out the smaller scales leaving them to 
a sub grid scale closure model.  General Circulation Models (\acs{GCM}) can model global processes, so their subgrid scale can span the
an LES domain, and they use a grid based on spherical coordinates.\\

(paper from Douws 571c justifying LES use)

\acs{LES} has steadily, repeatedly been used to better understand the \acs{CBL} since \citeauthor{Deardorff72} applied this relatively 
new method in \cite{Deardorff72} for this purpose.  \citeauthor{SullMoengStev} in \cite{SullMoengStev}, \citeauthor{FedConzMir04} in \cite{FedConzMir04}
and \citeauthor{BrooksFowler2} in cite{BrooksFowler2} used it to observe the structure and scaling behaviour of the \acs{EL}.\\

The technique has been widely used to closely simulate measurement campaigns as well as more idealized conditions whereby scaling relations, parametrizations
and analystical models can be tested. (references? tie in to what i'm using it for)
\\

\subsection{Choice of Grid Sizes}

\citeauthor{SullPat} in \cite{SullPat} investigated \acs{LES} solution convergence for modelling the \acs{CBL} over a range of grid sizes.
They found that the shapes of key variable vertical profiles ($\overline{\theta}$ and $\overline{w^{,}\theta^{,}}$), particularly in the \acs{EL} 
as well as entrainment rate ($w_{e}$) varied and began to converge at higher resolution.  Of particular interest is the variation
in shape of the vertical $\overline{\theta}$ profile within the \acs{EL}.  The height over which $\overline{\theta}$ increases
from the \acs{ML} value to the free atmosphere value is elongates with decreased resolution.\\

Of highest importance is the resolution in the vertical to resolve the steep gradients within the \acs{EL}. 
Although comparable horizontal resolution is needed to maintain reasonable aspect ratio, 3D isotropy is thought to be unecessary.
2D isotropy is sufficient.\\

Choice of grid size must also consider the scaling distance between the eddies containing the most energy
and the smallest resolved eddies.  Domain size must also be considered in order to allow fully turbulent fields to develop independently of the periodic sidewall 
boundary conditions. In particular at later simulation times ($t > 8hrs$) when large structures develop.  
(I'm afraid my domain is too small) These consideratons must be balances with the computational time and 
resource avaiability.\\

\subsection{Choice of Initial Conditions}

Both \citeauthor{SullMoengStev} and \citeauthor{BrooksFowler1} in \cite{SullMoengStev} and \cite{BrooksFowler1} chose to
start with a finite layer of uniform tempertaure and or tracer concentration topped by a temperature jump.  These inital conditions
are an idealization of a realistic potential temperature sounding.  Due to entrainment a temperature jump
also evolves when a \acs{CBL} grows against stable linear stratification ($\gamma$) (\cite{Turner86}, \cite{FedConzMir04} and \cite{GarciaMellado}).
This jump is a direct result of the \acs{CBL} growth and warming rates and $\gamma$.
 
\subsection{An Ensemble}

Using a true ensemble of individual cases varying by a pseudo-random small temerature perturbation from unform distribution,
means that true ensemble averages and so temperature peturbations can be calculated.  The turbulent veloctiy perturbations are 
prognosed by the \acs{LES} and the temperature is diagnosed.  Given the individual cases produce realistic turbulence,
 ie resolved, (statistically independent?) eddies, thermals and plumes, and ensemble provides an advantage in terms
 of point-sample size.       

\subsection{The Equilibrium Entrainment Regime}

The equilibrium or quasi-equilibrium entrainment regime is reached when the time evolution of turbulence kinetic energy (\acs{TKE})
 and its dissipation are negligible and there is negligible energy lost through gravity waves.  This regime is assumed for the derivation of
the zero order model based entrainment relation as in \cite{FedConzMir04}.  \citeauthor{GarciaMellado} in \cite{GarciaMellado} plot the 
terms of the terms of the evolution equation for \acs{TKE}, integrated from the surface to the top of the domain (? $z_{\infty}$) 
and show the the balancing of the buoyancy (driven TKE) term by the dissipation term.  They say this regime is also characterised 
by a rate of \acs{CBL} growth that is much slower than their time scale ($\frac{z_{enc}}{L_{0}}$, $L_{0} = \left(\frac{B_{0}}{N^{3}} \right)^{\frac{1}{2}}$)
and proportional to the square root of time.  % phil wants me to show this  


The energy term profiles become more or less self similar when scaled by $w^{*}$. Finally
\citeauthor{FedConzMir04} in \cite{FedConzMir04} say it is characterized by a(n approximately) constant entrainment ratio ($\frac{B_{i}}{B_{s}}$) and
and a constant stability ratio ($\frac{N^{2}\delta z_{i}}{\delta b} \approx 1.2$).   

\section{Seeing the \acs{CBL} and \acs{EL}}
\label{SeingCBL}
\subsection{plumes, thermals}
To establish that the simulated \acs{CBL} is behaving realitically visual representations of the key variables are observed.
\citeauthor{SchmidtSchu} in \cite{SchmidtSchu} used vertical profiles of the horizontally averaged (equivalent to ensemble averaged
by ergodic assumption) $\theta$, $w^{,}\theta^{,}$ and $w^{,2}$.  For example using their $\overline{\theta}$ profiles they observed
a developing \acs{ML} and the effects of heating by entrainment of warm air from above, in addition to from the surface,  
and the resulting development of an inversion, ie a point at which the gradient is steeper than the inititial lapse rate ($\gamma$).  Vertical profiles of the velocities and heat flux
showed a \acs{CBL} above which turbulence sharply decreases and the $\overline{w^{}\theta^{,}}$ profiles provided a clear
visual for the location and depth of the \acs{EL} where it became negative as entrainment became more important as well
as a definition for \acs{CBL} height at the point where it is minimum.\\

A more direct visual is obtained using horizontal contour plots of the turbulent quantities. \citeauthor{SchmidtSchu} in \cite{SchmidtSchu}
and later \citeauthor{SullMoengStev} in \cite{SullMoengStev} showed the spoke like patterns of the upward moving thermals as they become plumes impinging on
the stable air aloft.\\    
\subsection{energy spectra, ffts, the circular integration to one a scalar spectrum}

To confirm there is sufficient scale separation between the dominant eddies and the grid size
the \acs{FFT} energy of the velocity fields should be plotted against wavenumber.  For example
 the power spectrum of the 3 dimentional domain should obey Kolmagorov's $\frac{-11}{3}$ power law

\begin{equation}
\phi(k) = Ck^{\frac{-11}{3}}
\end{equation}

Analoguosly in 2D is a $\frac{-8}{3}$ power law but it is difficult to represent visually.
If radial isotropy is assumed and the the power function is integrated around a circle 
(or semicircle if only considering the positive wavenumbers), the scalar result should
obey the $\frac{-5}{3}$ power law for 1 dimension.


\begin{equation}
\Phi(r) = \int \phi(r) r d\theta = \int Cr^{\frac{-8}{3}} r d\theta = \pi \frac{-3}{8} C r^{\frac{-5}{3}}, \ r=\sqrt{k_{x}^{2} + k_{y}^{2}}
\end{equation}
 
Whether there is sufficient scale separation between the dominant turbulent structures
and the grid sizes would be obvious from a  plot of $log(\Phi(r))$ vs $log(r)$.

\section{Boundary Layer Height and Entrainment Layer Limits: Definitions}
\label{sec:BLhdeltah}
In \cite{Deardorff72}, \citeauthor{Deardorff72} confirmed using \acs{LES} that the inversion height ($z_{i}$) was the most important
lengh scale for the \acs{CBL}.  In earlier studies, the later of which for example would be \cite{SchmidtSchu}, they use this
term to describe the point of minimum $\overline{w^{,}\theta^{,}}$ which was more recently (\cite{SullMoengStev}) shown to be 
distinct from the vertical location of the maximum $\overline{\theta}$ gradient.   There are differences between the \acs{CBL}
height definitions in analytical and numerical studies as compared with those used for measurements.  For example \citeauthor{FedConzMir04}
in \cite{FedConzMir04} use the point of minimum $\overline{w^{,}\theta^{,}}$ and lable it $z_{i}$, whereas \citeauthor{Traum11} in \cite{Traum11}
use three methods based on LIDAR backscatter profile.  Other definitions include those that are turbulence based such as the height below which
the vertical velocity is significantly correlatedd with itself (\cite{Traum11}), ie the height of the dominant turbulent structures
and those based on statistics of locally determined heights (\cite{BrooksFowler2})\\

Likewise, definitions of the \acs{EL} limits vary, as is summarized well by both \cite{BrooksFowler2} and \cite{Traum11}.  There is the idea that
it is a local vertical distance over which a quantity such as $\theta$ or tracer concentration transitions from an \acs{CBL} value to an upper atmosphere
value.  This might be explained by plumes reaching different heights, with intermittent periods of relative inactivity over time.  At any point in time
there will be a range of local \acs{ML} heights and the \acs{EL} can be viewed as some function of the range over which they vary (\cite{StullNelEl} ).
These two concepts can then be combined as in \cite{BrooksFowler1}.\\     

(Need to say what I'm useing and why, mention of Sullivan and Moeng!)
\section{Boundary Layer height and Entrainment Layer Limits: Measurement Techniques}
\label{sec:BLMeas}

A full and detailed discussion about \acs{CBL} height determination is beyond the scope of this
thesis. (cite traumner and brooks and fowler) LIDAR backscatter profiles seem to dominate recent 
measurement campaigns \cite{Traum11}.  They clearly represent the difference between the \acs{CBL} 
and the upper atmosphere in terms of aerosol concentration.  So techniques such as 
\citeauthor{BrooksFowler2}'s wavelet algorithm and \citeauthor{SteynBaldHoff}'s idealized profile 
fit are important.\\

Numerical models have the advantage of easily producing $\overline{w^{,}\theta^{,}}$ profiles
on which analytical models are based as well as Tracer concentration, and potential temperature profiles.
So it's important to form a logical bridge from the analytical models from which scaling relations
are derived to actual measurements. Obtaining the $\overline{w^{,}\theta^{,}}$ based height(s) (minimum and zero
crossings) from LES output is trivial.  LES produced \acs{CBL} Potential temperature profiles have a 
real-world counterpart (soundings).  \citeauthor{SullMoengStev} in \cite{SullMoengStev} identified the 
point of maximum gradient in the profile and used this as their \acs{CBL} height.  But this method
fails when there is local variation in the profile for example due to waves, and the resulting
gradients exceed that marking the transition from \acs{ML} to the upper atmosphere.\\

For a uniform $\overline{w^{,}\theta^{,}}$ which causes a \acs{CBL} to rise against a constant $\gamma$
the local $\theta$ profiles can be approximated by three regions similar to a first order analytical model:
the \acs{ML} (a contstant $\theta$), \acs{EL} and the upper atmopshere ($\gamma$).  At least two 
of these three regions are approximately linear.  \citeauthor{Vieth} in \cite{Vieth} applied a two line
fit to data according to the minimum residual sum of squares (\acs{RSS}).  This can easily be extended
to three lines where the key points are those where one line ends and another begins ie, the top of the 
\acs{ML}, and the top of the \acs{EL}.  If strictly adhering to the analytical model \acs{EL} should be
approximated by a curve (3rd degree polynomial) but hundreds of local profiles where observed, and none
had a distinct \acs{EL} similar to that of the idealized or averaged profile.\\

\subsection{Justification of $h_{0}$ threshold}

Brooks and Fowler state that for the difinition of an entrainment zone using a mean scalar profile there
 is not clear definition of what a significant gradient is.  Given my mean profiles of the vertical potential
temperature gradient remain close to zero in the mixed layer, but decreases from the surface and then
increases gradually.  A threshold value must be chosen to separate the layer where it increases gradually
from that where the increase is sharper (EL.).  This threshold value must be greater than zero, and less
than $\gamma$ and the same for all runs.  The lowest $\gamma$ is .0025.  I chose a value of .0002, 
but verified that my main result  remained consistant with values of .0001 and .0004.
see scaleddeltahinvri plots 5 and 6, and theta grad profs 6 and 5.


\section{Entrainment Layer Statistics}
\label{sec:EntLayStat}
Since the concept of the \acs{CBL} \acs{EL} is basd on plume heights distributed over time
or space, a statistical perspective is natural.  Relationship between the distrubutions of 
\acs{CBL} heights ($h$) and Richardson Number is expected.  A visualization of
 the 2D distributions of the local $w^{,}\theta^{,}$ quadrants and their behaviour 
with respect to Richardson Number would compliment and further the insight given
by \citeauthor{SullMoengStev} in \cite{SullMoengStev}.\\

\citeauthor{BrooksFowler2} found the statistically based definitions $h$ yielded clearer 
relationships to Richardson number in \cite{BrooksFowler2} and proposed a measure of \acs{EL}
depth based on statistics of the local \acs{EL}.  \citeauthor{SullMoengStev} obsereved 
Richardson Number dependece on the second order statistics of the local $h$.\\

Having a large sample of local data points might be an advantage.  A comparison of the
local $h$  surface could be compared with upper horizontal slices of $\theta^{,}$ and 
$w^{,}$.  Visualization of the distributions of local $h$ using histograms, paying attention
to for example their spread and skew with relation to $\gamma$ and $\overline{w^{'}\theta^{'}}$
would lend support to scaling relationships based on the averaged profiles.\\

\section{Parameters, Scales and Scaling Relations -- Velocity, time, height, surface flux, stability}

\subsection{Convective Velocity Scale: $w^{*}$}
Given an average surface heat flux ($\overline{w^{,}\theta^{,}}_{s}$) a surface buoyancy flux can be defined as 
$\frac{g}{\overline{\theta}}\overline{w^{,}\theta^{,}}$ from which the convective velocity scale is obtained by
multiplying by the appropriate length scale.  Since the result is in $\frac{m^{3}}{s^{3}}$ a cube root is applied.\\

\begin{equation}
w^{*} = \left( \frac{gh}{\overline{\theta}}\overline{w^{,}\theta^{,}} \right)
\end{equation}

\citeauthor{Deardorff70} (\cite{Deardorff70}) confirmed that this effectively scaled the vertical turbulent velocity
perturbations in the \acs{CBL}.\\

It logically follows that the time for a plume to reach the top of the \acs{CBL} is

\begin{equation}
\tau = \frac{h}{\left( \frac{gh}{\overline{\theta}}\overline{w^{,}\theta^{,}} \right)}
\end{equation}

where $h$ is \acs{CBL} height.

\subsection{Richardson Numbers}

The Flux Richardson ($R_{f}$) expresses the balance between turbulent mechanical energy and buoyancy.  It's obtained from the ration of these two terms
 in the Turbulent Kinetic Energy budget equation (\cite{Stull-BLMetIntro}):

\begin{equation}
\frac{\partial \overline{e}}{\partial t} + \overline{U}_{j} \frac{\partial \overline{e}}{\partial x_{j}} = \delta_{i3}  \frac{g}{\overline{\theta}} \left( \overline{u_{i}^{,}\theta^{,}} \right) - \overline{u_{i}^{,}u_{j}^{,}}\frac{\partial \overline{U}_{i}}{\partial x_{j}} - \frac{ \partial \left( u_{j}^{,}e^{,} \right)}{\partial x_{j}} - \frac{1}{\overline{\rho}} \frac{\partial \left( u_{i}^{,} p^{,} \right) }{\partial x_{i}} - \epsilon
\end{equation}

\begin{equation}
R_{f} = \frac{\frac{g}{\overline{\theta}} \left( \overline{w^{,}\theta^{,}} \right)}{\overline{u_{i}^{,}u_{j}^{,}}\frac{\partial \overline{U}_{i}}{\partial x_{j}}}
\end{equation}
 
Assuming horizontal homogenaiety and neglecting subsidence
  
\begin{equation}
R_{f} = \frac{\frac{g}{\overline{\theta}} \left( \overline{w^{,}\theta^{,}} \right)}{\overline{u^{,}w^{,}}\frac{\partial \overline{U}}{\partial z} + \overline{v^{,}w^{,}}\frac{\partial \overline{V}}{\partial z}}
\end{equation}

Applying first order closer to the flux terms (i.e. assuming they are proportional to the vertical gradients)  gives the gradient Richardson Number ($R_{g}$)

\begin{equation}
R_{g} = \frac{ \frac{g}{\overline{\theta}} \frac{\partial \overline{\theta}}{\partial z}}{\left( \frac{ \partial \overline{U}}{\partial z} \right)^{2} + \left( \frac{\partial \overline{V}}{\partial z} \right)^{2}} 
\end{equation}

which is usually used to express the balance between shear and buoyancy driven turbulence, but in the \acs{EL} buoancy acts to supress turbulence.  
Applying a bulk approximation the to the numerator, and expressing it in terms of scales yields a ratio of two square of time scales

\begin{equation}
R_{g} = \frac{\frac{g}{\overline{\theta}} \frac{\partial \overline{\theta}}{\partial z}}{\frac{U^{*2}}{L^{2}}} = N^{2}\frac{L^{2}}{U^{*}}
\end{equation}


and applying the bulk approximation to both the numerator and the denomnitor yields

\begin{equation}
R_{b} = \frac{\frac{g}{\overline{\theta}} \Delta \overline{\theta} L}{U^{*}}
\end{equation}
A natural choice of length and veloctiy scales for the \acs{CBL} are $h$ and $w^{*}$.  \citeauthor{EllTurn} (\cite{EllTurn}) suggested and confirmed a relationship between the entrainment rate and this form of 
Richardson number based on tank experiments.  This parameter can be justified and arrived at by considering the principal forcings of the system, or from non-dimensionalizing the entrainment relation  deriived
analyitically, i.e. (\cite{Deardorff72})

\begin{equation}
w_{e} \propto \frac{\overline{w^{,}\theta^{,}}_{s}}{\Delta \overline{\theta}}
\end{equation}

\begin{equation}
\frac{w_{e}}{w^{*}} \propto  \frac{\overline{w^{,}\theta^{,}}_{s}}{\Delta \overline{\theta} w*{*}} = Ri_{b}^{-1}
\end{equation}
 

In one or other of its forms this parameter has become central to any study on \acs{CBL} entrainment (\cite{SullMoengStev}, \cite{FedConzMir04}, \cite{Traum11}, \cite{BrooksFowler2})

\subsection{Scaling Relationships}

(relationship of height to scaled time, check sull moeng and fed conz)

The relationship between the scaled entrainment rate $w_{e}$ and Richardson Number as summarized by \citeauthor{Traum11} in 
\cite{Traum11} can in general be expressed as follows:

\begin{equation}
\frac{w_{e}}{V} \propto Ri^{-a}
\end{equation}

where a ranges from 1 to 2.  The value of 1.5 predominates among the water-tank experiments, and 
it is equal to 1 in the \acs{LES} studies.  Although, \citeauthor{FedConzMir04} perceive deviation
towards 1.5 at higher stability (or stronger inversion) in both theirs and \citeauthor{SullMoengStev}'s 
results.  This is in line with \citeauthor{Turner86}'s discussion in \cite{Turner86}.\\

\citeauthor{DearWill80} suggested the importance considering the scaled depth of the \acs{EL} in \acs{CBL} 
entrainment analysis or prediction and put forward a relationship between it and the Richardson Number
which has been further examined in subsequent related studies.  

\begin{equation}
\frac{\Delta h}{h} \propto Ri^{-b}
\end{equation}   

As summarized in \cite{Traum11} b ranges from .25 to 1.  (unclear what Brooks and Fowler got for either of these
relationships) Of the three principle \acs{LES} studies (\cite{BrooksFowler2}, \cite{FedConzMir04}, \cite{SullMoengStev})
one (\cite{FedConzMir04}) varied the upper lapse rate ($\gamma$).  However, it seems their grid spacing 
was large compared to those at wich solutions converged in \cite{SullPat}

(connectionn to models, sero, first order jump models, for example sull moengs section on how the 1rst order jump model
better matches the enntrainment rate)

(gather materials for statistics section)

\begin{table}[!ht]
    \begin{center}
    \begin{tabular}{ | l | p{5cm} | p{5cm} |}
    \hline
     Parameter & Definition & Explaination \\ \hline
     $h$&  height of maximum $\frac{\partial \overline{\theta}}{\partial z}$& \acs{CBL} height \\ \hline
     $h_{0}$& height where $\frac{\partial \overline{\theta}}{\partial z}$ first exceeds 0 by a threshold& lower \acs{EL} limit\\ \hline
     $h_{1}$& height where $\frac{\partial \overline{\theta}}{\partial z}$ resumes $\gamma$& upper \acs{EL} limit\\ \hline
     $\Delta h$& $h_{1} - h_{0}$ & \acs{EL} depth\\ \hline
     $\Delta \theta$ & $\overline{\theta}(h_{1})-\overline{\theta}(h_{0})$ & Temperature Jump over the \acs{EL}\\ \hline
     $h^{l}_{0}$& see section \label{sec:BLMeas}  & local \acs{ML} height\\ \hline
     $w^{*}$ & $\left( \frac{gh}{\overline{\theta}}\overline{w^{,}\theta^{,}} \right)$ & Convective Velocty Scale\\ \hline
     $\tau$ & $\frac{h}{w^{*}}$ & Time scale for plume to reach \acs{CBL} top\\ \hline
     $Ri_{g}$ & $\frac{g}{\overline{\theta_{ML}}} \frac{\gamma h^{2} }{w^{*2}}$ & Gradient Richardson Number for \acs{CBL} entrainment \\ \hline
     $Ri_{b}$ & $\frac{g}{\overline{\theta_{ML}}} \frac{\Delta \theta h }{w^{*2}}$ & Bulk Richardson Number for \acs{CBL} entrainment \\ \hline
     % \end

   
\end{tabular}
\caption{}
\label{fig:}   
\end{center}    
\end{table}
%\footnotetext{Incomplete run: EL exceded high resolution vertical grid after 7 hours}

(construct a table of the scales used and scaling relations considered in this thesis)


\endinput

Any text after an \endinput is ignored.
You could put scraps here or things in progress.

%%% The following is a directive for TeXShop to indicate the main file
%%!TEX root = diss.tex

\chapter{Research Approach and Tools}

\label{ch:tools}
\setlength{\parindent}{0cm}
\section{Approach to Research Questions}
\label{sec:Approach}
\subsubsection{General Setup}

I modelled the dry shear free \acs{CBL} and \acs{EZ} using \acs{LES}, specifically the cloud resolving model System for Atmospheric Modelling (SAM) to be outlined in Chapter 2.  An ensemble of 10 cases was run to obtain true ensemble averages and turbulent potential temperature variances ($\theta^{'}$), each case had a domain of area 3.2 x 4.8 Km$^{2}$. Grid spacing was influenced by the resolution study of \citeauthor{SullPat} (\citeyear{SullPat}) and the vertical grid within the \acs{EZ} was of higher resolution than that applied in other comparable work.  The runs were initialized with a constant $(\overline{w^{'}\theta^{'}})_{s}$ acting a against a uniform $\gamma$.  So, the  $\theta$ jump arose from the overshoot of the thermals, rather than being initially imposed as in \citeauthor{SullMoengStev} (\citeyear{SullMoengStev}) and \citeauthor{BrooksFowler2} (\citeyear{BrooksFowler2}).\\

\subsubsection{Model (\acs{LES}) Valuation}

Before addressing the questions stated in Section \ref{sec:resquest} I will examine the modeled output to make sure it represents a realistic turbulent \acs{CBL} in Chapter \ref{ch:outver}. I will verify that the averaged vertical profiles are as expected and coherent thermals are being produced.  FFT energy density spectra will show if there is adequate scale separation between the structures of greatest energy and the grid spacing, and if realistic, isotropic turbulence is being modelled.  

\subsubsection{Entrainment Zone Structure}     
The \acs{EZ} can be thought of in terms of the distribution of individual thermal heights, or local heights. \citeauthor{SullMoengStev} (\citeyear{SullMoengStev}) measured local height by locating the vertical point of maximum $\theta$ gradient, and observed the effects of varying \acs{Ri} on the resulting distributions. However this method is problematic when gradients in the upper profile exceed that at the inversion (\citeauthor{BrooksFowler2} \citeyear{BrooksFowler2}).  \citeauthor{SteynBaldHoff} (\citeyear{SteynBaldHoff}) fitted an idealized curve to a Lidar backscatter profile.  This method produces a smooth curve based on the full original profile on which a maximum can easily be located.  I will apply a multi-linear regression method outlined in \citeauthor{Vieth} (\citeyear{Vieth}) to the local $\theta$ profile, representing the \acs{ML}, \acs{EZ} and \acs{FA} each with a separate line segment. From this fit, I will locate the \acs{ML} top ($h^{l}_{0}$).  I'll observe how the resulting distributions are effected by changes in $(\overline{w^{'}\theta^{'}})_{s}$ and $\gamma$ using histograms in Section \ref{subsec:locmlh}\\

\citeauthor{SullMoengStev} (\citeyear{SullMoengStev}) broke $w^{'}\theta^{'}$ into four quadrants and used this combined with local flow visualizations to show how \acs{CBL} thermals impinge and draw down warm air from above. \citeauthor{MahrtPaum} (\citeyear{MahrtPaum}) used 2 dimensional contour plots of local $w^{'}$ and $\theta^{'}$ measurements to analyze their joint distributions.  In his \citeyear{Sorbjan1} \acs{LES} study \citeauthor{Sorbjan1} concluded that in the \acs{EZ}, $\theta^{'}$ is strongly influenced by $\gamma$  whereas $w^{'}$ is practically independent thereof.  Influenced by these three studies, I will use 2 dimensional histograms at $h$ and so within the \acs{EZ} to look at how the distributions of local $w^{'}$ and $\theta^{'}$ are effected by changes in $\gamma$ and $(\overline{w^{'}\theta^{'}})_{s}$ .  I will magnify the effects of $\gamma$, by applying the convective scales, $\theta^{*}$ and $w^{*}$ and hone in specifically on the entrained air at $h$ in Section \ref{subsec:downwarm}.\\    

\subsubsection{Entrainment Zone Boundaries}
       
Here I define the \acs{CBL} height as the location of maximum vertical $\overline{\theta}$ gradient as in Figure \ref{fig:hdefs}.  The lower and upper \acs{EZ} boundaries are then the points at which $\frac{\partial \overline{\theta}}{\partial z}$ significantly exceeds zero and where it resumes $\gamma$.  The lower boundary requires choice of a threshold value which should be small, positive and less than $\gamma$. Since it is somewhat arbitrary I will compare results based on three different threshold values in Section \ref{sec:deltahri}.  \citeauthor{FedConzMir04} (\citeyear{FedConzMir04}) and \citeauthor{BrooksFowler2} (\citeyear{BrooksFowler2}) defined the \acs{EZ} in terms of the vertical $\overline{w^{'}\theta^{'}}$ profiles as in Figure \ref{fig:hdefs} but disagreed on the shape of the relationship of scaled \acs{EZ} depth to $\acs{Ri}$ (equation 2.1).  As well as observing this relationship using the height definitions based on the $\overline{\theta}$ profile, I will apply the definitions based on the $\overline{w^{'}\theta^{'}}$ profile for comparison with \citeauthor{BrooksFowler2} (\citeyear{BrooksFowler2}) and \citeauthor{FedConzMir04} (\citeyear{FedConzMir04}) in Section \ref{susec:fluxbound}.\\  

\subsubsection{Entrainment Rate Parameterization}
As discussed in see Section \ref{subsec:erri} the form of the entrainment relation is thought to vary based on the mechanism that initiates entrainment, which in turn depends on the magnitude of $\acs{Ri}$.  Furthermore the ways in which the height and $\theta$ jump are defined have an effect. I will vary the definition of the $\theta$ jump as outlined in Table \ref{tab:reldefs} in order to discern between how this, and variation in initial conditions, influence the entrainment relation and in particular $a$. I will reproduce this analysis using height definitions based on $\overline{w^{'}\theta^{'}}$ for comparison with the results of \citeauthor{FedConzMir04} (\citeyear{FedConzMir04}).

\begin{figure}[htbp]
    \centering
    %plot_height.py[master 1573b9d] h vs time plot
    \includegraphics[scale=.5]{/newtera/tera/phil/nchaparr/python/Plotting/Dec252013/pngs/height_defs.pdf}
    \caption[Height Definitions]{Height definitions based on the average vertical profiles. $\theta_{0}$ is the initial potential temperature.}
    \label{fig:hdefs}   % label should change
\end{figure}

\begin{table}[htbp]
\caption[Height definitions]{Definitions based on the vertical $\overline{\theta}$ profile in Figure \ref{fig:hdefs}.  To obtain those based on the $\overline{w^{'}\theta^{'}}$ profile, replace $h_{0}$, $h$ and $h_{0}$ with $z_{f0}$, $z_{f}$ and $z_{f1}$}
    \begin{center}
%\centerline{
    \begin{tabular}{ p{1.2cm} p{3.3cm}  p{3.2cm}  p{3cm} p{2.5cm} }
    %\hline
      \acs{CBL} Height & \acs{ML} $\overline{\theta}$ & $\theta$ Jump &$\acs{Ri}$\\ \hline 
       $h$ & $\overline{\theta}_{ML} = \frac{1}{h}\int^{h}_{0}\overline{\theta}(z)dz$ & $\Delta \theta=\overline{\theta}(h_{1})-\overline{\theta}(h_{0})$ & \acs{Ri}$_{\Delta}=\frac{\frac{g}{\overline{\theta}_{ML}}\Delta \theta h}{w^{*2}}$  \\ [.3cm] %\hline
        
       & &$\delta \theta = \overline{\theta}_{0}(h)- \overline{\theta}_{ML}$ & \acs{Ri}$_{\delta}=\frac{\frac{g}{\overline{\theta}_{ML}} \delta \theta h}{w^{*2}}$ \\ \hline
      \end{tabular}
%}
\label{tab:reldefs}   
\end{center}    
\end{table}

\section{Large Eddy Simulation (\acs{LES})}
\label{sec:LargeEddieSimulation}

System for Atmospheric Modelling (SAM) is a Large Eddy Simulation with cloud resolving capability (\citeauthor{KhairRand} \citeyear{KhairRand}). The dynamical framework uses the anelastic equations of motion, which in tensor notation are:

\begin{equation}
\frac{\partial u_{i}}{\partial t} = -\frac{1}{\overline{\rho}}\frac{\partial}{\partial x_{i}}(\overline{\rho}u_{i}u_{j} + \tau_{ij}) - \frac{\partial}{\partial x_{i}}\frac{p^{'}}{\overline{\rho}} + \delta_{i3}B + \epsilon_{ij3}f(u_{j} - U_{gj}) + \left( \frac{\partial u_{i}}{\partial t} \right)_{l.s.}
\end{equation}

and

\begin{equation}
\frac{\partial}{\partial x_{i}}\overline{\rho}u_{i}=0
\end{equation}


The over-bar denotes the horizontal average and prime denotes fluctuations from the average. B is buoyancy $=-g\frac{\rho^{'}}{\rho}$,  $U_{g}$ is the prescribed geostrophic wind and $f$ is the Coriolis parameter.  $\tau_{ij}$ is the sub-grid scale stress tensor and the subscript $l.s.$ denotes the prescribed large scale tendency.\\      

The prognostic thermodynamical variable is the liquid water/ice moist static energy ($h_{L}$). 

\begin{equation}
\frac{\partial h_{L}}{\partial t} = -\frac{1}{\overline{\rho}}\frac{\partial}{\partial x_{i}}(\overline{\rho} u_{i}h_{L} + F_{h_{L}i}) - \frac{1}{\overline{\rho}}\frac{\partial}{\partial z}(L_{c}P_{r} + L_{s}P_{s} + L_{s}P_{g}) + \left( \frac{\partial h_{L}}{\partial t} \right)_{rad} + \left( \frac{\partial h_{L}}{\partial t} \right)_{mic}
\end{equation}

$L_{c}$ and $L_{s}$ are the latent heats of condensation and sublimation.  $P_{r}$, $P_{s}$ and $P_{g}$ are precipitation fluxes of rain, snow and graupel.  These terms reduce to zero in the absence of condensed water and precipitation.  The subscripts $rad$ and $mic$ denote tendencies due to radiation and microphysics.  The liquid/ice water static energy is

\begin{equation}
h_{L} = c_{p}T + gz - L_{c}(q_{c} + q_{r}) - L_{s}(q_{i} + q_{s} + q_{g}) 
\end{equation}

where $q_{c}$, $q_{r}$, $q_{i}$, $q_{s}$ and $q_{g}$ are the mixing ratios for cloud water, rain, ice, snow and graupel.  Again, these reduce to zero in the absence of condensed water.  Temperature and potential temperature are diagnosed based on this variable, at each time-step.  A simple first-order Smagorinski closure scheme is used to parameterize the sub-grid stresses and scalar fluxes. The eddy diffusivity coefficient is based on the grid scale.\\

The model equations are represented discretely on a fully staggered Arakawa C-type grid which is uniform in the horizontal and stretched in the vertical. Integration is performed using a third-order Adams-Bashforth scheme with variable time step.  Momentum is advected in flux form with second order differencing and conservation of kinetic energy. 
Prognosed scalars are advected using a three dimensional positive definite, monotonic scheme.  Lateral boundaries are periodic.  The top is bounded by a rigid lid, and
Newtonian damping is applied in the top third of the domain to reduce the effects of gravity waves.  Surface fluxes are computed using Monin-Obvukhov similarity.\\

\section{Handling of Output}

The model was run on parallel computers in a Linux environment using Message Passing Interface (MPI). 3d variable fields were output every 10 - 15 minutes in binary form and converted to Network Common Data Form (NetCDF).  The use of Python was enabled using the netcdf4 interface.  Plotting was done using matplotlib. Most analyses were performed using NumPy and SciPy.  The tri-linear regression method, described in Appendix \ref{sec:trilin}, for determining local \acs{ML} height was implimented using Cython.

\endinput

Any text after an \endinput is ignored.
You could put scraps here or things in progress.

%%% The following is a directive for TeXShop to indicate the main file
%%!TEX root = diss.tex

\chapter{Results}
\label{ch:results}
\setlength{\parindent}{0cm}

\section{Description of Runs}
\FloatBarrier

All 10 member cases of the ensemble were carried out on a 3.2 x 4.8 Km horizontal 
domain ($\Delta x = \Delta y = 25m$, $nx=128$, $ny=192$).  
$nx$, $ny$ were chosen based on the optimal distribution accross processor nodes.  
The vertical grid ($nz=312$) was of higher resolution around the 
entrainment layer (\acs{EL}) ($\Delta z = 5m$), and lower below and above it 
($\Delta z = 10 \ to \ 100 m$). Grid size was chosen so that
a full spectrum of turbulence would be resolved within the \acs{EL} 
in line with the findings of \citeauthor{SullPat} in \cite{SullPat}.  The 7 runs
vary depending on surface heat flux ($\overline{w^{'}\theta^{'}_{s}}$) 
and initial lapse rate ($\gamma$).

%description of runs ie 10 member ensembles each had delta x, delta y=25 and a region of delta z=5m enclosing the 
%Entrainment zone, z=25 below  and streched to 100 above. 

\label{sec:Runs}

\begin{table}[!ht]
    \begin{center}
    \begin{tabular}{ | l | l | l | l |}
    \hline
    $\overline{w^{'}\theta^{'}_{s}}$ / $\gamma$ & 10 (K/Km) & 5 (K/Km) & 2.5 (K/Km) \\ \hline
     150 (W/m2)& \hspace{5mm} \ding{51} &\hspace{5mm} \ding{51}\footnotemark &  \\ \hline
     100 (W/m2)& \hspace{5mm} \ding{51} & \hspace{5mm} \ding{51} & \\ \hline
     60 (W/m2) & \hspace{5mm} \ding{51} & \hspace{5mm} \ding{51} & \hspace{5mm} \ding{51}\\ \hline
     
%\end

   
\end{tabular}
\caption{Runs in terms of $\overline{w^{'} \theta^{'}_{s}}$ and initial lapse rate $\gamma$}
\label{fig:tableofruns}   
\end{center}    
\end{table}
\footnotetext{Incomplete run: EL exceded high resolution vertical grid after 7 hours}

\clearpage

\section{Relevant Definitions}
\FloatBarrier
In large eddy simulation (\acs{LES}) studies, the \acs{CBL} height is usually defined as either the point of minimum 
$\overline{w^{'} \theta^{'}}$ or maximum $\frac{\partial \overline{\theta}}{\partial z}$.
A notable exception is the work of \citeauthor{BrooksFowler2} in \cite{BrooksFowler2} 
where the authors favoured a statistically based definition using local tracer profiles.  Similarly, they define the 
entrainment layer (\acs{EL}) in terms of the statistics of local profiles, 
whereas elsewhere in the literature it is usually defined according to the zero crossings in 
the vertical $\overline{w^{'} \theta^{'}}$ profile.\\

Here, the \acs{CBL} height and \acs{EL} limits are defined based on the vertical  $\frac{\partial \overline{\theta}}{\partial z}$
profile.  Namely, the \acs{CBL} height $h$ is the point where  $\frac{\partial \overline{\theta}}{\partial z}$ is maximum, the 
lower \acs{EL} limit is the point at which  $\frac{\partial \overline{\theta}}{\partial z}$ first increases
significantly from zero i.e. exceeds a threshold value above the surface layer, and the upper \acs{EL} limit $h_{1}$ is the point where
$\frac{\partial \overline{\theta}}{\partial z}$ resumes $\gamma$. (Figure \ref{fig:hdefs})\\

As \citeauthor{BrooksFowler2} point out in \cite{BrooksFowler2}, when using an average vertical 
tracer profile there is no universal critereon for a significant gradient.  So a threshold value for the
lower \acs{EL} limit ($h_{0}$) was chosen such that it was positive, small i.e. an order of magnitude 
less than $\gamma$ and the same for all runs.  For the sake of rigor, the main corresponding result 
was calculated based on two additional threshold values in Section \ref{subsec:deltahri}.\\

The temperature jump is defined here as the difference in $\overline{\theta}$ accross the \acs{EL}.
So, it is larger than those used by  \citeauthor{FedConzMir04} in \cite{FedConzMir04} to verify 
their zero order model and \citeauthor{SullMoengStev} in \cite{SullMoengStev} (Table \ref{table:reldefs}).\\ 
 
%table outlining my definitions, and other comparable ones, ie brooks and fowlers aren't comparable
% re justifying my arbitrary choice of threshold for the lower limit:  seee page 250 Brooks and Fowler. 
%there must be other places in my key papers where the 'arbitraryness' is referred to.
\begin{figure}[htbp]
    \centering
    %plot_height.py[master 1573b9d] h vs time plot
    \includegraphics[scale=.5]{/tera/phil/nchaparr/python/Plotting/Dec252013/pngs/height_defs}
    \caption{Height Definitions}
    \label{fig:hdefs}   % label should change
\end{figure}

\begin{table}[htbp]
    %\begin{center}
\centerline{
    \begin{tabular}{| p{3cm}| p{3cm} | p{3cm} | p{3cm}| }
    \hline
      Description & This Study & Sullivan et al. \cite{SullMoengStev} & Fedorovich et al.\cite{FedConzMir04}  \\ \hline %&  Garcia and Mellado \cite{GarciaMellado}
     CBL Height&$\overline{h_{l}}$&$h$ & $z_{f}$ \\ \hline   %& $z_{enc} \approx z_{f0}$
     Temperature Jump&$\Delta \theta = \overline{\theta}(h_{1})-\overline{\theta}(h_{0})$&$\Delta \theta = \overline{\theta}(z_{f1})-\overline{\theta}(z_{f})$ & $\Delta b = b_{0}(z_{f}) -b(z_{f0})$ \\ \hline %& $\Delta b = b_{0}(z_{f}) - b(z_{f}) $
     &&&$\delta b = b(z_{f1})$ - $b(z_{f0})$ \\ \hline %& $\delta b = b(z_{f1})$ - $b(z_{f0})$
     Convective Velocity Scale&$w_{*}= (h B_{s})^{\frac{1}{3}}$, $B_{s} = \frac{g}{\overline{\theta_{ML}}}\overline{w^{'}\theta^{'}}_{s}$&$w_{*}= (h B_{s})^{\frac{1}{3}}$, $B_{s} = \frac{g}{\overline{\theta_{ML}}}\overline{w^{'}\theta^{'}}_{s}$ & $w_{*}= (z_{f} B_{s})^{\frac{1}{3}}$\\ \hline %& $w_{*}= (z_{f0} B_{s})^{\frac{1}{3}}$
     Time Scale&$\tau = \frac{h}{w^{*}}$ &$\tau = \frac{h}{w^{*}}$ & $\tau=N^{-1}$ \\ \hline %& $\frac{t}{\tau}=\frac{z_{enc}}{L_{0}}$, $L_{0}=\left(\frac{B_{s}}{N^{3}}\right)^{\frac{1}{2}}$
  \end{tabular}
}
\caption{Comparison of relevant definitions with those from key publications.  
$b=\frac{g}{\overline{\theta_{ML}}}\overline{\theta}$}
\label{table:reldefs}   
%\end{center}    
\end{table}

\clearpage

\section{Verifying the Model Output}
\label{sec:CheckingtheModel}
\subsection{Time till well-mixed}%Spin Up Time according to the Convective Time Scale $\tau$}
\FloatBarrier

Time must be allowed to establish statistically steady turbulent flow.  \citeauthor{SullMoengStev} in 
\cite{SullMoengStev} recommended 10 eddie turnover times based on the convective time scale 
$\tau = \frac{h}{w^{*}} = \frac{h}{ \left( \frac{gh}{\overline{\theta}_{ML}}(\overline{w^{'} \theta^{'}_{s}}) \right)^{\frac{1}{3}} } $, 
and \citeauthor{BrooksFowler2} in \cite{BrooksFowler2} chose a simulated time of 2 hours.  For all of 
the runs, at least 10 eddie turnover times were completed by 2 simulated hours (Figure \ref{fig:ScaledTimevsTime}).  
Although each run has a distinct convective velocity scale that increases with time ($w^{*}(time)$), 
dividing boundary layer height ($h$) by it to obtain $\tau$ results in a collapse from 7 to 3 curves, 
one for each $\gamma$.\\

A measureable well mixed layer (\acs{ML}) and \acs{EL} based on the horizontaly averaged, ensemble averaged
potential temperature ($\overline{\theta}$) profile develops after 2 hours 
(Figure \ref{fig:tempgradfluxprofs15010}).  After 2 or 3 hours the \acs{EL} is fully contained within the vertical 
region of high resolution.\\

Averaged heat fluxes ($\overline{w^{'}\theta^{'}}$) (Figure \ref{fig:scaledfluxprofs15010}) and 
root mean squared vertical velocity perturbations ($\sqrt{w^{'2}}$) (Figure \ref{fig:rmswvelprofs15010})
become self similar and are scaled well by the surface heat flux ($\overline{w^{'}\theta^{'}}_{s}$) 
and the convective velocity scale ($w^{*}$) respectively after 2 hours.\\


\begin{figure}[!h]
    \centering
    % plot_height.py [master 03d2835] Round 1 of Plots in Results 
    \includegraphics[scale=.5]{/tera/phil/nchaparr/python/Plotting/Dec252013/pngs/scaledtimevstime}
    \caption{Plots of scaled time vs time for all runs.  Scaled time is based on the convective time scale 
    and can be thought of as the number of times an eddie has reached the top of the CBL. }
    \label{fig:ScaledTimevsTime}   
\end{figure}

\begin{figure}[htbp]
    \centering
    % plot_dthetaflux.py [master 03d2835] Round 1 of Plots in Results
    \includegraphics[scale=.5]{/tera/phil/nchaparr/python/Plotting/Mar52014/pngs/theta_flux_profs}
    \caption{Vertical profiles of the ensemble and horizontally averaged potential temperature ($\overline{\theta}$), its vertical gradient ($\frac{\partial \overline{\theta}}{\partial z}$)  
     and heat flux ($\overline{w^{'}\theta^{'}}$) for the 150/10 run}
    \label{fig:tempgradfluxprofs15010}   % label should change
\end{figure}

\begin{figure}[htbp]
    \centering
    % plot_dthetaflux.py [master 9883fda] Round 2 of Plots in Results
    %\includegraphics[scale=.5]{/tera/phil/nchaparr/python/Plotting/Mar52014/pngs/scaled_theta_flux_profs}
    %[master 41d073a] Mar52014 scaled_flux_profs by plot_dthetaflux.py
    \includegraphics[scale=.5]{/tera/phil/nchaparr/python/Plotting/Mar52014/pngs/scaled_flux_profs}
    \caption{$\overline{w^{'}\theta^{'}}$ and scaled $\overline{w^{'}\theta^{'}}$  vs scaled height for the 150/10 run}
    \label{fig:scaledfluxprofs15010}   % label should change
\end{figure}

\begin{figure}[htbp]
    \centering
    %w_analysis.py [master 03d2835] Round 1 of Plots in Results
    \includegraphics[scale=.5]{/tera/phil/nchaparr/python/Plotting/Mar52014/pngs/rmswvels}
    \caption{$\sqrt{w^{,2}}$ vs scaled height for the 150/10 run}
    \label{fig:rmswvelprofs15010}   % label should change
\end{figure}

\clearpage
\subsection{FFT Energy Spectra}
\FloatBarrier

Two dimensional \acs{FFT} power spectra taken of horizontal slices of $w^{'}$ 
(Figure \ref{fig:2fftw602point5}) at three different levels ($h_{0}$, $h$ and $h_{1}$) are collapsed to 
one dimension by integrating around a circle of wave-number radius $k$.  Isotropy in all radial 
directions is assumed and $k = \sqrt{k_{x}^{2} + k_{y}^{2}}$.  \\

The resulting scalar density spectra show peaks in  energy at the larger scales, cascading to the lower 
scales roughly according to a $\frac{-5}{3}$ slope, lower in the \acs{EL}.  At the top of the \acs{EL} 
where turbulence is supressed by stability, the slope is steeper.  The peak in energy occurs at smaller 
scales at the inversion ($h$) as compared to at the bottom of the \acs{EL} ($h_{0}$), indicating a 
change in the size of the dominant turbulent structures further into the entrainment layer (\acs{EL}).\\

\begin{figure}[htbp]
    \centering
    % fft_chap.py [master 03d2835] Round 1 of Plots in Results
    \includegraphics[scale=.5]{/tera/phil/nchaparr/python/Plotting/Dec252013/pngs/scalarfftpow}
    \caption{Scalar FFT  energy vs wavenumber ($k = \sqrt{k_{x}^{2}+k_{y}^{2}}$) for the 60/2.5 run
at 2 hours.  $E(k)$ is $E(k_{x}, k_{y})$ integrated around circles of radius $k$.  
   $E(k_{x}, k_{y})$ is the total integrated energy over the 2D domain.  
   $k_{x}$ and $k_{y}$ are number of waves per domain length.}
    \label{fig:2fftw602point5}   % label should change
\end{figure}

\clearpage

\subsection{Ensemble and horizontally averaged vertical Potential Temperature $\overline{\theta}$ 
and Heat Flux profiles $\overline{w^{'}\theta^{'}}$}
%Average Potential Temperature, Heat Flux and Kinetic Energy}
\FloatBarrier

The $\overline{\theta}$ profiles exhibit an \acs{ML} above which  $\frac{\partial\overline{\theta}}{\partial z}>0$ 
and reaches a maximum value at $h$ before resuming $\gamma$  at $h_{1}$ 
(Figures \ref{fig:tempgradfluxprofs15010} and \ref{fig:pottempprofs2hrs}).  Convective boundary layer \acs{CBL} growth is stimulated 
by $\overline{w^{'}\theta^{'}}_{s}$ and inhibited by $\gamma$.\\

The horizonally averaged, ensemble averaged heat flux ($\overline{w^{'}\theta^{'}}$) profiles decrease 
from the surface value ($\overline{w^{'}\theta^{'}_{s}}$) passing through zero to a minumum before 
increasing to zero (Figures \ref{fig:tempgradfluxprofs15010} and  \ref{fig:fluxprofs2hrs}).  All minima are 
less  in magnitude than the zero order approximation ($-.2 \times \overline{w^{'}\theta^{'}_{s}}$).\\


\begin{figure}[htbp]
    \centering
    % plot_theta_profs.py [master 9883fda] Round 2 of Plots in Results
    \includegraphics[scale=.5]{/tera/phil/nchaparr/python/Plotting/Dec252013/pngs/theta_profs2hrs}
    \caption{$\overline{\theta}$ profiles at 2 hours}
    \label{fig:pottempprofs2hrs}   % label should change
\end{figure}

\begin{figure}[htbp]
    \centering
    %plot_theta_profs.py [master 03d2835] Round 1 of Plots in Results
    \includegraphics[scale=.5]{/tera/phil/nchaparr/python/Plotting/Dec252013/pngs/flux_profs2hrs}
    \caption{Scaled $\overline{w^{'}\theta^{'}}_{s}$ profiles at 2 hours}
    \label{fig:fluxprofs2hrs}   % label should change
\end{figure}

%Each of the $\overline{\theta}$ and $\overline{w^{'}\theta^{'}}$ profiles has a region that can be defined as an \acs{EL}.
%Here we use the former definition.  The point of minimum $\overline{w^{'}\theta^{'}}$ is lower than $h$.  
%\citeauthor{SullMoengStev} in \cite{SullMoengStev} noted that the upper extrema of the individual flux quadrant profiles 
%are closer or even coincide with $h$.\\

%($\sqrt{u^{,2}}$) profiles show a dominance of vertical velocity perturbation ($w^{'}$) in the \acs{ML}, with a peak in horizontal velocity
%($u^{'}$ and $v^{'}$) within the \acs{EL} where the buoyancy driven $w^{'}$ is inhibited by stability (Figure \ref{fig:rmsvel150102hrs}). \\

%\begin{figure}[htbp]
%    \centering
    %plot_vars.py [master 03d2835] Round 1 of Plots in Results
%    \includegraphics[scale=.5]{/tera/phil/nchaparr/python/Plotting/Mar52014/pngs/rmsvel2}
%    \caption{Vertical $\frac{\sqrt[]{u^{,2}}}{w^{*}}$ profiles at 2 hours for the 150/10 run}
%    \label{fig:rmsvel150102hrs}   % label should change
%\end{figure}

\clearpage

\subsection{Visualization of Structures Within the Entrainment Layer}
\FloatBarrier

Horizontal slices, at the three entrainment layer (\acs{EL}) levels, of the potential temperature 
and vertical velocity perturbations are plotted to see the turbulent structures.  At the bottom of the \acs{EL} ($h_{0}$) 
in the 150/10 run (Figure \ref{fig:conts} (a) and (d)) coherent areas of positive and negative temperature perturbations 
correspond to areas of upward and downward moving air.\\

The individual plumes of relatively cool air are more evident at the inversion ($h$) and their 
locations correspond to areas of upward motion ((b) and (e)).  Most of the upward moving cool areas are adjacent to and even 
encircled by smaller areas of downward moving warm air.  At $h_{1}$ ((c) and (f)) peaks of cool air are associated 
with both up and down-welling.\\  

In the 60/2.5 run (Figure \ref{fig:conts1}) a similar progression is evident but the impinging, cool upward moving
plumes are more defined.  This is to be expected since stronger stability inhibits deformation of the 
inversion interface.\\   

\begin{figure}[htbp]
\caption{$\theta^{'}$ (left) and $w^{'}$ (right) at 2 hours at $h_{0}$ (a,d), $h$ (c,e) and $h_{1}$ (d,f)}
\begin{minipage}[b]{0.5\linewidth}
  
        %Flux_Quads.py [master 03d2835] Round 1 of Plots in Results
        \subfloat[]{\label{main:a}
                \includegraphics[scale=.36]{/tera/phil/nchaparr/python/Plotting/Mar52014/pngs/theta_cont0}}\\
        \subfloat[]{\label{main:b}      
                \includegraphics[scale=.36]{/tera/phil/nchaparr/python/Plotting/Mar52014/pngs/theta_cont1}}\\ 
        \subfloat[]{\label{main:c}      
                \includegraphics[scale=.36]{/tera/phil/nchaparr/python/Plotting/Mar52014/pngs/theta_cont2}} 
 \end{minipage}             
\quad
\begin{minipage}[b]{0.5\linewidth}
        %Flux_Quads.py [master 9883fda] Round 2 of Plots in Results
        \subfloat[]{\label{main:d}
                \includegraphics[scale=.36]{/tera/phil/nchaparr/python/Plotting/Mar52014/pngs/wvel_cont0}}\\
       
       \subfloat[]{\label{main:e}
                \includegraphics[scale=.36]{/tera/phil/nchaparr/python/Plotting/Mar52014/pngs/wvel_cont1}}\\
        
       \subfloat[]{\label{main:f}
                \includegraphics[scale=.36]{/tera/phil/nchaparr/python/Plotting/Mar52014/pngs/wvel_cont2}}                 
\end{minipage}
        
        \label{fig:conts}
\end{figure}

\begin{figure}[htbp]
\caption{$\theta^{'}$ (left) and $w^{'}$ (right) at 2 hours at $h_{0}$ (a,d), $h$(b,e) and $h_{1}$(c,f)}
\begin{minipage}[b]{0.5\linewidth} 
        
        \subfloat[]{\label{main:a}
                \includegraphics[scale=.36]{/tera/phil/nchaparr/python/Plotting/Dec252013/pngs/theta_cont0}}\\
        \subfloat[]{\label{main:b}      
                \includegraphics[scale=.36]{/tera/phil/nchaparr/python/Plotting/Dec252013/pngs/theta_cont1}}\\ 
        \subfloat[]{\label{main:c}      
                \includegraphics[scale=.36]{/tera/phil/nchaparr/python/Plotting/Dec252013/pngs/theta_cont2}} 
 \end{minipage}             
\quad
\begin{minipage}[b]{0.5\linewidth}
        \subfloat[]{\label{main:d}
                \includegraphics[scale=.36]{/tera/phil/nchaparr/python/Plotting/Dec252013/pngs/wvel_cont0}}\\
       
       \subfloat[]{\label{main:e}
                \includegraphics[scale=.36]{/tera/phil/nchaparr/python/Plotting/Dec252013/pngs/wvel_cont1}}\\
        
       \subfloat[]{\label{main:f}
                \includegraphics[scale=.36]{/tera/phil/nchaparr/python/Plotting/Dec252013/pngs/wvel_cont2}}                 
\end{minipage}
        
        \label{fig:conts1}
\end{figure}

\clearpage


\section{Local Mixed Layer Heights ($h_{0}^{l}$)}
\label{sec:locmlh}     
\FloatBarrier

Local $\theta$ profiles (Figures \ref{fig:rssfitshigh} and \ref{fig:rssfitslow}) exhibit a distinct 
\acs{ML} before resuming $\gamma$ but 
not always a clearly defined \acs{EL}.  There are sharp changes in the profile well into the free 
atmosphere, due possibly to waves, which render the gradient method for determining $h^{l}$ 
unusable.  Instead a linear regression method is used, whereby three lines representing: the
 \acs{ML}, the \acs{EL} and the upper lapse rate ($\gamma$), are fit to the profile according 
to the minimum residual sum of squares (RSS).  Determining local \acs{ML} height ($h_{0}^{l}$) was 
more straight forward than the local height of maximum potential temperature gradient 
($h^{l}$) for the reasons stated above.\\  

Figure \ref{fig:rssfitshigh} shows two local $\theta$ profiles where $h_{0}^{l}$ is relatively high.  
A sharp interface is evident indicating that this is within an active plume impinging on the stable layer.
In Figure \ref{fig:rssfitslow} where $h_{0}^{l}$ is relatively low a less defined interface indicates 
a point now outside a rising plume.  Contour plots (Figure \ref{fig:conts2}) show regions of high 
$h_{0}^{l}$ corresponding to regions of upward moving relatively cool air at $h$.\\

The distribution of $h_{0}^{l}$ is related to the depth of the entrainment layer (\acs{EL}).
Spread increases with increasing $\overline{w^{'}\theta^{'}_{s}}$ and decreases with increasing $\gamma$
(Figure \ref{fig:localhhist}).  When scaled by $h$  (Figure \ref{fig:localhpdf}), the local \acs{ML} height distribution 
has spread that narrows with increased $\gamma$ and seems relatively uninfluenced by change in $\overline{w^{'}\theta^{'}}_{s}$.  
The upper limit seems to be constant at about 1.1($\times h$) , whereas the lower limit varies 
depending on $\gamma$.   Runs with lower $h$ and narrower $\Delta h$ have relativiely 
larger spacing between bins and so higher numbers in each bin.  The above supports the results outlined in
Section \ref{subsec:deltahri}.\\


\begin{figure}[htbp]
%Pcolor_Peaks.py [master 61491be] rss_fit plots
\begin{minipage}[b]{0.5\linewidth}
        %
        \subfloat[]{\label{main:a}
                \includegraphics[scale=.36]{/tera/phil/nchaparr/python/Plotting/Dec252013/pngs/rss_fit_high}}\\
        \end{minipage}             
\quad
\begin{minipage}[b]{0.5\linewidth}
        \subfloat[]{\label{main:b}          
          
                \includegraphics[scale=.36]{/tera/phil/nchaparr/python/Plotting/Mar52014/pngs/rss_fit_high}}\\
       
       \end{minipage}
        \caption{Local vertical $\theta$ profiles with 3-line fit for the 60/2.5 (a) and 150/10 (b) runs at 
points where $h^{l}_{0}$ is high.}
        \label{fig:rssfitshigh}
\end{figure}

\begin{figure}[htbp]
%Pcolor_Peaks.py [master 61491be] rss_fit plots
\begin{minipage}[b]{0.5\linewidth}
        %
        \subfloat[]{\label{main:a}
                \includegraphics[scale=.36]{/tera/phil/nchaparr/python/Plotting/Dec252013/pngs/rss_fit_low}}\\
        \end{minipage}             
\quad
\begin{minipage}[b]{0.5\linewidth}
        \subfloat[]{\label{main:b}          
          
                \includegraphics[scale=.36]{/tera/phil/nchaparr/python/Plotting/Mar52014/pngs/rss_fit_low}}\\
       
       \end{minipage}
        \caption{Local vertical $\theta$ profiles with 3-line fit for the 60/2.5 (a) and 150/10 (b) runs at 
points where $h^{l}_{0}$ is low.}
        \label{fig:rssfitslow}
\end{figure}

\begin{figure}[htbp]
\caption{$\theta^{'}$ (a,d), $w^{'}$(b,e) at $h_{1}$(c,f) and local ML height $h^{l}_{0}$ at 2 hours for 60/2.5 (left) and 150/10 (right) runs}
\begin{minipage}[b]{0.5\linewidth} 
        
        \subfloat[]{\label{main:a}
                \includegraphics[scale=.36]{/tera/phil/nchaparr/python/Plotting/Dec252013/pngs/theta_cont1}}\\
        \subfloat[]{\label{main:b}      
                \includegraphics[scale=.36]{/tera/phil/nchaparr/python/Plotting/Dec252013/pngs/wvel_cont1}}\\ 
        \subfloat[]{\label{main:c}
          %Get_ML_Heights.py [master b7c3e5b] h_cont
                \includegraphics[scale=.36]{/tera/phil/nchaparr/python/Plotting/Dec252013/pngs/h_cont}} 
 \end{minipage}             
\quad
\begin{minipage}[b]{0.5\linewidth}
        \subfloat[]{\label{main:d}
                \includegraphics[scale=.36]{/tera/phil/nchaparr/python/Plotting/Mar52014/pngs/theta_cont1}}\\
       
       \subfloat[]{\label{main:e}
                \includegraphics[scale=.36]{/tera/phil/nchaparr/python/Plotting/Mar52014/pngs/wvel_cont1}}\\
        
       \subfloat[]{\label{main:f}
                \includegraphics[scale=.36]{/tera/phil/nchaparr/python/Plotting/Mar52014/pngs/h_cont}}                 
\end{minipage}
        
        \label{fig:conts2}
\end{figure}

%could put these sideways.  could do with checking the 2.5/60 low hs

\begin{figure}[htbp]
\caption{Histograms of $h^{l}_{0}$ for $\overline{w^{'}\theta^{'}_{s}} = 150$ to $60 (W / m^{2})$ (a to c) and $ \gamma = 10$ to $2.5 (K/Km)$ (c to g) at 5 hours}
%ML_Height_hist.py(sp?) master f992942ade
\begin{minipage}[b]{0.32\linewidth} 
        
        \subfloat[]{\label{main:a}
                \includegraphics[scale=.22]{/tera/phil/nchaparr/python/Plotting/Mar52014/pngs/ML_Height_hist}}\\
        \subfloat[]{\label{main:b}      
                \includegraphics[scale=.22]{/tera/phil/nchaparr/python/Plotting/Dec142013/pngs/ML_Height_hist}}\\ 
        \subfloat[]{\label{main:c}          
                \includegraphics[scale=.22]{/tera/phil/nchaparr/python/Plotting/Mar12014/pngs/ML_Height_hist}} 
 \end{minipage}             
%\quad
\begin{minipage}[b]{0.32\linewidth}
        \subfloat[]{\label{main:d}
                \includegraphics[scale=.22]{/tera/phil/nchaparr/python/Plotting/Jan152014_1/pngs/ML_Height_hist}}\\
       \subfloat[]{\label{main:e}
                \includegraphics[scale=.22]{/tera/phil/nchaparr/python/Plotting/Nov302013/pngs/ML_Height_hist}}\\
       \subfloat[]{\label{main:f}
                \includegraphics[scale=.22]{/tera/phil/nchaparr/python/Plotting/Dec202013/pngs/ML_Height_hist}}                 
\end{minipage}
\begin{minipage}[b]{0.32\linewidth}
        %\subfloat[]{\label{main:d}
        %        \includegraphics[scale=.18]{/tera/phil/nchaparr/python/Plotting/Mar52014/pngs/ML_Height_hist}}\\
       
       %\subfloat[]{\label{main:e}
       %         \includegraphics[scale=.18]{/tera/phil/nchaparr/python/Plotting/Mar52014/pngs/ML_Height_hist}}\\
       \vspace{10mm} 
       \subfloat[]{\label{main:f}
                \includegraphics[scale=.22]{/tera/phil/nchaparr/python/Plotting/Dec252013/pngs/ML_Height_hist}}                 
\end{minipage}

        
        \label{fig:localhhist}
\end{figure}

\begin{figure}[htbp]
\caption{PDFs of $\frac{h^{l}_{0}}{h}$ for $\overline{w^{'}\theta^{'}_{s}} = 150$ to $60 (W / m^{2})$ (a to c) and $ \gamma = 10$ to $2.5 (K/Km)$ (c to g) at 5 hours}
%ML_Height_hist.py(sp?) master f992942ade
\begin{minipage}[b]{0.32\linewidth} 
        
        \subfloat[]{\label{main:a}
                \includegraphics[scale=.22]{/tera/phil/nchaparr/python/Plotting/Mar52014/pngs/scaled_ML_Height_hist}}\\
        \subfloat[]{\label{main:b}      
                \includegraphics[scale=.22]{/tera/phil/nchaparr/python/Plotting/Dec142013/pngs/scaled_ML_Height_hist}}\\ 
        \subfloat[]{\label{main:c}          
                \includegraphics[scale=.22]{/tera/phil/nchaparr/python/Plotting/Mar12014/pngs/scaled_ML_Height_hist}} 
 \end{minipage}             
%\quad
\begin{minipage}[b]{0.32\linewidth}
        \subfloat[]{\label{main:d}
                \includegraphics[scale=.22]{/tera/phil/nchaparr/python/Plotting/Jan152014_1/pngs/scaled_ML_Height_hist}}\\
       \subfloat[]{\label{main:e}
                \includegraphics[scale=.22]{/tera/phil/nchaparr/python/Plotting/Nov302013/pngs/scaled_ML_Height_hist}}\\
       \subfloat[]{\label{main:f}
                \includegraphics[scale=.22]{/tera/phil/nchaparr/python/Plotting/Dec202013/pngs/scaled_ML_Height_hist}}                 
\end{minipage}
\begin{minipage}[b]{0.32\linewidth}
        %\subfloat[]{\label{main:d}
        %        \includegraphics[scale=.18]{/tera/phil/nchaparr/python/Plotting/Mar52014/pngs/ML_Height_hist}}\\
       
       %\subfloat[]{\label{main:e}
       %         \includegraphics[scale=.18]{/tera/phil/nchaparr/python/Plotting/Mar52014/pngs/ML_Height_hist}}\\
       \vspace{10mm} 
       \subfloat[]{\label{main:f}
                \includegraphics[scale=.22]{/tera/phil/nchaparr/python/Plotting/Dec252013/pngs/scaled_ML_Height_hist}}                 
\end{minipage}

        
        \label{fig:localhpdf}
\end{figure}

%\begin{figure}[htbp]
 %   \centering
    %plot_height.py [master 199de9a7cf]  
  %  \includegraphics[scale=.5]{/tera/phil/nchaparr/python/Plotting/Dec252013/pngs/varvsinvri}
  %  \caption{Variance vs \acs{Ri}$^{-1}$ at 5 hours}
   % \label{fig:varsvsinvri}   % label should change
%\end{figure}

\clearpage

\section{Flux Quadrants}
\label{sec:fluxquadrants}     
\FloatBarrier

As \citeauthor{SullMoengStev} point out in \cite{SullMoengStev} when broken out into four quadrants 
(Figure \ref{fig:fluxqadprofs}) the $\overline{w^{'}\theta^{'}}$ profiles have upper extrema above 
that of the total average profile ($z_{f}$).  2D histograms of the four quadrants are plotted at $h_{0}$, $h$ 
and $h_{1}$ to see how the distributions are influenced by changes in $\overline{w^{'} \theta^{'}}$ 
and $\gamma$.\\

 At $h_{0}$ (Figure \ref{fig:fluxquadsh0}) fast updraughts are relatively warm.  The spread in $w^{'}$ 
increases with increasing $\overline{w^{'}\theta^{'}}_{s}$ and decreases with increased $\gamma$.
At $h$ (Figure \ref{fig:fluxqadprofs}) the faster updraughts are now relatively cool and movement 
(both up and down) of warmer air from aloft becomes more prominent.  The spread of $w^{'}$ 
and $\theta^{'}$ both increase with increasing $\overline{w^{'}\theta^{'}}$ whereas that of
$\theta^{'}$ increases only slightly with increased stability.  As expected stability inhibits both
upward and downward $w^{'}$. \\ 

Although the quadrant of overall largest magnitude is that of
upward moving cool air, \citeauthor{SullMoengStev}'s assertion in \cite{SullMoengStev}
that in the \acs{EL} the heat flux is effectively due to downward moving warm air because
the other three quadrants cancel, is found to be approximately true.  At the top of the EL (Figure \ref{fig:fluxquadsh1}) 
velocities are damped and the distributions approach symmetry appart from some slow, cool, impinging up- 
and down-draughts as in Figure \ref{fig:conts2}. \\

\begin{figure}[htbp]
\begin{minipage}[b]{0.5\linewidth}
        %
        \subfloat[]{\label{main:a}
                \includegraphics[scale=.36]{/tera/phil/nchaparr/python/Plotting/Dec252013/pngs/fluxquadprofs}}\\
        \end{minipage}             
\quad
\begin{minipage}[b]{0.5\linewidth}
        \subfloat[]{\label{main:d}          
          %Flux_Quads.py [master 0eaf94e] fluxquadprofs
                \includegraphics[scale=.36]{/tera/phil/nchaparr/python/Plotting/Mar52014/pngs/fluxquadprofs}}\\
       \end{minipage}
        \caption{Scaled $\overline{w^{'} \theta^{'}}$ quadrant profiles at 5 hours for the 60/2.5 (a) and 150/10 (b) run}
        \label{fig:fluxqadprofs}
\end{figure}

%redo these with one colorbar see 
%http://stackoverflow.com/questions/13784201/matplotlib-2-subplots-1-colorbar  

\begin{figure}[htbp]
%
%\begin{minipage}[b]{0.32\linewidth} 
        
%        \subfloat[]{\label{main:a}
 %               \includegraphics[scale=.22]{/tera/phil/nchaparr/python/Plotting/Mar52014/pngs/fluxquadhist0}}\\
  %      \subfloat[]{\label{main:b}      
   %             \includegraphics[scale=.22]{/tera/phil/nchaparr/python/Plotting/Dec142013/pngs/fluxquadhist0}}\\ 
    %    \subfloat[]{\label{main:c}          
     %           \includegraphics[scale=.22]{/tera/phil/nchaparr/python/Plotting/Mar12014/pngs/fluxquadhist0}} 
 %\end{minipage}             
%
%\begin{minipage}[b]{0.32\linewidth}
 %       \subfloat[]{\label{main:d}
  %              \includegraphics[scale=.22]{/tera/phil/nchaparr/python/Plotting/Jan152014_1/pngs/fluxquadhist0}}\\
   %    \subfloat[]{\label{main:e}
    %            \includegraphics[scale=.22]{/tera/phil/nchaparr/python/Plotting/Nov302013/pngs/fluxquadhist0}}\\
    %   \subfloat[]{\label{main:f}
     %           \includegraphics[scale=.22]{/tera/phil/nchaparr/python/Plotting/Dec202013/pngs/fluxquadhist0}}                 
%\end{minipage}
%\begin{minipage}[b]{0.32\linewidth}
        %\subfloat[]{\label{main:d}
        %        \includegraphics[scale=.18]{/tera/phil/nchaparr/python/Plotting/Mar52014/pngs/ML_Height_hist}}\\
       
       %\subfloat[]{\label{main:e}
       %         \includegraphics[scale=.18]{/tera/phil/nchaparr/python/Plotting/Mar52014/pngs/ML_Height_hist}}\\
 %      \vspace{10mm} 
  %     \subfloat[]{\label{main:f}
\centering
 \includegraphics[scale=.8]{/tera/phil/nchaparr/python/Plotting/Dec252013/pngs/fluxquadhists0}                 

%\end{minipage}
\label{fig:fluxquadsh0}
\caption{ $\overline{w^{'}\theta^{'}}$ quadrants at $h_{0}$ for $w^{'}\theta^{'} = 150 - 60 (W/m^{2}$) (top-bottom) and $\gamma = 10 - 2.5 (K/Km)$ (left-right) at 5 hours}
\end{figure}

\begin{figure}[htbp]
%
\centering
 \includegraphics[scale=.8]{/tera/phil/nchaparr/python/Plotting/Dec252013/pngs/fluxquadhists1}                 

\caption{ $\overline{w^{'}\theta^{'}}$ quadrants at $h$ for $w^{'}\theta^{'} = 150 \ - \ 60$(W/$m^{2}$) (top - bottom) and $\gamma = 10 \ - \  2.5$(K/Km) (left - right) at 5 hours}

\label{fig:fluxquadsh}

\end{figure}

\begin{figure}[htbp]
%
\caption{ $\overline{w^{'}\theta^{'}}$ quadrants at $h_{1}$ for $w^{'}\theta^{'} = 150 \ to \ 60$(W/$m^{2}$) (top to bottom) and $\gamma = 10 \ to \ 2.5$(K/Km) (left to right) at 5 hours}
\centering
 \includegraphics[scale=.8]{/tera/phil/nchaparr/python/Plotting/Dec252013/pngs/fluxquadhists2}                 


\label{fig:fluxquadsh1}
\end{figure}

\clearpage

\section{$h$ and  $\Delta h$ based on Average Profiles}
\label{sec:hdeltahavprofs}

\FloatBarrier
\subsection{Reminder of Relevant Definitions}
\FloatBarrier
Here we define \acs{CBL} height $h$  as the point at which 
$\frac{\partial \overline{\theta}}{\partial z}$ is maximum and the \acs{EL} limits: $h_{0}$
the point at which $\frac{\partial \overline{\theta}}{\partial z}$ first exceeds a threshold
and $h_{1}$ the point at which $\frac{\partial \overline{\theta}}{\partial z}$ resumes $\gamma$.
The temperature jump $\Delta \theta$ is the difference accross the \acs{EL}.\\

%definitions
\begin{figure}[htbp]
    \centering
    %plot_height.py[master 1573b9d] h vs time plot
    \includegraphics[scale=.5]{/tera/phil/nchaparr/python/Plotting/Dec252013/pngs/height_defs}
    \caption{Height Definitions}
    \label{fig:hdefs1}   % label should change
\end{figure}

\begin{table}[htbp]
    %\begin{center}
\centerline{
    \begin{tabular}{| p{3cm}| p{3cm} | p{3cm} | p{3cm} | p{3cm} |}
    \hline
     Description & This Study & Sullivan et al. \cite{SullMoengStev} & Fedorovich et al.\cite{FedConzMir04} \\ \hline %&  Garcia and Mellado \cite{GarciaMellado}
     CBL Height&$h$&$h$ & $z_{f}$ \\ \hline %& $z_{enc} \approx z_{f0}$
     Temperature Jump&$\Delta \theta = \overline{\theta}(h_{1})-\overline{\theta}(h_{0})$&$\Delta \theta = \overline{\theta}(z_{f1})-\overline{\theta}(z_{f})$ & $\Delta b = b_{0}(z_{f}) -b(z_{f})$ \\ \hline %& $\Delta b = b_{0}(z_{f}) - b(z_{f}) $
     && &$\delta b = b(z_{f1})$ - $b(z_{f0})$\\ \hline %& $\delta b = b(z_{f1})$ - $b(z_{f0})$&
     Convective Velocity Scale&$w_{*}= (h B_{s})^{\frac{1}{3}}$, $B_{s} = \frac{g}{\overline{\theta_{ML}}}\overline{w^{'}\theta^{'}}_{s}$&$w_{*}= (h B_{s})^{\frac{1}{3}}$, $B_{s} = \frac{g}{\overline{\theta_{ML}}}\overline{w^{'}\theta^{'}}_{s}$ & $w_{*}= (z_{f} B_{s})^{\frac{1}{3}}$\\ \hline %& $w_{*}= (z_{f0} B_{s})^{\frac{1}{3}}$
     Richardson Number&$Ri = \frac{\Delta \theta h}{{w^{*2}}}$ &$Ri = \frac{\Delta \theta h}{w^{*2}}$ & $Ri_{\Delta b} = \frac{\Delta b z_{f}}{w^{*2}}$, $Ri_{\delta b} = \frac{\delta b_{i}z_{f}}{w^{*2}}$\\ \hline %& $Ri_{i, f} = \frac{\Delta b z_{enc}}{w^{*2}}$, $Ri_{*} = \frac{\delta b_{i} z_{enc}}{w^{*2}}$
  \end{tabular}
}
\caption{Comparison of relevant definitions with those from key publications}
\label{fig:}   
%\end{center}    
\end{table}

%\footnotetext[1]{Fedorovich et al and Garcia et al use buoancy instead of $\theta$, but they are interchangeable for this purpose}
\clearpage
\subsection{$\frac{w_{e}}{w^{*}}$ vs $Ri^{-1}$}
\FloatBarrier
Covective Boundary Layer (CBL) height ($h$) (Figure \ref{fig:hvstime}) grows rapidly initially with a 
steadily decreasing rate and relates to the square root of time (Figure\ref{fig:hvstimeloglog}).  
It is found to be proportionate to the height of minimum flux ($z_{f}$) (Figure \ref{fig:hvstime1}).\\
% thus 
%indicating it is in the quasi-steady state regime outlined by \citeauthor{FedConzMir04} in \cite{FedConzMir04} 
%in which the zero order relationship of the scaled entrainment rate to Richardson Number
%(\acs{Ri}) is expected.\\

Inverse Richardson Number (\acs{Ri}$^{-1}$) decreases with respect to time 
and clusters according to $\gamma$. (Figure \ref{fig:invristime}).  The entrainment rate ($w_{e}= \frac{dh}{dt}$) 
is determined from the slope of a second order polynomial fit to $h(time)$ (Figure \ref{fig:hvstime}).  
When scaled by ($w^{*}$) it is a roughly linear function of  \acs{Ri}$^{-1}$ (Figure \ref{fig:scaledweinvri}).\\    
  
\begin{figure}[htbp]
    \centering
    %plot_height.py[master 1573b9d] h vs time plot
    \includegraphics[scale=.5]{/tera/phil/nchaparr/python/Plotting/Dec252013/pngs/hvstime}
    \caption{$h$ vs time for all runs}
    \label{fig:hvstime}   % label should change
\end{figure}

\begin{figure}[htbp]
    \centering
    %plot_height.py[master 1573b9d] h vs time plot
    \includegraphics[scale=.5]{/tera/phil/nchaparr/python/Plotting/Dec252013/pngs/hstimelog}
    \caption{Log-Log plot of $h$ vs time for all runs}
    \label{fig:hvstimeloglog}   % label should change
\end{figure}

\begin{figure}[htbp]
    \centering
    %plot_height.py[master 1573b9d] h vs time plot
    \includegraphics[scale=.5]{/tera/phil/nchaparr/python/Plotting/Dec252013/pngs/scaled_h_f_time}
    \caption{$\frac{z_{f}}{h}$ vs Time}
    \label{fig:hvstime1}   % label should change
\end{figure}

\begin{figure}[htbp]

%\begin{minipage}[b]{0.5\linewidth}
 \centering        
        %\subfloat[]{\label{main:a}
                \includegraphics[scale=.5]{/tera/phil/nchaparr/python/Plotting/Dec252013/pngs/invristime}
        %\end{minipage}             
%\quad
%\begin{minipage}[b]{0.5\linewidth}
 %       \subfloat[]{\label{main:d}
          %plot_height [master fbd2dfd] invristime1, see branches named accordingly
  %              \includegraphics[scale=.36]{/tera/phil/nchaparr/python/Plotting/Dec252013/pngs/invristime1}}\\
       
   %    \end{minipage}
        \caption{Inverse bulk Richardson Number vs time}
        \label{fig:invristime}
\end{figure}

\begin{figure}[htbp]
\centering
%\begin{minipage}[b]{0.5\linewidth}
        %plot_height [master fbd2dfd] invristime1
 %       \subfloat[]{\label{main:a}
       \includegraphics[scale=.5]{/tera/phil/nchaparr/python/Plotting/Dec252013/pngs/scaledweinvri}
  %      \end{minipage}             
%\quad
%\begin{minipage}[b]{0.5\linewidth}
 %       \subfloat[]{\label{main:d}          
          %plot_height [master 01c3721] scaledweinvri1
  %              \includegraphics[scale=.36]{/tera/phil/nchaparr/python/Plotting/Dec252013/pngs/scaledweinvri1}}\\
       
   %    \end{minipage}
        \caption{Scaled Entrainment rate vs inverse Richardson Number (\acs{Ri})}
        \label{fig:scaledweinvri}
\end{figure}

\clearpage

\subsection{$\frac{\Delta h}{h}$ vs $Ri^{-1}$}
\label{subsec:deltahri}
\FloatBarrier

The scaled upper EL limits ($\frac{h_{1}}{h}$) collapse well in Figure \ref{fig:scaledELlims} 
to an initial value of approximately 1.15, decreasing to about 1.1.  $\frac{h_{0}}{h}$s appear 
grouped according to $\gamma$ and increase with respect to time.  So overall the scaled \acs{EL} appears
to narrow with time.   The scaled flux based \acs{EL} ($z_{f0}$ and $z_{f1}$) appears to remain constant 
with respect to time in Figure \ref{fig:scaledELlims1}.\\

The lower entrainment layer limit $h_{0}$ is the point at which the vertical 
$\frac{\partial \overline{\theta}}{\partial z}$ exceeds a threshold (.0002) chosen such that
it is positive, and at least an order of magnitude smaller than $\gamma$.   Although the resulting 
scaled \acs{EL} depth decreases with increasing \acs{Ri} grouping according to $\gamma$ is evident 
in Figure \ref{fig:scaledeltahinvri}.\\

To explore how varying the threshold value affects the relationship between scaled \acs{EL} depth
and Richardson number (\acs{Ri}), plots analogous to Figure \ref{fig:scaledeltahinvri} were produced at two 
additional thresholds.  A higher threshold value (.0004) results in a higher $h_{0}$ (Figure \ref{fig:thresh1})   
and so a narrower \acs{EL} but a similar grouping according to $\gamma$ (Figure\ref{fig:scaledeltahinvri1}).
A lower threshold value (.0001) results in a lower $h_{0}$ (Figure \ref{fig:thresh2})
but also similar grouping according to $\gamma$ (Figure \ref{fig:scaledeltahinvri2}.\\

When the height definitions are based on the scaled vertical $\frac{\partial \overline{\theta}}{\partial z}$
i.e. $\frac{\partial \overline{\theta}}{\partial z} / \gamma$ profile, only $h_{0}$ changes and for clarity we 
call this \acs{EL} depth $\Delta h^{*}$ and the revised Richardson number Ri$^{*}$.   The curves now collapse and 
scaled \acs{EL} depth is seen to decrease with increasing \acs{Ri}$^{*}$ (Figures \ref{fig:thresh3} to \ref{fig:ELvsri}).\\

There is a slight collapsing effect on the scaled entrainment rate vs \acs{Ri} relationship when
the heights are defined based on the scaled vertical potential temperature gradient 
$\frac{\partial \overline{\theta}}{\partial z} / \gamma$ profile in Figure \ref{fig:scaledweinvri2}.\\

\begin{figure}[htbp]
    \centering
    %plot_height.py [master fd5c6b1] delta h vs time1  
    \includegraphics[scale=.5]{/tera/phil/nchaparr/python/Plotting/Dec252013/pngs/deltahstime1}
    \caption{Scaled Entrainment Layer limits ($\frac{h_{1}}{h}$ and $\frac{h_{0}}{h}$) vs time}
    \label{fig:scaledELlims}   % label should change
\end{figure}

%need flux based el limits scaled by either h or hf
\begin{figure}[htbp]
    \centering
    %plot_height.py [master fd5c6b1] delta h vs time1  
    \includegraphics[scale=.5]{/tera/phil/nchaparr/python/Plotting/Dec252013/pngs/scaled_h_f0_time}
    \caption{Scaled Entrainment Layer limits ($z_{f1}$ and $z_{f0}$) vs time}
    \label{fig:scaledELlims1}   % label should change
\end{figure}

\begin{figure}[htbp]
    \centering
    %plot_height.py [master fd5c6b1] delta h vs time1  
    \includegraphics[scale=.5]{/tera/phil/nchaparr/python/Plotting/Dec252013/pngs/theta_grad_profs}
    \caption{Vertical $\frac{\partial \overline{\theta}}{\partial z}$ profiles with threshold at .0002}
    \label{fig:thresh}   % label should change
\end{figure}

\begin{figure}[htbp]
\centering
%\begin{minipage}[b]{0.5\linewidth}
        %plot_height [master c7af4de] scaleddeltahinvri
 %       \subfloat[]{\label{main:a}
 \includegraphics[scale=.5]{/tera/phil/nchaparr/python/Plotting/Dec252013/pngs/scaleddeltahinvri}
  %      \end{minipage}             
%\quad
%\begin{minipage}[b]{0.5\linewidth}
 %       \subfloat[]{\label{main:d}          
          %plot_height [master b9c30ad] scaleddeltahinvri1
  %              \includegraphics[scale=.36]{/tera/phil/nchaparr/python/Plotting/Dec252013/pngs/scaleddeltahinvri1}}\\
       
   %    \end{minipage}
        \caption{Scaled EL depth vs inverse bulk Richardson Number with threshold at .0002}
         \label{fig:scaledeltahinvri}
\end{figure}

\begin{figure}[htbp]
    \centering
    %plot_height.py [master fd5c6b1] delta h vs time1  
    \includegraphics[scale=.5]{/tera/phil/nchaparr/python/Plotting/Dec252013/pngs/theta_grad_profs5}
    \caption{Vertical $\frac{\partial \overline{\theta}}{\partial z}$ profiles with threshold at .0004}
    \label{fig:thresh1}   % label should change
\end{figure}

\begin{figure}[htbp]
    \centering
    %plot_height.py [master fd5c6b1] delta h vs time1  
    \includegraphics[scale=.5]{/tera/phil/nchaparr/python/Plotting/Dec252013/pngs/scaleddeltahinvri_5}
    \caption{Scaled EL depth vs inverse Richardson Number with threshold at .0004}
    \label{fig:scaledeltahinvri1}   % label should change
\end{figure}

\begin{figure}[htbp]
    \centering
    %plot_height.py [master fd5c6b1] delta h vs time1  
    \includegraphics[scale=.5]{/tera/phil/nchaparr/python/Plotting/Dec252013/pngs/theta_grad_profs6}
    \caption{Vertical $\frac{\partial \overline{\theta}}{\partial z}$ profiles with threshold at .0001}
    \label{fig:thresh2}   % label should change
\end{figure}

\begin{figure}[htbp]
    \centering
    %plot_height.py [master fd5c6b1] delta h vs time1  
    \includegraphics[scale=.5]{/tera/phil/nchaparr/python/Plotting/Dec252013/pngs/scaleddeltahinvri_6}
    \caption{Scaled EL depth vs inverse bulk Richardson Number with threshold at .0001}
    \label{fig:scaledeltahinvri2}   % label should change
\end{figure}

\begin{figure}[htbp]
    \centering
    %plot_height.py [master fd5c6b1] delta h vs time1  
    \includegraphics[scale=.5]{/tera/phil/nchaparr/python/Plotting/Dec252013/pngs/scaled_theta_grad_profs}
    \caption{Scaled vertical $\frac{\partial \overline{\theta}}{\partial z}$ profiles with threshold at .03}
    \label{fig:thresh3}   % label should change
\end{figure}

\begin{figure}[htbp]
    \centering
    %plot_height.py [master fd5c6b1] delta h vs time1  
    \includegraphics[scale=.5]{/tera/phil/nchaparr/python/Plotting/Dec252013/pngs/height_defs_1}
    \caption{Revised height definitions based on scaled $\frac{\partial \overline{\theta}}{\partial z}$ profiles with threshold at .03}
    \label{fig:heightdefs1}   % label should change
\end{figure}


\begin{figure}[htbp]
\begin{minipage}[b]{0.5\linewidth}
        %plot_height [master c7af4de] scaleddeltahinvri
        \subfloat[]{\label{main:a}
                \includegraphics[scale=.36]{/tera/phil/nchaparr/python/Plotting/Dec252013/pngs/scaleddeltahinvri_4}}\\
        \end{minipage}             
\quad
\begin{minipage}[b]{0.5\linewidth}
        \subfloat[]{\label{main:d}          
          %plot_height [master b9c30ad] scaleddeltahinvri1
                \includegraphics[scale=.36]{/tera/phil/nchaparr/python/Plotting/Dec252013/pngs/scaleddeltahinvri}}\\
       
       \end{minipage}
        \caption{Scaled EL Depths vs inverse bulk Richardson number based on scaled $\frac{\partial \overline{\theta}}{\partial z}$ (a) and $\frac{\partial \overline{\theta}}{\partial z}$ (b)}
        \label{fig:ELvsri}
\end{figure}

\begin{figure}[htbp]
\begin{minipage}[b]{0.5\linewidth}
        %plot_height [master c7af4de] scaleddeltahinvri
        \subfloat[]{\label{main:a}
                \includegraphics[scale=.36]{/tera/phil/nchaparr/python/Plotting/Dec252013/pngs/scaledweinvri_1}}\\
        \end{minipage}             
\quad
\begin{minipage}[b]{0.5\linewidth}
        \subfloat[]{\label{main:d}          
          %plot_height [master b9c30ad] scaleddeltahinvri1
                \includegraphics[scale=.36]{/tera/phil/nchaparr/python/Plotting/Dec252013/pngs/scaledweinvri}}\\
       
       \end{minipage}
        \caption{Scaled Entrainment Rate vs inverse bulk Richardson number based on scaled $\frac{\partial \overline{\theta}}{\partial z}$ (a) and $\frac{\partial \overline{\theta}}{\partial z}$ (b)}
        \label{fig:scaledweinvri2}
\end{figure}

\endinput

Any text after an \endinput is ignored.
You could put scraps here or things in progress.

%% The following is a directive for TeXShop to indicate the main file
%%!TEX root = diss.tex

\chapter{discussion}
\label{ch:results}
\setlength{\parindent}{0cm}

\section{Description of Runs}
\FloatBarrier

The domain for each individual case is small relative to that used by \citeauthor{SullMoengStev} in \cite{SullMoengStev}, \citeauthor{FedConzMir04} in \cite{FedConzMir04} and \citeauthor{BrooksFowler2} in \cite{BrooksFowler2} i.e. $5Km \times 5 Km$ in the horizontal.  \citeauthor{SullMoengStev} (\cite{SullMoengStev}) did a higher resolution run on a $3 Km \times 3 Km$ horizontal domain and noticed a lower convective boundary layer height ($h$) but similar slope in $h$ with respect to scaled time when compared with the analogous run on a larger domain with lower resolution.  They speculated the smaller domain enforced a smaller limit on plume size, thus influencing $h$. But according to \citeauthor{SullPat} (\cite{SullPat}) grid size also impacts $h$. \\

\begin{table}[htbp]
    \begin{center}
%\centerline{
    \begin{tabular}{ p{2cm} p{2cm}  p{2cm}  p{2cm} p{2cm} }
    %\hline
Publication & $\Delta z$ in\acs{EL} & Domain Size\\ \hline
      \citeauthor{SullMoengStev} (\citeyear{SullMoengStev}) & &  \\ \hline 
      \citeauthor{FedConzMir04} (\citeyear{FedConzMir04}) &  &  \\ [.3cm] %\hline
      \citeauthor{BrooksFowler2} (\citeyear{BrooksFowler2}) &  &  \\ \hline
    \end{tabular}
%}
\caption[]{Vertical Resolution and domain size in key publications}
\label{table:reldefs}   
\end{center}    
\end{table}


\citeauthor{SullMoengStev}'s (\cite{SullMoengStev}) grid spacing for most of their runs was $\Delta x, y= 33.3$, $\Delta z=10$ except for the run on the smaller domain which had $\Delta x, y = 15$, $\Delta z=6.67$.  The highest resolution \citeauthor{FedConzMir04} used in \cite{FedConzMir04} was $\Delta x, y = 50$ and $\Delta z = 20$.  \citeauthor{BrooksFowler2} in \cite{BrooksFowler2} used $\Delta x, y = 50$ and $\Delta z = 12$ except in resolution test runs where they used $\Delta x, y = 25$ and $\Delta z = 7.27$.  So the vertical resolution around the entrainment region in this study ($\Delta z= 5m$) is higher that the other LES studies. Both \citeauthor{SullMoengStev} (\cite{SullMoengStev}) and \citeauthor{BrooksFowler2} (\cite{BrooksFowler2}) use varying grids in the vertical, such that the region around the entrainment layer (\acs{EL}) is of higher resolution than elsewhere. We do the same in this study and noticed slight kinks in some of the profiles where the transition to and form higher resolution occurs. We will perform one run on a uniform vertical grid at $\Delta z=5m$ to verify that this does not effect the results.\\          

\citeauthor{SullMoengStev}'s (\cite{SullMoengStev}) initialized with a layer of constant potential temperature topped by an inversion topped by a constant lapse rate ($\gamma \approx 2.5 K/Km$). They applied constant average surface heat fluxes ($\overline{w^{'}\theta^{'}}_{s}$) ranging from about $20 - 450 \ Watts/m^{2}$. \citeauthor{BrooksFowler2} (\cite{BrooksFowler2}) followed suit, in that their range of Richardson numbers (\acs{Ri}) resulted from variation of initial inversion ($\Delta \theta$) strength and average surface heat flux ($\overline{w^{'}\theta^{'}}_{s}$).  \citeauthor{FedConzMir04} in \cite{FedConzMir04} start with a finite layer of constant average potential temperature ($\overline{\theta}$) above which there was a constant lapse rate which they varied from  $1 - 10 \ K/Km$. In this study we begin with a constant $\overline{w^{'}\theta^{'}}_{s}$ acting against uniform potential temperature lapse rate.  \citeauthor{SchmidtSchu} point out in \cite{SchmidtSchu} that as a convectively mixed layer (\acs{ML}) grows against a stable lapse rate ($\gamma$) overshoot of the plumes to buoyancy levels above their own, and subsequent entrainment causes a sharp temperature gradient. (see Table \ref{fig:tableofruns}) 

%\clearpage

\section{Relevant Definitions}

\FloatBarrier

See Table \ref{table:reldefs}.\\

\citeauthor{SullMoengStev} (\cite{SullMoengStev}) compared four methods of determining \acs{CBL} height, two of which they based on the vertical average heat flux ($\overline{w^{'}\theta^{'}}$) profile. For both, the time-series were a lot less smooth than that for $z_{f}$ determined in this study.  Their gradient and contour methods produced smoother time-series plots.  The former, they determined from the horizontal average of the local heights of maximum vertical potential temperature gradient.  Description of the contour method will be omitted since it is not directly useful. Their gradient based height is consistently higher than the heat flux based definitions i.e. the flux based definition overall is about 0.9 times the gradient definition. This is in line with the findings of this study. They did not focus on \acs{EL} depth. For their Richardson number (\acs{Ri}) they calculated $\Delta \theta = \overline{\theta}(z_{f1})-\overline{\theta}(z_{f})$.  This value is likely to be smaller than, and not necessarily proportionally to the $\Delta \theta$ used in this study. \\

%TODO: tables for comparison of heights and delta thetatas for 1 the delta h ri relationship and 2 the we ri reltanship.

%TODO: set up for comparisons of the two relationships.  so most of this has to go -- waffley.  must mention bandfs delta h based on flux
%and feds comment about this ie z_f - z_f0 is constant but z_f1-z_f0 changes slowly.  

\citeauthor{FedConzMir04} in \cite{FedConzMir04} determined \acs{CBL} height and \acs{EL} depth from the horizontal and time ($100 \times 2s$) averaged vertical $\overline{w^{'}\theta^{'}}$ profiles.  They used two difference buoyancy ($\frac{g\overline{\theta}}{\overline{\theta_{ML}}}$) jumps: $\Delta b = \overline{\theta}_{0}z_{f} - \overline{\theta}z_{f0}$ for comparison with the zero order model and $\delta \theta = \overline{\theta}z_{f1} - \overline{\theta}z_{f0}$ for comparison with the first order model and analysis of the \acs{EL}.\\  

%They observed a decrease in \acs{EL} depth with increasing \acs{Ri} with possibly a $-1$ or $-\frac{1}{2}$ power law.  In our study all three flux based heights seem to remain a constant fraction of $h$, so at this point our results diverge from theirs.  Their grid spacing could be compared to two of the lower resolutions runs from \citeauthor{SullPat}'s (\cite{SullPat}) resolution study, whereas ours compares with the lowest resolution run of those with vertical flux profiles that converge.  Assuming higher resolution produces more physical results, their flux profiles may be unrealistically wide.  According to \citeauthor{SullPat} (\cite{SullPat}) \acs{CBL} growth rate may also be impacted by such differences in resolution.  Furthermore to compare their plots of scaled entrainment rate vs \acs{Ri}, we would need to reproduce their \acs{Ri} s in our framework.

\citeauthor{BrooksFowler2} (\cite{BrooksFowler2}) used tracer concentration profiles and compare a number of different corresponding \acs{CBL} height definitions.  Although their height and temperature jump used to calculate the Richardson number (\acs{Ri}) are quite different, their scaling relations based on the fluxed based definitions can be compared to those in this study. For example the corresponding scaled entrainment rate vs \acs{Ri} plot has a lot of scatter.\\

The definitions that perform best in relation to \acs{Ri} for \citeauthor{BrooksFowler2} (\cite{BrooksFowler2}) are those based on the means of locally determined heights.  That based on the domain averaged tracer profile, ie the point of maximum vertical gradient, is directly comparable to our $h$. Although, this last definition does not produce a plot as correlated as ours.\\  

Their scaled statistical \acs{EL} definitions based on the local vertical gradient and the local wavelet covariance decrease with increasing \acs{Ri} similarly to ours, but their flux based definition ($2\times(z_{f1}-z_{f})$) show slight and opposite trends when averaged differently.  The latter is in line with what we found.\\


The height definitions in this study are all based on the average vertical potential temperature gradient ($\overline{\theta}$).  It seems to be assumed that the region, where the average potential temperature increases significantly from its mixed layer (\acs{ML}) value through the maximum to that of the free atmosphere, corresponds to the \acs{EL} as enclosed by the zero levels in the average potential temperature flux profiles (\citeauthor{Deardorff79} \cite{Deardorff79}, \citeauthor{FedConzMir04} 
\cite{FedConzMir04}, \citeauthor{GarciaMellado} \cite{GarciaMellado}).  But the average potential temperature profile is not used to quantitatively define the \acs{EL}.\\

\citeauthor{BrooksFowler2} (\cite{BrooksFowler2}) discuss the draw-backs of defining the \acs{EL} based on the 
gradient of an average tracer profile.  Specifically the inconsistency in the size of the gradient
relative to a maximum, at the average \acs{EL} limits as defined based on the local limits. They
found the relative size had significant scatter and varied according to \acs{Ri}. Their maximum and the manner in which they determine is not reproducible in our framework but their conclusion could serve as a caution.\\  

Since in the \acs{ML} on average there is a gradual increase through zero in average potential temperature above the surface layer, rather than a region where the gradient is zero.  So a threshold value must be chosen to identify the lower limit of the \acs{EL}.  This threshold should be less than the upper lapse rate ($\gamma$), positive and consistent for all runs.  It was chosen by looking at the gradient profile and selecting a point which looked reasonable. The principal result was plotted at three different thresholds based on the unscaled gradient ($\frac{\partial \overline{\theta}}{\partial z}$) profiles.\\

The upper \acs{EL} limit is defined as the point at which the average vertical potential temperature gradient resumes $\gamma$.  These two limits then represent: the point above the surface layer at which the air on average begins to be less turbulently mixed, and the lowest point at which the air is unaffected as yet by the convected turbulence.  Our principal length scale $h$ is the point at which the gradient is maximum i.e. the point at which on average the air differs greatest from that directly above it. Our $\Delta \theta$ is the difference in average potential temperature ($\overline{\theta}$) over the \acs{EL}.  We compare $h$ with the fluxed based definitions.
  
%\clearpage

\section{Verifying the Model Output}
\label{sec:CheckingtheModel}
\subsection{Time till well-mixed}%Spin Up Time according to the Convective Time Scale $\tau$}
\FloatBarrier

To establish statistically steady turbulent flow \citeauthor{SullMoengStev} in \cite{SullMoengStev}
ran from the same random initial conditions on their coarse grid for more than ten eddy turnover times.
Then they switched on the nested high resolution grid and continued for another 4 Odie turnovers.  
\citeauthor{BrooksFowler2} (\cite{BrooksFowler2}) waited 2 simulated hours before they
judged the turbulence to be fully developed.  To initialize turbulence they added a small random perturbation
to the temperature field.\\ 

\citeauthor{FedConzMir04} (\cite{FedConzMir04}) focus on the attainment of a quasi-steady
state regime within which their zero order entrainment equation holds.  Their derivation also hinges
upon parametrizations for turbulent kinetic energy ($e$) and dissipation ($\epsilon$):

\begin{equation}
e=w^{*2}\Psi_{e}\left( \frac{z}{z_{i}} \right) \ \epsilon=\frac{w^{*3}}{z_{i}}\Psi_{\epsilon}\left( \frac{z}{z_{i}}\right)
\end{equation}

Where the two functions of dimensionless height integrate over the \acs{CBL} to constants, for example

\begin{equation}
\int^{z_{i}}_{0}\frac{e}{w^{*2}}dz = C_{e}
\end{equation}

In the referenced regime, \acs{CBL} growth is much slower than the convective velocity scale ($w^{*}$),
there is a constant entrainment ratio $-\frac{\overline{w^{'}\theta^{'}}_{min}}{\overline{w^{'}\theta^{'}}_{s}}$
and change in the total $e$ and it's escape from the boundary layer through waves are negligible relative
to the buoyant production and dissipation rate.  The resulting entrainment equation predicts a $\frac{1}{2}$
power law relationship between the normalized height, $z_{i}B_{s}^{-\frac{1}{2}} N^{\frac{3}{2}}$ and time
$tN$.  Since variation in $\overline{\theta}$ results in
less than 3 percent variation in $N$, when the surface heat flux $B_{s}$ and $\gamma$ are constant
this roughly translates to a $\frac{1}{2}$ power law relationship between $h$ and time.  In our study we find this to be the case (see Figure \ref{fig:hvstime1}).\\
  
We also observe self similarity of the scaled flux profiles, and so a constant entrainment ratio (see Figure \ref{fig:scaledfluxprofs15010}). By 2 hours of simulated time, at least 10 eddy turnover times have elapsed and by 3 hours the \acs{EL} is fully within the region of high vertical resolution.  Worth noting is the collapse in scaled time curves from 7 to 3 according to upper lapse rate ($\gamma$) (see Figure \ref{fig:ScaledTimevsTime}).

%\clearpage
\subsection{FFT Energy Spectra}
\FloatBarrier

Based on the scalar \acs{FFT} energy plots taken at the top of the \acs{ML} there is a cascade from the larger
to the smaller scales following the $-\frac{5}{3}$ power law (see Figure \ref{fig:2fftw602point5}). The \acs{CBL} is fully turbulent at this point
but further into the entrainment layer (\acs{EL}) there are large areas of little or no vertical velocity
interspersed with isolated impinging plumes.  So the dominant structures are smaller
and there is a steeper decay to the lower scales.  In this the \acs{FFT} plots and the contour plots in Figures \ref{fig:conts1} and \ref{fig:conts} compliment
each-other.  Furthermore there seems to be adequate scale separation between the dominant turbulent structures and the grid size, as well as isotropic turbulence.

%\clearpage

\subsection{Ensemble and horizontally averaged vertical Potential Temperature $\overline{\theta}$ 
and Heat Flux profiles $\overline{w^{'}\theta^{'}}$}
%Average Potential Temperature, Heat Flux and Kinetic Energy}
\FloatBarrier

\citeauthor{SchmidtSchu} point out in \cite{SchmidtSchu} that as a convectively mixed layer (\acs{ML})
grows against a stable lapse rate ($\gamma$) overshoot of the plumes to levels above their buoyancy causes
a sharp temperature gradient.  The sharpest vertical gradient in the area averaged potential temperature ($\overline{\theta}$)
profile corresponds to the vertical level at which the average potential temperature (Figure \ref{fig:pottempprofs2hrs}) differs greatest from that one level above.  Once a plume has overshot, envelopment or pinching off (\citeauthor{SullMoengStev} 
\cite{SullMoengStev}) of warm air from above causes a more gradual increase in temperature.  
Where this occurs is regarded here as the entrainment layer \acs{EL}.  In the averaged potential temperature 
profile it is represented by an increase in the vertical gradient. On the horizontal plane it would be composed 
of areas of \acs{ML} air interspersed with pockets of warmer air from above.  The ratio of \acs{ML} to stable air 
increases with proximity to the \acs{ML}.  This progression is seen in the average profile as a decrease in the
vertical gradient to close to zero (Figure \ref{fig:hdefs}).  Our average potential temperature profiles in Figure \ref{fig:pottempprofs2hrs} show a well mixed \acs{ML} overshooting
and growing against $\gamma$.  \acs{CBL} growth increases with $\overline{w^{'}\theta^{'}_{s}}$ and is inhibited 
by $\gamma$.  The \acs{ML} warming rate is strongly influenced by $\overline{w^{'}\theta^{'}_{s}}$ and $\gamma$.\\

The vertical $\overline{w^{'}\theta^{'}}$ profiles in Figure \ref{fig:fluxprofs2hrs} assume the expected shape becoming negative in the \acs{EL}
where the upward moving thermals are relatively cooler than the horizontal average and there is also downward
moving warmer air that has been pinched off or folded in.  Like \citeauthor{SullMoengStev} in \cite{SullMoengStev}
and \citeauthor{FedConzMir04} in \cite{FedConzMir04} we notice the entrainment ratio is less than .2 ($\approx .1$) 
for all runs but seems to increase with increased $\gamma$ inline with \citeauthor{Sorbjan}'s assertion in \cite{Sorbjan} 
that moments of $\theta^{'}$ depend on $\gamma$. Otherwise, there seems to be self similarity in time and across runs
when scaled by $\overline{w^{'}\theta^{'}_{s}}$ and plotted against scaled height.  So the scaled depth of the 
region of negative  $\overline{w^{'}\theta^{'}}$ seems more or less constant whereas \citeauthor{FedConzMir04} in 
\cite{FedConzMir04} seem to show a decrease from about .6 to about .2 with increasing \acs{Ri} and \citeauthor{BrooksFowler2}
with their slightly different definition in \cite{BrooksFowler2} seem to observe slight and contrasting trends with
respect to \acs{Ri} depending on whether the output is time averaged or not.



%\clearpage

\subsection{Visualization of Structures within the Entrainment Layer}
\FloatBarrier

\citeauthor{SullMoengStev} in \cite{SullMoengStev} show both horizontal and vertical cross sections
of their domain within the \acs{EL} around the inversion ($h$).  Horizontal cross sections of vertical
velocity and temperature perturbations clearly show coherent structures with both relatively
warm and cool air, associated with up-and-downward velocity.  Vertical cross sections show
impinging plumes and pockets of trapped warmer air.  The weak inversion case seems to show
convective overturning with apparent folding of warm stable air.  The strong inversion
case shows less deformation of the inversion interface and the entrainment event 
shown in the vertical cross section seems to occur via a narrow downward wisp associated
with an impinging plume.  In both cases, the downward motion of air from above is closely associated
with upward moving impinging plumes.\\

In our contours of $w^{'}$ and $\theta^{'}$ we see the almost spoke like pattern characteristic of the
mixed layer (\citeauthor{SchmidtSchu} \cite{SchmidtSchu}) at the lower limit of the \acs{EL}
and then distinct plumes become clearer at the inversion and above, where there are coherent
areas of warmer and cooler air associated up and downward vertical velocity perturbations (Figures \ref{fig:conts} and \ref{fig:conts1}).  
This progression is similar to that seen in \cite{GarciaMellado} by \citeauthor{GarciaMellado}.  
We do see bigger clearer regions of upward moving air in the weak stability case as compared to the
the strong stability case. There are pockets of warmer air close to and around the impinging cooler plumes, in line with the concept of wisps being pinched off, or enfolded.

%\clearpage


\section{Local Mixed Layer Heights ($h_{0}^{l}$)}
\label{sec:locmlh}     
\FloatBarrier

%The gradient method of determining \acs{CBL} height proved to be flawed due to the high gradients
%well into the stable layer which exceeded that of the transition
%from \acs{ML} to free atmosphere at points outside an active impinging plume.  

\citeauthor{SullMoengStev} \cite{SullMoengStev} used a centred differencing gradient method for determining local \acs{CBL} height  and observed the distributions
of $z_{i}^{'} = z_{i}-<z_{i}>$.  They observed positive skew in their weak stability cases which
they speculated was due to a small number of high reaching plumes.  We initially tried a similar
method and noticed positive skew, which we found corresponded to local points where the upper variability
exceeded the gradient between the \acs{ML} and the upper atmosphere. So for our purposes the gradient method was rendered unusable\\

The point of maximum vertical gradient in  a tracer profile should correspond to that in a potential 
temperature profile but the profiles can be quite different.  For example a Lidar back-scatter profile
which corresponds directly to tracer concentration profile, has a high value in the \acs{ML} and a low
value in the upper atmosphere, similar to step function.  Usually the variability within these regions 
is a lotsmaller than that over the transition region between the two.  So the transition region can be 
identified using a wavelet of dilation corresponding the the depth of the transition zone.  This is 
clearly shown by \citeauthor{Brooks} in \cite{Brooks} who uses such a wavelet to identify the local \acs{EL}
and then one with narrower dilation to identify the \acs{EL} limits.  The gradient method can also be applied
to a Lidar profile but again this can be noisy. %[references p247 \cite{BrooksFowler2}].  
\citeauthor{SteynBaldHoff}
in \cite{SteynBaldHoff} overcame this by fitting smooth idealized curve to the profile.\\

In line with this last method, we fit a three lines to the local profile representing the \acs{ML}, 
\acs{EL} and upper layer of constant $\gamma$ based on the multilinear regression method outlined by \citeauthor{Vieth} in 
\cite{Vieth}.  This works well with our very simple set up, IE, each local profile consists
of a distinct \acs{ML} and upper region of constant $\gamma$. Locally there is not always
a clear \acs{EL}.  At points where there is neither a sharp gradient nor a clear \acs{EL} 
and some variation in the slope within the \acs{ML}, a test was needed on the slope of 
the second line to see if it was significantly less $\gamma$.  If so, it was considered
to be part of the \acs{ML}.\\

\citeauthor{BrooksFowler2}'s three statistically based entrainment zone limits in 
\cite{BrooksFowler2} showed decreasing trend with increase in \acs{Ri}. Their resulting
scaled \acs{EL} is a lotnarrower than that based on our $\frac{\partial \overline{\theta}}{\partial z}$ profile i.e. .05 - 1.5, and even seems narrower than
what would be the 5th and 95th percentile of our local \acs{ML} heights (see Figure \ref{fig:localhpdf}).  Their lowest
inversion strength seems to be 1 degree over 100 meters (IE .01 per meter) which is
the same as our maximum stability, except of course ours is constant, and their highest is
10 times that. But their lapse rate above is a lot lower ($3k/Km$).  So, this difference cannot simply be explained in terms of inversion strength.\\  %These authors dismiss using the averaged tracer profile based on how the gradients, at the limits based on the local height percentiles on the average profile, relate to the, further scaled by the peak gradient determined from the local peaks in gradient, decrease with decreasing \acs{Ri}. This roughly corresponds to the conceptual difference between the heights of the lowest plumes, and the region where air is mostly (for example on average 95 percent) \acs{ML} air.  It serves as a cautionary note for our study.\\

We see that the local profiles are very different to the average profile and that local
profiles differ from each other (Figures \ref{fig:rssfitshigh} and \ref{fig:rssfitslow}).  The \acs{EL} is an inherently average phenomena i.e.
the range in space or, the range in time, over which the plume heights vary.  So
it is possible to see a local \acs{EL}.  For example in Figure \ref{fig:rssfitslow} (a)
we see a region above the \acs{ML} which is clearly not part of the stable
air above.  Here, we can speculate that a plume previously had reached that point
and some entrainment of warmer air from above had occurred.\\
 
Overall \citeauthor{SullMoengStev} 
\cite{SullMoengStev} show decreased variation in the local heights, with increased \acs{Ri} as we do.  Based on the histograms of our local \acs{ML} heights in Figure \ref{fig:localhhist} we see the range or spread
increases with increased $\overline{w^{'}\theta^{'}}_{s}$ and decreases with increased
$\gamma$.  When scaled by $h$ in Figure \ref{fig:localhpdf} the spread seems only influenced by $\gamma$.  So once
again there is a cancellation of the effects of $\overline{w^{'}\theta^{'}}_{s}$ once
$h$ is introduced.

%\clearpage

\section{Flux Quadrants}
\label{sec:fluxquadrants}     
\FloatBarrier

The shape of the average potential temperature profile evolves according to the temperature flux
profile. In particular warming in the entrainment layer (\acs{EL}), and upper mixed layer (\acs{ML})
is related to the flux of warmer air up or down to that region.  Lower in the \acs{ML} warming is 
from the thermals or plumes originating at the surface.  These plumes become cooler than the horizontal
average in the \acs{EL} where upper stability above the inversion interface causes them to turn downward.
Here there are accompanying downward moving pockets of warm air associated with the upward moving
plumes.  All of this was seen in the visual aids presented by \citeauthor{SullMoengStev} in \cite{SullMoengStev}.\\

In \cite{MahrtPaum} \citeauthor{MahrtPaum} examined the joint distributions of $w^{'}$ and $\theta^{'}$ 
from measurements taken of mixed layers developed in the flow of cold air masses over a warm current.
Their two dimensional representations clearly show the four quadrants: upward warm, upward cool, downward cool
and downward warm.\\

\citeauthor{Sorbjan} in \cite{Sorbjan} asserted and demonstrated that the moments involving $\theta^{'}$
particularly in the upper \acs{ML} and \acs{EL} are strongly influenced by the upper lapse rate $\gamma$.
Whereas moments of $w^{'}$ were less so.  These effects were seen when the corresponding vertical profiles
were scaled by the convective scales ($\theta^{*}$ and $w^{*}$).\\

Bearing the above three studies in mind we separate the $w^{'}\theta^{'}$ into the four quadrants and plot
the average vertical scaled profiles as well as the 2d histograms at $h$ and the \acs{EL} limits.  We
can confirm that the upper extrema of the four individual quadrants exceed that of the average
and are higher i.e. close to $h$ (Figure \ref{fig:fluxqadprofs}).  Higher stability results in a more pronounced peak particularly in the
upward cool quadrant profile which corresponds to increased damping and a sharper decrease in velocity.
Since warming in this region is associated with downward movement of air from above, the downward warm
quadrant is important.\\

The 2d histograms at each level show increased spread of both $\theta^{'}$ and $w^{'}$ with increased 
$\overline{w^{'}\theta^{'}}_{s}$ (Figures \ref{fig:fluxquadsh}, \ref{fig:fluxquadsh0}, \ref{fig:fluxquadsh1}).  There is damping of $w^{'}$ with increased $\gamma$.  To isolate the 
effects of increased $\gamma$ we should scale by the convective scales ($\theta^{*}$ and $w^{*}$) .

%\clearpage

\section{$h$ and  $\Delta h$ based on Average Profiles}
\label{sec:hdeltahavprofs}
\subsection{Reminder of Relevant Definitions}
\FloatBarrier

Our heights are defined based on the average vertical temperature gradient the principle length scale being $h$ the vertical location of the maximum.  Flux based heights are scaled by $h$ to enable comparison with the frameworks of other studies.  

%\clearpage
\subsection{$\frac{\Delta h}{h}$ vs $Ri^{-1}$}
\label{subsec:deltahri}
\FloatBarrier

The \acs{EL} tops as defined by the point at which the temperature gradient resumes $\gamma$ seem to be scaled well by $h$ (Figure \ref{fig:scaledELlims}).  This seems in contrast to the assertion of \citeauthor{GarciaMellado} in \cite{GarciaMellado} about the upper \acs{EL} i.e. that length and buoyancy in this region are not scaled by the the \acs{CBL} convective scales. The \acs{EL} top as defined where the point at which the buoyancy flux decreases to close to zero, when scaled by $h$ is comparable, but has greater scatter (Figure \ref{fig:scaledELlims1}). But in both cases, the top limit is about 1.15 $\times \ h$, and there is a barely perceptible, possible negative trend.\\  

The scaled lower \acs{EL} limits based on the increase in potential temperature gradient from zero, show a clearer increase but don't show the same kind of collapse across runs as the upper limit does (Figure \ref{fig:scaledELlims}).  The scaled lower limit based on the flux profiles however, do collapse well (Figure \ref{fig:scaledELlims1}).  So we could say with some confidence that $\frac{h-z_{f0}}{h} \approx .2$ and this is comparable to \citeauthor{GarciaMellado}'s lower \acs{EL} sublayer.\\
  
So the scaled \acs{EL} as defined by the vertical gradient in the potential temperature profile certainly decreases with respect to time.  The scaled \acs{EL} based on the flux profiles shows slight or no change with respect to time.  This is in line to the findings of \citeauthor{BrooksFowler2} in \cite{BrooksFowler2} even though their definition is slightly different IE $2 \times (z_{f}-z_{f0})$.  But it is in stark contrast to what \citeauthor{FedConzMir04} show in \cite{FedConzMir04} i.e. $\frac{z_{f1}-z_{f0}}{z_{f}}$ decreasing from about .6 to about 0.1.  This could in part be explained by the difference in vertical resolution since according to \citeauthor{SullPat} in \cite{SullPat} the shape of average heat flux profile in the \acs{EL} is sensitive to grid size.\\

\citeauthor{Sorbjan} in \cite{Sorbjan} and \cite{Sorbjan1} demonstrates how the surface and lower \acs{ML} portions of the temperature gradient profile is scaled well by the convective scales but $\gamma$ becomes more important in the \acs{EL}.  From our potential temperature profiles in Figure \ref{fig:pottempprofs2hrs} we see that both $\gamma$ and $\overline{w^{'}\theta^{'}}_{s}$ influence the warming of the \acs{ML}. So this should reflect in particular in the downward flux of warm air from the inversion IE at $h$. That is, increasing $\gamma$ seems to result in an increased slightly positive gradient in the upper \acs{ML} and this should relate to an increase in the downward flux warm air above it, for example at $h$.\\

So, first we define the \acs{EL} lower limit as the point at which the vertical gradient exceeds a positive threshold that's less than $\gamma$ and the same for all runs, at all times. We try three different values and note that there is a seeming decrease in the scaled magnitude with respect to \acs{Ri}, bearing in mind the definition of the \acs{EL} is included in the calculation of $\Delta \theta$ for $Ri$. Grouping according to $\gamma$ is evident.\\

Scaling the vertical potential temperature gradient profiles by $\gamma$ results in collapse to more or less one curve.  The gradient profiles seem to show an increase in the peak gradient as the \acs{EL} seems to narrow. This trend is apparent with respect to time and across runs.  This portion of the profile has been scaled effectively by \citeauthor{Sorbjan1} in \cite{Sorbjan1} using $\frac{\Delta \theta}{\Delta h}$ and \citeauthor{GarciaMellado} using their buoyancy scale $b \approx N^{2} \delta + [\overline{b_{0}}(h) - \overline{b}(h)]$ where $\delta \propto \frac{w^{*}}{N}$ is their length-scale for the upper \acs{EL} sublayer.  Related to $\frac{\Delta \theta}{\Delta h}$ is the entrainment layer stratification parameter $G = \gamma \frac{\Delta h}{\Delta \theta}$ which \citeauthor{FedConzMir04} found to be constant throughout the quasi-steady state regime IE, $\Delta \theta \propto \Delta h$.  This seems to contradict the apparent increase in maximum gradient with decrease in \acs{EL} depth.

\subsection{$\frac{w_{e}}{w^{*}}$ vs $Ri^{-1}$}
\FloatBarrier

In Figure \ref{fig:hvstimeloglog} $h$ shows a $\frac{1}{2}$ power law relationship to time indicating we are in the regime outlined by \citeauthor{FedConzMir04} in \cite{FedConzMir04}.  Self similarity of the scaled heat flux profiles vs scaled height in Figure \ref{fig:scaledfluxprofs15010} indicate a more or less constant entrainment ratio, but also a more or less constant scaled entrainment depth with respect to time. Our Richardson numbers (\acs{Ri}s) increase with respect to time and again grouping according to $\gamma$ is evident (Figure \ref{fig:invristime}).\\


\citeauthor{KatoPhil} successfully related
the scaled entrainment rate of penetrative shear driven turbulence
in their water-tank experiment in \cite{KatoPhil}
to a  dimensionless group formed from the three main characteristics
of the flow : the buoyancy jump across the interface, the turbulent velocity of the \acs{ML} and the depth of the \acs{ML}. IE

\begin{equation}
\frac{u_{e}}{u^{*}} \propto \frac{\rho_{0} u^{*2}}{g \delta \rho D} 
\end{equation}

\citeauthor{DearWill80} related their scaled entrainment of penetrative convection to this dimensionless group, substituting the shear driven velocity scale for the convective one, thus forming the now commonly used Richardson number (\acs{Ri})for the \acs{CBL}.  Their heights were determined from the vertical heat flux profiles.  The heat flux profiles in turn were derived from two successive potential temperature profiles.  The resulting relationship between scaled entrainment rate and \acs{Ri} appears to potentially exhibit both $-1$ and $\frac{-3}{2}$ power laws.\\

\citeauthor{SullMoengStev}'s data in \cite{SullMoengStev} showed some scatter and they speculated that a power law other than $-1$ may have described the relationship at \acs{Ri}s smaller than 14.  They compare the data to this fit:

\begin{equation}
\frac{w_{e}}{w^{*}}=0.2 Ri^{-1}
\end{equation}

\citeauthor{Turner86} in \cite{Turner86} attribute the $-\frac{3}{2}$ power law to mixing that depends on the recoil of impinging eddies.  Whereas \citeauthor{FedConzMir04} in \cite{FedConzMir04} derive it from a best fit approximation of the \acs{Ri} calculated using the buoyancy jump across the \acs{EL} to scaled time (after $tN>100$) and applying the zero order model relationship.\\

\citeauthor{BrooksFowler2}'s plot in \cite{BrooksFowler2} has relatively little scatter and exhibits a linear relationship ($-1$ power law) whereas \citeauthor{GarciaMellado}'s data in \cite{GarciaMellado} seems asymptotic to a linear relationship.\\

Our data based on the temperature jump across the entire \acs{EL} shows a seemingly linear relationship (Figure \ref{fig:scaledweinvri}). 

%\clearpage


\endinput

Any text after an \endinput is ignored.
You could put scraps here or things in progress.

%% The following is a directive for TeXShop to indicate the main file
%%!TEX root = diss.tex

\chapter{Results in Context}
\label{ch:results}
\setlength{\parindent}{0cm}

Much work has been done to develop our understanding of \acs{CBL} entrainment, so this chapter will focus on how my results fit into the discussion established in the literature.  I hone in on five closely related publications for comarison.  \citeauthor{SullMoengStev}'s \citeyear{SullMoengStev} \acs{LES} study was seminal in shedding light on \acs{CBL} entrainment zone structure.  Whereas \citeauthor{BrooksFowler2}'s (\citeyear{BrooksFowler2}) work contains the most recent \acs{LES} results on the topic framed within an up to date review of \acs{CBL} height and \acs{EZ} definitions.  \citeauthor{FedConzMir04} (\citeyear{FedConzMir04}) bridges \acs{LES} and bulk models, while the closely related \acs{DNS} study of \citeauthor{GarciaMellado} (\citeyear{GarciaMellado}) introduces the two-layer \acs{EZ} concept and answers questions regarding the scale resolution required to realistically capture \acs{CBL} growth and \acs{EZ} structure.  \citeauthor{SullPat} (\citeyear{SullPat}) addressed this last point using an \acs{LES}, and was pivotal in guiding the choice of grid-size in the study described in this thesis.  Finally \citeauthor{Sorbjan1} (\citeyear{Sorbjan1}) focused on the effects of upper lapse rate on the turbulence in the upper \acs{CBL} and provided ideas upon which I based Section \ref{sec:q1}.\\

Section \ref{sec:gensetup} draws upon the results of \citeauthor{SullPat} (\citeyear{SullPat}) to address the need for high resolution in the entrainment zone (\acs{EZ}) such that the steep gradients are sufficiently represented. I present and compare those of my results that are pertinent and refer to how \citeauthor{GarciaMellado} (\citeyear{GarciaMellado}) speaks to this point.  Finally, I touch upon how my domain size and initial conditions compare with those of the similar \acs{LES} studies the bearing in mind possible implications of the similarities and differences.\\

In Section \ref{sec:entzonestruc} I describe problems encountered when using the gradient method, as well as discuss the results obtained using my chosen method of determining local \acs{ML} height.  All of this is set in context with the results of \citeauthor{SullMoengStev}'s \citeyear{SullMoengStev} and \citeauthor{BrooksFowler2}'s (\citeyear{BrooksFowler2}).  The influence of $\gamma$ on the turbulent fluctuations of vertical velocity and potential temperature is discussed and compared with the results of \citeauthor{Sorbjan1} (\citeyear{Sorbjan1}) before addressing the dependence of the downward moving positive potential temperature fluctuations at $h$ on this parameter.  An explaination of the potential temperature fluctuation scale $\delta h \gamma$ follows.\\ 

A primary goal of this thesis was to test the average $\theta$ profile as a basis for defining the \acs{EZ} boundaries.  Before comparing the results using heights thus defined in Section \ref{sec:ezbound}, I base all heights on the vertical heat flux ($\overline{w^{'}\theta^{'}}$) profile to enable direct comparison with the results of \citeauthor{BrooksFowler2} (\citeyear{BrooksFowler2}) and \citeauthor{FedConzMir04} (\citeyear{FedConzMir04}).  I discuss similarities, differences and possible reasons for the latter. I then compare results based on the potential temperature profile focusing on the exponent $b$ in Equation \ref{eq:dhvsri} and how it varies depending on $\acs{Ri}$.\\


Section \ref{sec:erparam} contains an analogous comparison to that described above.  Heights are defined, first based on the vertical heat flux profile for direct comparison with the results of \citeauthor{FedConzMir04} (\citeyear{FedConzMir04}) and \citeauthor{GarciaMellado} (\citeyear{GarciaMellado}) and then based on the average $\theta$ profile.  Each of these two comparisons is further broken into two, in order to address the effect of, definining the $\theta$ jump accross the \acs{EZ} as in Figure \ref{fig:1storder}, vs at $h$ as in Figure \ref{fig:0order}.  In all, there are four plots of Equation \ref{eq:ervsri} for the purpose of observing how the exponent $a$ varies depending on $\theta$ jump definition, as well as with $\acs{Ri}$.  This variation is discussed in the context of the results of the other comparable studies.\\          
%need chapter map here

\section{Comparison of general Set-up}
\label{sec:gensetup}
\FloatBarrier

\subsection{Significance of Grid-size}

\citeauthor{SullPat} (\citeyear{SullPat}) found that the shapes of the average potential temperature ($\overline{\theta}$) and heat flux ($\overline{w^{'}\theta^{'}}$) profiles, as well as the measured \acs{CBL} height vary depending on grid size.  The resolution at which convergence begins is listed in Table \ref{table:gridcomp}.  At lower resolution the $\overline{\theta}$ and $\overline{w^{'}\theta^{'}}$ profiles are such that the entrainment zone (\acs{EZ}) is a larger portion of the \acs{CBL} and measured \acs{CBL} height is higher.  Overall they concluded that vertical resolution was critical.  This compliments the conclusion \citeauthor{BrooksFowler2} (\citeyear{BrooksFowler2}) reached when discussing their resolution test.  That is, to capture the steep vertical gradients in the \acs{EZ} requires high vertical resolution. \\

\begin{table}[htbp]
\caption[]{Grid spacing around the \acs{EL} used in comparable \acs{LES} studies. Those used for resolution tests are not listed here.  For \citeauthor{SullPat}'s \citeyear{SullPat} resolution study I list the grid sizes at which profiles within the \acs{EL} and \acs{CBL} height evolution began to converge.}

    \begin{center}
%\centerline{
    \begin{tabular}{ p{5cm} p{3cm} p{3cm}}
    %\hline
Publication & $\Delta x$, $\Delta y$, $\Delta z$ & Horizontal \\ \hline
Publication & in the \acs{EZ} (m)& Domain (km$^{2}$) \\ \hline
      \citeauthor{SullMoengStev} (\citeyear{SullMoengStev}) & 33, 33, 10 & 5 x 5 \\ %\hline 
      \citeauthor{FedConzMir04} (\citeyear{FedConzMir04}) & 100, 100, 20 & 5 x 5 \\ [.3cm] %\hline
      \citeauthor{BrooksFowler2} (\citeyear{BrooksFowler2}) & 50, 50, 12 & 5 x 5 \\%\hline
      \citeauthor{SullPat} (\citeyear{SullPat}) &  20, 20, 8 & 5 x 5\\ %\hline
      This study & 25, 25, 5 &  3.4 x 4.8\\ \hline       
    \end{tabular}
%}
\label{table:gridcomp}   
\end{center}    
\end{table}


As \citeauthor{Turner86} discusses in his \citeyear{Turner86} review of turbulent entrainment, smaller scale processes such as those at the molecular level are relatively unimportant.  Large scale engulfment and trapping between thermals dominates.  If the ergodic assumption holds and potential temperature variance ($\overline{\theta^{'2}}$) is calculated based on the difference at a point from the horizontal average, it is a measure of horizontal variance at a point in time.  \citeauthor{SullPat} (\citeyear{SullPat}) found that the vertical distance over which $\overline{\theta^{'2}}$ varied significantly, more or less converged at the resolution shown in Table \ref{table:gridcomp}.  But the maximum $\overline{\theta^{'2}}$ continued to increase up to their finest grid spacing ($\Delta x=5$, $Delta y = 5$, $Delta z = 2$).\\

The question as to whether mixing and gradients within the \acs{EZ} are adequately resolved motivates \acs{DNS}  studies such as that of \citeauthor{GarciaMellado} (\citeyear{GarciaMellado}). These authors found the entrainment ratio ($\frac{\overline{w^{'}\theta^{'}}_{z_{f}}}{\overline{w^{'}\theta^{'}}_{s}}$) to be about 0.1 which is lower than observed by \citeauthor{FedConzMir04} (\citeyear{FedConzMir04}), but close to what was seen here in Figures \ref{fig:fluxprofs2hrs} and \ref{fig:tempgradfluxprofs1005}.  Based on their $\overline{w^{'}\theta^{'}}$ profiles the depth of the region of negative flux is comparable to what is shown in Figure \ref{fig:scaledELlims}.  Furthermore, these authors concluded that the production and destruction rates of turbulence kinetic energy (\acs{TKE}), as well as the entrainment ratio used to calculate the entrainment rate, were effectively independent of molecular scale processes.\\  
  
The \acs{FFT} energy spectra of the turbulent velocities at the top of the \acs{ML} show a substantial resolved inertial subrange giving confidence in the choice of horizontal grid size used. In the \acs{EZ} where turbulence is intermittent, the dominant energy containing structures are smaller, and decay to the smallest resolved turbulent structures is steeper. This confirms the assertion of \citeauthor{GarciaMellado} (\citeyear{GarciaMellado}) that the \acs{EZ} is separated into two sub-layers in terms of turbulence scales.\\

\subsection{Horizontal Domain}

The horizontal domain in this study is relatively small (see Table \ref{table:gridcomp}). However, visualizations of horizontal and vertical slices clearly showed multiple resolved thermals.  Their diameters increased with \acs{CBL} height, but remained less than or on the order of 100 meters.  \citeauthor{SullMoengStev} (\citeyear{SullMoengStev}) carried out one run on a smaller domain with higher resolution, noticed it resulted in lower \acs{CBL} height and concluded this was due to restricted horizontal thermal size. However, given the results of \citeauthor{SullPat} (\citeyear{SullPat}) it could have been an effect of grid-size.\\   

When defining heights based on average profiles \citeauthor{SullMoengStev} (\citeyear{SullMoengStev}) produced jagged, oscillating time-series and \citeauthor{BrooksFowler2} (\citeyear{BrooksFowler2}) encountered significant scatter in plots of Equation \ref{eq:ervsri}.  But the heights based on average profiles here, using an ensemble of cases, varied smoothly in time.  This could be attributed to a smoother profile based on a greater number horizontal points (10*128*192).\\

\subsection{Initial Conditions}

The principle parameter describing the balance of forces in dry, idealized \acs{CBL} entrainment is the Richardson number ($\acs{Ri}$) and its magnitude depends on the way in which the $\theta$ jump is defined.  Varying the $\theta$ jump definition causes identical conditions to be described by different $\acs{Ri}$ values.  The $\acs{Ri}$ range in this study was dependent on variation in $\gamma$ (see Figure \ref{fig:invristime}).  \citeauthor{BrooksFowler2} (\citeyear{BrooksFowler2}) and \citeauthor{SullMoengStev} (\citeyear{SullMoengStev}) imposed a $\theta$ jump of varying strength topped by a constant $\gamma$.  Whereas \citeauthor{FedConzMir04} (\citeyear{FedConzMir04}) initialized with a layer of uniform $\theta$.  They varied $\gamma$ and kept $\overline{w^{'}\theta^{'}}_{s}$ constant for each run.  Their initial conditions, definitions of the $\theta$ jump and $\acs{Ri}$ range are directly comparable to those of this study, whereas those of \citeauthor{BrooksFowler2} (\citeyear{BrooksFowler2}) and \citeauthor{SullMoengStev} (\citeyear{SullMoengStev}) are quite different.\\    

\begin{table}[htbp]
\caption[]{Initial conditions used in comparable \acs{LES} studies.}

    \begin{center}
%\centerline{
    \begin{tabular}{ p{4cm} p{1.4cm} p{1.4cm} p{1.7cm} p{1.8cm}}
    %\hline
Publication & $\overline{w^{'}\theta^{'}}_{s}$& $\gamma$& Initial $\theta$ & $\acs{Ri}$ \\ 
& $Wm^{-2}$ & $Kkm^{-1}$ & Jump K & range \\ \hline
      \citeauthor{SullMoengStev} (\citeyear{SullMoengStev}) & 20 - 450& 3  &.436 - 5.17 & 1 - 100\\ %\hline 
      \citeauthor{FedConzMir04} (\citeyear{FedConzMir04}) & 300 & 1 - 10 & NA & 10 - 40\\ %[.3cm] %\hline
      \citeauthor{BrooksFowler2} (\citeyear{BrooksFowler2}) &  10 -100 &  3& 1 - 10 &10 - 100 \\ %\hline
      This study & 60 - 150 & 2.5 - 10& NA & 10 - 30\\ \hline 
      
    \end{tabular}
%}
\label{table:initconditcomp}   
\end{center}    
\end{table}

%\clearpage

\section{Entrainment Zone Structure}
\label{sec:entzonestruc}
%\subsection{The Gradient Method is problematic}
\subsection{Local \acs{ML} Heights}

\citeauthor{SullMoengStev} (\citeyear{SullMoengStev}) determined local \acs{CBL} height by locating the point of maximum $\theta$ gradient.  Analysis of the resulting distributions showed dependence of standard deviation and skewness on Richardson number ($\acs{Ri}$).  The normalized standard deviation decreased with increased \acs{Ri} whereas skewness was almost bimodal; being negative at high $\acs{Ri}$ and positive and low \acs{Ri}.  Initially in this study, I applied a similar method and found local \acs{CBL} height distributions with lower $\acs{Ri}$ to have positive skew.  Upon exhaustive inspection of local vertical $\theta$  profiles such as those in Figure \ref{fig:rssfitslow}, it became evident that at certain horizontal points high gradients well into the free atmosphere exceeded those closer to the location of the \acs{CBL} height reasonably identified by eye.\\

\subsection{Local \acs{ML} Height Distributions}

Locating the local \acs{ML} height ($h^{l}_{0}$) using the multi-linear regression method described in Chapter 2 proved more reliable.  The resulting distributions, normalized by \acs{CBL} height ($h$) in Figure \ref{fig:localhpdf},  showed a decrease in the lowest $\frac{h^{l}_{0}}{h}$ resulting in an apparent increased negative skew with decreasing stability (decreasing $\acs{Ri}$). This, combined with a widening of the distribution agrees, with the findings of \citeauthor{SullMoengStev} (\citeyear{SullMoengStev}) and supports the results based on the average profiles in Section \ref{sec:deltahri}.  The approximate scaled \acs{EZ} based on the $\frac{h^{l}_{0}}{h}$ distributions is about 0.2 - 0.4 whereas that based on distributions of local maximum tracer gradients by  \citeauthor{BrooksFowler2} (\citeyear{BrooksFowler2}) was smaller (.05 - .2).  However, the local maximum gradient of the tracer profile would likely be within the \acs{EZ} at points outside an actively impinging plume and so higher than $h^{l}_{0}$ defined here. \\  

%Potential temperature and vertical velocity fluctuations ($\theta^{'}$ and $w^{'}$) at several vertical levels around the \acs{EL} were plotted as 2 dimensional histograms.  Increased $\overline{w^{'}\theta^{'}}_{s}$ causes an increase in positive temperature fluctuations and verical velocity as thermals become more vigorous, causing a higher $h$ so and a deeper \acs{EZ} over which relatively warmer air is pulled down.  The convective velocity scales ($\theta^{*}$ and $w^{*}$) were appplied to isolate the effects of $\gamma$, although it is accounted for indirectly via $h$ (see Section \ref{subsec:scales}). As shown in \citeauthor{Sorbjan}'s (\citeyear{Sorbjan}) $\theta^{'}$ is influenced  by $\gamma$.  For example at $h$ there is an apparant increase in the spread, as well as a shift thowards the positive.  So downward moving positive potential temperature fluctuations representing air from the \acs{FA} are more positive and negative fluctuations representing thermals are less negative, when scaled.  The former can easily be attributed to relatively warmer air over a shorter vertical distance (\acs{EZ}) being brought down but an explaination for the latter remains elusive.  I conclude that with incresed $\gamma$ there is a positive deviation from the convective temperature scale.  Also, there is an apparant damping of the scaled velocities associated with positive temperature fluctuations.  So the $w^{*}$ scales these less effectively with increased $\gamma$.  \\

\subsection{Local vertical Velocity and Potential Temperature Fluctuations}

As expected, with increased $\overline{w^{'}\theta^{'}}_{s}$ the variance and magnitude of the vertical velocity fluctuations within and at the limits of the \acs{EZ} increase.  Greater turbulent velocity causes a higher \acs{CBL} and a deeper \acs{EZ} over which: relatively warmer air from higher up is brought down; and relatively cooler air from below is brought up.  So the magnitude of the potential temperature fluctuations ($\theta^{'}$) and the width of their distribution increases. All of this agrees with the findings of \citeauthor{Sorbjan} (\citeyear{Sorbjan}), but the portion of the scaled $w^{'}$ ($\frac{w^{'}}{w^{*}}$) distribution where scaled $\theta^{'}$ ($\frac{\theta^{'}}{\theta^{*}}$) is positive, in Figure \ref{fig:scaled_fluxquadsh1}, appears to narrow as $\gamma$ increases. This seems to contradict his assertion that $w^{'}$ is independent of this parameter while the effectiveness of $w^{*}$ as a scale for $w^{'-}$ where $\theta^{'}>0$ in Figure \ref{fig:downwarm} supports it.\\

\subsection{Downward moving warm Air at $h$}

 Although the motion of the thermals dominates within the \acs{EZ}, the $\overline{w^{'-}\theta^{'-}}$, $\overline{w^{'+}\theta^{'-}}$ and $\overline{w^{'-}\theta^{'+}}$ quadrants do approximately cancel leaving $\overline{w^{'-}\theta^{'+}}$ as the net dynamic, as \citeauthor{SullMoengStev} (\citeyear{SullMoengStev}) concluded. The downward moving warm quadrant at $h$ ($\overline{w^{'-}\theta^{'+}}_{h}$) represents warmer free atmosphere (\acs{FA}) air that is being entrained.  So its magnitude, at a certain point in time, is an indication of how much the region below will be warmed due to entrainment at a successive time.  The increase of $\overline{w^{'-}\theta^{'+}}_{h}$ in time is primarily due to the increased average positive potential temperature fluctuation at $h$ ($\overline{\theta^{'+}}_{h}$) which is effectively scaled by the temperature scale $(h_{1}-h)\gamma = \delta h \gamma$ (see the Figures of Section \ref{subsec:downwarm}).  A similar scale was introduced by \citeauthor{GarciaMellado} (\citeyear{GarciaMellado}) to further their line of reasoning that the buoyancy in the upper \acs{EZ} is determined by $\gamma$. Figure \ref{fig:deltagamma} illustrates a broad qualitative explanation.  At $h$ much of the air is at the background (or initial) potential temperature $\overline{\theta}_{0}(h)$. 

\begin{figure}[htbp]
    \centering
    %plot_height.py[master 1573b9d] h vs time plot
    \includegraphics[scale=0.4]{/newtera/tera/phil/nchaparr/tera2_cp/nchaparr/ubcdiss/pngs/deltagamma}
    \caption[Illustration of \acs{EZ} Potential Temperature Scale based on $\gamma$]{Illustration of the potential temperature scale $(h_{1}-h)\gamma = \delta h \gamma$: The curves represent a vertical cross-section of thermal tops.  Between them is stable air at the initial lapse rate $\gamma$. $h_{1}$ and $h$ correspond to the highest and average thermal height respectively and $h_{0}$ is the top of the well mixed region (\acs{ML}).  The horizontally uniform, initial potential temperature is $\theta_{0} = \overline{\theta}_{0}$. A thermal will initiate the downward movement of air from $h_{1}$ to $h$, and the difference between its potential temperature and that of the background stable air at $h$ is $(h_{1}-h)\gamma = \delta h \gamma$.}
    \label{fig:deltagamma}   % label should change
\end{figure}
%diagram with plumes and as and $\gamma$

Some air at potential temperature $\theta = \overline{\theta}_{0}(h_{1})$ is brought down from the upper \acs{EZ} limit ($h_{1}$) resulting in positive potential temperature fluctuations ($\theta^{'+}$) at $h$.\\

\citeauthor{GarciaMellado} (\citeyear{GarciaMellado}) suggest that the buoyancy in the lower portion of the \acs{EZ}, i.e. from a point just below $h$ down, is more strongly influenced by the vigorous turbulence of the \acs{ML} than by $\gamma$.  So mixing reduces the difference between, the potential temperature at the top of the \acs{ML}, and that at or just below $h$.  However the observation in Section \ref{subsec:ellimscaledprof}, that the magnitude of the average vertical potential temperature gradient ($\frac{\partial \overline{\theta}}{\partial z}$) in the upper \acs{ML} increases with increasing $\gamma$, indicates that the influence of this parameter extends further down.  On a related note, the magnitude of the minimum heat flux ($\overline{w^{'}\theta^{'}}_{z_{f}}$) is seen to increase with increased $\gamma$, here and in both \citeauthor{Sorbjan} (\citeyear{Sorbjan}) and \citeauthor{FedConzMir04} (\citeyear{FedConzMir04}).  It is reasonable to suggest this leads to an increased negative vertical heat flux gradient ($-\frac{\partial \overline{w^{'}\theta^{'}}}{\partial z}$) in the lower \acs{EZ} and so increased warming per Equation \ref{eq:warming1}.

\begin{equation}
\frac{\partial \overline{\theta}}{\partial t} = -\frac{\partial}{\partial z}\overline{w^{'}\theta^{'}} \tag{\ref{eq:warming1}}
\end{equation}

\section{Entrainment Zone Boundaries}
\label{sec:ezbound}
The \acs{EZ} is inhomogeneous, but on average is a region of transition as clearly represented by the $\overline{\theta}$ profile.  It's where relatively cooler thermals overturn or recoil initiating entrainment as represented by the vertical heat flux ($\overline{w^{'}\theta^{'}}$) profile.  The $\overline{\theta}$ profile partially characterizes the thermodynamic state of the \acs{CBL} as well defining its three layer structure.  It is directly comparable to both bulk models and local $\theta$ profiles which in turn are comparable to a sounding, unlike a $\overline{w^{'}\theta^{'}}$ profile which is an inherently average quantity.\\

\subsection{Direct Comparison based on the vertical Heat Flux Profile}

Neither of the two comparable \acs{LES} studies in Table 3.3 define the \acs{EL} based on the $\frac{\partial \overline{\theta}}{\partial z}$ profile.  So to enable direct comparison, heights were based on the heat flux ($\overline{w^{'}\theta^{'}}$) profile as in Figure \ref{fig:hdefs1}.  In this framework \citeauthor{FedConzMir04} (\citeyear{FedConzMir04}) show decreasing scaled \acs{EZ} with increasing $\acs{Ri}$ and conclude an exponent of of $b = -\frac{1}{2}$.  They attribute the decrease in the overall scaled depth to a slight decrease in the scaled upper boundary over time.  However based on their plot in Figure \ref{fig:FedEZRi} the decrease seems more than slight, varying from about 0.5 to 0.2.\\

\begin{figure}[htbp]
    \centering
    %plot_height.py[master 1573b9d] h vs time plot
    \includegraphics[scale=1]{/newtera/tera/phil/nchaparr/tera2_cp/nchaparr/ubcdiss/pngs/FedEZRi}
    \caption[Plot of the relationship between scaled \acs{EZ} depth and Richardson number from \citeauthor{FedConzMir04}'s (\citeyear{FedConzMir04})]{Figure 9 from \citeauthor{FedConzMir04} (\citeyear{FedConzMir04}) representing Equation \ref{eq:dhvsri} using three different Richardson numbers, in log-log coordinates.  Heights are based on the $\overline{w^{'}\theta^{'}}$ profile as in Figure \ref{fig:hdefs1} and their $z_{i}$ is my $z_{f}$. $\frac{\delta z_{i}}{z_{i}}$. $Ri_{\Delta b}$ (circles) and $Ri_{\delta b}$ (crosses) correspond directly to those determined here using $\delta \theta$ and $\Delta \theta$.  Note that their $\Delta$ refers to the smaller jump measured at $z_{f}$, whereas I use it for the larger.  $Ri_{N}$ (triangles) is the Richardson number defined in Equation \ref{eq:gradri}, with $w^{*}$ and $z_{f}$ as the velocity and length scale.}
    \label{fig:FedEZRi}   % label should change
\end{figure}


 \citeauthor{BrooksFowler2} (\citeyear{BrooksFowler2}) found no clear $\acs{Ri}$ dependence of the scaled \acs{EZ} depth defined based on the $\overline{w^{'}\theta^{'}}$ profile.  But their definition hinged solely upon the lower part ($z_{f1} - z_{f}$) which according to \citeauthor{FedConzMir04} (\citeyear{FedConzMir04}) does not vary in time.  Figure \ref{fig:deltahinvri_scaled} of this thesis shows that when I defined the \acs{EZ} based on the $\overline{w^{'}\theta^{'}}$ profile as \citeauthor{FedConzMir04} (\citeyear{FedConzMir04}) did, the scaled \acs{EZ} depth had no clear dependence on $\acs{Ri}$. This is supported by the similarity in time and across runs of the vertical turbulent heat flux profiles when scaled by $(\overline{w^{'}\theta^{'}})_{s}$ in Figures \ref{fig:scaledfluxprofs15010} and \ref{fig:fluxprofs2hrs}.\\

The most obvious possible cause for disagreement with the results of \citeauthor{FedConzMir04} (\citeyear{FedConzMir04}) is the difference in grid size shown in Table \ref{table:gridcomp}.  Inspection of their $\overline{w^{'}\theta^{'}}$ profiles confirms a relatively deeper scaled region of negative flux as compared with those seen here (~.4 vs ~.25). Their surface heat flux $\overline{w^{'}\theta^{'}}_{s}$ was twice the highest used here, but their range of $\acs{Ri}$ is comparable to that of this study.  The latter point although not directly relevant here, serves as confirmation that $\gamma$ is the more influential parameter.\\              

\begin{table}[htbp]
\label{table:elandri}
\caption[\acs{EZ} Definitions used in comparable Studies]{\acs{EZ} Definitions used in comparable Studies}

\begin{center}
%\centerline{
\begin{tabular}{ p{4cm} p{2cm} p{1.5cm} p{3cm}}
    %\hline

Publication & \acs{EZ} Depth & \acs{CBL} height & $\theta$ Jump\\ \hline
\citeauthor{FedConzMir04} (\citeyear{FedConzMir04}) & $z_{f1} - z_{f0}$ & $z_{f}$ &  $\overline{\theta}(z_{f1})-\overline{\theta}(z_{f0})$\\ [.3cm] %\hline
\citeauthor{BrooksFowler2} (\citeyear{BrooksFowler2}) & $2 \times (z_{f} - z_{f0})$ & $z_{f}$ & average of local values\\ \hline

\end{tabular}
\end{center}    
\end{table}

\subsection{General Comparison using the Potential Temperature Profile}

Here, when heights are defined based on the scaled vertical potential temperature gradient profile $\frac{\frac{\partial \overline{\theta}}{\partial z}}{\gamma}$ the curve representing Equation \ref{eq:dhvsri} 

\begin{equation}
\frac{\Delta h}{h} \propto Ri^{b} \tag{\ref{eq:dhvsri}}
\end{equation}

shows an exponent $b$ which increases in magnitude, from about $-\frac{1}{2}$ as predicted and seen by \citeauthor{Boers89} (\citeyear{Boers89}), to about $-1$ as justified in \citeauthor{StullNelEl} (\citeyear{StullNelEl}),  with increasing $\acs{Ri}$ (decreasing $\acs{Ri}^{-1}$).  Overall there is a clear narrowing of the scaled \acs{EZ} depth with increasing $\acs{Ri}$ (decreasing $\acs{Ri}^{-1}$) as supported by the local height distributions in Section \ref{subsec:locmlh}.  Although based on different height definitions, \citeauthor{FedConzMir04} (\citeyear{FedConzMir04}) concluded an exponent $b = -\frac{1}{2}$ and \citeauthor{BrooksFowler2}'s (\citeyear{BrooksFowler2}) plots show curves with an apparent exponent less in magnitude than $-1$, in Figure \ref{fig:BandFEZ}. \\

\begin{figure}[htbp]
    \centering
    %plot_height.py[master 1573b9d] h vs time plot
    \includegraphics[scale=1.3]{/newtera/tera/phil/nchaparr/tera2_cp/nchaparr/ubcdiss/pngs/BandFEZ}
    \caption[Relationship of Scaled \acs{EZ} depth to Richardson number from \citeauthor{BrooksFowler2}'s (\citeyear{BrooksFowler2})]{Panel (a) from Figure 5 in \citeauthor{BrooksFowler2} (\citeyear{BrooksFowler2}) representing Equation \ref{eq:dhvsri}:The normalized \acs{EZ} depth is determined in three ways (i) the upper and lower percentiles from the distribution of local \acs{CBL} height (maximum tracer gradient), normalized by the average of the local heights (pale grey) (ii) the average of local scaled \acs{EZ} depths based on wavelet covariance (dark grey) and (iii) the average of the locally determined \acs{EZ} depths scaled by the average of the locally determined heights (black), based on wavelet covariance.  Their $\theta$ jump is an average of the potential temperature differences across the local \acs{EZ} depths.}
    \label{fig:BandFEZ}   % label should change
\end{figure}

The curves representing each run in Figure \ref{fig:BandFEZ} fan out.  In Figure \ref{fig:scaledeltahinvri} of this thesis, before scaling, the $\frac{\partial \overline{\theta}}{\partial z}$ profile curves separate out, but in the reverse order.  \acs{CBL}s under higher stability, and so higher $\acs{Ri}$, have larger scaled \acs{EZ} depths.  Whereas \citeauthor{BrooksFowler2}'s (\citeyear{BrooksFowler2}) runs with initially lower $\acs{Ri}$ have larger scaled \acs{EZ} depths than those with higher, even where $\acs{Ri}$ values overlap. Nonetheless, that there appears a family of separate but similar curves rather than a single curve hints at an underlying scaling parameter.\\     

Neither study referenced in Table 3.3 addresses the change in exponent with increased $\acs{Ri}$ that I observe in Figure \ref{fig:loglogdeltahinvri}.  It is reasonable to suggest that this represents a change in entrainment mechanism. \citeauthor{SullMoengStev} (\citeyear{SullMoengStev}) observed enfolding and engulfment at lower $\acs{Ri}$.  Whereas at higher $\acs{Ri}$ when motion is more restricted, entrainment seemed to occur via trapping of thinner wisps at the edge of an upward moving thermal.  \citeauthor{Turner86} (\citeyear{Turner86}) also distinguishes between entrainment by convective overturning and recoil. \citeauthor{GarciaMellado} (\citeyear{GarciaMellado}) refer to a change in entrainment rate due to the effects of increased stability on the upper \acs{EZ} sub-layer.  In this study, the narrowing of the \acs{EZ} depends predominantly on the magnitude of the average vertical potential temperature gradient $\frac{\partial \overline{\theta}}{\partial z}$ in the lower \acs{EZ} and upper \acs{ML}.  However, the scaled magnitude of upper limit in Figure \ref{fig:scaledELlims} (a) does appear to decrease slightly in time.  This could correspond to the slowly decreasing upper sub layer of the \acs{EZ} mentioned in both \citeauthor{GarciaMellado} (\citeyear{GarciaMellado}) and \citeauthor{FedConzMir04} (\citeyear{FedConzMir04}).\\

\section{Entrainment Rate Parameterization}
\label{sec:erparam}
$\acs{Ri}$ magnitude determined in this and the comparable studies is primarily influenced by the magnitude of the $\theta$ jump.  Here, I define it in two ways as \citeauthor{FedConzMir04} (\citeyear{FedConzMir04}) did.  I do this based on the $\overline{w^{'}\theta^{'}}$ profile, as in Figure \ref{fig:hdefs1} and Table \ref{table:reldefs} for the purpose of direct comparison and to observe how the change in definition effects Equation \ref{eq:ervsri}.\\

\begin{equation}
\frac{w_{e}}{w^{*}} \propto Ri^{a} \tag{\ref{eq:ervsri}}
\end{equation}

\subsection{Direct Comparison based on the vertical Heat Flux Profile}
The larger jump, i.e. that taken across the \acs{EZ} ($\Delta \theta$) as in Figure \ref{fig:1storder}, yields a larger value of $a$ as \citeauthor{FedConzMir04} (\citeyear{FedConzMir04}) conclude.  \citeauthor{GarciaMellado} (\citeyear{GarciaMellado}) interpret both curves as asymptotic to straight lines ($a=-1$) as the upper \acs{EZ} sub-layer narrows. Based on their plots in Figure \ref{fig:GarcMelERRi}, in the absence of their justification based on the derivation of the entrainment relation, for $\Delta \theta$ I see a curve (grey and blue) with increasing exponent exceeding magnitude $-1$ at higher $\acs{Ri}$.  For their version of $\delta \theta$ I see a curve (grey and red) with exponent less in magnitude than $-1$.\\

\begin{figure}[htbp]
    \centering
    %plot_height.py[master 1573b9d] h vs time plot
    \includegraphics[scale=1]{/newtera/tera/phil/nchaparr/tera2_cp/nchaparr/ubcdiss/pngs/GaMeERRi}
    \caption[Plots of scaled entrainment rate vs Richardson number from \citeauthor{GarciaMellado} (\citeyear{GarciaMellado})]{Figure 11 from \citeauthor{GarciaMellado} (\citeyear{GarciaMellado}) and representing equation \ref{eq:ervsri} based on the two $\theta$ jumps.  The grey and blue curve is based on $\Delta \theta$ and the (grey and) red curve is based on $\overline{\theta}(h) - \overline{\theta}_{0}(h)$ which is slightly different to the $\delta \theta$ defined here and in \citeauthor{FedConzMir04} (\citeyear{FedConzMir04}). The dashed and continuous black lines represent the straight lines to which the curves asymptote according to their analysis. Their heights are comparable to those based on the heat flux ($\overline{w^{'}\theta^{'}}$) profile in Figure \ref{fig:hdefs1}.}
    \label{fig:GarcMelERRi}   % label should change
\end{figure}

\subsection{Extending Comparison to the average Potential Temperature Profile}

There is an analogous distinction between curves representing Equation \ref{eq:ervsri} using $\Delta \theta$ and those using $\delta \theta$, when all heights are based on the $\frac{\frac{\partial \overline{\theta}}{\partial z}}{\gamma}$ profile.  Scatter is least when the $\theta$ jump is defined across the \acs{EZ}.  In Figure \ref{fig:weinvri} $a=-\frac{3}{2}$ fits at higher $\acs{Ri}$ (lower $\acs{Ri}^{-1}$) and $a=-1$ seems to fit at lower $\acs{Ri}$.  Combined with the apparent change in $b$ for Equation \ref{eq:dhvsri} I interpret this as an indication of a change in entrainment regime at increasing $\acs{Ri}$.\\ 

\FloatBarrier


\endinput

Any text after an \endinput is ignored.
You could put scraps here or things in progress.

%\include{impl}
%%% The following is a directive for TeXShop to indicate the main file
%%!TEX root = diss.tex

\chapter{Results in Context}
\label{ch:results}
\setlength{\parindent}{0cm}

Much work has been done to develop our understanding of \acs{CBL} entrainment, so this chapter will focus on how my results fit into the discussion established in the literature.  I hone in on five closely related publications for comarison.  \citeauthor{SullMoengStev}'s \citeyear{SullMoengStev} \acs{LES} study was seminal in shedding light on \acs{CBL} entrainment zone structure.  Whereas \citeauthor{BrooksFowler2}'s (\citeyear{BrooksFowler2}) work contains the most recent \acs{LES} results on the topic framed within an up to date review of \acs{CBL} height and \acs{EZ} definitions.  \citeauthor{FedConzMir04} (\citeyear{FedConzMir04}) bridges \acs{LES} and bulk models, while the closely related \acs{DNS} study of \citeauthor{GarciaMellado} (\citeyear{GarciaMellado}) introduces the two-layer \acs{EZ} concept and answers questions regarding the scale resolution required to realistically capture \acs{CBL} growth and \acs{EZ} structure.  \citeauthor{SullPat} (\citeyear{SullPat}) addressed this last point using an \acs{LES}, and was pivotal in guiding the choice of grid-size in the study described in this thesis.  Finally \citeauthor{Sorbjan1} (\citeyear{Sorbjan1}) focused on the effects of upper lapse rate on the turbulence in the upper \acs{CBL} and provided ideas upon which I based Section \ref{sec:q1}.\\

Section \ref{sec:gensetup} draws upon the results of \citeauthor{SullPat} (\citeyear{SullPat}) to address the need for high resolution in the entrainment zone (\acs{EZ}) such that the steep gradients are sufficiently represented. I present and compare those of my results that are pertinent and refer to how \citeauthor{GarciaMellado} (\citeyear{GarciaMellado}) speaks to this point.  Finally, I touch upon how my domain size and initial conditions compare with those of the similar \acs{LES} studies the bearing in mind possible implications of the similarities and differences.\\

In Section \ref{sec:entzonestruc} I describe problems encountered when using the gradient method, as well as discuss the results obtained using my chosen method of determining local \acs{ML} height.  All of this is set in context with the results of \citeauthor{SullMoengStev}'s \citeyear{SullMoengStev} and \citeauthor{BrooksFowler2}'s (\citeyear{BrooksFowler2}).  The influence of $\gamma$ on the turbulent fluctuations of vertical velocity and potential temperature is discussed and compared with the results of \citeauthor{Sorbjan1} (\citeyear{Sorbjan1}) before addressing the dependence of the downward moving positive potential temperature fluctuations at $h$ on this parameter.  An explaination of the potential temperature fluctuation scale $\delta h \gamma$ follows.\\ 

A primary goal of this thesis was to test the average $\theta$ profile as a basis for defining the \acs{EZ} boundaries.  Before comparing the results using heights thus defined in Section \ref{sec:ezbound}, I base all heights on the vertical heat flux ($\overline{w^{'}\theta^{'}}$) profile to enable direct comparison with the results of \citeauthor{BrooksFowler2} (\citeyear{BrooksFowler2}) and \citeauthor{FedConzMir04} (\citeyear{FedConzMir04}).  I discuss similarities, differences and possible reasons for the latter. I then compare results based on the potential temperature profile focusing on the exponent $b$ in Equation \ref{eq:dhvsri} and how it varies depending on $\acs{Ri}$.\\


Section \ref{sec:erparam} contains an analogous comparison to that described above.  Heights are defined, first based on the vertical heat flux profile for direct comparison with the results of \citeauthor{FedConzMir04} (\citeyear{FedConzMir04}) and \citeauthor{GarciaMellado} (\citeyear{GarciaMellado}) and then based on the average $\theta$ profile.  Each of these two comparisons is further broken into two, in order to address the effect of, definining the $\theta$ jump accross the \acs{EZ} as in Figure \ref{fig:1storder}, vs at $h$ as in Figure \ref{fig:0order}.  In all, there are four plots of Equation \ref{eq:ervsri} for the purpose of observing how the exponent $a$ varies depending on $\theta$ jump definition, as well as with $\acs{Ri}$.  This variation is discussed in the context of the results of the other comparable studies.\\          
%need chapter map here

\section{Comparison of general Set-up}
\label{sec:gensetup}
\FloatBarrier

\subsection{Significance of Grid-size}

\citeauthor{SullPat} (\citeyear{SullPat}) found that the shapes of the average potential temperature ($\overline{\theta}$) and heat flux ($\overline{w^{'}\theta^{'}}$) profiles, as well as the measured \acs{CBL} height vary depending on grid size.  The resolution at which convergence begins is listed in Table \ref{table:gridcomp}.  At lower resolution the $\overline{\theta}$ and $\overline{w^{'}\theta^{'}}$ profiles are such that the entrainment zone (\acs{EZ}) is a larger portion of the \acs{CBL} and measured \acs{CBL} height is higher.  Overall they concluded that vertical resolution was critical.  This compliments the conclusion \citeauthor{BrooksFowler2} (\citeyear{BrooksFowler2}) reached when discussing their resolution test.  That is, to capture the steep vertical gradients in the \acs{EZ} requires high vertical resolution. \\

\begin{table}[htbp]
\caption[]{Grid spacing around the \acs{EL} used in comparable \acs{LES} studies. Those used for resolution tests are not listed here.  For \citeauthor{SullPat}'s \citeyear{SullPat} resolution study I list the grid sizes at which profiles within the \acs{EL} and \acs{CBL} height evolution began to converge.}

    \begin{center}
%\centerline{
    \begin{tabular}{ p{5cm} p{3cm} p{3cm}}
    %\hline
Publication & $\Delta x$, $\Delta y$, $\Delta z$ & Horizontal \\ \hline
Publication & in the \acs{EZ} (m)& Domain (km$^{2}$) \\ \hline
      \citeauthor{SullMoengStev} (\citeyear{SullMoengStev}) & 33, 33, 10 & 5 x 5 \\ %\hline 
      \citeauthor{FedConzMir04} (\citeyear{FedConzMir04}) & 100, 100, 20 & 5 x 5 \\ [.3cm] %\hline
      \citeauthor{BrooksFowler2} (\citeyear{BrooksFowler2}) & 50, 50, 12 & 5 x 5 \\%\hline
      \citeauthor{SullPat} (\citeyear{SullPat}) &  20, 20, 8 & 5 x 5\\ %\hline
      This study & 25, 25, 5 &  3.4 x 4.8\\ \hline       
    \end{tabular}
%}
\label{table:gridcomp}   
\end{center}    
\end{table}


As \citeauthor{Turner86} discusses in his \citeyear{Turner86} review of turbulent entrainment, smaller scale processes such as those at the molecular level are relatively unimportant.  Large scale engulfment and trapping between thermals dominates.  If the ergodic assumption holds and potential temperature variance ($\overline{\theta^{'2}}$) is calculated based on the difference at a point from the horizontal average, it is a measure of horizontal variance at a point in time.  \citeauthor{SullPat} (\citeyear{SullPat}) found that the vertical distance over which $\overline{\theta^{'2}}$ varied significantly, more or less converged at the resolution shown in Table \ref{table:gridcomp}.  But the maximum $\overline{\theta^{'2}}$ continued to increase up to their finest grid spacing ($\Delta x=5$, $Delta y = 5$, $Delta z = 2$).\\

The question as to whether mixing and gradients within the \acs{EZ} are adequately resolved motivates \acs{DNS}  studies such as that of \citeauthor{GarciaMellado} (\citeyear{GarciaMellado}). These authors found the entrainment ratio ($\frac{\overline{w^{'}\theta^{'}}_{z_{f}}}{\overline{w^{'}\theta^{'}}_{s}}$) to be about 0.1 which is lower than observed by \citeauthor{FedConzMir04} (\citeyear{FedConzMir04}), but close to what was seen here in Figures \ref{fig:fluxprofs2hrs} and \ref{fig:tempgradfluxprofs1005}.  Based on their $\overline{w^{'}\theta^{'}}$ profiles the depth of the region of negative flux is comparable to what is shown in Figure \ref{fig:scaledELlims}.  Furthermore, these authors concluded that the production and destruction rates of turbulence kinetic energy (\acs{TKE}), as well as the entrainment ratio used to calculate the entrainment rate, were effectively independent of molecular scale processes.\\  
  
The \acs{FFT} energy spectra of the turbulent velocities at the top of the \acs{ML} show a substantial resolved inertial subrange giving confidence in the choice of horizontal grid size used. In the \acs{EZ} where turbulence is intermittent, the dominant energy containing structures are smaller, and decay to the smallest resolved turbulent structures is steeper. This confirms the assertion of \citeauthor{GarciaMellado} (\citeyear{GarciaMellado}) that the \acs{EZ} is separated into two sub-layers in terms of turbulence scales.\\

\subsection{Horizontal Domain}

The horizontal domain in this study is relatively small (see Table \ref{table:gridcomp}). However, visualizations of horizontal and vertical slices clearly showed multiple resolved thermals.  Their diameters increased with \acs{CBL} height, but remained less than or on the order of 100 meters.  \citeauthor{SullMoengStev} (\citeyear{SullMoengStev}) carried out one run on a smaller domain with higher resolution, noticed it resulted in lower \acs{CBL} height and concluded this was due to restricted horizontal thermal size. However, given the results of \citeauthor{SullPat} (\citeyear{SullPat}) it could have been an effect of grid-size.\\   

When defining heights based on average profiles \citeauthor{SullMoengStev} (\citeyear{SullMoengStev}) produced jagged, oscillating time-series and \citeauthor{BrooksFowler2} (\citeyear{BrooksFowler2}) encountered significant scatter in plots of Equation \ref{eq:ervsri}.  But the heights based on average profiles here, using an ensemble of cases, varied smoothly in time.  This could be attributed to a smoother profile based on a greater number horizontal points (10*128*192).\\

\subsection{Initial Conditions}

The principle parameter describing the balance of forces in dry, idealized \acs{CBL} entrainment is the Richardson number ($\acs{Ri}$) and its magnitude depends on the way in which the $\theta$ jump is defined.  Varying the $\theta$ jump definition causes identical conditions to be described by different $\acs{Ri}$ values.  The $\acs{Ri}$ range in this study was dependent on variation in $\gamma$ (see Figure \ref{fig:invristime}).  \citeauthor{BrooksFowler2} (\citeyear{BrooksFowler2}) and \citeauthor{SullMoengStev} (\citeyear{SullMoengStev}) imposed a $\theta$ jump of varying strength topped by a constant $\gamma$.  Whereas \citeauthor{FedConzMir04} (\citeyear{FedConzMir04}) initialized with a layer of uniform $\theta$.  They varied $\gamma$ and kept $\overline{w^{'}\theta^{'}}_{s}$ constant for each run.  Their initial conditions, definitions of the $\theta$ jump and $\acs{Ri}$ range are directly comparable to those of this study, whereas those of \citeauthor{BrooksFowler2} (\citeyear{BrooksFowler2}) and \citeauthor{SullMoengStev} (\citeyear{SullMoengStev}) are quite different.\\    

\begin{table}[htbp]
\caption[]{Initial conditions used in comparable \acs{LES} studies.}

    \begin{center}
%\centerline{
    \begin{tabular}{ p{4cm} p{1.4cm} p{1.4cm} p{1.7cm} p{1.8cm}}
    %\hline
Publication & $\overline{w^{'}\theta^{'}}_{s}$& $\gamma$& Initial $\theta$ & $\acs{Ri}$ \\ 
& $Wm^{-2}$ & $Kkm^{-1}$ & Jump K & range \\ \hline
      \citeauthor{SullMoengStev} (\citeyear{SullMoengStev}) & 20 - 450& 3  &.436 - 5.17 & 1 - 100\\ %\hline 
      \citeauthor{FedConzMir04} (\citeyear{FedConzMir04}) & 300 & 1 - 10 & NA & 10 - 40\\ %[.3cm] %\hline
      \citeauthor{BrooksFowler2} (\citeyear{BrooksFowler2}) &  10 -100 &  3& 1 - 10 &10 - 100 \\ %\hline
      This study & 60 - 150 & 2.5 - 10& NA & 10 - 30\\ \hline 
      
    \end{tabular}
%}
\label{table:initconditcomp}   
\end{center}    
\end{table}

%\clearpage

\section{Entrainment Zone Structure}
\label{sec:entzonestruc}
%\subsection{The Gradient Method is problematic}
\subsection{Local \acs{ML} Heights}

\citeauthor{SullMoengStev} (\citeyear{SullMoengStev}) determined local \acs{CBL} height by locating the point of maximum $\theta$ gradient.  Analysis of the resulting distributions showed dependence of standard deviation and skewness on Richardson number ($\acs{Ri}$).  The normalized standard deviation decreased with increased \acs{Ri} whereas skewness was almost bimodal; being negative at high $\acs{Ri}$ and positive and low \acs{Ri}.  Initially in this study, I applied a similar method and found local \acs{CBL} height distributions with lower $\acs{Ri}$ to have positive skew.  Upon exhaustive inspection of local vertical $\theta$  profiles such as those in Figure \ref{fig:rssfitslow}, it became evident that at certain horizontal points high gradients well into the free atmosphere exceeded those closer to the location of the \acs{CBL} height reasonably identified by eye.\\

\subsection{Local \acs{ML} Height Distributions}

Locating the local \acs{ML} height ($h^{l}_{0}$) using the multi-linear regression method described in Chapter 2 proved more reliable.  The resulting distributions, normalized by \acs{CBL} height ($h$) in Figure \ref{fig:localhpdf},  showed a decrease in the lowest $\frac{h^{l}_{0}}{h}$ resulting in an apparent increased negative skew with decreasing stability (decreasing $\acs{Ri}$). This, combined with a widening of the distribution agrees, with the findings of \citeauthor{SullMoengStev} (\citeyear{SullMoengStev}) and supports the results based on the average profiles in Section \ref{sec:deltahri}.  The approximate scaled \acs{EZ} based on the $\frac{h^{l}_{0}}{h}$ distributions is about 0.2 - 0.4 whereas that based on distributions of local maximum tracer gradients by  \citeauthor{BrooksFowler2} (\citeyear{BrooksFowler2}) was smaller (.05 - .2).  However, the local maximum gradient of the tracer profile would likely be within the \acs{EZ} at points outside an actively impinging plume and so higher than $h^{l}_{0}$ defined here. \\  

%Potential temperature and vertical velocity fluctuations ($\theta^{'}$ and $w^{'}$) at several vertical levels around the \acs{EL} were plotted as 2 dimensional histograms.  Increased $\overline{w^{'}\theta^{'}}_{s}$ causes an increase in positive temperature fluctuations and verical velocity as thermals become more vigorous, causing a higher $h$ so and a deeper \acs{EZ} over which relatively warmer air is pulled down.  The convective velocity scales ($\theta^{*}$ and $w^{*}$) were appplied to isolate the effects of $\gamma$, although it is accounted for indirectly via $h$ (see Section \ref{subsec:scales}). As shown in \citeauthor{Sorbjan}'s (\citeyear{Sorbjan}) $\theta^{'}$ is influenced  by $\gamma$.  For example at $h$ there is an apparant increase in the spread, as well as a shift thowards the positive.  So downward moving positive potential temperature fluctuations representing air from the \acs{FA} are more positive and negative fluctuations representing thermals are less negative, when scaled.  The former can easily be attributed to relatively warmer air over a shorter vertical distance (\acs{EZ}) being brought down but an explaination for the latter remains elusive.  I conclude that with incresed $\gamma$ there is a positive deviation from the convective temperature scale.  Also, there is an apparant damping of the scaled velocities associated with positive temperature fluctuations.  So the $w^{*}$ scales these less effectively with increased $\gamma$.  \\

\subsection{Local vertical Velocity and Potential Temperature Fluctuations}

As expected, with increased $\overline{w^{'}\theta^{'}}_{s}$ the variance and magnitude of the vertical velocity fluctuations within and at the limits of the \acs{EZ} increase.  Greater turbulent velocity causes a higher \acs{CBL} and a deeper \acs{EZ} over which: relatively warmer air from higher up is brought down; and relatively cooler air from below is brought up.  So the magnitude of the potential temperature fluctuations ($\theta^{'}$) and the width of their distribution increases. All of this agrees with the findings of \citeauthor{Sorbjan} (\citeyear{Sorbjan}), but the portion of the scaled $w^{'}$ ($\frac{w^{'}}{w^{*}}$) distribution where scaled $\theta^{'}$ ($\frac{\theta^{'}}{\theta^{*}}$) is positive, in Figure \ref{fig:scaled_fluxquadsh1}, appears to narrow as $\gamma$ increases. This seems to contradict his assertion that $w^{'}$ is independent of this parameter while the effectiveness of $w^{*}$ as a scale for $w^{'-}$ where $\theta^{'}>0$ in Figure \ref{fig:downwarm} supports it.\\

\subsection{Downward moving warm Air at $h$}

 Although the motion of the thermals dominates within the \acs{EZ}, the $\overline{w^{'-}\theta^{'-}}$, $\overline{w^{'+}\theta^{'-}}$ and $\overline{w^{'-}\theta^{'+}}$ quadrants do approximately cancel leaving $\overline{w^{'-}\theta^{'+}}$ as the net dynamic, as \citeauthor{SullMoengStev} (\citeyear{SullMoengStev}) concluded. The downward moving warm quadrant at $h$ ($\overline{w^{'-}\theta^{'+}}_{h}$) represents warmer free atmosphere (\acs{FA}) air that is being entrained.  So its magnitude, at a certain point in time, is an indication of how much the region below will be warmed due to entrainment at a successive time.  The increase of $\overline{w^{'-}\theta^{'+}}_{h}$ in time is primarily due to the increased average positive potential temperature fluctuation at $h$ ($\overline{\theta^{'+}}_{h}$) which is effectively scaled by the temperature scale $(h_{1}-h)\gamma = \delta h \gamma$ (see the Figures of Section \ref{subsec:downwarm}).  A similar scale was introduced by \citeauthor{GarciaMellado} (\citeyear{GarciaMellado}) to further their line of reasoning that the buoyancy in the upper \acs{EZ} is determined by $\gamma$. Figure \ref{fig:deltagamma} illustrates a broad qualitative explanation.  At $h$ much of the air is at the background (or initial) potential temperature $\overline{\theta}_{0}(h)$. 

\begin{figure}[htbp]
    \centering
    %plot_height.py[master 1573b9d] h vs time plot
    \includegraphics[scale=0.4]{/newtera/tera/phil/nchaparr/tera2_cp/nchaparr/ubcdiss/pngs/deltagamma}
    \caption[Illustration of \acs{EZ} Potential Temperature Scale based on $\gamma$]{Illustration of the potential temperature scale $(h_{1}-h)\gamma = \delta h \gamma$: The curves represent a vertical cross-section of thermal tops.  Between them is stable air at the initial lapse rate $\gamma$. $h_{1}$ and $h$ correspond to the highest and average thermal height respectively and $h_{0}$ is the top of the well mixed region (\acs{ML}).  The horizontally uniform, initial potential temperature is $\theta_{0} = \overline{\theta}_{0}$. A thermal will initiate the downward movement of air from $h_{1}$ to $h$, and the difference between its potential temperature and that of the background stable air at $h$ is $(h_{1}-h)\gamma = \delta h \gamma$.}
    \label{fig:deltagamma}   % label should change
\end{figure}
%diagram with plumes and as and $\gamma$

Some air at potential temperature $\theta = \overline{\theta}_{0}(h_{1})$ is brought down from the upper \acs{EZ} limit ($h_{1}$) resulting in positive potential temperature fluctuations ($\theta^{'+}$) at $h$.\\

\citeauthor{GarciaMellado} (\citeyear{GarciaMellado}) suggest that the buoyancy in the lower portion of the \acs{EZ}, i.e. from a point just below $h$ down, is more strongly influenced by the vigorous turbulence of the \acs{ML} than by $\gamma$.  So mixing reduces the difference between, the potential temperature at the top of the \acs{ML}, and that at or just below $h$.  However the observation in Section \ref{subsec:ellimscaledprof}, that the magnitude of the average vertical potential temperature gradient ($\frac{\partial \overline{\theta}}{\partial z}$) in the upper \acs{ML} increases with increasing $\gamma$, indicates that the influence of this parameter extends further down.  On a related note, the magnitude of the minimum heat flux ($\overline{w^{'}\theta^{'}}_{z_{f}}$) is seen to increase with increased $\gamma$, here and in both \citeauthor{Sorbjan} (\citeyear{Sorbjan}) and \citeauthor{FedConzMir04} (\citeyear{FedConzMir04}).  It is reasonable to suggest this leads to an increased negative vertical heat flux gradient ($-\frac{\partial \overline{w^{'}\theta^{'}}}{\partial z}$) in the lower \acs{EZ} and so increased warming per Equation \ref{eq:warming1}.

\begin{equation}
\frac{\partial \overline{\theta}}{\partial t} = -\frac{\partial}{\partial z}\overline{w^{'}\theta^{'}} \tag{\ref{eq:warming1}}
\end{equation}

\section{Entrainment Zone Boundaries}
\label{sec:ezbound}
The \acs{EZ} is inhomogeneous, but on average is a region of transition as clearly represented by the $\overline{\theta}$ profile.  It's where relatively cooler thermals overturn or recoil initiating entrainment as represented by the vertical heat flux ($\overline{w^{'}\theta^{'}}$) profile.  The $\overline{\theta}$ profile partially characterizes the thermodynamic state of the \acs{CBL} as well defining its three layer structure.  It is directly comparable to both bulk models and local $\theta$ profiles which in turn are comparable to a sounding, unlike a $\overline{w^{'}\theta^{'}}$ profile which is an inherently average quantity.\\

\subsection{Direct Comparison based on the vertical Heat Flux Profile}

Neither of the two comparable \acs{LES} studies in Table 3.3 define the \acs{EL} based on the $\frac{\partial \overline{\theta}}{\partial z}$ profile.  So to enable direct comparison, heights were based on the heat flux ($\overline{w^{'}\theta^{'}}$) profile as in Figure \ref{fig:hdefs1}.  In this framework \citeauthor{FedConzMir04} (\citeyear{FedConzMir04}) show decreasing scaled \acs{EZ} with increasing $\acs{Ri}$ and conclude an exponent of of $b = -\frac{1}{2}$.  They attribute the decrease in the overall scaled depth to a slight decrease in the scaled upper boundary over time.  However based on their plot in Figure \ref{fig:FedEZRi} the decrease seems more than slight, varying from about 0.5 to 0.2.\\

\begin{figure}[htbp]
    \centering
    %plot_height.py[master 1573b9d] h vs time plot
    \includegraphics[scale=1]{/newtera/tera/phil/nchaparr/tera2_cp/nchaparr/ubcdiss/pngs/FedEZRi}
    \caption[Plot of the relationship between scaled \acs{EZ} depth and Richardson number from \citeauthor{FedConzMir04}'s (\citeyear{FedConzMir04})]{Figure 9 from \citeauthor{FedConzMir04} (\citeyear{FedConzMir04}) representing Equation \ref{eq:dhvsri} using three different Richardson numbers, in log-log coordinates.  Heights are based on the $\overline{w^{'}\theta^{'}}$ profile as in Figure \ref{fig:hdefs1} and their $z_{i}$ is my $z_{f}$. $\frac{\delta z_{i}}{z_{i}}$. $Ri_{\Delta b}$ (circles) and $Ri_{\delta b}$ (crosses) correspond directly to those determined here using $\delta \theta$ and $\Delta \theta$.  Note that their $\Delta$ refers to the smaller jump measured at $z_{f}$, whereas I use it for the larger.  $Ri_{N}$ (triangles) is the Richardson number defined in Equation \ref{eq:gradri}, with $w^{*}$ and $z_{f}$ as the velocity and length scale.}
    \label{fig:FedEZRi}   % label should change
\end{figure}


 \citeauthor{BrooksFowler2} (\citeyear{BrooksFowler2}) found no clear $\acs{Ri}$ dependence of the scaled \acs{EZ} depth defined based on the $\overline{w^{'}\theta^{'}}$ profile.  But their definition hinged solely upon the lower part ($z_{f1} - z_{f}$) which according to \citeauthor{FedConzMir04} (\citeyear{FedConzMir04}) does not vary in time.  Figure \ref{fig:deltahinvri_scaled} of this thesis shows that when I defined the \acs{EZ} based on the $\overline{w^{'}\theta^{'}}$ profile as \citeauthor{FedConzMir04} (\citeyear{FedConzMir04}) did, the scaled \acs{EZ} depth had no clear dependence on $\acs{Ri}$. This is supported by the similarity in time and across runs of the vertical turbulent heat flux profiles when scaled by $(\overline{w^{'}\theta^{'}})_{s}$ in Figures \ref{fig:scaledfluxprofs15010} and \ref{fig:fluxprofs2hrs}.\\

The most obvious possible cause for disagreement with the results of \citeauthor{FedConzMir04} (\citeyear{FedConzMir04}) is the difference in grid size shown in Table \ref{table:gridcomp}.  Inspection of their $\overline{w^{'}\theta^{'}}$ profiles confirms a relatively deeper scaled region of negative flux as compared with those seen here (~.4 vs ~.25). Their surface heat flux $\overline{w^{'}\theta^{'}}_{s}$ was twice the highest used here, but their range of $\acs{Ri}$ is comparable to that of this study.  The latter point although not directly relevant here, serves as confirmation that $\gamma$ is the more influential parameter.\\              

\begin{table}[htbp]
\label{table:elandri}
\caption[\acs{EZ} Definitions used in comparable Studies]{\acs{EZ} Definitions used in comparable Studies}

\begin{center}
%\centerline{
\begin{tabular}{ p{4cm} p{2cm} p{1.5cm} p{3cm}}
    %\hline

Publication & \acs{EZ} Depth & \acs{CBL} height & $\theta$ Jump\\ \hline
\citeauthor{FedConzMir04} (\citeyear{FedConzMir04}) & $z_{f1} - z_{f0}$ & $z_{f}$ &  $\overline{\theta}(z_{f1})-\overline{\theta}(z_{f0})$\\ [.3cm] %\hline
\citeauthor{BrooksFowler2} (\citeyear{BrooksFowler2}) & $2 \times (z_{f} - z_{f0})$ & $z_{f}$ & average of local values\\ \hline

\end{tabular}
\end{center}    
\end{table}

\subsection{General Comparison using the Potential Temperature Profile}

Here, when heights are defined based on the scaled vertical potential temperature gradient profile $\frac{\frac{\partial \overline{\theta}}{\partial z}}{\gamma}$ the curve representing Equation \ref{eq:dhvsri} 

\begin{equation}
\frac{\Delta h}{h} \propto Ri^{b} \tag{\ref{eq:dhvsri}}
\end{equation}

shows an exponent $b$ which increases in magnitude, from about $-\frac{1}{2}$ as predicted and seen by \citeauthor{Boers89} (\citeyear{Boers89}), to about $-1$ as justified in \citeauthor{StullNelEl} (\citeyear{StullNelEl}),  with increasing $\acs{Ri}$ (decreasing $\acs{Ri}^{-1}$).  Overall there is a clear narrowing of the scaled \acs{EZ} depth with increasing $\acs{Ri}$ (decreasing $\acs{Ri}^{-1}$) as supported by the local height distributions in Section \ref{subsec:locmlh}.  Although based on different height definitions, \citeauthor{FedConzMir04} (\citeyear{FedConzMir04}) concluded an exponent $b = -\frac{1}{2}$ and \citeauthor{BrooksFowler2}'s (\citeyear{BrooksFowler2}) plots show curves with an apparent exponent less in magnitude than $-1$, in Figure \ref{fig:BandFEZ}. \\

\begin{figure}[htbp]
    \centering
    %plot_height.py[master 1573b9d] h vs time plot
    \includegraphics[scale=1.3]{/newtera/tera/phil/nchaparr/tera2_cp/nchaparr/ubcdiss/pngs/BandFEZ}
    \caption[Relationship of Scaled \acs{EZ} depth to Richardson number from \citeauthor{BrooksFowler2}'s (\citeyear{BrooksFowler2})]{Panel (a) from Figure 5 in \citeauthor{BrooksFowler2} (\citeyear{BrooksFowler2}) representing Equation \ref{eq:dhvsri}:The normalized \acs{EZ} depth is determined in three ways (i) the upper and lower percentiles from the distribution of local \acs{CBL} height (maximum tracer gradient), normalized by the average of the local heights (pale grey) (ii) the average of local scaled \acs{EZ} depths based on wavelet covariance (dark grey) and (iii) the average of the locally determined \acs{EZ} depths scaled by the average of the locally determined heights (black), based on wavelet covariance.  Their $\theta$ jump is an average of the potential temperature differences across the local \acs{EZ} depths.}
    \label{fig:BandFEZ}   % label should change
\end{figure}

The curves representing each run in Figure \ref{fig:BandFEZ} fan out.  In Figure \ref{fig:scaledeltahinvri} of this thesis, before scaling, the $\frac{\partial \overline{\theta}}{\partial z}$ profile curves separate out, but in the reverse order.  \acs{CBL}s under higher stability, and so higher $\acs{Ri}$, have larger scaled \acs{EZ} depths.  Whereas \citeauthor{BrooksFowler2}'s (\citeyear{BrooksFowler2}) runs with initially lower $\acs{Ri}$ have larger scaled \acs{EZ} depths than those with higher, even where $\acs{Ri}$ values overlap. Nonetheless, that there appears a family of separate but similar curves rather than a single curve hints at an underlying scaling parameter.\\     

Neither study referenced in Table 3.3 addresses the change in exponent with increased $\acs{Ri}$ that I observe in Figure \ref{fig:loglogdeltahinvri}.  It is reasonable to suggest that this represents a change in entrainment mechanism. \citeauthor{SullMoengStev} (\citeyear{SullMoengStev}) observed enfolding and engulfment at lower $\acs{Ri}$.  Whereas at higher $\acs{Ri}$ when motion is more restricted, entrainment seemed to occur via trapping of thinner wisps at the edge of an upward moving thermal.  \citeauthor{Turner86} (\citeyear{Turner86}) also distinguishes between entrainment by convective overturning and recoil. \citeauthor{GarciaMellado} (\citeyear{GarciaMellado}) refer to a change in entrainment rate due to the effects of increased stability on the upper \acs{EZ} sub-layer.  In this study, the narrowing of the \acs{EZ} depends predominantly on the magnitude of the average vertical potential temperature gradient $\frac{\partial \overline{\theta}}{\partial z}$ in the lower \acs{EZ} and upper \acs{ML}.  However, the scaled magnitude of upper limit in Figure \ref{fig:scaledELlims} (a) does appear to decrease slightly in time.  This could correspond to the slowly decreasing upper sub layer of the \acs{EZ} mentioned in both \citeauthor{GarciaMellado} (\citeyear{GarciaMellado}) and \citeauthor{FedConzMir04} (\citeyear{FedConzMir04}).\\

\section{Entrainment Rate Parameterization}
\label{sec:erparam}
$\acs{Ri}$ magnitude determined in this and the comparable studies is primarily influenced by the magnitude of the $\theta$ jump.  Here, I define it in two ways as \citeauthor{FedConzMir04} (\citeyear{FedConzMir04}) did.  I do this based on the $\overline{w^{'}\theta^{'}}$ profile, as in Figure \ref{fig:hdefs1} and Table \ref{table:reldefs} for the purpose of direct comparison and to observe how the change in definition effects Equation \ref{eq:ervsri}.\\

\begin{equation}
\frac{w_{e}}{w^{*}} \propto Ri^{a} \tag{\ref{eq:ervsri}}
\end{equation}

\subsection{Direct Comparison based on the vertical Heat Flux Profile}
The larger jump, i.e. that taken across the \acs{EZ} ($\Delta \theta$) as in Figure \ref{fig:1storder}, yields a larger value of $a$ as \citeauthor{FedConzMir04} (\citeyear{FedConzMir04}) conclude.  \citeauthor{GarciaMellado} (\citeyear{GarciaMellado}) interpret both curves as asymptotic to straight lines ($a=-1$) as the upper \acs{EZ} sub-layer narrows. Based on their plots in Figure \ref{fig:GarcMelERRi}, in the absence of their justification based on the derivation of the entrainment relation, for $\Delta \theta$ I see a curve (grey and blue) with increasing exponent exceeding magnitude $-1$ at higher $\acs{Ri}$.  For their version of $\delta \theta$ I see a curve (grey and red) with exponent less in magnitude than $-1$.\\

\begin{figure}[htbp]
    \centering
    %plot_height.py[master 1573b9d] h vs time plot
    \includegraphics[scale=1]{/newtera/tera/phil/nchaparr/tera2_cp/nchaparr/ubcdiss/pngs/GaMeERRi}
    \caption[Plots of scaled entrainment rate vs Richardson number from \citeauthor{GarciaMellado} (\citeyear{GarciaMellado})]{Figure 11 from \citeauthor{GarciaMellado} (\citeyear{GarciaMellado}) and representing equation \ref{eq:ervsri} based on the two $\theta$ jumps.  The grey and blue curve is based on $\Delta \theta$ and the (grey and) red curve is based on $\overline{\theta}(h) - \overline{\theta}_{0}(h)$ which is slightly different to the $\delta \theta$ defined here and in \citeauthor{FedConzMir04} (\citeyear{FedConzMir04}). The dashed and continuous black lines represent the straight lines to which the curves asymptote according to their analysis. Their heights are comparable to those based on the heat flux ($\overline{w^{'}\theta^{'}}$) profile in Figure \ref{fig:hdefs1}.}
    \label{fig:GarcMelERRi}   % label should change
\end{figure}

\subsection{Extending Comparison to the average Potential Temperature Profile}

There is an analogous distinction between curves representing Equation \ref{eq:ervsri} using $\Delta \theta$ and those using $\delta \theta$, when all heights are based on the $\frac{\frac{\partial \overline{\theta}}{\partial z}}{\gamma}$ profile.  Scatter is least when the $\theta$ jump is defined across the \acs{EZ}.  In Figure \ref{fig:weinvri} $a=-\frac{3}{2}$ fits at higher $\acs{Ri}$ (lower $\acs{Ri}^{-1}$) and $a=-1$ seems to fit at lower $\acs{Ri}$.  Combined with the apparent change in $b$ for Equation \ref{eq:dhvsri} I interpret this as an indication of a change in entrainment regime at increasing $\acs{Ri}$.\\ 

\FloatBarrier


\endinput

Any text after an \endinput is ignored.
You could put scraps here or things in progress.

%\include{conclusions}

%    3. Notes
%    4. Footnotes

%    5. Bibliography
\begin{singlespace}
\raggedright
\bibliographystyle{ametsoc}
\makeatletter
\renewcommand\@biblabel[1]{}
\makeatother
\bibliography{biblio}

\end{singlespace}

\appendix
%    6. Appendices (including copies of all required UBC Research
%       Ethics Board's Certificates of Approval)
%\include{reb-coa}	% pdfpages is useful here
\chapter{Appendices}
\section{Potential Temperature: $\theta$}

\label{sec:pottemp}

\begin{equation}
\theta = T \left(\frac{p_{0}}{p} \right)^{\frac{R_{d}}{c_{p}}} 
\end{equation}

$p_{0}$ and $p$ are a reference pressure and pressure respectively. 

\begin{equation}
\frac{c_{p}}{\theta}\frac{d\theta}{dt} = \frac{c_{p}}{T} \frac{dT}{dt} - \frac{R_{d}}{p}\frac{dp}{dt} 
\end{equation}

If changes in pressure are negligible compared to overall pressure, as in the case of that part atmosphere that extends from the surface to 2km above it. 

\begin{equation}
c_{p}\frac{d\theta}{\theta} = c_{p}\frac{dT}{T} - \frac{R_{d}}{p}\frac{dp}{p} 
\end{equation}

\begin{equation}
\frac{d\theta}{\theta} = \frac{dT}{T} 
\end{equation}

and if 

\begin{equation}
\frac{\theta}{T} \approx 1 
\end{equation}

then small changes in temperature are approximated by small changes in potential temperature

\begin{equation}
d\theta \approx dT \ or \ \theta^{'} \approx T^{'}
\end{equation}

and at constant pressure change in enthalpy ($H$) is 

\begin{equation}
dH = c_{p}dT.
\end{equation}

This serves as justification for defining $\overline{w^{'}\theta^{'}}$ as the vertical heat flux.
\section{Second Law of Thermodynamics}
\begin{equation}
\frac{ds}{dt} \ge \frac{q}{T}
\end{equation}

For a reversible process

\begin{equation}
\frac{ds}{dt} = \frac{q}{T}
\end{equation}

Using the first law and the equation of state for an ideal gas

\begin{equation}
\frac{q}{T} = \frac{1}{T} \left(\frac{dh}{dt} - \alpha \frac{dp}{dt}\right) =  \frac{c_{p}}{T} \frac{dT}{dt} - \frac{R_{d}}{p} \frac{dp}{dt}
\end{equation}

so

\begin{equation}
\frac{ds}{dt} = \frac{q}{T} =  \frac{c_{p}}{\theta}\frac{d\theta}{dt}
\end{equation}

For a dry adiabatic atmosphere

\begin{equation}
\frac{ds}{dt} =  \frac{c_{p}}{\theta}\frac{d\theta}{dt} = 0
\end{equation}

\section{Reynolds Decomposition and Simplification of Conservation of Enthalpy (or Entropy) for a dry Atmosphere}
\label{sec:rdent}
\begin{equation}
\frac{\partial \theta}{\partial t} + u_{i}\frac{\partial \theta}{\partial x_{i}} = \nu_{\theta} \frac{\partial^{2}\theta}{\partial x_{i}^{2}} - \frac{1}{c_{p}}\frac{\partial Q^{*}}{\partial x_{i}}
\end{equation}

$\nu$ and $Q^{*}$ are the thermal diffusivity and net radiation respectively.  If we ignore these two effects then

\begin{equation}
\frac{\partial \theta}{\partial t} + u_{i}\frac{\partial \theta}{\partial x_{i}} = 0
\end{equation}

\begin{equation}
\theta = \overline{\theta} + \theta^{'}, \theta = \overline{u_{i}} + u_{i}^{'} 
\end{equation}

\begin{equation}
\frac{\partial \overline{\theta}}{\partial t} + \frac{\partial \theta^{'}}{\partial t} + \overline{u_{i}}\frac{\partial \overline{\theta}}{\partial x_{i}} + u_{i}^{'}\frac{\partial \overline{\theta}}{\partial x_{i}} + \overline{u_{i}}\frac{\partial \theta^{'}}{\partial x_{i}} + u_{i}^{'}\frac{\partial \theta^{'}}{\partial x_{i}} = 0
\end{equation}

Averaging and getting rid of average variances and their linear products

\begin{equation}
\frac{\partial \overline{\theta}}{\partial t} + \overline{u_{i}}\frac{\partial \overline{\theta}}{\partial x_{i}} + u_{i}^{'}\frac{\partial \overline{\theta^{'}}}{\partial x_{i}} = 0
\end{equation}

Ignoring mean winds

\begin{equation}
\frac{\partial \overline{\theta}}{\partial t} + u_{i}^{'}\frac{\partial \overline{\theta^{'}}}{\partial x_{i}} = 0
\end{equation}

using flux form

\begin{equation}
\frac{\partial \overline{\theta}}{\partial t} + \frac{\partial(\overline{u_{i}^{'}\theta^{'}})}{\partial x_{i}} - \theta^{'}\frac{\partial \overline{u_{i}^{'}}}{\partial x_{i}}= 0
\end{equation}

under the bousinesq assumption $\Delta \cdot u_{i} = 0$

\begin{equation}
\frac{\partial \overline{\theta}}{\partial t} = -\frac{\partial(u_{i}^{'}\theta^{'})}{\partial z}
\end{equation}

ignoring horizontal fluxes

\begin{equation}
\label{eq:warming1}
\frac{\partial \overline{\theta}}{\partial t} = -\frac{\partial(\overline{w^{'}\theta^{'}})}{\partial z}
\end{equation}

%\section{Reynolds averaged Turbulence Kinetic Energy Equation}

%\begin{equation}
%\frac{\partial \overline{e}}{\partial t} + \overline{U}_{j} \frac{\partial \overline{e}}{\partial x_{j}} = \delta_{i3}  \frac{g}{\overline{\theta}} \left( \overline{u_{i}^{'}\theta^{'}} \right) - \overline{u_{i}^{'}u_{j}^{'}}\frac{\partial \overline{U}_{i}}{\partial x_{j}} - \frac{ \partial \left( \overline{u_{j}^{'}e^{'}} \right)}{\partial x_{j}} - \frac{1}{\overline{\rho}} \frac{\partial \left( \overline{u_{i}^{'} p^{'}} \right) }{\partial x_{i}} - \epsilon
%\end{equation}

%$e$ is turbulence kinetic energy (TKE).  $p$ is pressure.  $\rho$ is density.  $\epsilon$ is viscous dissipation.

%This would be any supporting material not central to the dissertation.
%For example:
%\begin{itemize}
%\item Authorizations from Research Ethics Boards for the various
%    experiments conducted during the course of research.
%\item Copies of questionnaires and survey instruments.
%\end{itemize}

\endinput


%\backmatter
%    7. Index
% See the makeindex package: the following page provides a quick overview
% <http://www.image.ufl.edu/help/latex/latex_indexes.shtml>


\end{document}
