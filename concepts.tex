%% The following is a directive for TeXShop to indicate the main file
%%!TEX root = diss.tex

\chapter{Concepts}
\label{ch:Conceptss}
\setlength{\parindent}{0cm}

\section{Modelling the Convective Boundary Layer}
\label{CBLModels}

\subsection{Analytical Models}
The most common analytical models for the Convective Boundary layer (\acs{CBL}) can be subdivided into: (i) zero order
and (ii) first order bulk models.\\

Zero order bulk models assume a Mixed Layer (\acs{ML}) of constant potential temperature topped
by an infinitessimally thin layer accross with there is a temperature jump.  The assumed vertical heat flux ($w^{,}\theta^{,}$) profiles are
linearly decreasing from the surface up, reaching a maximum negative value proportional to the surface value (usually -.2)
at the temperature inversion and going to zero over an infinitessimally small layer.  Equations for the evolution of ML or inversion height (Entrainment Relation),
and average mixed layer temperature and temperature jump at the inversion are derived from the reynolds average conservation of enthalpy
euqation integrated over the ML, simplified and closed according to the entrainment relation and assumptions such as those mentioned above.  
So relevant quantities are ensemble averaged.\\

\cite{Deardorff79}

First order models assume an Entrainment layer (\acs{EL}) of finite depth at the top of the ML, defined by two heights:
the top of the ML ($h_{0}$) and the point where free atmospheric characteristics are resumed ($h_{1}$).  The derivations are more complex and 
examples of implifying assumptions about the \acs{EL} are: $\Delta h = h_{1} - h_{0} = 0$ (Betts, see \cite{Deardorff79}), $\Delta h$ or maximum 
overshoot distance $d \propto \frac{w^{*}}{N}$ where $w^{*}$ is the relevant vertical velocity scale and $N = \sqrt{\frac{g}{\overline{\theta}} \frac{\partial \overline{\theta}}{\partial z}}$
is the Brunt-Vaisalla frequency (Stull 1972 see dropbox folder) and that between $h_{0}$ and $h_{1}$ $\overline{\theta} = \overline{\theta}_{ML} + f(x,t) \Delta \overline{\theta}$ where
$f(x,t) \Delta \overline{\theta}$ is a dimensionless shape factor \cite{Deardorff79}.\\

Although validation or development of these models is beyond the scope of thisthesis, mention of them is necessary to give context to the scaling 
parameters and relationships to be considered. \\         

\subsection{Numerical Models using the Navier Stokes Equations: Large Eddie Simulation}

Direct Numerical Simulataions (\acs{DNS}) solve the Navier Stokes Equations without use of sub grid scale closure, i.e. all range of spatial
and temportal turbulence must be resolved.  A Large Eddy Simulation (\acs{LES}) filters out the smaller scales leaving them to 
a sub grid scale closure model.  General Circulation Models (\acs{GCM}) can model global processes, so their subgrid scale can span the
an LES domain, and they use a grid based on spherical coordinates.\\

(paper from Douws 571c justifying LES use)

\acs{LES} has steadily, repeatedly been used to better understand the \acs{CBL} since \citeauthor{Deardorff72} applied this relatively 
new method in \cite{Deardorff72} for this purpose.  \citeauthor{SullMoengStev} in \cite{SullMoengStev}, \citeauthor{FedConzMir04} in \cite{FedConzMir04}
and \citeauthor{BrooksFowler2} in cite{BrooksFowler2} used it to observe the structure and scaling behaviour of the \acs{EL}.\\

The technique has been widely used to closely simulate measurement campaigns as well as more idealized conditions whereby scaling relations, parametrizations
and analystical models can be tested. (references? tie in to what i'm using it for)
\\

\subsection{Choice of Grid Sizes}

\citeauthor{SullPat} in \cite{SullPat} investigated \acs{LES} solution convergence for modelling the \acs{CBL} over a range of grid sizes.
They found that the shapes of key variable vertical profiles ($\overline{\theta}$ and $\overline{w^{,}\theta^{,}}$), particularly in the \acs{EL} 
as well as entrainment rate ($w_{e}$) varied and began to converge at higher resolution.  Of particular interest is the variation
in shape of the vertical $\overline{\theta}$ profile within the \acs{EL}.  The height over which $\overline{\theta}$ increases
from the \acs{ML} value to the free atmosphere value is elongates with decreased resolution.\\

Of highest importance is the resolution in the vertical to resolve the steep gradients within the \acs{EL}. 
Although comparable horizontal resolution is needed to maintain reasonable aspect ratio, 3D isotropy is thought to be unecessary.
2D isotropy is sufficient.\\

Choice of grid size must also consider the scaling distance between the eddies containing the most energy
and the smallest resolved eddies.  Domain size must also be considered in order to allow fully turbulent fields to develop independently of the periodic sidewall 
boundary conditions. In particular at later simulation times ($t > 8hrs$) when large structures develop.  
(I'm afraid my domain is too small) These consideratons must be balances with the computational time and 
resource avaiability.\\

\subsection{Choice of Initial Conditions}

Both \citeauthor{SullMoengStev} and \citeauthor{BrooksFowler1} in \cite{SullMoengStev} and \cite{BrooksFowler1} chose to
start with a finite layer of uniform tempertaure and or tracer concentration topped by a temperature jump.  These inital conditions
are an idealization of a realistic potential temperature sounding.  Due to entrainment a temperature jump
also evolves when a \acs{CBL} grows against stable linear stratification ($\gamma$) (\cite{Turner86}, \cite{FedConzMir04} and \cite{GarciaMellado}).
This jump is a direct result of the \acs{CBL} growth and warming rates and $\gamma$.
 
\subsection{An Ensemble}

Using a true ensemble of individual cases varying by a pseudo-random small temerature perturbation from unform distribution,
means that true ensemble averages and so temperature peturbations can be calculated.  The turbulent veloctiy perturbations are 
prognosed by the \acs{LES} and the temperature is diagnosed.  Given the individual cases produce realistic turbulence,
 ie resolved, (statistically independent?) eddies, thermals and plumes, and ensemble provides an advantage in terms
 of point-sample size.       

\subsection{The Equilibrium Entrainment Regime}

The equilibrium or quasi-equilibrium entrainment regime is reached when the time evolution of turbulence kinetic energy (\acs{TKE})
 and its dissipation are negligible and there is negligible energy lost through gravity waves.  This regime is assumed for the derivation of
the zero order model based entrainment relation as in \cite{FedConzMir04}.  \citeauthor{GarciaMellado} in \cite{GarciaMellado} plot the 
terms of the terms of the evolution equation for \acs{TKE}, integrated from the surface to the top of the domain (? $z_{\infty}$) 
and show the the balancing of the buoyancy (driven TKE) term by the dissipation term.  They say this regime is also characterised 
by a rate of \acs{CBL} growth that is much slower than their time scale ($\frac{z_{enc}}{L_{0}}$, $L_{0} = \left(\frac{B_{0}}{N^{3}} \right)^{\frac{1}{2}}$)
and proportional to the square root of time.  % phil wants me to show this  


The energy term profiles become more or less self similar when scaled by $w^{*}$. Finally
\citeauthor{FedConzMir04} in \cite{FedConzMir04} say it is characterized by a(n approximately) constant entrainment ratio ($\frac{B_{i}}{B_{s}}$) and
and a constant stability ratio ($\frac{N^{2}\delta z_{i}}{\delta b} \approx 1.2$).   

\section{Seeing the \acs{CBL} and \acs{EL}}
\label{SeingCBL}
\subsection{plumes, thermals}
To establish that the simulated \acs{CBL} is behaving realitically visual representations of the key variables are observed.
\citeauthor{SchmidtSchu} in \cite{SchmidtSchu} used vertical profiles of the horizontally averaged (equivalent to ensemble averaged
by ergodic assumption) $\theta$, $w^{,}\theta^{,}$ and $w^{,2}$.  For example using their $\overline{\theta}$ profiles they observed
a developing \acs{ML} and the effects of heating by entrainment of warm air from above, in addition to from the surface,  
and the resulting development of an inversion, ie a point at which the gradient is steeper than the inititial lapse rate ($\gamma$).  Vertical profiles of the velocities and heat flux
showed a \acs{CBL} above which turbulence sharply decreases and the $\overline{w^{}\theta^{,}}$ profiles provided a clear
visual for the location and depth of the \acs{EL} where it became negative as entrainment became more important as well
as a definition for \acs{CBL} height at the point where it is minimum.\\

A more direct visual is obtained using horizontal contour plots of the turbulent quantities. \citeauthor{SchmidtSchu} in \cite{SchmidtSchu}
and later \citeauthor{SullMoengStev} in \cite{SullMoengStev} showed the spoke like patterns of the upward moving thermals as they become plumes impinging on
the stable air aloft.\\    
\subsection{energy spectra, ffts, the circular integration to one a scalar spectrum}

To confirm there is sufficient scale separation between the dominant eddies and the grid size
the \acs{FFT} energy of the velocity fields should be plotted against wavenumber.  For example
 the power spectrum of the 3 dimentional domain should obey Kolmagorov's $\frac{-11}{3}$ power law

\begin{equation}
\phi(k) = Ck^{\frac{-11}{3}}
\end{equation}

Analoguosly in 2D is a $\frac{-8}{3}$ power law but it is difficult to represent visually.
If radial isotropy is assumed and the the power function is integrated around a circle 
(or semicircle if only considering the positive wavenumbers), the scalar result should
obey the $\frac{-5}{3}$ power law for 1 dimension.


\begin{equation}
\Phi(r) = \int \phi(r) r d\theta = \int Cr^{\frac{-8}{3}} r d\theta = \pi \frac{-3}{8} C r^{\frac{-5}{3}}, \ r=\sqrt{k_{x}^{2} + k_{y}^{2}}
\end{equation}
 
Whether there is sufficient scale separation between the dominant turbulent structures
and the grid sizes would be obvious from a  plot of $log(\Phi(r))$ vs $log(r)$.

\section{Boundary Layer Height and Entrainment Layer Limits: Definitions}
\label{sec:BLhdeltah}
In \cite{Deardorff72}, \citeauthor{Deardorff72} confirmed using \acs{LES} that the inversion height ($z_{i}$) was the most important
lengh scale for the \acs{CBL}.  In earlier studies, the later of which for example would be \cite{SchmidtSchu}, they use this
term to describe the point of minimum $\overline{w^{,}\theta^{,}}$ which was more recently (\cite{SullMoengStev}) shown to be 
distinct from the vertical location of the maximum $\overline{\theta}$ gradient.   There are differences between the \acs{CBL}
height definitions in analytical and numerical studies as compared with those used for measurements.  For example \citeauthor{FedConzMir04}
in \cite{FedConzMir04} use the point of minimum $\overline{w^{,}\theta^{,}}$ and lable it $z_{i}$, whereas \citeauthor{Traum11} in \cite{Traum11}
use three methods based on LIDAR backscatter profile.  Other definitions include those that are turbulence based such as the height below which
the vertical velocity is significantly correlatedd with itself (\cite{Traum11}), ie the height of the dominant turbulent structures
and those based on statistics of locally determined heights (\cite{BrooksFowler2})\\

Likewise, definitions of the \acs{EL} limits vary, as is summarized well by both \cite{BrooksFowler2} and \cite{Traum11}.  There is the idea that
it is a local vertical distance over which a quantity such as $\theta$ or tracer concentration transitions from an \acs{CBL} value to an upper atmosphere
value.  This might be explained by plumes reaching different heights, with intermittent periods of relative inactivity over time.  At any point in time
there will be a range of local \acs{ML} heights and the \acs{EL} can be viewed as some function of the range over which they vary (\cite{StullNelEl} ).
These two concepts can then be combined as in \cite{BrooksFowler1}.\\     

(Need to say what I'm useing and why, mention of Sullivan and Moeng!)
\section{Boundary Layer height and Entrainment Layer Limits: Measurement Techniques}
\label{sec:BLMeas}

A full and detailed discussion about \acs{CBL} height determination is beyond the scope of this
thesis. (cite traumner and brooks and fowler) LIDAR backscatter profiles seem to dominate recent 
measurement campaigns \cite{Traum11}.  They clearly represent the difference between the \acs{CBL} 
and the upper atmosphere in terms of aerosol concentration.  So techniques such as 
\citeauthor{BrooksFowler2}'s wavelet algorithm and \citeauthor{SteynBaldHoff}'s idealized profile 
fit are important.\\

Numerical models have the advantage of easily producing $\overline{w^{,}\theta^{,}}$ profiles
on which analytical models are based as well as Tracer concentration, and potential temperature profiles.
So it's important to form a logical bridge from the analytical models from which scaling relations
are derived to actual measurements. Obtaining the $\overline{w^{,}\theta^{,}}$ based height(s) (minimum and zero
crossings) from LES output is trivial.  LES produced \acs{CBL} Potential temperature profiles have a 
real-world counterpart (soundings).  \citeauthor{SullMoengStev} in \cite{SullMoengStev} identified the 
point of maximum gradient in the profile and used this as their \acs{CBL} height.  But this method
fails when there is local variation in the profile for example due to waves, and the resulting
gradients exceed that marking the transition from \acs{ML} to the upper atmosphere.\\

For a uniform $\overline{w^{,}\theta^{,}}$ which causes a \acs{CBL} to rise against a constant $\gamma$
the local $\theta$ profiles can be approximated by three regions similar to a first order analytical model:
the \acs{ML} (a contstant $\theta$), \acs{EL} and the upper atmopshere ($\gamma$).  At least two 
of these three regions are approximately linear.  \citeauthor{Vieth} in \cite{Vieth} applied a two line
fit to data according to the minimum residual sum of squares (\acs{RSS}).  This can easily be extended
to three lines where the key points are those where one line ends and another begins ie, the top of the 
\acs{ML}, and the top of the \acs{EL}.  If strictly adhering to the analytical model \acs{EL} should be
approximated by a curve (3rd degree polynomial) but hundreds of local profiles where observed, and none
had a distinct \acs{EL} similar to that of the idealized or averaged profile.\\

\subsection{Justification of $h_{0}$ threshold}

Brooks and Fowler state that for the difinition of an entrainment zone using a mean scalar profile there
 is not clear definition of what a significant gradient is.  Given my mean profiles of the vertical potential
temperature gradient remain close to zero in the mixed layer, but decreases from the surface and then
increases gradually.  A threshold value must be chosen to separate the layer where it increases gradually
from that where the increase is sharper (EL.).  This threshold value must be greater than zero, and less
than $\gamma$ and the same for all runs.  The lowest $\gamma$ is .0025.  I chose a value of .0002, 
but verified that my main result  remained consistant with values of .0001 and .0004.
see scaleddeltahinvri plots 5 and 6, and theta grad profs 6 and 5.


\section{Entrainment Layer Statistics}
\label{sec:EntLayStat}
Since the concept of the \acs{CBL} \acs{EL} is basd on plume heights distributed over time
or space, a statistical perspective is natural.  Relationship between the distrubutions of 
\acs{CBL} heights ($h$) and Richardson Number is expected.  A visualization of
 the 2D distributions of the local $w^{,}\theta^{,}$ quadrants and their behaviour 
with respect to Richardson Number would compliment and further the insight given
by \citeauthor{SullMoengStev} in \cite{SullMoengStev}.\\

\citeauthor{BrooksFowler2} found the statistically based definitions $h$ yielded clearer 
relationships to Richardson number in \cite{BrooksFowler2} and proposed a measure of \acs{EL}
depth based on statistics of the local \acs{EL}.  \citeauthor{SullMoengStev} obsereved 
Richardson Number dependece on the second order statistics of the local $h$.\\

Having a large sample of local data points might be an advantage.  A comparison of the
local $h$  surface could be compared with upper horizontal slices of $\theta^{,}$ and 
$w^{,}$.  Visualization of the distributions of local $h$ using histograms, paying attention
to for example their spread and skew with relation to $\gamma$ and $\overline{w^{'}\theta^{'}}$
would lend support to scaling relationships based on the averaged profiles.\\

\section{Parameters, Scales and Scaling Relations -- Velocity, time, height, surface flux, stability}

\subsection{Convective Velocity Scale: $w^{*}$}
Given an average surface heat flux ($\overline{w^{,}\theta^{,}}_{s}$) a surface buoyancy flux can be defined as 
$\frac{g}{\overline{\theta}}\overline{w^{,}\theta^{,}}$ from which the convective velocity scale is obtained by
multiplying by the appropriate length scale.  Since the result is in $\frac{m^{3}}{s^{3}}$ a cube root is applied.\\

\begin{equation}
w^{*} = \left( \frac{gh}{\overline{\theta}}\overline{w^{,}\theta^{,}} \right)
\end{equation}

\citeauthor{Deardorff70} (\cite{Deardorff70}) confirmed that this effectively scaled the vertical turbulent velocity
perturbations in the \acs{CBL}.\\

It logically follows that the time for a plume to reach the top of the \acs{CBL} is

\begin{equation}
\tau = \frac{h}{\left( \frac{gh}{\overline{\theta}}\overline{w^{,}\theta^{,}} \right)}
\end{equation}

where $h$ is \acs{CBL} height.

\subsection{Richardson Numbers}

The Flux Richardson ($R_{f}$) expresses the balance between turbulent mechanical energy and buoyancy.  It's obtained from the ration of these two terms
 in the Turbulent Kinetic Energy budget equation (\cite{Stull-BLMetIntro}):

\begin{equation}
\frac{\partial \overline{e}}{\partial t} + \overline{U}_{j} \frac{\partial \overline{e}}{\partial x_{j}} = \delta_{i3}  \frac{g}{\overline{\theta}} \left( \overline{u_{i}^{,}\theta^{,}} \right) - \overline{u_{i}^{,}u_{j}^{,}}\frac{\partial \overline{U}_{i}}{\partial x_{j}} - \frac{ \partial \left( u_{j}^{,}e^{,} \right)}{\partial x_{j}} - \frac{1}{\overline{\rho}} \frac{\partial \left( u_{i}^{,} p^{,} \right) }{\partial x_{i}} - \epsilon
\end{equation}

\begin{equation}
R_{f} = \frac{\frac{g}{\overline{\theta}} \left( \overline{w^{,}\theta^{,}} \right)}{\overline{u_{i}^{,}u_{j}^{,}}\frac{\partial \overline{U}_{i}}{\partial x_{j}}}
\end{equation}
 
Assuming horizontal homogenaiety and neglecting subsidence
  
\begin{equation}
R_{f} = \frac{\frac{g}{\overline{\theta}} \left( \overline{w^{,}\theta^{,}} \right)}{\overline{u^{,}w^{,}}\frac{\partial \overline{U}}{\partial z} + \overline{v^{,}w^{,}}\frac{\partial \overline{V}}{\partial z}}
\end{equation}

Applying first order closer to the flux terms (i.e. assuming they are proportional to the vertical gradients)  gives the gradient Richardson Number ($R_{g}$)

\begin{equation}
R_{g} = \frac{ \frac{g}{\overline{\theta}} \frac{\partial \overline{\theta}}{\partial z}}{\left( \frac{ \partial \overline{U}}{\partial z} \right)^{2} + \left( \frac{\partial \overline{V}}{\partial z} \right)^{2}} 
\end{equation}

which is usually used to express the balance between shear and buoyancy driven turbulence, but in the \acs{EL} buoancy acts to supress turbulence.  
Applying a bulk approximation the to the numerator, and expressing it in terms of scales yields a ratio of two square of time scales

\begin{equation}
R_{g} = \frac{\frac{g}{\overline{\theta}} \frac{\partial \overline{\theta}}{\partial z}}{\frac{U^{*2}}{L^{2}}} = N^{2}\frac{L^{2}}{U^{*}}
\end{equation}


and applying the bulk approximation to both the numerator and the denomnitor yields

\begin{equation}
R_{b} = \frac{\frac{g}{\overline{\theta}} \Delta \overline{\theta} L}{U^{*}}
\end{equation}
A natural choice of length and veloctiy scales for the \acs{CBL} are $h$ and $w^{*}$.  \citeauthor{EllTurn} (\cite{EllTurn}) suggested and confirmed a relationship between the entrainment rate and this form of 
Richardson number based on tank experiments.  This parameter can be justified and arrived at by considering the principal forcings of the system, or from non-dimensionalizing the entrainment relation  deriived
analyitically, i.e. (\cite{Deardorff72})

\begin{equation}
w_{e} \propto \frac{\overline{w^{,}\theta^{,}}_{s}}{\Delta \overline{\theta}}
\end{equation}

\begin{equation}
\frac{w_{e}}{w^{*}} \propto  \frac{\overline{w^{,}\theta^{,}}_{s}}{\Delta \overline{\theta} w*{*}} = Ri_{b}^{-1}
\end{equation}
 

In one or other of its forms this parameter has become central to any study on \acs{CBL} entrainment (\cite{SullMoengStev}, \cite{FedConzMir04}, \cite{Traum11}, \cite{BrooksFowler2})

\subsection{Scaling Relationships}

(relationship of height to scaled time, check sull moeng and fed conz)

The relationship between the scaled entrainment rate $w_{e}$ and Richardson Number as summarized by \citeauthor{Traum11} in 
\cite{Traum11} can in general be expressed as follows:

\begin{equation}
\frac{w_{e}}{V} \propto Ri^{-a}
\end{equation}

where a ranges from 1 to 2.  The value of 1.5 predominates among the water-tank experiments, and 
it is equal to 1 in the \acs{LES} studies.  Although, \citeauthor{FedConzMir04} perceive deviation
towards 1.5 at higher stability (or stronger inversion) in both theirs and \citeauthor{SullMoengStev}'s 
results.  This is in line with \citeauthor{Turner86}'s discussion in \cite{Turner86}.\\

\citeauthor{DearWill80} suggested the importance considering the scaled depth of the \acs{EL} in \acs{CBL} 
entrainment analysis or prediction and put forward a relationship between it and the Richardson Number
which has been further examined in subsequent related studies.  

\begin{equation}
\frac{\Delta h}{h} \propto Ri^{-b}
\end{equation}   

As summarized in \cite{Traum11} b ranges from .25 to 1.  (unclear what Brooks and Fowler got for either of these
relationships) Of the three principle \acs{LES} studies (\cite{BrooksFowler2}, \cite{FedConzMir04}, \cite{SullMoengStev})
one (\cite{FedConzMir04}) varied the upper lapse rate ($\gamma$).  However, it seems their grid spacing 
was large compared to those at wich solutions converged in \cite{SullPat}

(connectionn to models, sero, first order jump models, for example sull moengs section on how the 1rst order jump model
better matches the enntrainment rate)

(gather materials for statistics section)

\begin{table}[!ht]
    \begin{center}
    \begin{tabular}{ | l | p{5cm} | p{5cm} |}
    \hline
     Parameter & Definition & Explaination \\ \hline
     $h$&  height of maximum $\frac{\partial \overline{\theta}}{\partial z}$& \acs{CBL} height \\ \hline
     $h_{0}$& height where $\frac{\partial \overline{\theta}}{\partial z}$ first exceeds 0 by a threshold& lower \acs{EL} limit\\ \hline
     $h_{1}$& height where $\frac{\partial \overline{\theta}}{\partial z}$ resumes $\gamma$& upper \acs{EL} limit\\ \hline
     $\Delta h$& $h_{1} - h_{0}$ & \acs{EL} depth\\ \hline
     $\Delta \theta$ & $\overline{\theta}(h_{1})-\overline{\theta}(h_{0})$ & Temperature Jump over the \acs{EL}\\ \hline
     $h^{l}_{0}$& see section \label{sec:BLMeas}  & local \acs{ML} height\\ \hline
     $w^{*}$ & $\left( \frac{gh}{\overline{\theta}}\overline{w^{,}\theta^{,}} \right)$ & Convective Velocty Scale\\ \hline
     $\tau$ & $\frac{h}{w^{*}}$ & Time scale for plume to reach \acs{CBL} top\\ \hline
     $Ri_{g}$ & $\frac{g}{\overline{\theta_{ML}}} \frac{\gamma h^{2} }{w^{*2}}$ & Gradient Richardson Number for \acs{CBL} entrainment \\ \hline
     $Ri_{b}$ & $\frac{g}{\overline{\theta_{ML}}} \frac{\Delta \theta h }{w^{*2}}$ & Bulk Richardson Number for \acs{CBL} entrainment \\ \hline
     % \end

   
\end{tabular}
\caption{}
\label{fig:}   
\end{center}    
\end{table}
%\footnotetext{Incomplete run: EL exceded high resolution vertical grid after 7 hours}

(construct a table of the scales used and scaling relations considered in this thesis)


\endinput

Any text after an \endinput is ignored.
You could put scraps here or things in progress.
