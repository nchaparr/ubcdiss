%% The following is a directive for TeXShop to indicate the main file
%%!TEX root = diss.tex

\chapter{Introduction} 
\label{ch:Introduction}
\setlength{\parindent}{0cm}

\section{Motivation}
\label{sec:Mot}

The daytime convective atmospheric boundary layer (\acs{CBL}) over land starts to grow at sunrise when the surface becomes warmer than the air above it.  Coherent turbulent structures (thermals) begin to form and rise since their relative warmth causes them to be less dense and so buoyant.  The temperature profile of the residual nightime boundary layer is stable i.e. potential temperature ($\theta$) increases with height.  The thermals rise to their neutral bouoyancy level overshoot and then overturn or recoil concurrently dragging down warm stable air from above which is subsequently mixed into the growing turbulent mixed layer (\acs{ML}) (\citeauthor{Stull-BLMetIntro} \citeyear{Stull-BLMetIntro}).  This mixing at the top of the \acs{CBL} is known as entrainment and the region over which it occurs, the entrainment layer (\acs{EL}).\\

\acs{CBL} height (h) and the prediction thereof are important for calculating the concentration of any species as well as the sizes of the turbulent structures.  In combination with the lifting condensation level knowledge of \acs{EL} depth allows predictions pertaining to the formation of cumulous clouds.  For example cloud cover increases as more thermals rise above their lifting condensation level.  Parametrizations for both \acs{CBL} growth and \acs{EL} depth are required in mesoscale and general circulation models (\acs{GCM}s).  Furthermore it is an attractive goal to develop a robust set of scales for this region analogous to Monin-Obvukov Theory (\citeauthor{Stull-BLMetIntro} \citeyear{Stull-BLMetIntro}, \citeauthor{Traum11} \citeyear{Traum11}, \citeauthor{SteynBaldHoff} \citeyear{SteynBaldHoff}, \citeauthor{StullNelEl} \citeyear{StullNelEl}, \citeauthor{Sorbjan} \citeyear{Sorbjan})\\

Atmospheric \acs{CBL} entrainment has been studied as a separate phenomenon (BandF, SullMoeng, StulNelEl, Trau) as well as part of the broader topic of entrainment in geophysical flows (TurnerEl). There is broad agreement as to the fundamental scaling parameters and relationships involved.  However, their is ongoing discussion about how the important heights are defined and measured (bandf, trau) and the exact forms of the relationships (trau, fed, sull, turnel.)

%%%%%%%%%%%%%%%%%%%%%%%%%%%%%%%%%%%%%%%%%%%%%%%%%%%%%%%%%%%%%%%%%%%%%%

\section{Relevant Background}
\label{sec:}
\subsection{The Convective Boundary Layer (CBL)}

The \acs{CBL} starts to grow rapidly at sunrise, peaking at midday.  Convective turbulence and the dominant upward vertical motions then begin to subside as the surface cools.  While the surface is warm, buoyancy driven thermals of uniform potential temperature ($\theta$) and tracer concentration at their cores form and entrain surrounding air laterally as they rise, as well as trapping and mixing in stable warm from above (\citeauthor{Stull-BLMetIntro} \citeyear{Stull-BLMetIntro}, \citeauthor{CrumStullEl} \citeyear{CrumStullEl}).  Under conditions of strong convection, buoyantly driven turbulence dominates and shear is insignificant (\citeauthor{DirLEddy} \citeyear{DirLEddy}). Thermal overshoot relative to their neutral buoyancy level, and subsequent entrainment of the warmer air from aloft augments the warming caused by the surface heat flux ($\overline{w^{'}\theta^{'}}_{s}$) and results in a potential temperature jump ($\Delta \theta$) or inversion at the \acs{CBL} top (\citeauthor{SchmidtSchu} \citeyear{SchmidtSchu}, \citeauthor{Turner86} \citeyear{Turner86}).  There may also be a residual inversion from the day before possibly strengthened by subsidence, i.e. large scale downward movenent of warmer air from above (\citeauthor{Stull-BLMetIntro} \citeyear{Stull-BLMetIntro}, \citeauthor{SullMoengStev} \citeyear{SullMoengStev}).\\  

Lidar images show the overall structure of the \acs{CBL} with rising thermals, impinging on the air above (\citeauthor{CrumStullEl} \citeyear{CrumStullEl}, \citeauthor{Traum11} \citeyear{Traum11}).  This has been effectively modelled using large eddy simulation (\acs{LES}) by \citeauthor{SchmidtSchu} (\citeyear{SchmidtSchu}) who used horizontal slices of potential temperature and vertical velocity perturbations ($\theta^{'}$, $w^{'}$) at various vertical levels to show how the thermals form, merge and impinge at the \acs{CBL} top with concurrent peripheral downward motions.  The latter is supported in the visualizations of \citeauthor{SullMoengStev} (\citeyear{SullMoengStev}).  Vertical cross sections within the \acs{EL} show the relatively cooler thermals and trapped warmer air as well as the closely associated upward motion of cooler air and downward motion of warmer air.\\ 

On average these convective turbulent structures create a fully turbulent mixed layer (\acs{ML}) with eddie sizes cascading through an inertial subrange to the molecular scales at which energy is lost via viscous dissipation (\citeauthor{Stull-BLMetIntro} \citeyear{Stull-BLMetIntro}).  Here $\overline{\theta}$ is close to uniform and increases with respect to time due to $\overline{w^{'}\theta^{'}}_{s}$ and the flux of entrained stable air at the inversion ($\overline{w^{'}\theta^{'}}_{h}$).  \acs{ML} turbulence is dominated by warm updraughts and cool downdraughts.  With proximity to the top the updraughts become relatively cool and warmer air from above is drawn downward, so in the \acs{ML} $\overline{w^{'}\theta^{'}}$ is positive and decreasing.  Above the \acs{ML} the air becomes more stable with altitude and, on average, transitions from a uniform \acs{ML} potential temperature ($\frac{\partial \overline{\theta}}{\partial z} \approx 0$) to a stable lapse rate ($\gamma$).  A peak in the average vertical gradient ($\frac{\partial \overline{\theta}}{\partial z}$) at the inversion represents regions where thermals have exceeded their neutral buoyancy level. \\

\citeauthor{StullNelEl} (\citeyear{StullNelEl}) outline the stages of \acs{CBL} growth from when the sub-layers of the nocturnal boundary layer are entrained, until the previous day's capping inversion is reached and a quasi-steady growth is attained.  The \acs{EL} depth relative to \acs{CBL} height varies throughout these stages and its relationship to scaled entrainment is hysteresial.  Numerical studies typically represent this last quasi-steady phase involving a constant $\overline{w^{'}\theta^{'}}_{s}$ working against an inversion and or a stable $\gamma$ (\citeauthor{SchmidtSchu} \cite{SchmidtSchu}, \citeauthor{Sorbjan} \citeyear{Sorbjan}, \citeauthor{SullMoengStev} \citeyear{SullMoengStev}, \citeauthor{FedConzMir04} \citeyear{FedConzMir04}, \citeauthor{BrooksFowler2} \citeyear{BrooksFowler2}).  

\subsection{Convective Boundary Layer Height ($h$)}
\label{subsec:}

The \acs{ML} is fully turbulent with an on uniform average potential temperature ($\overline{\theta}$). Aerosol and water vapour concentrations decrease dramatically with transition to the stable upper free atmosphere (\acs{FA}).  So any of these characteristics can support
a definition of \acs{CBL} height ($h$).  \citeauthor{StullNelEl} (\citeyear{StullNelEl}) defined $h$ in terms of the percentage of \acs{ML} air
and identified it by eye from Lidar back-scatter images.  \citeauthor{Traum11} (\citeyear{Traum11}) compared four automated methods applied to Lidar images: a suitable threshold value above which the air is categorized as \acs{ML} air,  the point of minimum (largest negative) vertical gradient, the point of minimum vertical gradient based on a fitted idealized curve, 
and the maximum wavelet covariance. \acs{CBL} height detection is a wide and varied field.  Both \citeauthor{BrooksFowler2} (\citeyear{BrooksFowler2}) and \citeauthor{Traum11} \citeyear{Traum11} provide more thorough reviews.\\

Numerical models produce hundreds of local horizontal points
from which smooth averaged vertical profiles be obtained, and statistically robust relationships inferred. \citeauthor{BrooksFowler2} (\citeyear{BrooksFowler2}) applied a wavelet technique to identify the point of maximum covariance in a local vertical tracer profiles in their large eddy simulation (\acs{LES}) study.  They compared this method to the gradient method i.e. locating the point of maximum vertical gradient, as well as the point of minimum $\overline{w^{'}\theta^{'}}$.  This last definition is common among \acs{LES} and laboratory studies where it has been referred to as the inversion height (\citeauthor{DearWill80} \citeyear{DearWill80}, \citeauthor{Sorbjan1} \citeyear{Sorbjan1}, \citeauthor{FedConzMir04} \citeyear{FedConzMir04}).  \citeauthor{SullMoengStev} (\citeyear{SullMoengStev}) clarified that the extrema of the four $\overline{w^{'}\theta^{'}}$ quadrants ($\overline{w^{'+}\theta^{'+}}$), $\overline{w^{'-}\theta^{'+}}$, $\overline{w^{'+}\theta^{'-}}$, $\overline{w^{'-}\theta^{'-}}$) in the \acs{EL} more or less correspond to the average point of maximum $\frac{\partial \overline{\theta}}{\partial z}$, whereas the point of minimum $\overline{w^{'}\theta^{'}}$ was consistently lower. They defined \acs{CBL} height based on local $\frac{\partial \theta}{\partial z}$ and applied horizontal averaging as well as two methods based on $\overline{w^{'}\theta^{'}}$
for comparison.\\

So far, no published \acs{LES} study so far defines the height in terms of the $\overline{\theta}$ or $\frac{\partial \overline{\theta}}{\partial z}$ profile even though analyitical models, from which \acs{CBL} growth parametrizations stem, rely on an idealized version thereof. \citeauthor{GarciaMellado} \citeyear{GarciaMellado} do include it as one of their measures of \acs{CBL} height in their direct numerical
simulation (\acs{DNS}) study.       

\subsection{Convective Boundary Layer Growth by Entrainment}
\label{subsec:}

The \acs{CBL} grows by trapping pockets of warm stable air between
or adjacent to impinging thermal plumes.  \citeauthor{Traum11} (\citeyear{Traum11}) summarize two categories of \acs{CBL} entrainment:\\

\begin{itemize}

\item{Non turbulent fluid can be engulfed between or in the overturning of thermal plumes. This kind of event has been supported by the visualizations in \citeauthor{SullMoengStev}'s (\citeyear{SullMoengStev}) \acs{LES} study as well as in \citeauthor{Traum11}'s (\citeyear{Traum11}) observations. In both it appeared to occur under a weak inversion or upper lapse rate ($\gamma$)}

\item{
Impinging thermal plumes distort the inversion interface dragging wisps of warm stable air down at their edges or during recoil under a strong inversion or lapse rate. This type of event is supported by the findings  of both \citeauthor{SullMoengStev} (\citeyear{SullMoengStev}) and \citeauthor{Traum11} (\citeyear{Traum11}).}

\end{itemize}

Shear induced instabilities do occur at the top of the atmospheric boundary layer and in some laboratory studies of turbulent boundary layers, under conditions of very high stability, the breaking of internal waves have been observed.  Entrainment via, the former is relatively insignificant in strong convection, and the latter has not been directly observed in real or modeled atmospheric \acs{CBL}s over the range of \acs{Ri}s considered here (\citeauthor{Traum11} \citeyear{Traum11}, \citeauthor{SullMoengStev} \citeyear{SullMoengStev}).

\subsection{The Convective Boundary Layer Entrainment Layer}
\label{subsec:}

The \acs{ML} is fully turbulent but the top is characterised by stable air with intermittent turbulence due to the higher reaching thermals. \citeauthor{GarciaMellado} (\citeyear{GarciaMellado}) demonstrate that the \acs{EL} is subdivided in terms of length and buoyancy scales.  That is, the lower region is comprised of mostly turbulent air with pockets of stable warmer air that are quickly mixed, and so scales with the convective scales (see section \ref{sec:scales}). Whereas the upper region is mostly stable apart from the impinging thermals so scaling here is more influenced by the lapse rate ($\gamma$).  In the \acs{EL} the average vertical heat flux ($\overline{w^{'}\theta^{'}}$) switches sign relative to that in the \acs{ML}.  The fast updraughts are now relatively cool ($w^{'+}\theta^{'-}$).  In their analysis of the four $\overline{w^{'}\theta^{'}}$ quadrants \citeauthor{SullMoengStev} (\cite{SullMoengStev}) concluded that the overall net dynamic in this region is downward motion of warm air from the free atmosphere (\acs{FA}) ($\overline{w^{'-}\theta^{'+}}$) since the other three quadrants effectively cancel.\\

In terms of tracer concentration and for example based on a Lidar backscatter profile, there are two ways to conceptually define the entrainment layer (\acs{EL}).  It can be thought of as the range in space (or time) over which local \acs{CBL} height varies (\citeauthor{CrumStullEl} \citeyear{CrumStullEl}) or a local region over which the concentration (or back-scatter intensity) transitions from \acs{ML} to free atmospheric (\acs{FA}) values (\citeauthor{Traum11} \citeyear{Traum11}).  The latter can be estimated using either curve-fitting and wavelet techniques (\citeauthor{Traum11} \cite{Traum11}, \citeauthor{SteynBaldHoff} \cite{SteynBaldHoff}, \citeauthor{BrooksFowler2} \cite{BrooksFowler2}).\\

Although \citeauthor{BrooksFowler2} apply a wavelet technique to tracer profiles for the determination of \acs{EL} limits, in their \acs{LES} study (\cite{BrooksFowler2}) it is more common in numerical modelling and laboratory studies for the \acs{EL} limits to be defined based on the average vertical turbulent heat flux ($\overline{w^{'}\theta^{'}}$) i.e. the point at which it goes from positive to negative values, and the point at which it goes from a negative value to zero as shown in Figure \ref{fig:1stord} (\citeauthor{DearWill80} \citeyear{DearWill80}, \citeauthor{FedConzMir04} \citeyear{FedConzMir04}, \citeauthor{GarciaMellado} \citeyear{GarciaMellado}). Analytical models based on the representation in Figure \ref{fig:1stord} assume the region of negative $\overline{w^{'}\theta^{'}}$ coincides with the region where $\overline{\theta}$ transitions from the \acs{ML} value to the \acs{FA} value. (\citeauthor{Deardorff79} \citeyear{Deardorff79}, \citeauthor{FedConzMir04} \citeyear{FedConzMir04}) but no modelling studies use the vertical $\overline{\theta}$ profile to define the (\acs{EL}).\\

Since $\overline{\theta}$ modeled by an \acs{LES} is not strictly constant with respect to height in the \acs{ML} (\citeauthor{FedConzMir04} \citeyear{FedConzMir04}), a threshold value for $\overline{\theta}$ or its vertical gradient must be chosen to identify the lower \acs{EL} limit.  In their \citeyear{BrooksFowler2} \acs{LES} study \citeauthor{BrooksFowler2} encountered inconsistencies when determining the \acs{EL} limits from the average tracer profile.  Although their tracer profile was quite different to a simulated \acs{CBL} $\overline{\theta}$ profile, this could serve as cautionary note.\\             

Our understanding of the the characterstics and dynamics of the atmospheric \acs{CBL} entrainment layer (\acs{EL}) evolves with the increasing body of measurement (\citeauthor{Traum11} \citeyear{Traum11}, \citeauthor{StullNelEl} \citeyear{StullNelEl}), laboratory (\citeauthor{DearWill80} \citeyear{DearWill80}) and numerical studies (\citeauthor{Deardorff72} \citeyear{Deardorff72}, \citeauthor{Sorbjan} \citeyear{Sorbjan}, \citeauthor{SullMoengStev} \citeyear{SullMoengStev}, \citeauthor{FedConzMir04} \citeyear{FedConzMir04}, \citeauthor{BrooksFowler2} \citeyear{BrooksFowler2}, \citeauthor{GarciaMellado} \citeyear{GarciaMellado}). Parametrizations are derived based on analytical models and are validated using \acs{LES} output and measurements (\citeauthor{FedConzMir04} \citeyear{FedConzMir04}, \citeauthor{Boers89} \citeyear{Boers89}).  So the relationship between theory, numerical simulation and measurement is inextricable and any study based on one must refer to at least one of the others.\\  

%%%%%%%%%%%%%%%%%%%%%%%%%%%%%%%%%%%%%%%%%%%%%%%%%%%%%%%%%%%%%%%%%%%%%%
\subsection{Modelling the Convective Boundary Layer and Entrainment Layer}
\label{subsec:}

\subsubsection{Bulk Analytical Models}
\label{subsubsec:}
Bulk analytical models for the Convective Boundary layer (\acs{CBL}) derived based on average, vertical profiles
of \acs{ML} quantities can be subdivided into: (i) zero order as represented in Figure \ref{fig:0order} and 
(ii) first (and higher) order bulk models as represented in Figure \ref{fig:1stord}. The order refers to the 
shape of the assumed \acs{EL}.\\

\begin{figure}[htbp]
    \centering
    %plot_height.py[master 1573b9d] h vs time plot
    \includegraphics[scale=.5]{/newtera/tera/phil/nchaparr/python/Plotting/Dec252013/pngs/zero_order.pdf}
    \caption{A further simplified version of Figure \ref{fig:1stord}.  The \acs{EL} is infinitessimely thin.}
    \label{fig:0order}   % label should change
\end{figure}

Zero order bulk models assume a mixed layer (\acs{ML}) of uniform potential temperature ($\overline{\theta}_{ML}$) 
topped by an infinitesimally thin layer across which there is a temperature jump ($\delta \theta$) and above which 
is a constant lapse rate ($\gamma$).  The assumed average vertical turbulent heat flux ($\overline{w^{'}\theta^{'}}$) 
decreases linearly from the surface up, reaching a maximum negative value ($\overline{w^{'}\theta^{'}}_{h}$) which  
is a constant proportion of the surface value (usually $-.2\overline{w^{'}\theta^{'}}_{h}$) at the temperature inversion, 
and decreasing to zero accross the jump.  Equations for the evolution of \acs{CBL} height, $\overline{\theta}_{ML}$ 
and $\Delta \theta$ are derived on this basis (\citeauthor{Tennekes73} \citeyear{Tennekes73}).\\

If the \acs{CBL} height ($h$) is rising, air is being drawn in from the stable layer above and cooled i.e. it is
decreasing in enthalpy.  The rate of decrease in enthalpy with respect to time is $c_{p}\rho\Delta \theta \frac{dh}{dt}$ 
per unit of horizontal area.  Since the lapse rate above the inversion is stable \citeauthor{Tennekes73} 
(\citeyear{Tennekes73}) equates this enthalpy loss to the average vertical turbulent heat flux at the inversion

\begin{equation}
\Delta \theta \frac{dh}{dt} = -\overline{w^{'}\theta^{'}}_{h} 
\end{equation}.  

The \acs{ML} warming rate is arrived at via the simplified Reynolds averaged conservation of enthalpy, for which
the full derivation is shown in an appendix

\begin{equation}
\frac{\partial \overline{\theta}_{ML}}{\partial t} = -\frac{\partial}{\partial z}\overline{w^{'}\theta^{'}}
\end{equation}.

Assuming $\overline{w^{'}\theta^{'}}$ has a constant slope this becomes

\begin{equation}
\frac{\partial \overline{\theta}_{ML}}{\partial t} = \frac{\overline{w^{'}\theta^{'}}_{s}-\overline{w^{'}\theta^{'}}_{h}}{h}
\end{equation},

and the evolution of the temperature jump ($\delta \theta$) depends on the rate of \acs{CBL} height ($h$) increase, 
the upper lapse rate $\gamma$ and the \acs{ML} warming rate
  
\begin{equation}
\frac{d\delta \theta}{dt} = \gamma\frac{dh}{dt} - \frac{d\overline{\theta}_{ML}}{dt}.
\end{equation}

An assumption about the vertical heat flux at the inversion ($h$), such as the entrainment ratio, closes this set

\begin{equation}
\frac{\overline{w^{'}\theta^{'}}_{h}}{\overline{w^{'}\theta^{'}}_{s}} = -.2
\end{equation}.\\

The relevant quantities in equations 2.2 through 2.6 are idealized, ensemble averages. There is some variation within this class of model, for example the rate equation for $h$ (entrainment relation) can alternatively be derived based on the turbulent kinetic energy budget 
(\citeauthor{FedConzMir04} \cite{FedConzMir04}).  But they are all based on the simplified $\overline{\theta}$ and 
$\overline{w^{'}\theta^{'}}$ profiles outlined above.\\  

\begin{figure}[htbp]
    \centering
    %plot_height.py[master 1573b9d] h vs time plot
    \includegraphics[scale=.5]{/newtera/tera/phil/nchaparr/python/Plotting/Dec252013/pngs/first_order.pdf}
    \caption{A further simplified version of Figure \ref{fig:1stord}.  The \acs{EL} is infinitessimely thin.}
    \label{fig:1storder}   % label should change
\end{figure}


First (and higher) order models assume an entrainment layer (\acs{EL}) of finite depth at the top of the ML, defined by two heights:
the top of the ML ($h_{0}$) and the point where free atmospheric characteristics are resumed ($h_{1}$).  The derivations are more complex and 
involve assumptions about the \acs{EL} i.e.: 

\begin{itemize}
\item{$\Delta h = h_{1} - h_{0} = Constant$ (\citeauthor{Betts74} \citeyear{Betts74})}

\item{$\Delta h = h_{1} - h_{0}$ is related to the zero-order jump at $h$ by two right angled triangles with opposite sides
of lengths $h_{1} - h$ and $h - h_{0}$ (\citeauthor{BatchGryn} \citeyear{BatchGryn})}

\item{$\Delta h$ or maximum overshoot distance $d \propto \frac{w^{*}}{N}$ where $w^{*}$ is the convective vertical velocity scale to be defined in section \ref{subsec:convel} and $N = \sqrt{\frac{g}{\overline{\theta}} \frac{\partial \overline{\theta}}{\partial z}}$ is the Brunt-Vaisala frequency (\citeauthor{Stull73} \citeyear{Stull73})}
 
\item{For $h_{0}<z<h_{1}$ $\overline{\theta} = \overline{\theta}_{ML} + f(z,t) \Delta \theta$ where $f(z,t)$ is a dimensionless shape factor (\citeauthor{Deardorff79} \citeyear{Deardorff79}, \citeauthor{FedConzMir04} \citeyear{FedConzMir04})}
\end{itemize}
 .\\

Although development of these models is beyond the scope of this thesis, they are mentioned to give context to the parametrizations considered. \\         

\subsubsection{Numerical Simulations}
\label{subsubsec:}

Numerical simulation of the convective boundary layer (\acs{CBL}) is carried out by solving the Navier Stokes equations, simplified according to a suitable approximation, on a discrete grid.  Types of simulations can be grouped according to the scales of motion they resolve.  In direct numerical simulations (\acs{DNS}) the full range of spatial and temporal turbulence are resolved from the size of the domain down to the smallest dissipative scales i.e. the Kolmagorov micro-scales.  This requires a dense numerical grid and so can be computationally prohibitive.\\

In a large eddy simulation (\acs{LES}) smaller scales are filtered out and parametrized by sub grid scale closure model. General circulation models (\acs{GCM}) solve the Navier Stokes equations on a spherical grid and parametrize smaller scale processes including convection and cloud cover.  \acs{LES} has steadily, repeatedly been used to better understand the \acs{CBL} since \citeauthor{Deardorff72} (\citeyear{Deardorff72}) applied this relatively new method for this purpose.  \citeauthor{SullMoengStev} (\citeyear{SullMoengStev}), \citeauthor{FedConzMir04} (\citeyear{FedConzMir04}) and \citeauthor{BrooksFowler2} in (\citeyear{BrooksFowler2}) used it to observe the structure and scaling behaviour of the \acs{EL}.\\

%%%%%%%%%%%%%%%%%%%%%%%%%%%%%%%%%%%%%%%%%%%%%%%%%%%%%%%%%%%%%%%%%%%%%%
\subsection{Scales of the \acs{CBL} and \acs{EL}}
\label{subsec:scales}

\subsubsection{Length Scale ($h$)}
\label{subsubsec:}

\citeauthor{Deardorff72} (\citeyear{Deardorff72}) demonstrated that dominant turbulent structures in penetrative convection
scale with inversion height which was taken to be the height of minimum average vertical heat flux:$z_{f}$ 
(\citeauthor{DearWill80} \citeyear{DearWill80}).  Since then, the definition of \acs{CBL} height ($h$) has 
remained  unchanged i.e. it is the height of the inversion or the vertical point,
above the surface layer, of maximum change in tracer concentration or potential temperature ($\theta$). There are
alternatives. For example turbulence based definitions, such as the velocity variance and the distance over which 
velocity is correlated with itself, represent the current turbulent dynamics rather than the recent turbulence history
as does $h$ (\citeauthor{Traum11} \citeyear{Traum11}).\\

\subsubsection{Convective Velocity Scale ($w^{*}$)}
\label{subsubsec:convel}

Given an average surface vertical heat flux ($\overline{w^{'}\theta^{'}}_{s}$) a surface buoyancy flux can be defined as 
$\frac{g}{\overline{\theta}}\overline{w^{'}\theta^{'}}_{s}$ from which the convective velocity scale is obtained by
multiplying by the appropriate length scale.  Since the result is in $\frac{m^{3}}{s^{3}}$ a cube root is applied\\

\begin{equation}
w^{*} = \left( \frac{gh}{\overline{\theta}}\overline{w^{'}\theta^{'}}_{s} \right)^{\frac{1}{3}}
\end{equation}.\\

\citeauthor{Deardorff70} (\citeyear{Deardorff70}) confirmed that this effectively scaled the local vertical turbulent velocity
perturbations ($w^{'}$) in the \acs{CBL}.  \citeauthor{Sorbjan}'s (\citeyear{Sorbjan}) work  supports this, even at the 
\acs{CBL} top.  $\frac{dh}{dt}$ and $w^{'}$ are driven by $\overline{w^{'}\theta^{'}}_{s}$ and inhibited by  $\gamma$. 
The influence of $\gamma$ on $w^{'}$ is indirectly accounted for via $h$ in $w^{*}$.\\

\subsubsection{Convective Time Scale ($\tau$)}
\label{subsubsec:}

It logically follows that the time which a thermal travelling at velocity 
$w^{*}$ takes to reach the top of the \acs{CBL}, i.e. travel a distance $h$ is

\begin{equation}
\tau = \frac{h}{\left( \frac{gh}{\overline{\theta}}\overline{w^{'}\theta^{'}} \right)^{\frac{1}{3}}}
\end{equation}.

 This is also referred to as the convective overturn time scale.  \citeauthor{SullMoengStev} (\citeyear{SullMoengStev}) 
showed a linear relationship between $h$ and time scaled by $\tau$ . An alternative is the Brunt-Vaisala frequency i.e. the time scale
 associated with the buoyant thermals overshooting and sinking (\citeauthor{FedConzMir04} \cite{FedConzMir04}).  The ratio of these two time-scales forms a parameter which characterizes this system (see \citeauthor{Sorbjan}\citeyear{Sorbjan} and \citeauthor{Deardorff79} \citeyear{Deardorff79}). 

\subsubsection{Temperature Scales ($\theta^{*}$ and $\gamma$)}
\label{subsubsec:}

The \acs{CBL} temperature fluctuations $\theta^{'}$ are influenced by $\overline{w^{'}\theta^{'}}$ from both the surface and the \acs{CBL} top.
\citeauthor {Deardorff70} (\citeyear{Deardorff70}) showed that an effective scale based on the convective velocity scale is

\begin{equation}
\theta^{*} = \frac{\overline{w^{'}\theta^{'}}_{s}}{w^{*}}
\end{equation} 

Whereas \citeauthor{Sorbjan} (\citeyear{Sorbjan}) showed that with proximity to the \acs{CBL} top the effects of $\gamma$ become more important.
 
\subsubsection{Buoyancy Richardson Number (\acs{Ri})}
\label{subsubsec:}

The flux Richardson ($R_{f}$) number expresses the balance between mechanical and buoyant production of turbulent kinetic energy (\acs{TKE}) and is obtained from the ratio of these two terms in the \acs{TKE} budget equation (\citeauthor{Stull-BLMetIntro} \citeyear{Stull-BLMetIntro}):

\begin{equation}
\frac{\partial \overline{e}}{\partial t} + \overline{U}_{j} \frac{\partial \overline{e}}{\partial x_{j}} = \delta_{i3}  \frac{g}{\overline{\theta}} \left( \overline{u_{i}^{'}\theta^{'}} \right) - \overline{u_{i}^{'}u_{j}^{'}}\frac{\partial \overline{U}_{i}}{\partial x_{j}} - \frac{ \partial \left( \overline{u_{j}^{'}e^{'}} \right)}{\partial x_{j}} - \frac{1}{\overline{\rho}} \frac{\partial \left( \overline{u_{i}^{'} p^{'}} \right) }{\partial x_{i}} - \epsilon
\end{equation}

\begin{equation}
R_{f} = \frac{\frac{g}{\overline{\theta}} \left( \overline{w^{'}\theta^{'}} \right)}{\overline{u_{i}^{'}u_{j}^{'}}\frac{\partial \overline{U}_{i}}{\partial x_{j}}}
\end{equation}.
 
Assuming horizontal homogeneity and neglecting subsidence yields
  
\begin{equation}
R_{f} = \frac{\frac{g}{\overline{\theta}} \left( \overline{w^{'}\theta^{'}} \right)}{\overline{u^{'}w^{'}}\frac{\partial \overline{U}}{\partial z} + \overline{v^{'}w^{'}}\frac{\partial \overline{V}}{\partial z}}
\end{equation}.

Applying first order closure to the flux terms, i.e. assuming they are proportional to the vertical gradients, gives the gradient Richardson number ($R_{g}$)

\begin{equation}
R_{g} = \frac{ \frac{g}{\overline{\theta}} \frac{\partial \overline{\theta}}{\partial z}}{\left( \frac{ \partial \overline{U}}{\partial z} \right)^{2} + \left( \frac{\partial \overline{V}}{\partial z} \right)^{2}} 
\end{equation},

which expresses the balance between shear and buoyancy production of \acs{TKE}.  However, in the \acs{EL} buoyancy acts to suppress buoyant production of \acs{TKE}.  Applying a bulk approximation to the denominator, and expressing it in terms of scales yields a squared ratio of two time scales

\begin{equation}
R_{g} = \frac{\frac{g}{\overline{\theta}} \frac{\partial \overline{\theta}}{\partial z}}{\frac{U^{*2}}{L^{2}}} = N^{2}\frac{L^{2}}{U^{*2}}
\end{equation},


where $U^{*}$ and $L^{*}$ are appropriate velocity and length scales.  Applying the bulk approximation to both the numerator and denominator yields
the bulk Richardson number:

\begin{equation}
R_{b} = \frac{\frac{g}{\overline{\theta}} \Delta \theta L^{*}}{U^{*2}}
\end{equation}.

A natural choice of length and velocity scales for the \acs{CBL} are $h$ and $w^{*}$ giving the convective or buoyancy Richardson number:

\begin{equation}
Ri = \frac{\frac{g}{\overline{\theta}} \Delta \theta h}{w^{*2}}
\end{equation}.

\acs{Ri} can be arrived at by considering the principal forcings of the system, or from non-dimensionalizing the entrainment relation  derived analytically (\citeauthor{Tennekes73}  \citeyear{Tennekes73}, \citeauthor{Deardorff72} \citeyear{Deardorff72}). It is central to any study on \acs{CBL} entrainment (\citeauthor{SullMoengStev} \citeyear{SullMoengStev}, \citeauthor{FedConzMir04} \citeyear{FedConzMir04}, \citeauthor{Traum11} \citeyear{Traum11}, \citeauthor{BrooksFowler2} \citeyear{BrooksFowler2}).

\subsubsection{Relationship of Entrainment Rate to Richardson Number}

The relationship between scaled entrainment rate and the buoyancy Richardson number (\acs{Ri}) is arrived at according the zero order bulk
model through thermodynamic arguments, or by integration of the conservation of enthalpy or turbulent kinetic energy equations
over the growing \acs{CBL}. (\citeauthor{Tennekes73} \cite{Tennekes73}, \citeauthor{Deardorff79} \cite{Deardorff79}, 
\citeauthor{FedConzMir04} \cite{FedConzMir04}). 

\begin{equation}
\frac{w_{e}}{w^{*}} \propto  Ri^{-a}
\end{equation}

It has been verified in numerous laboratory and numerical studies (\citeauthor{DearWill80} \cite{DearWill80}, \citeauthor{SullMoengStev} \cite{SullMoengStev}, \citeauthor{FedConzMir04} \cite{FedConzMir04}, \citeauthor{BrooksFowler2} \cite{BrooksFowler2}).  But there is still some 
unresolved discussion as the the exact value of a.  It seems there are two possible values, $-\frac{3}{2}$ and $-1$, the first of which \citeauthor{EllTurn} (\citeyear{EllTurn}) suggested occurs at high stability when buoyant recoil of impinging thermals becomes more important than their convective overturning. Assume that an impinging thermal supplies kinetic energy per unit time and per unit area for entrainment, in terms of appropriate length and timescale $L^{*}$ and $t^{*}$ as follows 

\begin{equation}
K \propto \frac{\overline{\rho} L^{*3} U^{*2}}{L^{*2} t^{*}}
\end{equation},

and that the corresponding change in potential energy per unit time and area of the rising \acs{CBL} is

\begin{equation}
\Delta P \propto g \Delta \rho h \frac{d h}{ dt}  
\end{equation}.

If $L^{*}$ is the penetration depth of the thermals travelling at velocity $w^{*}$ against a decelerating force
$g \frac{\Delta \rho}{\overline{\rho}}$

\begin{equation}
L^{*} = \frac{w^{*2} \overline{\rho}}{\Delta \rho}  
\end{equation}

and the $t^{*}$ is the response time of the inversion layer to a thermal of length $h$ is

\begin{equation}
t^{*} = \sqrt{h \frac{\overline{\rho}}{g \Delta \rho}}  
\end{equation}

then assuming all of the kinetic energy ($K$) is transferred to the change in potential energy ($\Delta P$) and
using the \acs{CBL} veloctiy, yields

%\begin{equation}
%\frac{\overline{\rho} L^{*} U^{*2}}{t^{*}} \propto g \Delta \rho h \frac{dh}{dt}
%\end{equation},

%\begin{equation}
%\frac{\overline{\rho} \frac{w^{*2} \overline{\rho}}{\Delta \rho} U^{*2}}{\sqrt{h \frac{\overline{\rho}}{g \Delta \rho}}} \propto g \Delta \rho h \frac{dh}{dt}
%\end{equation},

%\begin{equation}
%\frac{\overline{\rho} \frac{w^{*2} \overline{\rho}}{\Delta \rho} U^{*2}}{\sqrt{h \frac{\overline{\rho}}{g \Delta \rho}} g \Delta \rho h} \propto  \frac{dh}{dt}
%\end{equation}

\begin{equation}
\frac{\frac{dh}{dt}}{w^{*}} \propto \frac{w^{*2} \overline{\rho}}{g \Delta \rho h} \sqrt{\frac{\overline{\rho} w^{*}}{g \Delta \rho h}}
\end{equation},

i.e.

\begin{equation}
\frac{w_{e}}{w^{*}} \propto Ri^{-\frac{3}{2}}
\end{equation}.

Adding further complexity to this discussion, \citeauthor{FedConzMir04} (\citeyear{FedConzMir04}) suggest that this power law relationship ($a = -\frac{3}{2}$) is arrived at through defining the $\theta$ jump accross the \acs{EL} (see Figure \ref{fig:1stord}).\\

\subsubsection{Relationship of Entrainment Layer Depth to Richardson Number}

A relationship of the scaled entrainment layer \acs{EL} depth to \acs{Ri} is arrived at by considering the deceleration of a thermal
as it overshoots its neutral buoyancy level (\citeauthor{StullNelEl} \citeyear{StullNelEl}).  If the velocity of the thermal is assumed to be
proportional to $w^{*}$ and the decelerating force is due to the buoyancy difference, or $\theta$ jump, then the distance the thermal overshoots
($d$) can be approximated by

\begin{equation}
d \propto \frac{w^{*2}}{\frac{g}{\overline{\theta}_{ML}} \Delta \theta} 
\end{equation}. 

If the \acs{EL} depth is proportional to the overshoot distance ($d$) then

\begin{equation}
\frac{\Delta h}{h} \propto \frac{w^{*2}}{\frac{g}{\overline{\theta}_{ML}} h \Delta \theta} = Ri^{-1} 
\end{equation} 

Alternitavely, \citeauthor{Boers89} \citeyear{Boers89} integrated the internal ($U$), potential ($P$) and kinetic ($K$) energy over a hydrostatic atmosphere

\begin{equation}
U = \frac{c_{v}}{g}\int^{p_{0}}_{0}Tdp
\end{equation},

\begin{equation}
P = \frac{R}{c_{v}}U
\end{equation}

and

\begin{equation}
K = \frac{1}{2} \int^{p_{0}}_{0}\frac{w^{2}}{g}dp
\end{equation}.

$p_{0}$ is the pressure the surface pressure, $R$ and $c_{v}$ are the gas constant and heat capacity of dry air at constant volume.
$T$ is temperature.  Initially there is a flat infinitessimely thin inversion interface  which is distorted by an
impinging thermal.  The resulting height difference is assumed sinusoidal and an average $\Delta h$ is obtained by integrating 
over a wavelentgh.  At this point, no entrainment is assumed to have occurred and all of the initial kinetic energy ($K_{i}$) has been transferred to the change in potential energy ($\Delta P$).

\begin{equation}
K_{i} = P_{f} - P_{i} = \Delta P
\end{equation}

Assuming a dry adiabatic atmosphere and that the vertical velocity in the layer below the inversion can be approximated by the convective velocity scale ($w^{*}$), the following expression is reached

\begin{equation}
\left(\frac{\Delta h}{h}\right)^{2} \propto \frac{T_{0} w^{*2}}{g \Delta \theta h}
\end{equation}

The reference temperature, $T_{0}$, can be replaced by $\overline{\theta}_{ML}$ to give

\begin{equation}
\frac{\Delta h}{h} \propto Ri^{-\frac{1}{2}}
\end{equation}.


%%%%%%%%%%%%%%%%%%%%%%%%%%%%%%%%%%%%%%%%%%%%%%%%%%%%%%%%%%%%%%%%%%%%%%

\section{Research Questions}
\label{sec:}

An effective, simplified conceptual model of the dry, shear-free \acs{CBL} in the absence of large scale winds is represented in Figure \ref{fig:1stord}.  Turbulent mixing via upward moving thermals and corresponding cooler dowdraughts renders the average potential temperature $\overline{\theta}$ effectively constant within the \acs{ML}.  The average vertical turbulent heat flux ($\overline{w^{'}\theta^{'}}$) is positive and decreases as the thermals approach the \acs{CBL} top.  In the \acs{EL} there is a mixture of turbulent thermals and relatively warmer stable air.  The proportion of the latter increasing with proximity to the free atmosphere (\acs{FA}) above.  In the \acs{EL} $\overline{w^{'}\theta^{'}}$ becomes negative as the thermals, which are now relatively cool, impinge on the \acs{FA} and turn downward pulling warmer air with them.  The dimensionless parameter that represents the forcings under these conditions, and is ubiquitous in the corresponding literature, is the convective or buoyancy Richardson number: \acs{Ri} $=\frac{\frac{g}{\overline{\theta}_{ML}}\Delta \theta h}{w^{*2}}$ ( \citeauthor{DearWill80} \citeyear{DearWill80}, \citeauthor{Stull-BLMetIntro} \citeyear{Stull-BLMetIntro}, \citeauthor{SullMoengStev} \citeyear{SullMoengStev}, \citeauthor{FedConzMir04} \citeyear{FedConzMir04}, \citeauthor{BrooksFowler2} \citeyear{BrooksFowler2}, \citeauthor{GarciaMellado} \citeyear{GarciaMellado}).  $h$ is the \acs{CBL} height and $w^{*}$ is the convective velocity scale (\citeauthor{Deardorff70} \citeyear{Deardorff70}) to be defined later.\\   


The two principal external parameters in this simplified case, i.e. the dry, shear-free \acs{CBL} in the absence of large scale winds, are the average vertical turbulent surace heat flux ($\overline{w^{'}\theta^{'}}_{s}$) and the upper lapse rate ($\gamma$) (\citeauthor{FedConzMir04} \citeyear{FedConzMir04},\citeauthor{Sorbjan} \citeyear{Sorbjan}).  They have opposing effects, that is $\overline{w^{'}\theta^{'}}_{s}$ drives upward turbulent velocity ($w^{'+}$) and so \acs{CBL} growth ($w_{e}$) wheras $\gamma$  suppresses it.  Conversely they both cause positive turbulent potential temperature perturbations ($\theta^{'+}$) and so warming of the \acs{CBL}.  In the \acs{EL} the thermals from the surface are now relatively cool.  They turn downwards as they interact with the stable \acs{FA} concurrently bringing down warmer stable are from above.  \citeauthor{SullMoengStev} (\citeyear{SullMoengStev}) demonstrated these dynamics by partitioning $w^{'} \theta^{'}$ into four quadrants:  upward moving warm ($w^{'+} \theta^{'+}$), downward moving warm ($w^{'-} \theta^{'+}$), upward moving cool ($w^{'+} \theta^{'-}$) and downward moving cool ($w^{'-} \theta^{'-}$).  \citeauthor{Sorbjan} \citeyear{Sorbjan} asserted and showed that in this region the potential temperature perturbations ($\theta^{'}$) are strongly influenced by $\gamma$ whereas the turbulent vertical velocity peturbations ($w^{'}$) are almost independent thereof. Inspired by these two studies and to gain some insight into the dynamics of this idealized \acs{CBL} I ask \textbf{Q1: How does the distrubutions of local \acs{CBL} height, and the joint distrubutions of $w^{'}$ and $\theta^{'}$ within the \acs{EL}, vary with $\overline{w^{'}\theta^{'}}_{s}$ and $\gamma$?}\\

The relationship between scaled \acs{EL} depth and \acs{Ri} 

\begin{equation}
\frac{\Delta h}{h} \propto  \acs{Ri}^{b}
\end{equation}

has been explored and justified in field measurement, laboratory, numerical studies.  There is disagreement with respect to its exact form, in part stemming from variation in height and $\theta$ jump defininitions, but general its magnitude relative to h decreases with increasing \acs{Ri}. Although referred to in most relevant studies to and relied upon in analytical models, the average potential temperature profile has not been used to define the \acs{EL} (\citeauthor{DearWill80} \citeyear{DearWill80}, \citeauthor{StullNelEl} \citeyear{StullNelEl}, \citeauthor{FedConzMir04} \citeyear{FedConzMir04}, \citeauthor{Boers89} \citeyear{Boers89}, \citeauthor{BrooksFowler2} \citeyear{BrooksFowler2}). This leads me to ask \textbf{Q2: Can the \acs{EL} limits be defined based on the $\overline{\theta}$ profile and what is the relationship of the resulting depth ($\Delta h$) to \acs{Ri}?}\\


A further simplification to the dry, shear-free, \acs{CBL} model without large scale velocities, is to regard the \acs{EL} depth as infinitessimly small as in Figure \ref{fig:0order}.  The relationship of the scaled, time rate of change of $h$ (entrainment rate: $w_{e}$) to \acs{Ri} can be derived based on this model (\citeauthor{Tennekes73} \citeyear{Tennekes73}, \citeauthor{Deardorff79} \citeyear{Deardorff79}, \citeauthor{FedConzMir04} \citeyear{FedConzMir04})

\begin{equation}
\frac{w_{e}}{w^{*}} \propto  \acs{Ri}^{a}
\end{equation}.
 
This will be referred to as the entrainment relation.  Although that there is such a relationship is well esablished, discussion as to the power exponent of \acs{Ri} is unresolved and results from studies justify values of both $-\frac{3}{2}$ and $-1$. See \citeauthor{Traum11} (\citeyear{Traum11}) for a summary and review.  \citeauthor{Turner86} (\citeyear{Turner86}) explains this disparity in terms of entrainment mechanism such that the higher value occurrs when thermals recoil rather than overturn in response to a stronger $\theta$ jump (or inversion).  Whereas \citeauthor{SullMoengStev} (\citeyear{SullMoengStev}) notice a deviation from the lower power ($-1$) at lower \acs{Ri} and attribute it to the effect of a the shape of $\overline{\theta}$ within a thicker \acs{EL}.  Both \citeauthor{FedConzMir04} (\citeyear{FedConzMir04}) and \citeauthor{GarciaMellado} (\citeyear{GarciaMellado}) show how the definition of the $\theta$ jump influences the time rate  of change of \acs{Ri} and so effects $a$. \textbf{Q3: How does defining the $\theta$ jump based on the vertical $\overline{\theta}$ accross the \acs{EL} as in Figure \ref{fig:1stord} vs at the inversion ($h$) as in Figure \ref{fig:0order}, effect the entrainment relation and in particular $a$?}\\


\section{Approach to Research Questions}
\label{sec:Approach}

Similar to \citeauthor{SullMoengStev} \citeyear{SullMoengStev}, \citeauthor{FedConzMir04} \citeyear{FedConzMir04} and \citeauthor{BrooksFowler2} \citeyear{BrooksFowler2} I will model the dry shear free \acs{CBL} and \acs{EL} using \acs{LES}, specifically the cloud resolving model System for Atmospheric Modelling (SAM) to be outlined in Chapter 3.  I will use a slightly smaller domain than usuall (3.2x4.8 Km2 vs 5x5Km2) but will run a 10 case ensemble to obtain true ensemble averages and turbulent potential temperature perturbations ($\theta^{'}$). Grid spacing was influenced by \citeauthor{SullPat} (\citeyear{SullPat}) and the vertical grid within the \acs{EL} is of higher resolution than the other comparable studies. Before addressing the questions stated in Section \ref{sec:Mot} I'll examine the modeled output to make sure it represents a realistic turbulent \acs{CBL} in Chapter 3 section 2. I will verify that the averaged vertical profiles are as expected and coherent thermals are being produced.  FFT energy density spectra will show if there is adequate scale separation between the most energy intense structures and the grid spacing, and that realistic, isotropic turbulence is being modelled.  The runs will be initiallized with a constant $\overline{w^{'}\theta^{'}}_{s}$ acting a against a uniform $\gamma$.  So here, the  $\theta$ jump arises from the overshoot of the thermals, rather than being initially imposed as in \citeauthor{SullMoengStev} (\citeyear{SullMoengStev}) and \citeauthor{BrooksFowler2} (\citeyear{BrooksFowler2}).  

\subsection{Q1: How do the distrubutions of local \acs{CBL} height, and $w^{'}\theta^{'}$ within the \acs{EL}, vary with $\overline{w^{'}\theta^{'}}_{s}$ and $\gamma$?}     

The \acs{EL} can be thought of in terms of the distribution of indivual thermal heights, or local heights. \citeauthor{SullMoengStev} (\citeyear{SullMoengStev}) measured local height by locating the vertical point of maximum $\theta$ gradient, and observed the effects of varying \acs{Ri} on the resulting distributions. However this method is problematic when gradients in the upper profile exceed that at the inversion (\citeauthor{BrooksFowler2} \citeyear{BrooksFowler2}).  \citeauthor{SteynBaldHoff} (\citeyear{SteynBaldHoff}) fitted an idealized curve to a lidar backscatter profile.  This method produces a smooth curve based on the full original profile on which a mximum can easily be located.  I will apply a multi-linear regression method outlined in \citeauthor{Vieth} (\citeyear{Vieth}) to the local $\theta$ profile, representing the \acs{ML}, \acs{EL} and \acs{FA} each with a separate line. From this fit, I will locate the \acs{ML} top.  I'll observe how the resulting distributions are effected by changes in $\overline{w^{'}\theta^{'}}_{s}$ and $\gamma$ using histograms and verify that the resulting surface corresponds to the location of the termals in the \acs{EL}, in Chapter 3 Section 3.\\

\citeauthor{SullMoengStev} (\citeyear{SullMoengStev}) broke the turbulent vertical heat flux ($w^{'}\theta^{'}$) into four quadrants and used this combined with local flow visualizations to show how thermals impinge and draw down warm air from above. \citeauthor{MahrtPaum} (\citeyear{MahrtPaum}) used 2 dimensional contour plots of local $w^{'}$ and $\theta^{'}$ measurements to analyse their joint distribubions.  In his \citeyear{Sorbjan} \acs{LES} study \citeauthor{Sorbjan} showed that in the \acs{EL} $\theta^{'}$ is strongly influenced by $\gamma$  whereas $w^{'}$ is independent thereof.  Influenced by these three studies, I will use 2 dimensional histograms at three levels within the \acs{EL} to look at how the distrubutions of local $w^{'}$ and $\theta^{'}$ are effected by changes in $\gamma$ and $\overline{w^{'}\theta^{'}}_{s}$ .  I will magnify the effects of $\gamma$, by applying the convective scales, $\theta^{*}$ and $w^{*}$ in Chapter 3 section 4.\\    

\subsection{Q2: How can the \acs{EL} limits be defined based on the $\overline{\theta}$ profile and what is the relationship of the resulting depth ($\Delta h$) to \acs{Ri}?}
       
Here I define the \acs{CBL} height as the location of maximum vertical $\overline{\theta}$ gradient as in Figure \ref{fig:hdefs}.  The lower and upper \acs{EL} limits are then, the points at which $\frac{\partial \overline{\theta}}{\partial z}$ significantly exceeds zero and where it resumes $\gamma$.  The lower limit requires choice of a threshold value which should be small, positive and less than $\gamma$. Since it is somewhat arbitrary I will compare results based on three different threshold values in Chapter 3 section 5.  \citeauthor{FedConzMir04} (\citeyear{FedConzMir04}) and \citeauthor{BrooksFowler2} (\citeyear{BrooksFowler2}) defined the \acs{EL} in terms of the vertical $\overline{w^{'}\theta^{'}}$ profiles as in Figure \ref{fig:hdefs} but disagreed on the shape of the relationship of scaled \acs{EL} depth to \acs{Ri} (equation 2.1).  As well as observing this relationship using the height defininitions based on the $\overline{\theta}$ profile, I will apply the definitions based on the $\overline{w^{'}\theta^{'}}$ profile for comparison with the \citeauthor{BrooksFowler2} (\citeyear{BrooksFowler2}) and \citeauthor{FedConzMir04} (\citeyear{FedConzMir04}) in Chapter 3 section 4.\\  

\begin{figure}[htbp]
    \centering
    %plot_height.py[master 1573b9d] h vs time plot
    \includegraphics[scale=.5]{/newtera/tera/phil/nchaparr/python/Plotting/Dec252013/pngs/height_defs.pdf}
    \caption{Height definitions based on the average vertical profiles. $\theta_{0}$ is the inital potential temperature.}
    \label{fig:hdefs}   % label should change
\end{figure}

\begin{table}[htbp]
    \begin{center}
%\centerline{
    \begin{tabular}{ p{2cm} p{4cm}  p{3cm}  p{3cm} p{3cm} }
    %\hline
      \acs{CBL} Height & \acs{ML} $\overline{\theta}$ & $\theta$ Jump & \acs{Ri} \\ \hline 
       $h$ & $\overline{\theta}_{ML} = \frac{1}{h}\int^{h}_{0}\overline{\theta}(z)dz$ & $\delta \theta=\overline{\theta}(h_{1})-\overline{\theta}(h_{0})$ & \acs{Ri}$_{\Delta}=\frac{\frac{g}{\overline{\theta}_{ML}}\delta \theta h}{w^{*2}}$  \\ [.3cm] %\hline
        
       & &$\Delta \theta = \overline{\theta}_{0}(h)- \overline{\theta}_{ML}$ & \acs{Ri}$_{\delta}=\frac{\frac{g}{\overline{\theta}_{ML}} \Delta \theta h}{w^{*2}}$ \\ \hline
      \end{tabular}
%}
\caption{Defintions based on the vertical $\overline{\theta}$ profile in Figure \ref{fig:hdefs}.  To obtain those based on the $\overline{w^{'}\theta^{'}}$ profile, replace $h_{0}$, $h$ and $h_{0}$ with $z_{f0}$, $z_{f}$ and $z_{f1}$}
\label{table:reldefs}   
\end{center}    
\end{table}

\subsection{Q3: How does defining the $\theta$ jump based on the vertical $\overline{\theta}$ accross the \acs{EL} as in Figure \ref{fig:1stord} vs at the inversion ($h$) as in Figure \ref{fig:0order}, effect the entrainment relation, in particular $a$?}

I will vary the definition of the $\theta$ jump and so \acs{Ri} to see how this effects the resulting entrainment relation, as showin in table \ref{table:reldefs}.  In particular I wish to see the value of $a$ so I'll plot in log coordinates.  I will reproduce this plot using height definitions based on $\overline{w^{'}\theta^{'}}$ for comparison with the results of \citeauthor{FedConzMir04} (\citeyear{FedConzMir04}).

\endinput

Any text after an \endinput is ignored.
You could put scraps here or things in progress.
%\begin{epigraph}
 %   \emph{If I have seen farther it is by standing on the shoulders of
  %  Giants.} ---~Sir Isaac Newton (1855)
%\end{epigraph}

%This document provides a quick set of instructions for using the
%\class{ubcdiss} class to write a dissertation in \LaTeX. 
%Unfortunately this document cannot provide an introduction to using
%\LaTeX.  The classic reference for learning \LaTeX\ is
%\citeauthor{lamport-1994-ladps}'s
%book~\cite{lamport-1994-ladps}.  There are also many freely-available
%tutorials online;
%\webref{http://www.andy-roberts.net/misc/latex/}{Andy Roberts' online
 %   \LaTeX\ tutorials}
%seems to be excellent.
%The source code for this docment, however, is intended to serve as
%an example for creating a \LaTeX\ version of your dissertation.

%We start by discussing organizational issues, such as splitting
%your dissertation into multiple files, in
%\autoref{sec:SuggestedThesisOrganization}.
%We then cover the ease of managing cross-references in \LaTeX\ in
%\autoref{sec:CrossReferences}.
%We cover managing and using bibliographies with \BibTeX\ in
%\autoref{sec:BibTeX}. 
%We briefly describe typesetting attractive tables in
%\autoref{sec:TypesettingTables}.
%We briefly describe including external figures in
%\autoref{sec:Graphics}, and using special characters and symbols
%in \autoref{sec:SpecialSymbols}.
%As it is often useful to track different versions of your dissertation,
%we discuss revision control further in
%\autoref{sec:DissertationRevisionControl}. 
%We conclude with pointers to additional sources of informat%ion in
%\autoref{sec:Conclusions}.

  
%The \acs{UBC} \acf{FoGS} specifies a particular arrangement of the
%components forming a thesis.\footnote{See
 %   \url{http://www.grad.ubc.ca/current-students/dissertation-thesis-preparation/order-components}}
%This template reflects that arrangement.

%In terms of writing your thesis, the recommended best practice for
%organizing large documents in \LaTeX\ is to place each chapter in
%a separate file.  These chapters are then included from the main
%file through the use of \verb+\include{file}+.  A thesis might
%be described as six files such as \file{intro.tex},
%\file{relwork.tex}, \file{model.tex}, \file{eval.tex},
%\file{discuss.tex}, and \file{concl.tex}.

%We also encourage you to use macros for separating how something
%will be typeset (\eg bold, or italics) from the meaning of that
%something. 
%For example, if you look at \file{intro.tex}, you will see repeated
%uses of a macro \verb+\file{}+ to indicate file names.
%The \verb+\file{}+ macro is defined in the file \file{macros.tex}.
%The consistent use of \verb+\file{}+ throughout the text not only
%indicates that the argument to the macro represents a file (providing
%meaning or semantics), but also allows easily changing how
%file names are typeset simply by changing the definition of the
%\verb+\file{}+ macro.
%\file{macros.tex} contains other useful macros for properly typesetting
%things like the proper uses of the latinate \emph{exempli grati\={a}}
%and \emph{id est} (\ie \verb+\eg+ and \verb+\ie+), 
%web references with a footnoted \acs{URL} (\verb+\webref{url}{text}+),
%as well as definitions specific to this documentation
%(\verb+\latexpackage{}+).

 
%\LaTeX\ make managing cross-references easy, and the \latexpackage{hyperref}
%package's\ \verb+\autoref{}+ command\footnote{%
 %   The \latexpackage{hyperref} package is included by default in this
 
%   template.}
%makes it easier still. 

%A thing to be cross-referenced, such as a section, figure, or equation,
%is \emph{labelled} using a unique, user-provided identifier, defined
%using the \verb+\label{}+ command.  
%The thing is referenced elsewhere using the \verb+\autoref{}+ command.
%For example, this section was defined using:
%\begin{lstlisting}
%    \section{Making Cross-References}
 %   \label{sec:CrossReferences}
%\end{lstlisting}
%References to this section are made as follows:
%\begin{lstlisting}
 %   We then cover the ease of managing cross-references in \LaTeX\
  %  in \autoref{sec:CrossReferences}.
%\end{lstlisting}
%\verb+\autoref{}+ takes care of determining the \emph{type} of the 
%thing being referenced, so the example above is rendered as
%\begin{quote}
%    We then cover the ease of managing cross-references in \LaTeX\
%    in \autoref{sec:CrossReferences}.
%\end{quote}

%The label is any simple sequence of characters, numbers, digits,
%and some punctuation marks such as ``:'' and ``--''; there should
%be no spaces.  Try to use a consistent key format: this simplifies
%remembering how to make references.  This document uses a prefix
%to indicate the type of the thing being referenced, such as \texttt{sec}
%for sections, \texttt{fig} for figures, \texttt{tbl} for tables,
%and \texttt{eqn} for equations.

%For details on defining the text used to describe the type
%of \emph{thing}, search \file{diss.tex} and the \latexpackage{hyperref}
%documentation for \texttt{autorefname}.
