%% The following is a directive for TeXShop to indicate the main file
%%!TEX root = diss.tex

\chapter{Introduction} 
\label{ch:Introduction}
\setlength{\parindent}{0cm}
Scaling simplifies an expression that describe a process by: elimination of terms deemed insignificant and units, 
and reducing magnitude to O(1).  Robust scales enable prediction, and parametrization of one set of phenomena 
in models whose focus is of a different scale.  Scales of the lower Convective Boundary Layer are well known.  Much work has been done on those of the Entrainment Zone 
but it does not seem to have been as yet organized in the form of a scaling diagram.   

%%%%%%%%%%%%%%%%%%%%%%%%%%%%%%%%%%%%%%%%%%%%%%%%%%%%%%%%%%%%%%%%%%%%%%


%%%%%%%%%%%%%%%%%%%%%%%%%%%%%%%%%%%%%%%%%%%%%%%%%%%%%%%%%%%%%%%%%%%%%%
\section{Scales of the Convective Boundary Layer}
\label{sec:ScalesoftheConnvectiveBoundaryLayer}

Under certain sets of conditions The Atmospheric Boundary Layer (ABL) exhibits predictable behaviour.  
Key variables can be identified from which dimensionless groups and scales are formed, and parametrizations derived.  One of such, the Convective Boundary Layer (CBL), is known to be driven by convection resulting from 
surface heat flux ($\overline{w^{'}\theta^{'}}_{s}$).  It has a time varying height (h) at which turbulent
kinetic energy (TKE), and tracer concentration (c) sharply decreases and potential temperature ($\theta$)
increases to that of the stable layer above.  Growth is driven by $\overline{w^{'}\theta^{'}}_{s}$ and 
impeded by the stable lapse rate ($\gamma$) above  or the potential temperature inversion jump ($\Delta \theta$).  $\Delta \theta$ is effectively strengthened
by subsidence $W_{L}$ under anticyclonic conditions; warm dry air advects downwards, reducing cloud formation
and clearing the way for solar radiation.

\subsection{Surface Similarity: from $Ri_{f}$ to $-\frac{z}{L}$}
\label{subsec:MoninObukhovLength}
The surface layer is usually defined as the layer where fluxes vary less than 10 percent of their 
magnitude with height.  So, variables are simplified by representing them at one height - the surface ($z_{0}$).
Here the TKE is from both convection and shear.  The Flux Richardson number, $Ri_{f}$ is obtained by taking the buoyancy term over the negative of the shear 
term from the TKE equation:\\

\begin{equation}
R_{f} = \frac{\left({\frac{g}{\overline{\theta_{v}}}} \right) \left(\overline{w^{'}\theta_{v}^{'}} \right)}{ \left(\overline{u_{i}^{'}u_{j}^{'}} \right) \frac{\partial \overline{U_{i}}}{\partial x_{j}}} \\
\end{equation}

Assuming horizontal homogeneity and ignoring the large scale vertical motions this is reduced to:\\

\begin{equation}
R_{f} = \frac{\left({\frac{g}{\overline{\theta_{v}}}} \right) \left(\overline{w^{'}\theta_{v}^{'}} \right)}{ \left(\overline{u^{,}w^{'}} \right) 
\frac{\partial \overline{U}}{\partial z} + \left(\overline{v^{'}w^{'}} \right) \frac{\partial \overline{V}}{\partial z}} \\
\end{equation}

In the surface layer the shear and stability terms can be nondimentionalized:\\

\begin{equation}
\psi_{M} = \frac{\partial M}{\partial z}\frac{kz}{u_{*}}, \psi_{H} = \frac{\partial \theta}{\partial z}\frac{kz}{\theta_{*}}\\
\end{equation}

and $\psi_{M} = 1$.  First order closure gives:\\

$\left( \overline{w^{'}\theta^{'}} \right) = K_{H} \frac{\partial \overline{\theta}}{\partial z} = u_{*} \theta_{*}$ 
and $\left( \overline{w^{'} u^{'}} \right) = K_{M} \frac{\partial M}{\partial z} = u_{*}^{2}$\\

This scaling is applied to the denominator of $Ri_{f}$ giving

\begin{equation}
\frac{\left({\frac{g}{\overline{\theta_{v}}}} \right) \left(\overline{w^{'}\theta_{v}^{'}} \right)}{u_{*}^{2}  \frac{\partial \overline{M}}{\partial z}}\\
\end{equation}

and the wind shear is exchanged with its dimensionless form above.  Rearranging gives\\

\begin{equation}
z \left( \frac{kg \left( \overline{w^{'} \theta^{'}} \right) }{\overline{\theta} u_{*}^{3}} \right) = -\frac{z}{L} = \xi      
\end{equation}

a commonly used dimensionless stability parameter for the neutral surface layer.%does this apply to the CBL.
%is there an imbalance in the amount of detail here wrt the amount of detail supporting research questions
%should there be a reference here -- from phils notes
\subsection{A Scaling Diagram for the Convective Boundary Layer}

In the lower CBL, L becomes small as convective turbulence dominates and is influenced more by the surface than the inversion.  Local free convection scales become important.  In the Mixed Layer (ML) convective turbulence is driven by the surface heat flux ($\overline{w^{'}\theta^{'}}_{0}$) and is impeded by the inversion or stable lapse rate at h. The ML temperature approaches a uniform, increasing value.   
\cite{Stull-BLMetIntro}\\

Scales for these three CBL regions are concisely represented over a typical range of $-\frac{h}{L}$ in \citeauthor{HoltNieu86}'s scaling diagram 
(\autoref{fig:Scaledi}).  A free convection layer begins to form when $-\frac{h}{L} > 10$ and dominates the surface layer when $-\frac{h}{L} \ge 100$.  
At the time of 
publication little was known of the scales above the convective ML so the authors assigned to the Entrainment Zone a constant scaled depth of 
0.4h.  But, for example, at the onset of significant convection the entire initial CBL could be considered an Entrainment Zone. So, a constant scaled depth 
with respect to time may not be representative.
%(need to ask about Rolands use of a local $\overline{w^{'}\theta^{'}}$ as opposed to the surface value in the diagram)
\begin{figure}[!ht]
    \centering
    % For the sake of this example, we'll just use text
    \includegraphics[scale=.5]{pngs/Scaledi}
    \caption{Scales of the CBL \cite{HoltNieu86}}
    \label{fig:Scaledi}   % label should change
\end{figure}

%%%%%%%%%%%%%%%%%%%%%%%%%%%%%%%%%%%%%%%%%%%%%%%%%%%%%%%%%%%%%%%%%%%%%%
\section{Scales for the Entrainment Zone (EZ)}
\label{sec:ScalesfortheEntrainmnetZone}

\subsection{Convective Boundary Layer Entrainment}
\label{subsec:ConvectiveBoundaryLayerEntrainment}

A turbulent fluid will entrain an adjoining or surrounding less turbulent fluid and the mean velocity at which the less turbulent fluid flows in is assumed proportional to a characteristic velocity of the turbulent region.  The entrainment velocity is then either: the rate at which the non turbulent fluid flows in or the rate at which the turbulent fluid expands.\\   

The principal vehicles for turbulence in the CBL are buoyant thermals or plumes and they act against an inversion or stable lapse rate. In \cite{Turner86} \citeauthor{Turner86} draws a distinction between how surrounding air is mixed into thermals vs plumes and argues that in case of thermals it depends on the overall dynamics rather than turbulent velocity or length scales. Nonetheless he states estimates for the rise heights of both plumes and thermals in terms of their initial buoyancy and the stability against which they act.\\ 

The CBL plumes or thermals rise, mixing surrounding air and expanding. They rise until their density is equal to or exceeds that of the surroundings. The plume then spreads, or overshoots and sinks with oscillation characterized by the Brunt Vasalla frequency $N = \sqrt{\frac{g}{\theta} \frac{d\theta}{dz}}$. Sinking can occur by convective overturning or recoil when $\Delta \theta$ or $\gamma$ is large.  Thus air from aloft is engulfed or trapped (entrained) and further mixing occurs at smaller scales.  Under these conditions the general consensus is that entrainment at the level of viscosity is unimportant. If there is no penetration of the layer above and the air reaches its density level unimpeded, this is considered encroachment rather than entrainment.\\ 
\cite{Turner86}

\subsection{Boundary Layer Growth (Entrainment) Rate $W_{e}$}
\label{subsec:BoundaryLayerGrowth}

The CBL grows via thermals or plumes impinging upon the stable air aloft and entraining it. In line with the above the overall entrainment velocity  is its increase in height less large scale vertical motion IE subsidence under anticyclonic conditions.

\begin{equation}
W_{e} = \frac{dh}{dt} - W_{L}  
\end{equation}

The speed at which the CBL advances vertically is proportional to the surface heat flux and inversely proportional to the inversion strength (or stability above).

\begin{equation}
W_{e} \propto \frac{\overline{w^{'}\theta^{'}}}{\Delta \theta}  
\end{equation}
\cite{Deardorff79}
When scaled by a characteristic velocity which in our case is the convective velocity scale: 

\begin{equation}
W^{*} = \left( B h \right)^{\frac{1}{3}}  
= \left( \frac{g}{\overline{\theta}} h \overline{w^{'}\theta^{'}_{s}} \right)^{\frac{1}{3}}
\end{equation}

This expression assumes the following form

\begin{equation}
\frac{W_{e}}{W^{*}} \propto Ri^{*a}  
\end{equation}

where $Ri^{*} = \frac{g}{\overline{\theta}} \Delta \theta \frac{h}{w^{*2}}$ is a buoyancy Richardson Number. a is usually $-1$ but sometimes $-\frac{3}{2}$ as summarized by \citeauthor{Turner86} when $\frac{LN}{U}$ in our case $\frac{hN}{W^{*}}$ becomes sufficiently large (eg $\approx 6$) \cite{Turner86}.
\\

Alternatively $S = \frac{g}{\overline{\theta}} \gamma \frac{h^{2}}{w^{*2}}$,  the square of the ratio of the convective time scale to the Brunt Vasalla time scale, is used. \cite{Deardorff79}  Both are applicable in experiments where a constant $\overline{w^{'}\theta^{'}}_{s}$ acts against a stable $\gamma$ producing a convective turbulent ML capped by a $\Delta \theta$ \cite{Turner86}. It follows that $\Delta \theta$ must be a function of $\gamma$ and $\overline{w^{'}\theta^{'}}_{s}$.\\
%the last reference to Turner may actually be a reference to a Deardorff study.
\subsection{What is the Entrainment Zone?} 

\begin{wrapfigure}{r}{0.55\textwidth}
\vspace{-12mm}
  \begin{center}
    \includegraphics[scale=.25]{pngs/BLPotTemp}
  \end{center}
\vspace{-5mm}
  \caption{CBL $\theta$ Profile \cite{Stull-BLMetIntro}}
  \label{fig:BLPotTemp}
\end{wrapfigure}

In terms of an average or idealized $\theta$ (or c profile) if h is the point at which the gradient is maximum, then the Entrainment Zone (EZ)
can be defined as the region bounded by the points at which; the profiles first deviate from the ML value, and resume the Free Atmospheric (FA) value.\\
%reference for this -- maybe one of the analytical model studies
\newpage
\begin{wrapfigure}{r}{0.6\textwidth}
    \begin{center}
      \vspace{-12mm}
    \includegraphics[scale=.8]{pngs/FluxProf}
    \vspace{-5mm}
    \end{center}
    \caption{CBL Flux Profile \cite{Stull-BLMetIntro}}
    \label{fig:CBLFluxProf}   % label should change
\end{wrapfigure}

The $\overline{w^{'}\theta^{'}}$ profile crosses zero, reaches a minimum (maximum negative) value and increases to zero again bounding a region of overall heating from above which also can be regarded as the EZ.\\

Since the CBL is composed of buoyant plumes (or thermals) the top of the EZ can be defined at the tops of the highest but the bottom is not so clear from this perspective.  It's often defined as the level at which a high percentage of air is Free Atmosphere (FA) air. \cite{StullNelEl} \\

\begin{figure}[!ht]
    \centering
    % For the sake of this example, we'll just use text
    \includegraphics[scale=.5]{pngs/ELStull_Plumes}
    \caption{CBL Plume Tops \cite{StullNelEl}}
    \label{fig:CBLPlumTops}   % label should change
\end{figure}


The degree to which (the highest) CBL plumes overshoot the top of the ML (or the bottom of the EZ) is obviously related to the buoyant momentum and inversely related to the FA $\gamma$ or $\Delta \theta$.  So a relationship to the buoyancy Richardson Number ($Ri^{*}$) and scaled $W_{e}$ follows, where suggested values for b  are  $-\frac{1}{2}$ or $-\frac{1}{4}$ \cite{StullNelEl}.

\begin{equation}
\frac{\Delta h}{h} \propto Ri^{* b} \propto \frac{W_{e}}{W_{*}}  
\end{equation}  

Clearly the h and $\Delta h$ based on the average $\theta$ profile is related to the distribution of local h values. So characteristics of the distribution should relate to $Ri^{*}$ \cite{SullMoengStev}, \cite{BrooksFowler2}.  

\section{Research Goals}
\label{sec:ResearchGoals}

The LES studies of \citeauthor{SullMoengStev} \cite{SullMoengStev}, \citeauthor{FedConzMir04} \cite{FedConzMir04} and more recently \citeauthor{BrooksFowler2} in \cite{BrooksFowler2} were carried out on grids of lower resolution than those which began to converge in \citeauthor{SullPat} \cite{SullPat}, except for the cases specifically for testing resolution effects. \citeauthor{SullPat} \cite{SullPat} influenced our choice of grid size, in particular within the \acs{EL}\\

 All of these were carried out on $5 \times 5 km$ domains and used a combination of both time and spatial averaging to obtain average profiles.  So far, we haven't seen a study involving an ensemble of cases, such that true ensemble averaging can be carried out.  We chose this setup to obtain true ensemble averages, and to have a wealth of local points for some basic statistical observations.\\

\citeauthor{FedConzMir04} \cite{FedConzMir04} varied their cases by upper lapse rate ($\gamma$) over a typical range found in the troposphere IE $1 - 10 K / Km$, whereas \citeauthor{SullMoengStev} \cite{SullMoengStev} and \citeauthor{BrooksFowler2} in \cite{BrooksFowler2} varied surface heat flux and initial inversion ($\Delta \theta$).  To obtain a range of Richardson numbers (\acs{Ri}) we varied both surface heat flux ($w^{'}\theta^{'}$) and upper lapse rate ($\gamma$).\\

\subsection{}

Since unlike the other LES studies, we initialize with a constant surface heat flux ($\overline{w^{'}\theta^{'}}_{s}$) working against a constant lapse rate ($\gamma$), it would be good to see the formation of a convective boundary layer with the expected average profiles.\\

Given our concern about a slightly smaller domain than usual, we would like to make sure each of the individual cases are producing coherent turbulent structures and that there is adequate scale separation between the structures with highest energy and the grid size.\\

\citeauthor{SullMoengStev} \cite{SullMoengStev} showed with effective visual aids some of the details of the dynamics in the \acs{EL}.  It would be important to at least confirm our setup is producing comparable motions of warm and cool air in this region.


  

\subsection{}

Analytical models for the CBL often start with the Reynold's averaged conservation equation for $\theta$ \cite{Deardorff79}

\begin{equation}
\frac{D \overline{\theta}}{Dt} = -\frac{\partial \overline{w^{'} \theta^{'}}}{\partial z}
\end{equation}

If advection is assumed unimportant, points on a resulting $\overline{w^{'} \theta^{'}}$ profile correspond to changes in the $\overline{\theta}$ profile with time.  h has been defined as both the point at which the vertical gradient of $\theta$ is maximum and the point at which $\overline{w^{'} \theta^{'}}$ is minumum but these points do not coincide \cite{SullMoengStev}.

\subsection{How does the distribution of $h_{max \frac{\partial \overline{\theta}}{\partial z}}$ determined locally behave?}
The spread and skew should differ depending on initial $\gamma$ and $\overline{w^{'}\theta^{'}}_{s}$ 
and throughout the run.  For example a large $\gamma$ would decrease any positive skew (less overshoot) 
and increase in both parameters would reduce spread.\\

Although related to, due to smoothing and discrete grid heights $h_{max \frac{\partial \overline{\theta}}{\partial z}}$ does not necessarily equal 
$\overline{h_{max \frac{\partial {\theta}}{\partial z}}}$. Equivalence of the height of the highest plumes to the height at which the air is mostly FA air (top of the EZ) has been demonstrated but the lower EZ limit is not so clearly defined in terms of local h (IE lowest plumes).%need to check this 
\cite{BrooksFowler2}. 

\subsection{Can anything meaningful be added to the top of the Scaling Diagram}
%\citeauthor{HoltNieu86} 's

In \cite{StullNelEl} \citeauthor{StullNelEl} used measurements from the BLX and CIRCE field campaigns 
to show a time varying $\frac{\Delta h}{h}$ ($\sim1.5$ to $\sim .25$).  It peaks in the late morning
and decreases to an almost steady value by midday when $\overline{w^{'}\theta^{'}}$ begins to decrease. Change could be attributed to the CBL encountering different components of the residual nigh-time BL ($\gamma$) and  $-\frac{h}{L}$ does not directly account for this.      

\subsection{Can previous relationships to $Ri^{*a}$ (or S) be verified, including both a = $-\frac{3}{2}$ and $-1$}

It might be interesting to explore the possibility of differing $Ri^{*}$ dependence regimes by choosing parameters 
($\gamma$ and $\overline{w^{'}\theta^{'}}$) such that $\frac{hN}{W^{*}}$ straddles 6 \cite{Turner86}. 

\endinput

Any text after an \endinput is ignored.
You could put scraps here or things in progress.
%\begin{epigraph}
 %   \emph{If I have seen farther it is by standing on the shoulders of
  %  Giants.} ---~Sir Isaac Newton (1855)
%\end{epigraph}

%This document provides a quick set of instructions for using the
%\class{ubcdiss} class to write a dissertation in \LaTeX. 
%Unfortunately this document cannot provide an introduction to using
%\LaTeX.  The classic reference for learning \LaTeX\ is
%\citeauthor{lamport-1994-ladps}'s
%book~\cite{lamport-1994-ladps}.  There are also many freely-available
%tutorials online;
%\webref{http://www.andy-roberts.net/misc/latex/}{Andy Roberts' online
 %   \LaTeX\ tutorials}
%seems to be excellent.
%The source code for this docment, however, is intended to serve as
%an example for creating a \LaTeX\ version of your dissertation.

%We start by discussing organizational issues, such as splitting
%your dissertation into multiple files, in
%\autoref{sec:SuggestedThesisOrganization}.
%We then cover the ease of managing cross-references in \LaTeX\ in
%\autoref{sec:CrossReferences}.
%We cover managing and using bibliographies with \BibTeX\ in
%\autoref{sec:BibTeX}. 
%We briefly describe typesetting attractive tables in
%\autoref{sec:TypesettingTables}.
%We briefly describe including external figures in
%\autoref{sec:Graphics}, and using special characters and symbols
%in \autoref{sec:SpecialSymbols}.
%As it is often useful to track different versions of your dissertation,
%we discuss revision control further in
%\autoref{sec:DissertationRevisionControl}. 
%We conclude with pointers to additional sources of informat%ion in
%\autoref{sec:Conclusions}.

  
%The \acs{UBC} \acf{FoGS} specifies a particular arrangement of the
%components forming a thesis.\footnote{See
 %   \url{http://www.grad.ubc.ca/current-students/dissertation-thesis-preparation/order-components}}
%This template reflects that arrangement.

%In terms of writing your thesis, the recommended best practice for
%organizing large documents in \LaTeX\ is to place each chapter in
%a separate file.  These chapters are then included from the main
%file through the use of \verb+\include{file}+.  A thesis might
%be described as six files such as \file{intro.tex},
%\file{relwork.tex}, \file{model.tex}, \file{eval.tex},
%\file{discuss.tex}, and \file{concl.tex}.

%We also encourage you to use macros for separating how something
%will be typeset (\eg bold, or italics) from the meaning of that
%something. 
%For example, if you look at \file{intro.tex}, you will see repeated
%uses of a macro \verb+\file{}+ to indicate file names.
%The \verb+\file{}+ macro is defined in the file \file{macros.tex}.
%The consistent use of \verb+\file{}+ throughout the text not only
%indicates that the argument to the macro represents a file (providing
%meaning or semantics), but also allows easily changing how
%file names are typeset simply by changing the definition of the
%\verb+\file{}+ macro.
%\file{macros.tex} contains other useful macros for properly typesetting
%things like the proper uses of the latinate \emph{exempli grati\={a}}
%and \emph{id est} (\ie \verb+\eg+ and \verb+\ie+), 
%web references with a footnoted \acs{URL} (\verb+\webref{url}{text}+),
%as well as definitions specific to this documentation
%(\verb+\latexpackage{}+).

 
%\LaTeX\ make managing cross-references easy, and the \latexpackage{hyperref}
%package's\ \verb+\autoref{}+ command\footnote{%
 %   The \latexpackage{hyperref} package is included by default in this
 
%   template.}
%makes it easier still. 

%A thing to be cross-referenced, such as a section, figure, or equation,
%is \emph{labelled} using a unique, user-provided identifier, defined
%using the \verb+\label{}+ command.  
%The thing is referenced elsewhere using the \verb+\autoref{}+ command.
%For example, this section was defined using:
%\begin{lstlisting}
%    \section{Making Cross-References}
 %   \label{sec:CrossReferences}
%\end{lstlisting}
%References to this section are made as follows:
%\begin{lstlisting}
 %   We then cover the ease of managing cross-references in \LaTeX\
  %  in \autoref{sec:CrossReferences}.
%\end{lstlisting}
%\verb+\autoref{}+ takes care of determining the \emph{type} of the 
%thing being referenced, so the example above is rendered as
%\begin{quote}
%    We then cover the ease of managing cross-references in \LaTeX\
%    in \autoref{sec:CrossReferences}.
%\end{quote}

%The label is any simple sequence of characters, numbers, digits,
%and some punctuation marks such as ``:'' and ``--''; there should
%be no spaces.  Try to use a consistent key format: this simplifies
%remembering how to make references.  This document uses a prefix
%to indicate the type of the thing being referenced, such as \texttt{sec}
%for sections, \texttt{fig} for figures, \texttt{tbl} for tables,
%and \texttt{eqn} for equations.

%For details on defining the text used to describe the type
%of \emph{thing}, search \file{diss.tex} and the \latexpackage{hyperref}
%documentation for \texttt{autorefname}.
