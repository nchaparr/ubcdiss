%% The following is a directive for TeXShop to indicate the main file
%%!TEX root = diss.tex

\chapter{Introduction}
\label{ch:Introduction}
\setlength{\parindent}{0cm}
Scales of the lower Convective Boundary Layer are well known.  Much work has been done on those of the Entrainment Region but it does not seem to have been as yet organized in the form of a scaling diagram. This thesis will focus on the dry convective case.  

%%%%%%%%%%%%%%%%%%%%%%%%%%%%%%%%%%%%%%%%%%%%%%%%%%%%%%%%%%%%%%%%%%%%%%
\section{Why do we care about Convective Boundary Layer and Entrainment Scales}
\label{sec:WhydowecareaboutConvectiveBoundaryLayerScales}

Scaling simplifies expressions that describe a process by: elimination of deemed insignificant terms, 
getting rid of units and reducing magnitude to O(1).\\

Robust scales enable prediction and parametrisations of one set of phenomena in models whose focus is
of a different scale.\\


%%%%%%%%%%%%%%%%%%%%%%%%%%%%%%%%%%%%%%%%%%%%%%%%%%%%%%%%%%%%%%%%%%%%%%
\section{Scales of the Convective Boundary Layer}
\label{sec:ScalesoftheConnvectiveBoundaryLayer}

\subsection{Similarity Theory per Rolands Book.  What is similarity theory.  Outline the main ones for the CBL.} 

Under a number of sets of conditions The Atmospheric Boundary Layer (ABL) exhibits predictable behavior
and key variables can be identified from which dimensionless groups and scales are formed, 
and parametrizations derived.
 
One of such, the Convective Boundary Layer (CBL), is known to be driven by convection resulting from 
suface heat flux ($\overline{w^{'}\theta^{'}}_{s}$).  It has a time varying height (h) at which turbulent
kinetic energy (TKE), and tracer concentrations (c) sharply decrease and potential temperature ($\theta$)
increases to $h \gamma$ where $\gamma$ is the lapse rate of the stable layer above.  Growth of the layer
is driven by $\overline{w^{'}\theta^{'}}_{s}$ and impeded by $\gamma$ or, as in the case of interest here,
the potential temperature inversion jump ($\Delta \theta$).  $Delta \theta$ is effectively strengthened
by subsidence $W_{s}$ under anticyclonic conditions; warm dry advects downwards, reducing cloud formation
and clearing the way for solar radiation.\\
 
(Monin-Obvukhov - Surface Similarity)
The surface layer is usually defined as the layer where the fluxes vary less than 10 percent of their 
magitude with height so variables are simplified by representing them at one height ie the surface ($z_{0}$).
Here the TKE is from both Convection and The Flux Richardson number, $R_{f}$ is a well established
dimensionless stability parameter and is obtained by taking the Bouancy term over the negative of the Shear 
term in the TKE equation:\\

$R_{f} = \frac{\left({\frac{g}{\overline{\theta_{v}}}} \right) \left(\overline{w^{'}\theta_{v}^{'}} \right)}{ \left(\overline{u_{i}^{,}u_{j}^{'}} \right) \frac{\partial \overline{U_{i}}}{\partial x_{j}}}$ \\

Assuming horizontal homogenaeity and ignoring the large scale vertical motions this is reduced to:\\

$R_{f} = \frac{\left({\frac{g}{\overline{\theta_{v}}}} \right) \left(\overline{w^{'}\theta_{v}^{'}} \right)}{ \left(\overline{u^{,}w^{'}} \right) 
\frac{\partial \overline{U}}{\partial z} + \left(\overline{v^{,}w^{'}} \right) \frac{\partial \overline{V}}{\partial z}}$ \\

In the (neutral? need to check with Phil if this applies to CBL) surface layer the shear and stability terms can be nondimentionalized:\\

$\psi_{M} = \frac{\partial M}{\partial z}\frac{kz}{u_{*}}, \psi_{M} = \frac{\partial \theta}{\partial z}\frac{kz}{\theta_{*}}$\\

and $\psi_{M} = 1$\\

and first order closure gives:\\

$\left( \overline{w^{'}\theta^{'}} \right) = K_{H} \frac{\partial \overline{\theta}}{\partial z} = u_{*} \theta_{*}$ 
and $\left( \overline{w^{'} u^{'}} \right) = K_{M} \frac{\partial M}{\partial z} = u_{*}^{2}$\\

This scaling is applied to the denominator of $Ri_{f}$\\ giving

$\frac{\left({\frac{g}{\overline{\theta_{v}}}} \right) \left(\overline{w^{'}\theta_{v}^{'}} \right)}{u_{*}^{2}  \frac{\partial \overline{M}}{\partial z}}$\\

and the windshear is exchaged with its dimensionless form above.  Rearranging gives\\

$z \left( \frac{kg \left( \overline{w^{'} \theta^{'}} \right) }{\overline{\theta} u_{*}^{3}} \right) = -\frac{z}{L} = \xi$      

a commonly used dimensionless stability parameter for the neutral surface layer (need to verify with Phil that this applies in CBL).
In the CBL, L becomes small as convective turbulence dominates and feels the surface more than the inversion.  Local free convection
scales become important (Need to ask Douw Phil about relevence of L, and scaling diagram in my case)\\

(Mixed Layer Similarity)\\

In the Mixed Layer (ML) the convective turbulence is driven by the surface heat flux ($\overline{w^{'}\theta^{'}}_{0}$), the mixed layer 
temperature ($\overline{\theta}$) and is impeded by the inversion or stable lapse rate above (ie h).   
\cite{Stull-BLMetIntro}

\subsection{scaling diagrams for the boundary layer}


%%%%%%%%%%%%%%%%%%%%%%%%%%%%%%%%%%%%%%%%%%%%%%%%%%%%%%%%%%%%%%%%%%%%%%
\section{Scales for the Entrainment Region}
\label{sec:ScalesfortheEntrainmnetRegion}

\subsection{Boundary Layer Growth (Entrainment) Rate $W_{e}$}

\subsection{What is the Entrainment Region} 
Define it in terms of the average profile potential temperature and flux.

Define in it terms of individual plumes, their local entrainment regions, and then in terms of the local height distributions.\\

\subsection{Buoyancy Richardson Number and S}

Talk about and maybe go through example of how richardson number is arrived at from nondimensionalizing an equation -- I think there's an old reference to help with this.Maybe Deardorf79/78?\\

Where does S come frome?

\subsection{Relating $Ri_{b}$, $S$ to Scales and Statistics}
Relate $Ri$ to $S$\\

\section{Research Goals}
\label{sec:ResearchGoals}

\subsection{How do the Flux and theta definitions of h and $\Delta h$ relate to eachother}

Define entrainment zone in terms of average profile.\\

\subsection{How do the Distributions of h behave and resulting definitions compare with those from the average profile}

\subsection{Does the Scaling Diagram Accurately Represent the Entrainment Zone}

Rolands study and explaination of the $L$ estimate and how $-\frac{h}{L}$ progresses throughout\\
the day.\\

\subsection{At higher resolution and more data points can previous relationships to scales/statistics and S and Ri be verified}



\endinput

Any text after an \endinput is ignored.
You could put scraps here or things in progress.
%\begin{epigraph}
 %   \emph{If I have seen farther it is by standing on the shoulders of
  %  Giants.} ---~Sir Isaac Newton (1855)
%\end{epigraph}

%This document provides a quick set of instructions for using the
%\class{ubcdiss} class to write a dissertation in \LaTeX. 
%Unfortunately this document cannot provide an introduction to using
%\LaTeX.  The classic reference for learning \LaTeX\ is
%\citeauthor{lamport-1994-ladps}'s
%book~\cite{lamport-1994-ladps}.  There are also many freely-available
%tutorials online;
%\webref{http://www.andy-roberts.net/misc/latex/}{Andy Roberts' online
 %   \LaTeX\ tutorials}
%seems to be excellent.
%The source code for this docment, however, is intended to serve as
%an example for creating a \LaTeX\ version of your dissertation.

%We start by discussing organizational issues, such as splitting
%your dissertation into multiple files, in
%\autoref{sec:SuggestedThesisOrganization}.
%We then cover the ease of managing cross-references in \LaTeX\ in
%\autoref{sec:CrossReferences}.
%We cover managing and using bibliographies with \BibTeX\ in
%\autoref{sec:BibTeX}. 
%We briefly describe typesetting attractive tables in
%\autoref{sec:TypesettingTables}.
%We briefly describe including external figures in
%\autoref{sec:Graphics}, and using special characters and symbols
%in \autoref{sec:SpecialSymbols}.
%As it is often useful to track different versions of your dissertation,
%we discuss revision control further in
%\autoref{sec:DissertationRevisionControl}. 
%We conclude with pointers to additional sources of informat%ion in
%\autoref{sec:Conclusions}.

  
%The \acs{UBC} \acf{FoGS} specifies a particular arrangement of the
%components forming a thesis.\footnote{See
 %   \url{http://www.grad.ubc.ca/current-students/dissertation-thesis-preparation/order-components}}
%This template reflects that arrangement.

%In terms of writing your thesis, the recommended best practice for
%organizing large documents in \LaTeX\ is to place each chapter in
%a separate file.  These chapters are then included from the main
%file through the use of \verb+\include{file}+.  A thesis might
%be described as six files such as \file{intro.tex},
%\file{relwork.tex}, \file{model.tex}, \file{eval.tex},
%\file{discuss.tex}, and \file{concl.tex}.

%We also encourage you to use macros for separating how something
%will be typeset (\eg bold, or italics) from the meaning of that
%something. 
%For example, if you look at \file{intro.tex}, you will see repeated
%uses of a macro \verb+\file{}+ to indicate file names.
%The \verb+\file{}+ macro is defined in the file \file{macros.tex}.
%The consistent use of \verb+\file{}+ throughout the text not only
%indicates that the argument to the macro represents a file (providing
%meaning or semantics), but also allows easily changing how
%file names are typeset simply by changing the definition of the
%\verb+\file{}+ macro.
%\file{macros.tex} contains other useful macros for properly typesetting
%things like the proper uses of the latinate \emph{exempli grati\={a}}
%and \emph{id est} (\ie \verb+\eg+ and \verb+\ie+), 
%web references with a footnoted \acs{URL} (\verb+\webref{url}{text}+),
%as well as definitions specific to this documentation
%(\verb+\latexpackage{}+).

 
%\LaTeX\ make managing cross-references easy, and the \latexpackage{hyperref}
%package's\ \verb+\autoref{}+ command\footnote{%
 %   The \latexpackage{hyperref} package is included by default in this
 
%   template.}
%makes it easier still. 

%A thing to be cross-referenced, such as a section, figure, or equation,
%is \emph{labelled} using a unique, user-provided identifier, defined
%using the \verb+\label{}+ command.  
%The thing is referenced elsewhere using the \verb+\autoref{}+ command.
%For example, this section was defined using:
%\begin{lstlisting}
%    \section{Making Cross-References}
 %   \label{sec:CrossReferences}
%\end{lstlisting}
%References to this section are made as follows:
%\begin{lstlisting}
 %   We then cover the ease of managing cross-references in \LaTeX\
  %  in \autoref{sec:CrossReferences}.
%\end{lstlisting}
%\verb+\autoref{}+ takes care of determining the \emph{type} of the 
%thing being referenced, so the example above is rendered as
%\begin{quote}
%    We then cover the ease of managing cross-references in \LaTeX\
%    in \autoref{sec:CrossReferences}.
%\end{quote}

%The label is any simple sequence of characters, numbers, digits,
%and some punctuation marks such as ``:'' and ``--''; there should
%be no spaces.  Try to use a consistent key format: this simplifies
%remembering how to make references.  This document uses a prefix
%to indicate the type of the thing being referenced, such as \texttt{sec}
%for sections, \texttt{fig} for figures, \texttt{tbl} for tables,
%and \texttt{eqn} for equations.

%For details on defining the text used to describe the type
%of \emph{thing}, search \file{diss.tex} and the \latexpackage{hyperref}
%documentation for \texttt{autorefname}.
