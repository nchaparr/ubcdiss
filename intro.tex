%% The following is a directive for TeXShop to indicate the main file
%%!TEX root = diss.tex

\chapter{Introduction} 
\label{ch:Introduction}
\setlength{\parindent}{0cm}

\section{Motivation and Research Questions}
\label{sec:Mot}
The daytime convective atmospheric boundary layer (\acs{CBL}) over land starts to grow at sunrise when the surface becomes warmer than the air above it.  Coherent turbulent structures (thermals) begin to form and rise since their relative warmth causes them to be less dense and so buoyant.  the temperature profile of the residual nigtime boundary layer is stable i.e. potential temperature ($\theta$) increases with height.  The thermals rise to their neutral bouoyancy level overshoot and then overturn or recoil concurrently dragging down warm stable air from above which is subsequently mixed into the growing turbulent mixed layer (\acs{ML}) (\citeauthor{Stull-BLMetIntro} \citeyear{Stull-BLMetIntro}).  This mixing at the top of the \acs{CBL} is known as entrainment and the region over which it occurs, the entrainment layer (\acs{EL}).  \acs{CBL} entrainment occurrs via distortion of buoyancy driven thermals by the buoyancy difference (jump) caused upon overshoot.  These opposing buoyant forcings make for interesting dynamics.\\

\acs{CBL} height (h) and the prediction thereof are important for calculating the concentration of any species and the sizes of the turbulent structures.  In combination with the lifting condensation level knowledge of \acs{EL} depth allows predictions pertaining to the formation of cumulous clouds.  For example cloud cover increases as more thermals rise above their lifting condensation level.  Parametrizations for both \acs{CBL} growth and \acs{EL} depth are required in mesoscale and general circulation models (\acs{GCM}s).  Furthermore it is an attractive goal to develop a robust set of scales for this region analogous to Monin-Obvukov Theory (\citeauthor{Stull-BLMetIntro} \citeyear{Stull-BLMetIntro}, \citeauthor{Traum11} \citeyear{Traum11}, \citeauthor{SteynBaldHoff} \citeyear{SteynBaldHoff}, \citeauthor{StullNelEl} \citeyear{StullNelEl}, \citeauthor{Sorbjan} \citeyear{Sorbjan})\\

An effective, simplified conceptual model of the dry, shear-free \acs{CBL} in the absence of large scale winds is represented in Figure \ref{fig:1stord}.  Turbulent mixing via upward moving thermals and corresponding cooler dowdraughts renders the average potential temperature $\overline{\theta}$ effectively constant within the \acs{ML}.  The average vertical turbulent heat flux ($\overline{w^{'}\theta^{'}}$) is positive and decreases as the thermals approach the \acs{CBL} top.  In the \acs{EL} there is a mixture of turbulent thermals and relatively warmer stable air the proportion of the latter increasing with proximity to the free atmosphere (\acs{FA}) above.  In the \acs{EL} $\overline{w^{'}\theta^{'}}$ becomes negative as the thermals which are now relatively cool impinge on the \acs{FA} and turn downward pulling warmer air with them.  The dimensionless parameter that represents the forcings under these conditions and is ubiquitous in the corresponding literature is the convective or buoyancy Richardson number (\acs{Ri} $=frac{h\frac{g}{\overline{\theta}_{ML}}\Delta \theta}{w^{*2}}$) (\citeauthor{DearWill80} \citeyear{DearWill80}, \citeauthor{Stull-BLMetIntro} \citeyear{Stull-BLMetIntro}, \citeauthor{SullMoengStev} \citeyear{SullMoengStev}, \citeauthor{FedConzMir04} \citeyear{FedConzMir04}, \citeauthor{BrooksFowler2} \citeyear{BrooksFowler2}, \citeauthor{GarciaMellado} \citeyear{GarciaMellado}).  $h$ is the \acs{CBL} height and $w^{*}$ is the convective velocity scale (\citeauthor{Deardorff70} \citeyear{Deardorff70}) to be defined later.\\   


\begin{figure}[htbp]
    \centering
    %plot_height.py[master 1573b9d] h vs time plot
    \includegraphics[scale=.5]{/newtera/tera/phil/nchaparr/python/Plotting/Dec252013/pngs/first_order.pdf}
    \caption{}
    \label{fig:1stord}   % label should change
\end{figure}

The two principal external parameters in this simplified case, i.e. the dry, shear-free \acs{CBL} in the absence of large scale winds, are the average vertical turbulent surace heat flux ($\overline{w^{'}\theta^{'}}_{s}$) and the upper lapse rate ($\gamma$) (\citeauthor{FedConzMir04} \citeyear{FedConzMir04},\citeauthor{Sorbjan} \citeyear{Sorbjan}).  They have opposing effects, that is $\overline{w^{'}\theta^{'}}_{s}$ drives upward turbulent velocity ($w^{'+}$) and so \acs{CBL} growth ($w_{e}$) wheras $\gamma$  suppresses it.  Conversely they both cause positive turbulent potential temperature perturbations ($\theta^{'+}$) and so warming of the \acs{CBL}.  In the \acs{EL} the thermals from the surface are now relatively cool.  They turn downwards as they interact with the stable \acs{FA} concurrently bringing down warmer stable are from above.  \citeauthor{SullMoengStev} \citeyear{SullMoengStev} demonstrated these tynamics by partitioning $w^{'} \theta^{'}$ into four quadrants:  upward moving warm ($w^{'+} \theta^{'+}$), downward moving warm ($w^{'-} \theta^{'+}$), upward moving cool ($w^{'+} \theta^{'-}$) and downward moving cool ($w^{'-} \theta^{'-}$).  \citeauthor{Sorbjan} \citeyear{Sorbjan} asserted and showed that in this region the potential temperature perturbations ($\theta^{'}$) are strongly influenced by $\gamma$ whereas the turbulent vertical velocity peturbations ($w^{'}$) are almost independent thereof. Inspired by these two studies and to gain some insight into the dynamics of this idealized \acs{CBL} I ask \textbf{Q1: How do the distrubutions of local \acs{CBL} height, and $\theta^{'}w^{'}$ within the \acs{EL}, vary with $\overline{w^{'}\theta^{'}}_{s}$ and $\gamma$?}

The relationship between scaled \acs{EL} depth and \acs{Ri} 

\begin{equation}
\frac{\Delta h}{h} \propto  \acs{Ri}^{b}
\end{equation}

has been explored and justified in measurement, laboratory, numerical and based studies.  There is some disagreement with respect to its exact form, in part stemming from variation in defininition, but general its magnitude relative to h decreases with increasing \acs{Ri}. Although referred in most relevant studies to and relied upon in analytical models, the average potential temperature profile has not been used to define the \acs{EL}(\citeauthor{DearWill80} \citeyear{DearWill80}, \citeauthor{StullNelEl} \citeyear{StullNelEl}, \citeauthor{FedConzMir04} \citeyear{FedConzMir04}, \citeauthor{Boers89} \citeyear{Boers89}, \citeauthor{BrooksFowler2} \citeyear{BrooksFowler2}). This leads me to ask \textbf{Q2: How can the \acs{EL} limits be defined based on the $\overline{\theta}$ profile and what is the relationship of the resulting depth ($\Delta h$) to \acs{Ri}?}\\

\begin{figure}[htbp]
    \centering
    %plot_height.py[master 1573b9d] h vs time plot
    \includegraphics[scale=.5]{/newtera/tera/phil/nchaparr/python/Plotting/Dec252013/pngs/zero_order.pdf}
    \caption{}
    \label{fig:0order}   % label should change
\end{figure}

A further simplification to the dry, shear-free, \acs{CBL} model without large scale velocities, is to regard the \acs{EL} depth as infinitessimly small as in Figure \ref{fig:0order}.  The relationship of the sca;ed, time rate of change of $h$ (entrainment rate $w_{e}$), to \acs{Ri} can be derived based on this model (\citeauthor{Tennekes73} \citeyear{Tennekes73}, \citeauthor{Deardorff79} \citeyear{Deardorff79}, \citeauthor{FedConzMir04} \citeyear{FedConzMir04})

\begin{equation}
\frac{w_{e}}{w^{*}} \propto  \acs{Ri}^{a}
\end{equation}.
 
This will be referred to as the entrainment relation.  Although that there is such a relationship is well esablished, discussion as to the power exponent of \acs{Ri} is unresolved and results from studies justify values of both $-\frac{3}{2}$ and $-1$. See \citeauthor{Traum11} \citeyear{Traum11} for a summary and review.  \citeauthor{Turner86} \citeyear{Turner86} explains this disparity in terms of entrainment mechanism such that the higher value occurrs when thermals recoil rather than overturn in response to a stronger $\theta$ jump (or inversion).  Whereas \citeauthor{SullMoengStev} (\citeyear{SullMoengStev}) notice a deviation from the lower power ($-1$) at lower \acs{Ri} and attribute it to the effect of a the shape of $\overline{\theta}$ within a thicker \acs{EL}.  Both \citeauthor{FedConzMir04} (\citeyear{FedConzMir04}) and \citeauthor{GarciaMellado} (\citeyear{GarciaMellado}) show how the definition of the $\theta$ jump influences the time rate  of change of \acs{Ri} and so effects $a$. \textbf{Q3: How does defining the $\theta$ jump based on the vertical $\overline{\theta}$, i.e. accross the \acs{EL} as in Figure \ref{fig:1stord} vs at the inversion ($h$) as in Figure \ref{fig:0order}, effect the entrainment relation, in particular $a$?}\\


%%%%%%%%%%%%%%%%%%%%%%%%%%%%%%%%%%%%%%%%%%%%%%%%%%%%%%%%%%%%%%%%%%%%%%
\section{Relevant Background}
\label{sec:}
\subsection{The Convective Boundary Layer (CBL)}

The \acs{CBL} starts to grow rapidly at sunrise, peaking at midday.  Convective turbulence and the dominant upward vertical motions then begin to subside as the surface cools.  While the surface is warm, buoyancy driven thermals of uniform potential temperature ($\theta$) and tracer concentration at their cores form and entrain surrounding air laterally as they rise, as well as trapping and mixing in stable warm from above. (\citeauthor{Stull-BLMetIntro} \citeyear{Stull-BLMetIntro}, \citeauthor{CrumStullEl} \citeyear{CrumStullEl}).  Under conditions of strong convection, buoyantly driven turbulence dominates and shear is insignificant (\citeauthor{DirLEddy} \citeyear{DirLEddy}). Thermal overshoot relative to their neutral buoyancy level, and subsequent entrainment of the warmer air from aloft augments the warming caused by the surface heat flux and results in a potential temperature jump ($\Delta \theta$) or inversion at the \acs{CBL} top (\citeauthor{SchmidtSchu} \citeyear{SchmidtSchu}, \citeauthor{Turner86} \citeyear{Turner86}).  There may also be a residual inversion from the day before possibly strengthened by subsidence, i.e. large scale downward movenent of warmer air from above (\citeauthor{Stull-BLMetIntro} \citeyear{Stull-BLMetIntro}, \citeauthor{SullMoengStev} \citeyear{SullMoengStev}).\\  

Lidar images show the overall structure of the \acs{CBL} with rising thermals, impinging on the air above. (\citeauthor{CrumStullEl} \citeyearr{CrumStullEl}, \citeauthor{Traum11} \citeyear{Traum11}) This has been effectively modelled using large eddy simulation (\acs{LES}) by \citeauthor{SchmidtSchu} (\citeyear{SchmidtSchu}) who used horizontal slices of potential temperature and vertical velocity perturbations ($\theta^{'}$, $w^{'}$) at various vertical levels to show how the thermals form, merge and impinge at the \acs{CBL} top with concurrent peripheral downward motions.  The latter is supported in the visualizations of \citeauthor{SullMoengStev} (\citeyear{SullMoengStev}).  Vertical cross sections within the \acs{EL} show the relatively cooler thermals and trapped warmer air as well as the closely associated upward motion of cooler air and downward motion of warmer air.\\ 

On average these convective turbulent structures create a fully turbulent mixed layer (\acs{ML}) with eddie sizes cascading via an inertial subrange to the molecular scales at which energy is lost through viscous dissipation (\citeauthor{Stull-BLMetIntro} \citeyear{Stull-BLMetIntro}).  Here $\overline{\theta}$ is close to uniform and increases due to surface heat flux ($\overline{w^{'}\theta^{'}}_{s}$) and the flux of entrained stable air at the inversion ($\overline{w^{'}\theta^{'}}_{h}$).  \acs{ML} turbulence is dominated by warm updraughts and cool downdraughts.  With proximity to the top the updraughts become relatively cool and warmer air from above is drawn downward, so $\overline{w^{'}\theta^{'}}$ is positive and decreasing.  Above the \acs{ML} the air becomes more stable with altitude and on average this reflects as a transition from a uniform \acs{ML} potential temperature ($\frac{\partial \overline{\theta}}{\partial z} \approx 0$) to a stable lapse rate ($\gamma$).  A peak in the average vertical gradient ($\frac{\partial \overline{\theta}}{\partial z}$) at the inversion represents regions where thermals have exceeded their neutral buoyancy level. \\

\citeauthor{StullNelEl} in \cite{StullNelEl} outline the stages of \acs{CBL} growth from when the sub-layers of the nocturnal
boundary layer are entrained, untill the previous day's capping inversion is reached and a quasi-steady state growth 
is attained.  The \acs{EL} depth relative to \acs{CBL} height varies throughout these stages and its relationship
to scaled entrainment is hysteresial.  Numerical studies typically represent this last quasi-steady
phase, since there is usually a constant heat flux working against an inversion and or a stable lapse rate. 
(\citeauthor{SchmidtSchu} \cite{SchmidtSchu}, \citeauthor{Sorbjan} \cite{Sorbjan}, \citeauthor{SullMoengStev} \cite{SullMoengStev}, 
\citeauthor{FedConzMir04} \cite{FedConzMir04}, \citeauthor{BrooksFowler2} \cite{BrooksFowler2})  

\subsection{Convective Boundary Layer Height ($h$)}
\label{subsec:}

The \acs{ML} is fully turbulent with an on average uniform potential temperature ($\theta$). Aerosol and water vapour concentrations 
decrease dramatically with transition to the stable upper free atmosphere (\acs{FA}).  So any of these characteristics can support
a definition of \acs{CBL} height ($h$).  \citeauthor{StullNelEl} define $h$ in terms of the percentage of \acs{ML} air
and identified it by eye from Lidar back-scatter images in \cite{StullNelEl}.  \citeauthor{Traum11} compared
four automated methods applied to Lidar images: a suitable threshold value 
above which the air is categorized as \acs{ML} air,  the point of minimum (largest negative) 
vertical gradient, the point of minimum vertical gradient based on a fitted idealized curve, 
and the maximum wavelet covariance in \cite{Traum11}.\\

The use of Lidar dominates studies based on measurement. Numerical modelling studies have hundreds of local horizontal points
from which smooth averaged vertical profiles be obtained, and statistically robust relationships inferred. 
\citeauthor{BrooksFowler2} \cite{BrooksFowler2} applied their wavelet technique to local vertical tracer profiles 
in their large eddy simulation (\acs{LES}) study and compared it to the gradient method (i.e. locating the point of minimum vertical gradient)
and the point of minimum ($\overline{w^{'}\theta^{'}}$).  This last definition has been common
in \acs{LES} and laboratory studies where it's been referred to as the inversion height (\citeauthor{DearWill80} 
\cite{DearWill80}, \citeauthor{Sorbjan1} \cite{Sorbjan1}, \citeauthor{FedConzMir04} \cite{FedConzMir04}).
 \citeauthor{SullMoengStev} \cite{SullMoengStev} clarified that this point does not correspond to the average point of maximum $\frac{\partial \overline{\theta}}{\partial z}$, whereas the upper extrema of the four $\overline{w^{'}\theta^{'}}$ quadrants: upward moving warm air ($\overline{w^{'+}\theta^{'+}}$), downward moving warm air ($\overline{w^{'-}\theta^{'+}}$), 
upward moving cool air ($\overline{w^{'+}\theta^{'-}}$), downward moving cool air ($\overline{w^{'-}\theta^{'-}}$) 
more or less did. They defined \acs{CBL} height based on local $\frac{\partial \theta}{\partial z}$
and applied horizontal averaging as well as two methods based on $\overline{w^{'}\theta^{'}}$
for comparison.\\

None of the published \acs{LES} studies so far define the height in terms of the $\overline{\theta}$ or 
$\frac{\partial \overline{\theta}}{\partial z}$ profile even though bulk models, from which 
\acs{CBL} growth parametrizations stem, rely on an idealized version thereof.
\citeauthor{GarciaMellado} do include it as one of their measures of \acs{CBL} height in their direct numerical
simulation study (\acs{DNS}) \cite{GarciaMellado}.       

\subsection{Convective Boundary Layer Growth by Entrainment}
\label{subsec:}

In the quasi-steady regime the \acs{CBL} grows by trapping pockets of warm stable air between
or adjacent to impinging thermal plumes.  \citeauthor{Traum11} \cite{Traum11} summarize two
relevant buoyancy driven regimes of entrainment:\\

\begin{itemize}

\item{Non turbulent fluid can be engulfed between or in the overturning of thermal plumes. This kind of
event was seen by \citeauthor{SullMoengStev} in \cite{SullMoengStev} when the inversion was weak. 
\citeauthor{Traum11}'s observations in \cite{Traum11} support this.
}
\item{
Impinging thermal plumes distort the inversion interface dragging wisps of warm stable air down
at their edges or during recoil under a strong inversion or lapse rate. This type of event is supported 
by the findings  of both \citeauthor{SullMoengStev} \cite{SullMoengStev} and \citeauthor{Traum11} \cite{Traum11}.}

\end{itemize}

Under atmospheric conditions shear induced instabilities do occur, and in some laboratory studies 
under conditions of very high stability the breaking of internal waves have been observed.  
Both processes are believed to result in some entrainment but we do not consider them here
since the former is relatively insignificant in strong convection and the latter has not so far been observed 
in measurements or modeled output of the atmospheric \acs{CBL}. 
(\citeauthor{Traum11} \cite{Traum11}, \citeauthor{SullMoengStev} \cite{SullMoengStev})

\subsection{The Convective Boundary Layer Entrainment Layer}
\label{subsec:}

The \acs{ML} is fully turbulent but the top is characterised by stable air with intermittent turbulence due
to the higher reaching thermal plumes. \citeauthor{GarciaMellado} demonstrate that the entrainment layer (\acs{EL})
 is subdivided in terms of length and buoyancy scales.  That is, the lower region is comprised of mostly
turbulent air with pockets of stable warmer air that are quickly mixed, and so scales with the convective scales
(see section \ref{sec:scales}). Whereas the upper region is mostly stable apart from the impinging thermal plumes 
so scaling here is more influenced by the lapse rate ($\gamma$).\\  

In the \acs{EL} the average vertical heat flux ($\overline{w^{'}\theta^{'}}$) switches sign relative to that in the \acs{ML}.
The fast updraughts are now relatively cool ($\overline{w^{'+}\theta^{'-}}$).  In their analysis of the four $\overline{w^{'}\theta^{'}}$
quadrants \citeauthor{SullMoengStev} \cite{SullMoengStev} concluded that the overall dynamic in this region is downward motion of 
warm air from the free atmosphere (\acs{FA}) ($\overline{w^{'-}\theta^{'+}}$) since the other three quadrants effectively cancel.\\

In terms of tracer concentration and for example based on a Lidar backscatter profile, there are two ways to conceptually
define the entrainment layer (\acs{EL}).  It can be thought of as the range in space (or time) over which local height
varies (\citeauthor{CrumStullEl} \cite{CrumStullEl}).  There is also a local region over which the concentration (or back-scatter intensity) transitions from \acs{ML} to free atmospheric (\acs{FA}) values (\citeauthor{Traum11} \cite{Traum11}).  The latter
can be estimated using both curve-fitting and wavelet techniques (\citeauthor{Traum11} \cite{Traum11}, \citeauthor{SteynBaldHoff} 
\cite{SteynBaldHoff}, \citeauthor{BrooksFowler2} \cite{BrooksFowler2}). \citeauthor{Traum11} \cite{Traum11} compared the
two concepts, found them to differ and seem to favour the latter based on how correlated the corresponding scaling relationships were.\\

\citeauthor{BrooksFowler2} apply a wavelet technique to tracer profiles for the determination of 
\acs{EL} limits, in their \acs{LES} study (\cite{BrooksFowler2}).  But it is more common in numerical modelling and laboratory studies
for the \acs{EL} limits to be defined based on the average vertical heat flux ($\overline{w^{'}\theta^{'}}$) i.e. the point
at which it goes from positive to negative values, and the point at which it goes from negative value to zero (\citeauthor{DearWill80} \cite{DearWill80}, \citeauthor{FedConzMir04} \cite{FedConzMir04}, \citeauthor{GarciaMellado} \cite{GarciaMellado}). Bulk first order models assume
the region of negative $\overline{w^{'}\theta^{'}}$ coincides with the region where $\overline{\theta}$
transitions from the \acs{ML} value to the \acs{FA} value. (\citeauthor{Deardorff79} \cite{Deardorff79}, 
\cite{FedConzMir04} \cite{FedConzMir04}).  But no modelling studies use the vertical $\overline{\theta}$ profile to define 
the entrainment layer (\acs{EL}).\\

Since the mixed layer $\overline{\theta}$ from a numerical model is not strictly constant (\citeauthor{FedConzMir04} 
\cite{FedConzMir04}), a threshold value for $\overline{\theta}$ or its vertical gradient must be chosen to identify the lower 
\acs{EL} limit.  \citeauthor{BrooksFowler2} encountered inconsistencies when determining the \acs{EL} limits from the average 
tracer profile \cite{BrooksFowler2}.  But the their tracer profile was different to a simulated $\overline{\theta}$ profile whose 
\acs{ML} value increases in time predictably based on the $\overline{w^{'}\theta^{'}}$ from the surface and the \acs{CBL} top or inversion.             

%%%%%%%%%%%%%%%%%%%%%%%%%%%%%%%%%%%%%%%%%%%%%%%%%%%%%%%%%%%%%%%%%%%%%%
\section{Modelling the Convective Boundary Layer and Entrainment Layer}
\label{sec:}

\subsection{Bulk Analytical Models}
\label{subsec:}
Bulk analytical models for the Convective Boundary layer (\acs{CBL}) can be subdivided into: (i) zero order
and (ii) first order bulk models.\\

Zero order bulk models assume a Mixed Layer (\acs{ML}) of uniform potential temperature ($\overline{\theta}_{ML}$) topped by an infinitesimally 
thin layer across which there is a temperature jump ($\Delta \theta$) and above which is a constant lapse rate ($\gamma$).  
The assumed vertical heat flux ($\overline{w^{'}\theta^{'}}$) profile is linearly decreasing from the surface up, reaching 
a maximum negative $\overline{w^{'}\theta^{'}}_{h}$ value which is a constant proportion of the surface value (usually -.2)
at the temperature inversion, and decreasing to zero ac-cross the jump.  Equations for the evolution of \acs{CBL} height,
 $\overline{\theta}_{ML}$ and $\Delta \theta$ are derived on this basis.\\

For example, if the \acs{CBL} height ($h$) is rising, air is being drawn in from the stable layer above and decreasing in enthalpy.
So, the decrease in enthalpy is $c_{p}\rho\Delta \theta \frac{dh}{dt}$ per unit of horizontal area.  Since above the 
inversion is stable \citeauthor{Tennekes73} in \cite{Tennekes73} equates this enthalpy loss to the average vertical flux at the inversion.

\begin{equation}
\Delta \theta \frac{dh}{dt} = -\overline{w^{'}\theta^{'}}_{h} 
\end{equation}  

The \acs{ML} warming rate is arrived at via the simplified reynolds averaged conservation of enthalpy

\begin{equation}
\frac{\partial \overline{\theta}_{ML}}{\partial t} = -\frac{\partial}{\partial z}\overline{w^{'}\theta^{'}}
\end{equation}

which based on the assumed constant slope of the vertical heat flux becomes

\begin{equation}
\frac{\partial \overline{\theta}_{ML}}{\partial t} = \frac{\overline{w^{'}\theta^{'}}_{s}-\overline{w^{'}\theta^{'}}_{h}}{h}
\end{equation}

and the evolution of the temperature jump ($\Delta \theta$) depends on the rate of \acs{CBL} height ($h$) increase, 
the upper lapse rate $\gamma$ and the \acs{ML} warming rate
  
\begin{equation}
\frac{d\Delta \theta}{dt} = \gamma\frac{dh}{dt} - \frac{d\overline{\theta}_{ML}}{dt}
\end{equation}

An assumption about the vertical heat flux at the inversion ($h$), such as the entrainment ratio, closes this set.

\begin{equation}
\frac{\overline{w^{'}\theta^{'}}_{h}}{\overline{w^{'}\theta^{'}}_{s}} = -.2
\end{equation}


(\citeauthor{Tennekes73} \cite{Tennekes73})\\

The relevant quantities are idealized ensemble averages. There is some variation within this class of model, for example 
the rate equation for $h$ (entrainment relation) can alternatively be derived based on the turbulent kinetic energy budget 
(\citeauthor{FedConzMir04} \cite{FedConzMir04}).  But they are all based on the simplified $\overline{\theta}$ and 
$\overline{w^{'}\theta^{'}}$ profiles outlined above.\\  

First order models assume an entrainment layer (\acs{EL}) of finite depth at the top of the ML, defined by two heights:
the top of the ML ($h_{0}$) and the point where free atmospheric characteristics are resumed ($h_{1}$).  The derivations are more complex and 
examples of simplifying assumptions about the \acs{EL} are: 
\begin{itemize}
\item{$\Delta h = h_{1} - h_{0} = Constant$
}
\item{$\Delta h$ or maximum 
overshoot distance $d \propto \frac{w^{*}}{N}$ where $w^{*}$ is the relevant vertical velocity scale and $N = \sqrt{\frac{g}{\overline{\theta}} \frac{\partial \overline{\theta}}{\partial z}}$ is the Brunt-Vaisalla frequency}
 \item{and that between $h_{0}$ and $h_{1}$ $\overline{\theta} = \overline{\theta}_{ML} + f(z,t) \Delta \theta$ where $f(z,t)$ is a dimensionless shape factor}
\end{itemize}
 (\citeauthor{Deardorff79} \cite{Deardorff79}, \citeauthor{Stull73} \cite{Stull73}).\\

Although development of these models is beyond the scope of this thesis, mention of them is necessary to give context to the scaling 
relationships or parametrizations considered. \\         

\subsection{Numerical Simulations}
\label{subsec:}

Numerical simulation of the convective boundary layer (\acs{CBL}) is carried out by solving the Navier Stokes equations, simplified according to a suitable approximation, on a discrete grid.  Types of simulations can be grouped according to the scales of motion they resolve.  In direct numerical simulations (\acs{DNS}) the full range of spatial and temporal turbulence are resolved from the size of the domain down to the smallest dissipative scales i.e. the Kolmagorov micro-scales.  This requires a dense numerical grid and so can be computationally prohibitive.  In a large eddy simulation (\acs{LES}) smaller scales are filtered out and parametrized by sub grid scale closure model. General circulation models (\acs{GCM}) solve the Navier Stokes equations on a spherical grid and parametrize smaller scale processes including convection and cloud cover.\\

\acs{LES} has steadily, repeatedly been used to better understand the \acs{CBL} since \citeauthor{Deardorff72} applied this relatively 
new method in \cite{Deardorff72} for this purpose.  \citeauthor{SullMoengStev} in \cite{SullMoengStev}, \citeauthor{FedConzMir04} in \cite{FedConzMir04} and \citeauthor{BrooksFowler2} in \cite{BrooksFowler2} used it to observe the structure and scaling behaviour of the \acs{EL}.\\

%%%%%%%%%%%%%%%%%%%%%%%%%%%%%%%%%%%%%%%%%%%%%%%%%%%%%%%%%%%%%%%%%%%%%%
\section{Scales of the CBL and Entrainment Layer}
\label{sec:scales}

\subsection{Length Scale ($h$)}
\label{subsec:}

\citeauthor{Deardorff72} in \cite{Deardorff72} demonstrated that the inversion base height
scales the sizes of the dominant turbulent structures in penetrative convection.
This was taken to be the height of minimum average vertical heat flux ($z_{f}$) 
(\citeauthor{DearWill80} \cite{DearWill80}).  Since then, the concept of 
\acs{CBL} height ($h$) has remained reasonably consistent in that it is measured at the inversion or point,
above the surface layer, of maximum change in tracer concentration or potential temperature ($\theta$). Turbulence based concepts, 
such as the velocity variance and the distance over which velocity is correlated with itself,
are related but represent the current turbulent dynamics rather than the turbulence history (\citeauthor{Traum11} \cite{Traum11}).\\

\subsection{Convective Velocity Scale ($w^{*}$)}
\label{subsec:}

Given an average surface vertical heat flux ($\overline{w^{'}\theta^{'}}_{s}$) a surface buoyancy flux can be defined as 
$\frac{g}{\overline{\theta}}\overline{w^{'}\theta^{'}}$ from which the convective velocity scale is obtained by
multiplying by the appropriate length scale.  Since the result is in $\frac{m^{3}}{s^{3}}$ a cube root is applied.\\

\begin{equation}
w^{*} = \left( \frac{gh}{\overline{\theta}}\overline{w^{'}\theta^{'}} \right)^{\frac{1}{3}}
\end{equation}\\

\citeauthor{Deardorff70} (\cite{Deardorff70}) confirmed that this effectively scaled the vertical turbulent velocity
perturbations ($w^{'}$) in the \acs{CBL}.  \citeauthor{Sorbjan}'s work in \cite{Sorbjan} supports this, even at the 
\acs{CBL} top.  $\frac{dh}{dt}$ and $w^{'}$ are driven by $\overline{w^{'}\theta^{'}}_{s}$ and inhibited by  $\gamma$. 
The influence of $\gamma$ on $w^{'}$ is indirectly accounted for via $h$ in $w^{*}$.\\

\subsection{Convective Time Scale ($\tau$)}
\label{sec:}

It logically follows that the time for a thermal to reach the top of the \acs{CBL} is

\begin{equation}
\tau = \frac{h}{\left( \frac{gh}{\overline{\theta}}\overline{w^{'}\theta^{'}} \right)^{\frac{1}{3}}}
\end{equation}

 \citeauthor{SullMoengStev} showed a linear relationship
 between $h$ and time scaled by this time scale in \cite{SullMoengStev}. The time scale
 associated with the buoyant thermals overshooting and sinking (Brunt-Vaisala
frequency) is another obvious choice (\citeauthor{FedConzMir04} \cite{FedConzMir04}).
The ratio of these two time-scales forms a parameter which characterizes this system.
(see \citeauthor{Sorbjan}\cite{Sorbjan} and \citeauthor{Deardorff79} \cite{Deardorff79}) 


\subsection{Convective Temperature Scale ($\theta^{*}$)}
\label{sec:}

The \acs{CBL} temperature fluctuations $\theta^{'}$ are influenced by $\overline{w^{'}\theta^{'}}$ from both the surface and the \acs{CBL} top.
\citeauthor {Deardorff70} (\cite{Deardorff70}) showed that an effective scale based on the convective velocity scale is

\begin{equation}
\theta^{*} = \frac{\overline{w^{'}\theta^{'}}}{w^{*}}
\end{equation} 

Whereas \citeauthor{Sorbjan} (\cite{Sorbjan}) showed that as with proximity to the \acs{CBL} top the effects of $\gamma$ become more important.
 
\subsection{Buoyancy Richardson Number (\acs{Ri})}
\label{sec:}

The flux Richardson ($R_{f}$) number expresses the balance between turbulent mechanical energy and buoyancy.  It's obtained from the ratio of these two terms in the turbulent kinetic energy budget equation (\citeauthor{Stull-BLMetIntro} \cite{Stull-BLMetIntro}):

\begin{equation}
\frac{\partial \overline{e}}{\partial t} + \overline{U}_{j} \frac{\partial \overline{e}}{\partial x_{j}} = \delta_{i3}  \frac{g}{\overline{\theta}} \left( \overline{u_{i}^{'}\theta^{'}} \right) - \overline{u_{i}^{'}u_{j}^{'}}\frac{\partial \overline{U}_{i}}{\partial x_{j}} - \frac{ \partial \left( u_{j}^{'}e^{'} \right)}{\partial x_{j}} - \frac{1}{\overline{\rho}} \frac{\partial \left( u_{i}^{'} p^{'} \right) }{\partial x_{i}} - \epsilon
\end{equation}

\begin{equation}
R_{f} = \frac{\frac{g}{\overline{\theta}} \left( \overline{w^{'}\theta^{'}} \right)}{\overline{u_{i}^{'}u_{j}^{'}}\frac{\partial \overline{U}_{i}}{\partial x_{j}}}
\end{equation}
 
Assuming horizontal homogeneity and neglecting subsidence
  
\begin{equation}
R_{f} = \frac{\frac{g}{\overline{\theta}} \left( \overline{w^{'}\theta^{'}} \right)}{\overline{u^{'}w^{'}}\frac{\partial \overline{U}}{\partial z} + \overline{v^{'}w^{'}}\frac{\partial \overline{V}}{\partial z}}
\end{equation}

Applying first order closer to the flux terms, i.e. assuming they are proportional to the vertical gradients, gives the gradient Richardson number ($R_{g}$)

\begin{equation}
R_{g} = \frac{ \frac{g}{\overline{\theta}} \frac{\partial \overline{\theta}}{\partial z}}{\left( \frac{ \partial \overline{U}}{\partial z} \right)^{2} + \left( \frac{\partial \overline{V}}{\partial z} \right)^{2}} 
\end{equation}

which expresses the balance between shear and buoyancy driven turbulence, but in the \acs{EL} buoyancy acts to suppress buoyancy driven turbulence.  
Applying a bulk approximation to the denominator, and expressing it in terms of scales yields a ratio of two square of time scales

\begin{equation}
R_{g} = \frac{\frac{g}{\overline{\theta}} \frac{\partial \overline{\theta}}{\partial z}}{\frac{U^{*2}}{L^{2}}} = N^{2}\frac{L^{2}}{U^{*2}}
\end{equation}


and applying the bulk approximation to both the numerator and the denominator yields

\begin{equation}
R_{b} = \frac{\frac{g}{\overline{\theta}} \Delta \theta L}{U^{*2}}
\end{equation}

A natural choice of length and velocity scales for the \acs{CBL} are $h$ and $w^{*}$.  \citeauthor{EllTurn} (\cite{EllTurn}) suggested and confirmed a relationship between the entrainment rate and this form of 
Richardson number (\acs{Ri}) based on tank experiments.  This parameter can be justified and arrived at by considering the principal forcings of the system, or from non-dimensionalizing the entrainment relation  derived
analytically (\citeauthor{Tennekes73}  \cite{Tennekes73}, \citeauthor{Deardorff72} \cite{Deardorff72}).

\begin{equation}
w_{e} \propto \frac{\overline{w^{'}\theta^{'}}_{s}}{\Delta \theta}
\end{equation}

\begin{equation}
\frac{w_{e}}{w^{*}} \propto  \frac{\overline{w^{'}\theta^{'}}_{s}}{\Delta \theta w^{*}} = Ri^{-1}
\end{equation}
 

In one or other of its forms this parameter has become central to any study on \acs{CBL} entrainment (\citeauthor{SullMoengStev} \cite{SullMoengStev}, \citeauthor{FedConzMir04} \cite{FedConzMir04}, \citeauthor{Traum11} \cite{Traum11}, \citeauthor{BrooksFowler2} \cite{BrooksFowler2})


\subsection{Relationship of Entrainment Rate and Entrainment Layer Depth to Richardson Number}

The relationship between scaled entrainment rate and the buoyancy Richardson number (\acs{Ri}) is arrived at according the zero order bulk
model through thermodynamic arguments, or by integration of the conservation of enthalpy or turbulent kinetic energy equations
over the growing \acs{CBL}. (\citeauthor{Tennekes73} \cite{Tennekes73}, \citeauthor{Deardorff79} \cite{Deardorff79}, 
\citeauthor{FedConzMir04} \cite{FedConzMir04}). 

\begin{equation}
\frac{w_{e}}{w^{*}} \propto  Ri^{-a}
\end{equation}


It has been verified in numerous laboratory and numerical studies (\citeauthor{DearWill80} \cite{DearWill80}, \citeauthor{SullMoengStev} \cite{SullMoengStev}, \citeauthor{FedConzMir04} \cite{FedConzMir04}, \citeauthor{BrooksFowler2} \cite{BrooksFowler2}).  But there is still some 
unresolved discussion as the the exact value of a.  It seems there are two possible values, $-\frac{3}{2}$ and $-1$, the first of which \citeauthor{EllTurn} (\cite{EllTurn}) suggested occurs at high stability when buoyant recoil of impinging thermals becomes more important than their convective overturning.  \citeauthor{FedConzMir04} (\cite{FedConzMir04}) arrive at this power law through an \acs{Ri} obtained using the potential temperature jump across the \acs{EL}.\\

A relationship of the scaled entrainment layer \acs{EL} depth to \acs{Ri} is arrived at by considering the deceleration of a thermal
as it overshoots its natural buoyancy level (\citeauthor{StullNelEl} \cite{StullNelEl}), such that its overshoot distance is

\begin{equation}
d \propto \frac{w^{*2}}{\frac{g}{\overline{\theta}_{ML}} \Delta \theta} 
\end{equation} 

If the \acs{EL} depth is proportional to the overshoot distance then

\begin{equation}
\frac{\Delta h}{h} \propto \frac{w^{*2}}{\frac{g}{\overline{\theta}_{ML}} h \Delta \theta} = Ri^{-1} 
\end{equation} 

\citeauthor{Boers89} \cite{Boers89} integrated the potential and thermal energy difference before and after distortion of 
the inversion interface with the assumption that the resulting variation in the shape is sinusoidal.  He equated this to the total 
kinetic energy of the \acs{CBL} and arrived at a $-\frac{1}{2}$ power law relationship

\begin{equation}
\frac{\Delta h}{h} \propto Ri^{-\frac{1}{2}} 
\end{equation} 


%%%%%%%%%%%%%%%%%%%%%%%%%%%%%%%%%%%%%%%%%%%%%%%%%%%%%%%%%%%%%%%%%%%%%%
\section{Approach to Research Questions}
\label{sec:Approach}

Similar to (list) I will model the dry shear free convective boundary layer using Large Eddie Simulation, specifically the cloud resolving model of Marat System for Atmospheric Modelling (SAM).  The set up will be slightly different in that I will use a slightly smaller domain than usuall (vs 5x5) but I will run a 10 case ensemble to obtain true ensemble averages and so turbulent potential temperature perturbations. The chosen grid sizes were very much influenced by the \citeauthor{SullPat} \citeyear{SullPat} and in the vertical is of higher resolution than the other comparable studies (Sull Moeng, Fed, BandF). Before addressing the questions stated in Section \ref{sec:Mot} I'll examine the model to make sure it represents a realistic turbulent \acs{CBL} in Section \ref{sec:CheckingtheModel}. I will look at the averaged vertical profiles of velocity and temperature, and tunrbulet kinetic energy as represented my the root mean squared velocity profiles and make sure they are in line with those seen before (SchmidtSchu, Sull, Fed).  I will look at the local temperature and velocity profiles to see that coherent thermals are being produced.  I will look at the fft energy density spectra to see that there is adequate scale separation between the most energy intense structures and the grid spacing and that there is a slope corresponding kolmagorov power law.  I will initiallize with a constant surface heat flux acting a against a uniform lapse rate.  This is different to Sullivan and Moeng and Brooks and Fowler.  They obtained a range of \acs{Ri} by varying an imposed a $\theta$ jump while keeping the same $\gamma$.  So here, the  $\theta$ jump arises from the overshoot of the thermals (references here, Garcia Mellado).  

\subsection{Q1: How do the distrubutions of local \acs{CBL} height, and $\theta^{'}w^{'}$ within the \acs{EL}, vary with $\overline{w^{'}\theta^{'}}_{s}$ and $\gamma$?}     

The \acs{EL} can be thought of in terms of the distribution of the thermal heights, or local heights.  Sullivan and moeng measured local height by locating the vertical point of maximum gradient, and observed the effects of varying \acs{Ri} on the resulting distributions. However this method is problematic when gradients in the upper profile exceed that at the inversion (cite BandF).  Steyn et al fitted an idealized curve to a lidar backscatter profile.  This method produces a smooth curve ased on the full original profile on which a mximum can easily be located.  Similarly, I will apply a multi-linear regression method outlined in \citeauthor{Vieth} to the local $\theta$ profiles one line representing the \acs{ML}, \acs{EL} and \acs{FA} and locate the \acs{ML} top.  I'll observe how the resulting distributions are effected by changes in \acs{Ri} using histograms and verify that the resulting surface corresponds to the location of the termals in the \acs{EL} (See Section %\ref{}
)
\\
Sullivan and Moeng broke the turbulent vertical heat flux into four quadrants and used this combined with local flow visualizations to show how the thermals impinge and draw down warm air from above.  Mahrt and Paulmier used 2 dimensional contour plots of local $w^{'}$ and $\theta^{'}$ on measurements and analysed their joint distribubions.  In his \acs{LES} study Sorbjan showed that in the \acs{CBL} in particular in the \acs{EL} $\theta^{'}$ is strongly influenced by $\gamma$  whereas $w^{'}$ is independent thereof.  Influenced by these three studies, I will use 2 dimensional histograms at three levels within the \acs{EL} to look at how the distrubutions of local $w^{'}$ and $\theta^{'}$ are effected by changes in $\gamma$ and $\overline{w^{'}\theta^{'}}_{s}$ .  I will try to isolate the effects of $\gamma$, other than on the heights reached, by applying the convective scales, $\theta^{*}$ and $w^{*}$.\\    

\subsection{\textbf{Q2: How can the \acs{EL} limits be defined based on the $\overline{\theta}$ profile and what is the relationship of the resulting depth ($\Delta h$) to \acs{Ri}?}}
       
Here I define the \acs{CBL} height as the location of the maximum in the vertical $\overline{\theta}$ gradient profile.  The lower and upper \acs{EL} limits are then, the points at which $\frac{\partial \overline{\theta}}{\partial z}$ significantly exceeds zero and where it resumes $\gamma$.  The lower limit requires choice of a threshold value which should be small, positive and less than $\gamma$. Since it is somewhat arbitrary I will produce the important plot based on three different threshold values in section \ref{sec:hdeltahavprofs}.  Fed et al defined the \acs{EL} in termps of the vertical average turbulent heat flux profiles as in Figure \ref{fig:hdefs}.  BandF compared results based on the heat flux profile and two statistical methods ie high and low percentiles of the pdf of local heights, and averages of the locally determined \acs{EL} limits. These two studies disagree on the shape of the relationship of scaled \acs{EL} depth to \acs{Ri}.  As well as plotting this relationship using my heights defined above and in Figure \ref{fig:hdefs} I will plot it using the flux based definitions to try and resolve the conflict I just mentioned.\\  

\begin{figure}[htbp]
    \centering
    %plot_height.py[master 1573b9d] h vs time plot
    \includegraphics[scale=.5]{/newtera/tera/phil/nchaparr/python/Plotting/Dec252013/pngs/height_defs.pdf}
    \caption{Height Definitions Based on the Average Vertical Profiles }
    \label{fig:hdefs}   % label should change
\end{figure}

\begin{table}[htbp]
    \begin{center}
%\centerline{
    \begin{tabular}{ p{2cm} p{2cm}  p{2cm}  p{2cm} p{2cm} }
    %\hline
      \acs{CBL} Height & \acs{ML} $\overline{\theta}$ & \acs{EL} Limits & $\theta$ Jump & \acs{Ri} \\ \hline 
       $h$ (see Figure \ref{fig:hdefs})& $\overline{\theta}_{ML} = \frac{1}{h}\int^{h}_{0}\overline{\theta}(z)dz$ & $h_{0}$, $h_{1}$ & $\delta \theta=\overline{\theta}(h_{1})-\overline{\theta}(h_{0})$ & \acs{Ri} $=\frac{\delta \theta}{\frac{\overline{w^{'}\theta^{'}}_{s}}{w^{*}}}$  \\ %\hline 
       & & &$\Delta \theta = \overline{\theta}_{0}(h)- \overline{\theta}_{ML}$ & \acs{Ri}$=\frac{\Delta \theta}{\frac{\overline{w^{'}\theta^{'}}_{s}}{w^{*}}}$ \\ \hline
      \end{tabular}
%}
\caption{Relevant Definitions used in this Study}
\label{table:reldefs}   
\end{center}    
\end{table}

\subsection{Q3: How does defining the $\theta$ jump based on the vertical $\overline{\theta}$, i.e. accross the \acs{EL} as in Figure \ref{fig:1stord} vs at the inversion ($h$) as in Figure \ref{fig:0order}, effect the entrainment relation, in particular $a$?}

I will vary the definition of $\Delta \theta$ and so \acs{Ri} to see how this effects the resulting entrainment relation.  In particular I wish to see the value of $a$ so I'll plot in log coordinates.  I will repoduce this plot using the flux profile definitions for comparison with the results of Fedorovich and Garcia.

\endinput

Any text after an \endinput is ignored.
You could put scraps here or things in progress.
%\begin{epigraph}
 %   \emph{If I have seen farther it is by standing on the shoulders of
  %  Giants.} ---~Sir Isaac Newton (1855)
%\end{epigraph}

%This document provides a quick set of instructions for using the
%\class{ubcdiss} class to write a dissertation in \LaTeX. 
%Unfortunately this document cannot provide an introduction to using
%\LaTeX.  The classic reference for learning \LaTeX\ is
%\citeauthor{lamport-1994-ladps}'s
%book~\cite{lamport-1994-ladps}.  There are also many freely-available
%tutorials online;
%\webref{http://www.andy-roberts.net/misc/latex/}{Andy Roberts' online
 %   \LaTeX\ tutorials}
%seems to be excellent.
%The source code for this docment, however, is intended to serve as
%an example for creating a \LaTeX\ version of your dissertation.

%We start by discussing organizational issues, such as splitting
%your dissertation into multiple files, in
%\autoref{sec:SuggestedThesisOrganization}.
%We then cover the ease of managing cross-references in \LaTeX\ in
%\autoref{sec:CrossReferences}.
%We cover managing and using bibliographies with \BibTeX\ in
%\autoref{sec:BibTeX}. 
%We briefly describe typesetting attractive tables in
%\autoref{sec:TypesettingTables}.
%We briefly describe including external figures in
%\autoref{sec:Graphics}, and using special characters and symbols
%in \autoref{sec:SpecialSymbols}.
%As it is often useful to track different versions of your dissertation,
%we discuss revision control further in
%\autoref{sec:DissertationRevisionControl}. 
%We conclude with pointers to additional sources of informat%ion in
%\autoref{sec:Conclusions}.

  
%The \acs{UBC} \acf{FoGS} specifies a particular arrangement of the
%components forming a thesis.\footnote{See
 %   \url{http://www.grad.ubc.ca/current-students/dissertation-thesis-preparation/order-components}}
%This template reflects that arrangement.

%In terms of writing your thesis, the recommended best practice for
%organizing large documents in \LaTeX\ is to place each chapter in
%a separate file.  These chapters are then included from the main
%file through the use of \verb+\include{file}+.  A thesis might
%be described as six files such as \file{intro.tex},
%\file{relwork.tex}, \file{model.tex}, \file{eval.tex},
%\file{discuss.tex}, and \file{concl.tex}.

%We also encourage you to use macros for separating how something
%will be typeset (\eg bold, or italics) from the meaning of that
%something. 
%For example, if you look at \file{intro.tex}, you will see repeated
%uses of a macro \verb+\file{}+ to indicate file names.
%The \verb+\file{}+ macro is defined in the file \file{macros.tex}.
%The consistent use of \verb+\file{}+ throughout the text not only
%indicates that the argument to the macro represents a file (providing
%meaning or semantics), but also allows easily changing how
%file names are typeset simply by changing the definition of the
%\verb+\file{}+ macro.
%\file{macros.tex} contains other useful macros for properly typesetting
%things like the proper uses of the latinate \emph{exempli grati\={a}}
%and \emph{id est} (\ie \verb+\eg+ and \verb+\ie+), 
%web references with a footnoted \acs{URL} (\verb+\webref{url}{text}+),
%as well as definitions specific to this documentation
%(\verb+\latexpackage{}+).

 
%\LaTeX\ make managing cross-references easy, and the \latexpackage{hyperref}
%package's\ \verb+\autoref{}+ command\footnote{%
 %   The \latexpackage{hyperref} package is included by default in this
 
%   template.}
%makes it easier still. 

%A thing to be cross-referenced, such as a section, figure, or equation,
%is \emph{labelled} using a unique, user-provided identifier, defined
%using the \verb+\label{}+ command.  
%The thing is referenced elsewhere using the \verb+\autoref{}+ command.
%For example, this section was defined using:
%\begin{lstlisting}
%    \section{Making Cross-References}
 %   \label{sec:CrossReferences}
%\end{lstlisting}
%References to this section are made as follows:
%\begin{lstlisting}
 %   We then cover the ease of managing cross-references in \LaTeX\
  %  in \autoref{sec:CrossReferences}.
%\end{lstlisting}
%\verb+\autoref{}+ takes care of determining the \emph{type} of the 
%thing being referenced, so the example above is rendered as
%\begin{quote}
%    We then cover the ease of managing cross-references in \LaTeX\
%    in \autoref{sec:CrossReferences}.
%\end{quote}

%The label is any simple sequence of characters, numbers, digits,
%and some punctuation marks such as ``:'' and ``--''; there should
%be no spaces.  Try to use a consistent key format: this simplifies
%remembering how to make references.  This document uses a prefix
%to indicate the type of the thing being referenced, such as \texttt{sec}
%for sections, \texttt{fig} for figures, \texttt{tbl} for tables,
%and \texttt{eqn} for equations.

%For details on defining the text used to describe the type
%of \emph{thing}, search \file{diss.tex} and the \latexpackage{hyperref}
%documentation for \texttt{autorefname}.
