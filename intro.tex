%% The following is a directive for TeXShop to indicate the main file
%%!TEX root = diss.tex

\chapter{Introduction} 
\label{ch:Introduction}
\setlength{\parindent}{0cm}

\section{Motivation}
\label{sec:}

Atmospheric boundary layer growth by entrainment driven by convecive thermals acting against a stable lapse rate
or inversion makes for an interesting topic.  (list of studies) It is a subsection of a broader topic of interest in geophysical fluid dynamics ie that of entrainment of a fluid into a turbulent fluid accross a density interface.  It involves the buoyant suppression of buoancy driven plumes, and subsequent trapping and mixing of the more buoyant air from accross the buoyancy/density gradient or jump.\\

Boundary layer height and prediction there of are important for calculating tracer IE pollutant concentrations
and the limiting sizes of turbulant structures.  (Steyn Baldi Hoff) Also the rate at which the boundary layer entrains air from above
enables calculation of the rate at which tracers transported long range are drawn in an mixed with the air we breathe.\\

In combination with the lifting condensation level, knowledge of entrainment layer depth allows predictions pertaining
to the formation of cumulous clouds for example if the \acs{EL} is above the lifting condensation a high percentage of cloud cover
would be expected.  Parametrizations for both \acs{CBL} growth and \acs{EL} depth are used in Mesoscale models and GCMs.  Further along the vein of prediction and parametrization it is an attractive goal to develop a robust set of scales such that a similarity theory as analogous to MO theory for the surface for this region could be developed (quote from GarciaMellado, Sorbjan1999).\\

Bulk analyitical models underly certain parametrizations and scaling relatiohships (Traumner) and are validated by measurements 
and output from LES and DNS models. The two important categories referenced here are the zero order whcih
assume an infinitessimly thin entrainment layer and the 1st order which includes an entrainment layer of finite depth
and there is discussion in the literature as to whether explicit consideration of the \acs{EL} depth is required. Both
types typically relies on the simplified reynolds averaged thermodynamic equation 

\begin{equation}
\frac{D \overline{\theta}}{Dt} = -\frac{\partial \overline{w^{'}\theta^{'}}}{\partial z}
\end{equation}

relating average potential temperature to average vertical turbulent potential temperature flux, as well as idealized vertical profiles for both.  So verification of the resulting entrainment equation by LES using ensemble averaged $\theta$ and $w^{'}\theta^{'}$ logically follows.  So far, horizontal and time averaging has been
used but we will run an ensemble of cases, and apply true ensemble averaging as well as horizontaly averaging to get smooth vertical profiles.\\  


Studies in which the scaling relation or parametrizations for $\Delta h$ is verified (Deardorf tank, SullMoengStev, Fed, BrooksFow, GarciaMellado) refer to the potential temperature or buoyancy inversion region but do not define the entrainment layer \acs{EL} limits it in terms of the 
vertical average potential temperature profile.  So it seemed like an obvious opportunity to try.\\

Convective boundary layer entrainment depends most critically on surface heat flux and upper lapse rate (Fed04 from Sorbjan1996) 
but so far similar studies (ie those verifying the entrainment rate and entrainment layer depth relations) have focused on acheiving 
a range of Richardson numbers by varying $\overline{w^{'}\theta^{'}}_{s}$ and the temperature jump (SullMoeng, BrooksFow), whereas Fedorovich varied the upper lapse rate while maintaining a constant surface heat flux.  We choose to vary these two key external parameters, allowing a $\Delta \theta$ to develop with penetrative convection and entrainment.  Garcia and mellado came out after our runs were done, but they seem to vary both. \\

Sullivan and Pattons resolution study showed the sensitivity of the heat flux and potential temperature profiles to grid size in particular in the vertical and in the entrainment layer.  This strongliy influenced our decision to apply a higher resolution within the entrainment layer than the previous similar LES studies.\\ 

%%%%%%%%%%%%%%%%%%%%%%%%%%%%%%%%%%%%%%%%%%%%%%%%%%%%%%%%%%%%%%%%%%%%%%
\section{Relevant Background}
\label{sec:}
\subsection{The Convective Boundary Layer (CBL)}

The convective boundary layer over land starts to grow at rapidly at sunrise reaching a peak at midday with the peak in solar irradiation.
Convective turbulence and the dominant upward vertical motions die down towards the evening and as surface cools. In the morning as the 
surface warms relative to the environment instability causes thermals to develop and rise with buoyancy driven momentum. The are adjacent to 
and closely associated with cooler slower downdraughts.  The thermals or thermal plumes are of uniform potential temperature and tracer 
concentration at their cores and entrain surrounding air laterally as they rise, as well as trapping and mixing in stable warm from above. 
(\citeathor{Stull-BLMetIntro} \cite{Stull-BLMetIntro}, \citeauthor{CrumStullEl} \cite{CrumStullEl}) In cases of stromg convection, 
Buoyantly driven turbulence dominates and shear is insignificant. (\citauthor{DirLEddy} \cite{DirLEddy}) Thermals rise, overshoot their
natural buouancy level and overturn or recoil, trapping pockets (or whisps) of warm stable air which then becomes turbulently mixed.  This
overshoot and subsequent entrainment of the warmer air from aloft augments the warming cased by the surface heat flux.  A potential
temperature jump results, but an inversion may also be imposed and strengthened for example by subsidence.  

Lidar images show the overall structure of the convective boundary layer (\acs{CBL}) with the rising thermals, impinging on the air above.
(\citeauthor{CrumStullEl} \cite{CrumStullEl}, \citeauthor{Traum11} \cite{Traum11}) This has been effectively modelled by large eddy simulation
by \citeauthor{SchmidtSchu} in \cite{SchmidtSchu} where using horizontal slices at various levels show how the thermals apparently merge 
with some distance from the surface.  At the top of the \acs{CBL} they observed thermal tips impinging with concurrent periferal downward motions.
This is supported in the visual tools used by \citeauthor{SullMoengStev} \cite{SullMoengStev}.  They show vertical cross sections of relatively cooler thermal plumes and trapped warmer air as well as the closely associated upward motion of cooler air and downward motion of warmer air.\\ 

On average these convective turbulent structures create a mixed layer \acs{ML} with eddy scales cascading from approximately the \acs{CBL} height 
to molecular diffusion according to the Kolmagorav power law.  Here potential temperature is close to uniform and average heat
flux is positive and decreasing. Warming is from both the surface heat flux ($w^{'}\theta^{'}_{s}$) and the flux of entrained stable air at the
 inversion ($w^{'}\theta^{'}_{i}$).  Since the \ac{ML} is mainly comprised of warm updraughts and cool downdraughts the $w^{'}\theta^{'}_{s}$ 
is on average positive, turning negative in the entrainment layer \acs{EL} where the updraughts are now relatively cool and there is downward
movement of warm air from above.  Above the \acs{ML} the air becomes more stable with altitude and on average this reflects as transition from
a uniform \acs{ML} potential temperature to the stable lapse rate.  A peak in the gradient represents regions where thermal plumes are at points 
higher than their natural buoyancy level. \\

\citeauthor{StullNelEl} in \cite{StullNelEl} outlines the stages of \acs{CBL} growth as the sublayers of the nocturnal
boundary layer are entrained, untill the previous days capping inversion is reached and a quasi-steady state growth 
is attained.  The \acs{EL} depth relative to \acs{CBL} height varies throughout these stages and its relationship
to scaled entrainment rate forms a hysteresis.  The studies based on LES output typically represent this last quasi-steady
phase, since there is usually a constant heat flux working against an inversion and or a stable lapse rate. 
(\citeauthor{SchmidtSchu} \cite{SchmidtSchu}, \citeauthor{Sorbjan} \cite{Sorbjan}, \citeauthor{SullMoengStev} \cite{SullMoengStev}, 
\citeauthor{FedConzMir04} \cite{FedConzMir04}, \citeauthor{BrooksFowler2} \cite{BrooksFowler2})  

\subsection{Convective Boundary Layer Height}
\label{subsec:}

The \acs{ML} is fully turbulent with an on average uniform temperature. Aerosol and water vapour concentrations 
decrease dramatically with transition to the stable upper atmpsphere.  So any of these characterstics can support
a definition of \acs{CBL} height.  \citeauthor{StullNelEl} define height in terms of percentage of \acs{ML} air
and identified by eye from Lidar backscatter images in \cite{StullNelEl}.  \citeauthor{Traum11} used
the four most commonly used automated methods applied to Lidar images: identifying a suitable threshold value 
above which the air is categorized as \acs{ML} air,  locating the point of minimum (largest negative) 
vertical gradient, fitting an idealized curve to the profile from which a location of minimum vertical
gradient can be easily determined, and using wavelet covariance in \cite{Traum11}.\\

The use of Lidar dominates studies based on measurement. In numerical modelling studies there
is the advantage of having plently of points to average over as well as the ability to directly
measure veritcal heat flux. \citeauthor{BrooksFowler2} \cite{BrooksFowler2}
applied their wavelet technique to local vertical tracer profiles in their \acs{LES} study as well as using the
gradient method (IE locating the point of minimum vertical gradient).  For comparison they also used the point of 
miniumum average vertical heat flux (\overline{w^{'}\theta^{'}}).  This last definition has been the common
in \acs{LES} studies and has frequently been referred to as $z_{i}$ (inversion height) (\citeathor{DearWill80} 
\cite{DearWill80}, \citeauthor{Sorbjan1} \cite{Sorbjan1}, \citeauthor{FedConzMir04} \cite{FedConzMir04}).
 \citeauthor{SullMoengStev} \cite{SullMoengStev} clarified that this point did not correspond to the
average point of maximum vertical potential temperature gradient, whereas the upper extrema of the four vertical
heat flux quadrants (upward moving warm air, dowward moving warm air, upward moving cool air, downward moving
cool air) more or less did. They defined \acs{CBL} height based on local vertical potential temperature gradients
and applied horizontal averaging as well as other methods, including two based on the average vertical heat flux
,for comparison.\\

None of the \acs{LES} studies so far define the height in terms of the average potential temperature profile
even though bulk models, from which \acs{CBL} growth parametrizations stem, rely on an idealized version thereof.
\citeauthor{GarciaMellado} do include it as one of their measures of \acs{CBL} height in their direct numerical
simulation study (\acs{DNS}) \cite{GarciaMellado}.       

\subsection{Convective Boundary Layer Growth by Entrainment}
\label{subsec:}

In the quasi-steady regime the \acs{CBL} grows by trapping pockets of warm stable air between
or adjacent to impinging thermal plumes.  \citeauthor{Traum11} \cite{Traum11} summarize two
relevant buoyancy driven regimes of entrainment:\\

Non turbulent fluid can be engulfed between or in the overturning of thermal plumes. This kind of
event was seen by \citeathor{SullMoengStev} in \cite{SullMoengStev} when the iversion was weak. 
\citeauthor{Traum11}'s observations in \cite{Traum11} support this.\\

Impinging thermal plumes distort the inversion interface dragging whisps of warm stable air down
at their edges or during recoil under a strong inversion or lapse rate. This type of event is supported 
by the finding  of both \citeathor{SullMoengStev} \cite{SullMoengStev} and \citeauthor{Traum11} \cite{Traum11}.\\

Under atmospheric conditions shear induced stabilities do occur, and in some laboratory studies 
under very high stability the breaking of internal waves have been observed.  Both processes are 
believed to result in some entrainment but we dismiss since the former is insignificant relative
to and the latter has not so far been observed in measurements or modeled output of the 
atmospheric \acs{CBL} (\citeauthor{Traum11} \cite{Traum11}, \citeathor{SullMoengStev} \cite{SullMoengStev})

\subsection{The Convective Boundary Layer Entrainment Layer}
\label{subsec:}

locally, ie horizontal 2d, to horizontal average, to idealized/model representation.

lower layer, more turbulent upper layer stability with intermittent turbulence
GarciaMellado's two layer structure makes perfect sense.

what this looks like from measurements eg lidar traum and LES sullmoeng.

how it looks in terms of temperature.  then the movement of up and down warm and cool air ie the 
fluxes.  

traumner talks about two concepts, statistical vs transition zone concept -- by the way it seems the transition zone concept
was implimented by fitting a curve of which the transition zone was a parameter.

%%%%%%%%%%%%%%%%%%%%%%%%%%%%%%%%%%%%%%%%%%%%%%%%%%%%%%%%%%%%%%%%%%%%%%
\section{Modelling the Convective Boundary Layer and Entrainment Layer}
\label{sec:}

\subsection{Bulk Analyitical Models}
\label{subsec:}

\subsection{Numerical Simulations}
\label{subsec:}

cover initial conditions here, ie delta theta vs constant lapse rate

%%%%%%%%%%%%%%%%%%%%%%%%%%%%%%%%%%%%%%%%%%%%%%%%%%%%%%%%%%%%%%%%%%%%%%
\section{Scales of the CBL and Entrainment Layer}
\label{sec:}

\subsection{Length Scale}
\label{subsec:}


\subsection{Convective Velocity Scale}
\label{subsec:}

Given an average surface heat flux ($\overline{w^{,}\theta^{,}}_{s}$) a surface buoyancy flux can be defined as 
$\frac{g}{\overline{\theta}}\overline{w^{,}\theta^{,}}$ from which the convective velocity scale is obtained by
multiplying by the appropriate length scale.  Since the result is in $\frac{m^{3}}{s^{3}}$ a cube root is applied.\\

\begin{equation}
w^{*} = \left( \frac{gh}{\overline{\theta}}\overline{w^{,}\theta^{,}} \right)
\end{equation}

\citeauthor{Deardorff70} (\cite{Deardorff70}) confirmed that this effectively scaled the vertical turbulent velocity
perturbations in the \acs{CBL}.\\

this scales the vertical velocity perturbations in the cbl. 
can i tie in something about the joint distributions of theta and w?
ie talk about the $\theta^{'}$ and $w^{'}$
for example $w^{'}$ are driven by $\overline{w^{'}\theta^{'}}$
must refer to Sorbjan

\subsection{Convective Time Scale}
\label{}

It logically follows that the time for a plume to reach the top of the \acs{CBL} is

\begin{equation}
\tau = \frac{h}{\left( \frac{gh}{\overline{\theta}}\overline{w^{,}\theta^{,}} \right)}
\end{equation}

where $h$ is \acs{CBL} height.

mention the brunt vaisalla time scale as used by Fed and Garcia

\subsection{Convective Temperature Scale}
\label{}

here I must refer to sorbjan.  $\theta^{'}$ is influenced by $\overline{w^{'}\theta^{'}}_{s}$
in the lower part of the \acs{CBL} but by $\gamma$ in the upper \acs{CBL} and \acs{EL}

\subsection{Richardson Number}
\label{}

The Flux Richardson ($R_{f}$) expresses the balance between turbulent mechanical energy and buoyancy.  It's obtained from the ration of these two terms
 in the Turbulent Kinetic Energy budget equation (\cite{Stull-BLMetIntro}):

\begin{equation}
\frac{\partial \overline{e}}{\partial t} + \overline{U}_{j} \frac{\partial \overline{e}}{\partial x_{j}} = \delta_{i3}  \frac{g}{\overline{\theta}} \left( \overline{u_{i}^{,}\theta^{,}} \right) - \overline{u_{i}^{,}u_{j}^{,}}\frac{\partial \overline{U}_{i}}{\partial x_{j}} - \frac{ \partial \left( u_{j}^{,}e^{,} \right)}{\partial x_{j}} - \frac{1}{\overline{\rho}} \frac{\partial \left( u_{i}^{,} p^{,} \right) }{\partial x_{i}} - \epsilon
\end{equation}

\begin{equation}
R_{f} = \frac{\frac{g}{\overline{\theta}} \left( \overline{w^{,}\theta^{,}} \right)}{\overline{u_{i}^{,}u_{j}^{,}}\frac{\partial \overline{U}_{i}}{\partial x_{j}}}
\end{equation}
 
Assuming horizontal homogenaiety and neglecting subsidence
  
\begin{equation}
R_{f} = \frac{\frac{g}{\overline{\theta}} \left( \overline{w^{,}\theta^{,}} \right)}{\overline{u^{,}w^{,}}\frac{\partial \overline{U}}{\partial z} + \overline{v^{,}w^{,}}\frac{\partial \overline{V}}{\partial z}}
\end{equation}

Applying first order closer to the flux terms (i.e. assuming they are proportional to the vertical gradients)  gives the gradient Richardson Number ($R_{g}$)

\begin{equation}
R_{g} = \frac{ \frac{g}{\overline{\theta}} \frac{\partial \overline{\theta}}{\partial z}}{\left( \frac{ \partial \overline{U}}{\partial z} \right)^{2} + \left( \frac{\partial \overline{V}}{\partial z} \right)^{2}} 
\end{equation}

which is usually used to express the balance between shear and buoyancy driven turbulence, but in the \acs{EL} buoancy acts to supress turbulence.  
Applying a bulk approximation the to the numerator, and expressing it in terms of scales yields a ratio of two square of time scales

\begin{equation}
R_{g} = \frac{\frac{g}{\overline{\theta}} \frac{\partial \overline{\theta}}{\partial z}}{\frac{U^{*2}}{L^{2}}} = N^{2}\frac{L^{2}}{U^{*}}
\end{equation}


and applying the bulk approximation to both the numerator and the denomnitor yields

\begin{equation}
R_{b} = \frac{\frac{g}{\overline{\theta}} \Delta \overline{\theta} L}{U^{*}}
\end{equation}
A natural choice of length and veloctiy scales for the \acs{CBL} are $h$ and $w^{*}$.  \citeauthor{EllTurn} (\cite{EllTurn}) suggested and confirmed a relationship between the entrainment rate and this form of 
Richardson number based on tank experiments.  This parameter can be justified and arrived at by considering the principal forcings of the system, or from non-dimensionalizing the entrainment relation  deriived
analyitically, i.e. (\cite{Deardorff72})

\begin{equation}
w_{e} \propto \frac{\overline{w^{,}\theta^{,}}_{s}}{\Delta \overline{\theta}}
\end{equation}

\begin{equation}
\frac{w_{e}}{w^{*}} \propto  \frac{\overline{w^{,}\theta^{,}}_{s}}{\Delta \overline{\theta} w*{*}} = Ri_{b}^{-1}
\end{equation}
 

In one or other of its forms this parameter has become central to any study on \acs{CBL} entrainment (\cite{SullMoengStev}, \cite{FedConzMir04}, \cite{Traum11}, \cite{BrooksFowler2})


\subsection{Relationship of Entrainment Rate and Entrainment Layer Depth to Richardson Number}

equilibrium entrainment regime

derivation of the scaled entrainment relation to Ri by integration of the thermodynamic equation over an infinitessimely small layer.

mention of two main power law relationships

sequence of publications katophilips, deardorff, sullmoeng, fed, bandf

two ways of deriving relation of scaled entrainment depth to ri: stull momentum balance ie plumes decelerate due to bouancy difference
and boers energy balance

list of publications where its verified, deardorf, fed, band f.  a bit about the disagreement between deardorf and bandf

%%%%%%%%%%%%%%%%%%%%%%%%%%%%%%%%%%%%%%%%%%%%%%%%%%%%%%%%%%%%%%%%%%%%%%
\section{Research Goals}
\label{sec:ResearchGoals}

The LES studies of \citeauthor{SullMoengStev} \cite{SullMoengStev}, \citeauthor{FedConzMir04} \cite{FedConzMir04} and more recently \citeauthor{BrooksFowler2} in \cite{BrooksFowler2} were carried out on grids of lower resolution than those which began to converge in \citeauthor{SullPat} \cite{SullPat}, except for the cases specifically for testing resolution effects. \citeauthor{SullPat} \cite{SullPat} influenced our choice of grid size, in particular within the \acs{EL}\\

 All of these were carried out on $5 \times 5 km$ domains and used a combination of both time and spatial averaging to obtain average profiles.  So far, we haven't seen a study involving an ensemble of cases, such that true ensemble averaging can be carried out.  We chose this setup to obtain true ensemble averages, and to have a wealth of local points for some basic statistical observations.\\

\citeauthor{FedConzMir04} \cite{FedConzMir04} varied their cases by upper lapse rate ($\gamma$) over a typical range found in the troposphere IE $1 - 10 K / Km$, whereas \citeauthor{SullMoengStev} \cite{SullMoengStev} and \citeauthor{BrooksFowler2} in \cite{BrooksFowler2} varied surface heat flux and initial inversion ($\Delta \theta$).  To obtain a range of Richardson numbers (\acs{Ri}) we varied both surface heat flux ($w^{'}\theta^{'}$) and upper lapse rate ($\gamma$).\\

\subsection{Verifying that the Model output is realistic}

Since unlike the other LES studies, we initialize with a constant surface heat flux ($\overline{w^{'}\theta^{'}}_{s}$) working against a constant lapse rate ($\gamma$), it would be good to see the formation of a convective boundary layer with the expected average profiles.\\

Given our concern about a slightly smaller domain than usual, we would like to make sure each of the individual cases are producing coherent turbulent structures and that there is adequate scale separation between the structures with highest energy and the grid size.\\

\citeauthor{SullMoengStev} \cite{SullMoengStev} showed with effective visual aids some of the details of the dynamics in the \acs{EL}.  It would be important to at least confirm our setup is producing comparable motions of warm and cool air in this region.\\  


\subsection{Local Mixed Layer Heights}

The average profiles must correspond to the average of the local profiles.  Local potential temperatures are expected to be less smooth and their variance in height corresponds to the depth of the average entrainment layer.  \citeauthor{SullMoengStev} \cite{SullMoengStev} used a centered differencing gradient method to find the local convective boundary layer (\acs{CBL}) heights. \citeauthor{BrooksFowler2} in \cite{BrooksFowler2} apply a wavelet of dilation comparable to \acs{EL} depth the to local tracer profiles to determine the location of the \acs{EL} and then a narrower wavelet to determine the limits.  Since the local profiles are not smooth, and there is often upper variabilty of similar and greater magnitude than that which separates the mixed layer (\acs{ML}) from the layer above it the gradient method is not reliable.  A typical tracer profile is quite different to a potential temperature profile IE it goes from a large value in the \acs{CBL} to a much lower value above.  So an ideal approximation might be a step function.  Usually the variation accross the \acs{EL} is much greater than any local variation.  The vertical potential temperature gradient profile in this study could be approximated by step function but there is variance usually above the \acs{EL} which is of comparable magnitude to that accross it.  So a wavelet technique might not be as well suited. \citeauthor{SteynBaldHoff} in \cite{SteynBaldHoff} fitted an idealized curve to tracer profiles (Lidar backscatter).  So we apply this idea and the multi-linear regression method outlined by \citeauthor{Vieth} in \cite{Vieth} to our local potential temperature profiles.  The result is a three line fit, one line for each of the layers IE the \acs{ML} of almost constant potentia temperature, the \acs{EL} over which transition to the stable layer above occurs, and the stable layer of constant lapse rate $\gamma$.  From
this the height of the mixed layer (\acs{ML}) can readily be determined and the distribution of local \acs{ML} heights ($h^{l}_{0}$) should correspond to the concept of the \acs{EL}.  We would expect increased stability ($\gamma$) to reduce the disortion of the inversion interface and so the variation in $h^{l}_{0}$. Increased surface heat flux ($w^{'}\theta^{'}_{s}$) would increase the overall magnitude of $h^{l}_{0}$ so histograms would represent these effects.  We would expect also the $h^{l}_{0}$ surface to correspond to coherent plumes of warm and cool air.  In particular hight $h^{l}_{0}$ should correspond to impinging cool plumes at the top of the \acs{EL}.

\subsection{Flux Quadrants}

Since the average potential temperature profile represents the results of warming, and warming occurrs via the flux of heat from below and above ($\overline{w^{'}\theta^{'}}_{s}$, $\overline{w^{'}\theta^{'}}_{h}$) observation and analyisis of the heat flux ($w^{'}\theta^{'}$)s should give some insight as to the shape of vertical average potential temperature profile.  For instance is there a clear trend in how the upper lapse rate ($\gamma$) influences the extent of downward moving warm air in the the \acs{EL} and so the heating rate from above of the \acs{ML}. Quadrant analyses have been carried on the fluxes from both LES output and measurement (\citeauthor{SullMoengStev}, \cite{SullMoengStev} and \citeauthor{MahrtPaum}, \cite{MahrtPaum}).  The effects of upper lapse rate ($\gamma$) on temperature and velocity perturbations in the \acs{CBL} were analyzed by \citeauthor{Sorbjan} in \cite{Sorbjan}.  Following these three studies we break the $w^{'}\theta^{'}$ into four quadrants to see the vertical average individual profiles and also the joint distributions at the \acs{EL} limits and at the inversion ($h$).  In particular to isolate the effects of $\gamma$ we will scale by the convective velocity and temperature variance scales $w^{*}$ and $\theta^{*}=\frac{\overline{w^{'}\theta^{'}}_{s}}{w^{*}}$.

\subsection{Choice of height definitions}

Bulk models assume that the region of transition from the \acs{ML} value to the free atmospheric value in the potential temperature or buoyancy profile corresponds to the region of negative potential temperature or buoyancy flux (\citeauthor{Deardorff79} \cite{Deardorff79}, \citeauthor{FedConzMir04} \cite{FedConzMir04}).  But although the  height of maximum average potential temperature gradient or average of local maximum potential temperature gradient used as \acs{CBL} height (\citeauthor{SullMoengStev}, \cite{SullMoengStev} and \citeauthor{GarciaMellado} \cite{GarciaMellado}) there does not seem to be an example in the literature where the \acs{EL} limits are defined in terms of the average potential temperature profile for the purpose of measurement. So here, we test this framework in terms of two scaling relationships, IE define both the \acs{CBL} height and \acs{EL} limits in terms of the average vertical potential temperature gradeient. It follows that the temperature jump ($\Delta \theta$) is defined as the difference accross the \acs{EL}.  This is 1st order type definition.  An approximation to the zero order definition would be the difference between the \acs{ML} value and the value at $h$ on the initial temperature profile.       

\subsection{$\frac{w_{e}}{w^{*}}$ vs $Ri$}

The relationship of scaled entrainment velocity to Richardson number \acs{Ri} is well established.  (\citeauthor{Deardorff79} \cite{Deardorff79} \citeauthor{DearWill80}, \cite{DearWill80}, \citeauthor{Stull-BLMetIntro} \cite{Stull-BLMetIntro}) and can be reached by first integrtating the reynolds averaged thermodynamic equation, neglecting radiation, and large scale advection, accross the inversion.  When the inversion depth is assumed infinitessimly small and there is no turbulence at the upper limit, and the flux at the lower limit is assumed a proportion of the surface flux this becomes

\begin{equation}
w_{e} \propto \frac{\overline{w^{'}\theta^{'}}}{\Delta \theta}
\end{equation}

and scaling by the convective velocity scale gives, the following

\begin{equation}
\frac{w_{e}}{w^{*}} \propto Ri^{-1}
\end{equation}

So this relationship is based on a simplified zero order framework.  Strictly the lengthscale and $\Delta \theta$ should be from within this framework.  But we will use our 1st order framework definition to test this scaling relationship for all runs.  We will compare our $\Delta \theta$ with the zero order approximation and those defined elsewhere in the literature.  This may enable comparison with other findings, in regards to this relationship.  There is some discussion in the literature about the power exponent of the Richarson number \acs{Ri} and it seems the two major contenders are $-1$ and $-\frac{3}{2}$.  The latter has been discussed in the context of a shift in entrainment mechanism from eddy turnover to recoil at higher \acs{Ri} (\citeauthor{Turner86} \cite{Turner86}), under conditions of a large upper lapse rate ($\gamma$) \citeauthor{DearWill80} \cite{DearWill80}. \citeauthor{FedConzMir04} \cite{FedConzMir04} asser that it is inferred when the first order $Ri$ is used IE when the temperature jump accross the \acs{EL} is used. Whereas \citeauthor{SullMoengStev} in \cite{SullMoengStev} speculate a power law other than $-1$ may apply at lower $Ri$.  We will plot our data in log-log coordinates to observe which if any power law applies to our results.

\subsection{$\frac{\Delta h}{h}$ vs $Ri$}

Relationship of the scaled \acs{EL} depth to Richardson number arises from a momentum balance where the depth is considered to be related to the overshoot of the plumes (\citeauthor{Stull73} \cite{Stull73}, \citeauthor{DearWill80} \cite{DearWill80}).  This in turn is a function of the plumes velocity at the top of the \acs{CBL} and the buoyancy difference between it and the surrounding stable air.  When the velocity at the top is assumed to be proportional the the convective velocity scale ($w^{*}$) this becomes

\begin{equation}
\frac{\Delta h}{h} \propto Ri^{-1}
\end{equation}

\citeauthor{Boers89} in \cite{Boers89} arrived at a $-\frac{1}{2}$ power law relationship using an energy balance.  IE they equate the difference in potential and thermal energy before and after distortion of the inversion interface to the kinetic energy before distortion. Output and measurements have resulted in power laws ranging from  $-1$ to $-\frac{1}{4}$ (\citeauthor{Traum11} \cite{Traum11}).\\

Given that our principal definition of the \acs{EL} has not been used before we would be happy to see any relationship to \acs{Ri}.  Once we establish there is one, we will use log-log coordinates to see which power law best fits.\\


\endinput

Any text after an \endinput is ignored.
You could put scraps here or things in progress.
%\begin{epigraph}
 %   \emph{If I have seen farther it is by standing on the shoulders of
  %  Giants.} ---~Sir Isaac Newton (1855)
%\end{epigraph}

%This document provides a quick set of instructions for using the
%\class{ubcdiss} class to write a dissertation in \LaTeX. 
%Unfortunately this document cannot provide an introduction to using
%\LaTeX.  The classic reference for learning \LaTeX\ is
%\citeauthor{lamport-1994-ladps}'s
%book~\cite{lamport-1994-ladps}.  There are also many freely-available
%tutorials online;
%\webref{http://www.andy-roberts.net/misc/latex/}{Andy Roberts' online
 %   \LaTeX\ tutorials}
%seems to be excellent.
%The source code for this docment, however, is intended to serve as
%an example for creating a \LaTeX\ version of your dissertation.

%We start by discussing organizational issues, such as splitting
%your dissertation into multiple files, in
%\autoref{sec:SuggestedThesisOrganization}.
%We then cover the ease of managing cross-references in \LaTeX\ in
%\autoref{sec:CrossReferences}.
%We cover managing and using bibliographies with \BibTeX\ in
%\autoref{sec:BibTeX}. 
%We briefly describe typesetting attractive tables in
%\autoref{sec:TypesettingTables}.
%We briefly describe including external figures in
%\autoref{sec:Graphics}, and using special characters and symbols
%in \autoref{sec:SpecialSymbols}.
%As it is often useful to track different versions of your dissertation,
%we discuss revision control further in
%\autoref{sec:DissertationRevisionControl}. 
%We conclude with pointers to additional sources of informat%ion in
%\autoref{sec:Conclusions}.

  
%The \acs{UBC} \acf{FoGS} specifies a particular arrangement of the
%components forming a thesis.\footnote{See
 %   \url{http://www.grad.ubc.ca/current-students/dissertation-thesis-preparation/order-components}}
%This template reflects that arrangement.

%In terms of writing your thesis, the recommended best practice for
%organizing large documents in \LaTeX\ is to place each chapter in
%a separate file.  These chapters are then included from the main
%file through the use of \verb+\include{file}+.  A thesis might
%be described as six files such as \file{intro.tex},
%\file{relwork.tex}, \file{model.tex}, \file{eval.tex},
%\file{discuss.tex}, and \file{concl.tex}.

%We also encourage you to use macros for separating how something
%will be typeset (\eg bold, or italics) from the meaning of that
%something. 
%For example, if you look at \file{intro.tex}, you will see repeated
%uses of a macro \verb+\file{}+ to indicate file names.
%The \verb+\file{}+ macro is defined in the file \file{macros.tex}.
%The consistent use of \verb+\file{}+ throughout the text not only
%indicates that the argument to the macro represents a file (providing
%meaning or semantics), but also allows easily changing how
%file names are typeset simply by changing the definition of the
%\verb+\file{}+ macro.
%\file{macros.tex} contains other useful macros for properly typesetting
%things like the proper uses of the latinate \emph{exempli grati\={a}}
%and \emph{id est} (\ie \verb+\eg+ and \verb+\ie+), 
%web references with a footnoted \acs{URL} (\verb+\webref{url}{text}+),
%as well as definitions specific to this documentation
%(\verb+\latexpackage{}+).

 
%\LaTeX\ make managing cross-references easy, and the \latexpackage{hyperref}
%package's\ \verb+\autoref{}+ command\footnote{%
 %   The \latexpackage{hyperref} package is included by default in this
 
%   template.}
%makes it easier still. 

%A thing to be cross-referenced, such as a section, figure, or equation,
%is \emph{labelled} using a unique, user-provided identifier, defined
%using the \verb+\label{}+ command.  
%The thing is referenced elsewhere using the \verb+\autoref{}+ command.
%For example, this section was defined using:
%\begin{lstlisting}
%    \section{Making Cross-References}
 %   \label{sec:CrossReferences}
%\end{lstlisting}
%References to this section are made as follows:
%\begin{lstlisting}
 %   We then cover the ease of managing cross-references in \LaTeX\
  %  in \autoref{sec:CrossReferences}.
%\end{lstlisting}
%\verb+\autoref{}+ takes care of determining the \emph{type} of the 
%thing being referenced, so the example above is rendered as
%\begin{quote}
%    We then cover the ease of managing cross-references in \LaTeX\
%    in \autoref{sec:CrossReferences}.
%\end{quote}

%The label is any simple sequence of characters, numbers, digits,
%and some punctuation marks such as ``:'' and ``--''; there should
%be no spaces.  Try to use a consistent key format: this simplifies
%remembering how to make references.  This document uses a prefix
%to indicate the type of the thing being referenced, such as \texttt{sec}
%for sections, \texttt{fig} for figures, \texttt{tbl} for tables,
%and \texttt{eqn} for equations.

%For details on defining the text used to describe the type
%of \emph{thing}, search \file{diss.tex} and the \latexpackage{hyperref}
%documentation for \texttt{autorefname}.
