%% The following is a directive for TeXShop to indicate the main file
%%!TEX root = diss.tex

\chapter{Introduction}
\label{ch:Introduction}

%\begin{epigraph}
 %   \emph{If I have seen farther it is by standing on the shoulders of
  %  Giants.} ---~Sir Isaac Newton (1855)
%\end{epigraph}

%This document provides a quick set of instructions for using the
%\class{ubcdiss} class to write a dissertation in \LaTeX. 
%Unfortunately this document cannot provide an introduction to using
%\LaTeX.  The classic reference for learning \LaTeX\ is
%\citeauthor{lamport-1994-ladps}'s
%book~\cite{lamport-1994-ladps}.  There are also many freely-available
%tutorials online;
%\webref{http://www.andy-roberts.net/misc/latex/}{Andy Roberts' online
 %   \LaTeX\ tutorials}
%seems to be excellent.
%The source code for this docment, however, is intended to serve as
%an example for creating a \LaTeX\ version of your dissertation.

%We start by discussing organizational issues, such as splitting
%your dissertation into multiple files, in
%\autoref{sec:SuggestedThesisOrganization}.
%We then cover the ease of managing cross-references in \LaTeX\ in
%\autoref{sec:CrossReferences}.
%We cover managing and using bibliographies with \BibTeX\ in
%\autoref{sec:BibTeX}. 
%We briefly describe typesetting attractive tables in
%\autoref{sec:TypesettingTables}.
%We briefly describe including external figures in
%\autoref{sec:Graphics}, and using special characters and symbols
%in \autoref{sec:SpecialSymbols}.
%As it is often useful to track different versions of your dissertation,
%we discuss revision control further in
%\autoref{sec:DissertationRevisionControl}. 
%We conclude with pointers to additional sources of informat%ion in
%\autoref{sec:Conclusions}.

Scales of the lower Convective Boundary Layer are well known.  Much work has been done on those of the Entrainment Region but it does not seem to have been as yet organized in the form of a scaling diagram.  

%%%%%%%%%%%%%%%%%%%%%%%%%%%%%%%%%%%%%%%%%%%%%%%%%%%%%%%%%%%%%%%%%%%%%%
\section{Why do we care about Convective Boundary Layer Scales}
\label{sec:WhydowecareaboutConvectiveBoundaryLayerScales}

In general scales enable simplification of the expressions that describe a process by elimination 
of deemed insignificant terms, getting rid of units and reducing magnitude to order 1.\\

They enable parametrisations?  Need to find out about this or come up with nice way to say it.\\

  
%The \acs{UBC} \acf{FoGS} specifies a particular arrangement of the
%components forming a thesis.\footnote{See
 %   \url{http://www.grad.ubc.ca/current-students/dissertation-thesis-preparation/order-components}}
%This template reflects that arrangement.

%In terms of writing your thesis, the recommended best practice for
%organizing large documents in \LaTeX\ is to place each chapter in
%a separate file.  These chapters are then included from the main
%file through the use of \verb+\include{file}+.  A thesis might
%be described as six files such as \file{intro.tex},
%\file{relwork.tex}, \file{model.tex}, \file{eval.tex},
%\file{discuss.tex}, and \file{concl.tex}.

%We also encourage you to use macros for separating how something
%will be typeset (\eg bold, or italics) from the meaning of that
%something. 
%For example, if you look at \file{intro.tex}, you will see repeated
%uses of a macro \verb+\file{}+ to indicate file names.
%The \verb+\file{}+ macro is defined in the file \file{macros.tex}.
%The consistent use of \verb+\file{}+ throughout the text not only
%indicates that the argument to the macro represents a file (providing
%meaning or semantics), but also allows easily changing how
%file names are typeset simply by changing the definition of the
%\verb+\file{}+ macro.
%\file{macros.tex} contains other useful macros for properly typesetting
%things like the proper uses of the latinate \emph{exempli grati\={a}}
%and \emph{id est} (\ie \verb+\eg+ and \verb+\ie+), 
%web references with a footnoted \acs{URL} (\verb+\webref{url}{text}+),
%as well as definitions specific to this documentation
%(\verb+\latexpackage{}+).



%%%%%%%%%%%%%%%%%%%%%%%%%%%%%%%%%%%%%%%%%%%%%%%%%%%%%%%%%%%%%%%%%%%%%%
\section{Scales of the Connvective Boundary Layer}
\label{sec:ScalesoftheConnvectiveBoundaryLayer}
The convective boundary layer is turbulent due to convection.\\

The dry case in the absence of horizontal winds.\\

Important Scales: Monin Obvukov length, Richardson number, $\frac{h}{L}$

scaling diagrams for the boundary layer.\\
 
%\LaTeX\ make managing cross-references easy, and the \latexpackage{hyperref}
%package's\ \verb+\autoref{}+ command\footnote{%
 %   The \latexpackage{hyperref} package is included by default in this
 
%   template.}
%makes it easier still. 

%A thing to be cross-referenced, such as a section, figure, or equation,
%is \emph{labelled} using a unique, user-provided identifier, defined
%using the \verb+\label{}+ command.  
%The thing is referenced elsewhere using the \verb+\autoref{}+ command.
%For example, this section was defined using:
%\begin{lstlisting}
%    \section{Making Cross-References}
 %   \label{sec:CrossReferences}
%\end{lstlisting}
%References to this section are made as follows:
%\begin{lstlisting}
 %   We then cover the ease of managing cross-references in \LaTeX\
  %  in \autoref{sec:CrossReferences}.
%\end{lstlisting}
%\verb+\autoref{}+ takes care of determining the \emph{type} of the 
%thing being referenced, so the example above is rendered as
%\begin{quote}
%    We then cover the ease of managing cross-references in \LaTeX\
%    in \autoref{sec:CrossReferences}.
%\end{quote}

%The label is any simple sequence of characters, numbers, digits,
%and some punctuation marks such as ``:'' and ``--''; there should
%be no spaces.  Try to use a consistent key format: this simplifies
%remembering how to make references.  This document uses a prefix
%to indicate the type of the thing being referenced, such as \texttt{sec}
%for sections, \texttt{fig} for figures, \texttt{tbl} for tables,
%and \texttt{eqn} for equations.

%For details on defining the text used to describe the type
%of \emph{thing}, search \file{diss.tex} and the \latexpackage{hyperref}
%documentation for \texttt{autorefname}.


%%%%%%%%%%%%%%%%%%%%%%%%%%%%%%%%%%%%%%%%%%%%%%%%%%%%%%%%%%%%%%%%%%%%%%
\section{Scales for the Entrainmnet Region}
\label{sec:ScalesfortheEntrainmnetRegion}


\endinput

Any text after an \endinput is ignored.
You could put scraps here or things in progress.
