%% The following is a directive for TeXShop to indicate the main file
%%!TEX root = diss.tex

\chapter{Abstract}
\setlength{\parindent}{0cm}
The atmospheric convective boundary layer has been studied for over thirty years in order to understand the dynamics and scaling behaviour of its growth by entrainment.  This enables prediction of its entrainment rate and entrainment zone depth, and so parameterizations thereof for use in global circulation models.\\

Fundamentals, such as the dependence of the entrainment rate and entrainment zone depth on the convective Richardson number, have been established but there is still unresolved discussion about the form of these relationships.  Details regarding the structure of the entrainment zone continue to emerge.  The variety of convective boundary layer height and entrainment zone depth definitions adds further complexity.  The study described in this thesis aims to join this ongoing discussion.\\

A dry, shear-free, idealized convective boundary layer in the absence of large scale winds was modeled using a large eddy simulation.  The use of ten ensemble cases enabled calculation of true ensemble averages and potential temperature fluctuations as well as providing smooth average profiles.  A range of convective Richardson numbers was achieved by varying the two principle external parameters: surface vertical heat flux and stable upper lapse rate.\\

The gradient method for determining local convective boundary layer height was found to be unreliable so a multi-linear regression method was used instead.  Distributions of the local heights thus determined were found to narrow with increased upper stability.  Height and entrainment zone depth were then defined based on the ensemble and horizontally averaged potential temperature profile.  The resulting relationships of entrainment rate and entrainment zone depth to Richardson number showed behaviour in general agreement with theory and the results of other studies.  The potential temperature gradient in the upper convective boundary layer and entrainment zone was seen to depend on the upper lapse rate, as was the positive downward moving temperature fluctuations at the \acs{CBL} top.  Overall, once the surface heat flux was accounted for by applying the \acs{CBL} height as a scale, the upper lapse rate emerged as the dominant parameter influencing scaled entrainment zone depth, and potential temperature variance in the entrainment zone and upper convective boundary layer.   
%This document provides brief instructions for using the \class{ubcdiss}
%class to write a \acs{UBC}-conformant dissertation in \LaTeX.  This
%document is itself written using the \class{ubcdiss} class and is
%intended to serve as an example of writing a dissertation in \LaTeX.
%This document has embedded \acp{URL} and is intended to be viewed
%using a computer-based \ac{PDF} reader.

%Note: Abstracts should generally try to avoid using acronyms.

%Note: at \ac{UBC}, both the \ac{FoGS} Ph.D. defence programme and the
%Library's online submission system restricts abstracts to 350
%words.

% Consider placing version information if you circulate multiple drafts
%\vfill
%\begin{center}
%\begin{sf}
%\fbox{Revision: \today}
%\end{sf}
%\end{center}
